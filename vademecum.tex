\documentclass[landscape,12pt]{report}

\usepackage[a5paper,hmargin=1.5cm]{geometry}
% tmargin=2.8cm,bmargin=0cm
%                ,hmargin=1.0cm,vmargin=1.0cm
%                tmargin=1.2cm, %top margin = 1.2 cm
%                bmargin=0.8cm, %bottom margin = 0.8 cm


\usepackage[utf8x]{inputenc}
\usepackage[spanish]{babel}
\usepackage[T1]{fontenc}
\usepackage{fancyhdr}
\usepackage{graphicx}

%\paperwidth=21cm
%\paperheight=14.85cm
%\voffset=-1.5cm
%\hoffset=-0.5cm
%\topmargin=0pt
%\oddsidemargin=0cm
%\evensidemargin=0cm


%\topskip=0pt
\headheight=28pt

%\textheight=7cm
%\textwidth=17cm


\newcommand{\changefont}[3]{\fontfamily{#1} \fontseries{#2} \fontshape{#3} \selectfont}

\begin{document}

\thispagestyle{empty}

\begin{center}
{\changefont{cmss}{bx}{n} \Huge VADEMECUM}\\
\vspace{6mm}
{\changefont{cmss}{bx}{n} \Huge de cantos litúrgicos}
\end{center}

%\vspace{20mm}

\begin{center}
%{\changefont{cmr}{bx}{n} \LARGE Parroquia Inmaculada Concepción (B)}
\end{center}

%------------------------
\newpage
\thispagestyle{empty}

%------------------------
\newpage
\thispagestyle{empty}

%\begin{center}
%{\changefont{cmss}{bx}{n} \Huge VADEMECUM}\\
%\vspace{6mm}
%{\changefont{cmss}{bx}{n} \Huge de cantos litúrgicos}
%\end{center}

\vspace*{30mm}

\begin{center}
{\changefont{cmss}{bx}{n} \Huge CICLO A  (Mateo)}
%{\changefont{cmr}{bx}{n} \Huge Ciclo A}
\end{center}
%------------------------
\newpage
\thispagestyle{empty}
%------------------------

\newpage
\setcounter{footnote}{0}
\thispagestyle{fancy}
\fancyhead{}
\fancyfoot{}
%\footskip=-1cm

\lhead{\changefont{cmss}{bx}{n} \small Revisión 2013}
\chead{\changefont{cmss}{bx}{n} \small Uso del vademecum}
\rhead{\changefont{cmss}{bx}{n} \small guillermo.frank@gmail.com}
\rfoot{\changefont{cmss}{bx}{n}\large\thepage}


\noindent Este \emph{vademecum} es una propuesta de salmos y cantos para la 
liturgia dominical y solemnidades. El criterio seguido es el siguiente

\begin{itemize}
\item Dentro de cada categoría (entrada, ofertorio, etc.), el orden en que se 
enumeran los cantos va desde el más recomendado hasta el más general para la 
liturgia de ese día.
\item Los salmos se basan en los \emph{72 salmos} del Padre Osvaldo Catena. 
Pueden no coincidir exactamente con el salmo del día ya que en muchos casos 
tuvieron que buscarse los reemplazos correspondientes dentro de estos \emph{72 
salmos}. 
\item Bajo la aclaración ``antífona de reemplazo'' se detalla un grupo de no 
más de 20 antífonas para el caso en que se lee la estrofa (sólo se canta la 
antífona). Esta facilidad está dirigida a asambleas que recién comienzan a 
cantar los salmos. Por ese motivo se seleccionaron antífonas simples y breves. 
Si no figura la aclaración ``antífona de reemplazo'', se utiliza la indicada 
para el día.
\item Al final de cada ciclo litúrgico se incluyen algunas solemnidades de 
fecha fija, que en algunos años pueden caer en domingo.
\item El repertorio usado corresponde a cantos muy conocidos. No es exhaustivo, 
ni intenta cerrarse sobre este único repertorio. \'Este surgió de la 
experiencia y compilación de muchos años realizada por el organista Gustavo 
Vargas ($\dagger$ junio de 2014). 
\end{itemize}


\begin{flushright}
{\changefont{cmss}{bx}{n}\small Guillermo Frank}
\end{flushright}

%------------------------
\newpage
\setcounter{footnote}{0}
\thispagestyle{fancy}
\fancyhead{}
\fancyfoot{}
%\footskip=-1cm

\lhead{\changefont{cmss}{bx}{n} \small Revisión 2013}
\chead{\changefont{cmss}{bx}{n} \small Domingo I - Adviento}
\rhead{\changefont{cmss}{bx}{n} \small Ciclo A}
\rfoot{\changefont{cmss}{bx}{n}\large\thepage}

\begin{center}
{\large\it Exhortación a la vigilancia y a la fidelidad}
\end{center}

\vspace{3mm}

\begin{tabbing}

{\changefont{cmss}{bx}{n} Entrada:\ \ \ \ \ }\= Despertemos llega Cristo\footnotemark[1], Juntos como hermanos (estr. 2)\footnotemark[2], Toda la tierra espera, \\ 
 \> Señor a ti clamamos (se puede usar para el encendido de la corona de Adviento). \\ \\

{\changefont{cmss}{bx}{n} Salmo:} \> 121 ant. 1 ``!`Cómo me alegré cuando me dijeron...'' (estr. 1, 2, 4 o 5). \\ 
\> (antífona de reemplazo: Sal 84 ant. 1 ``Señor, revélanos tu amor...'') \\ \\

{\changefont{cmss}{bx}{n} Ofrendas:} \> Este es nuestro pan\footnotemark[3], Ofrenda de amor, Toda la tierra espera.\\ \\

{\changefont{cmss}{bx}{n} Comunión:} \> Yo soy el camino, Bendeciré al Señor.\\ \\

{\changefont{cmss}{bx}{n} Post-com.:} \> Vaso nuevo (El alfarero)\footnotemark[4], Nada te turbe (de Taizé).\\ \\

{\changefont{cmss}{bx}{n} Salida:} \> Santa María del camino.  \\  \\

\end{tabbing}

\vspace{-10mm}

\footnotetext[1]{Debido al texto de la 2$^\circ$ lectura (Rom 13,11). }
\footnotetext[2]{Ver comentario a \textbf{Juntos como hermanos}. }
\footnotetext[3]{Porque en la estr. 3 habla de ``nuestra libertad'', apropiado para este tiempo de espera.}
\footnotetext[4]{Se hace eco de la 2$^\circ$ lectura (Rom 13,14).}

%------------------------
\newpage
\setcounter{footnote}{0}
\thispagestyle{fancy}
\fancyhead{}
\fancyfoot{}
%\footskip=-1cm


\lhead{\changefont{cmss}{bx}{n} \small Revisión 2013}
\chead{\changefont{cmss}{bx}{n} \small Domingo II - Adviento}
\rhead{\changefont{cmss}{bx}{n} \small Ciclo A}
\rfoot{\changefont{cmss}{bx}{n}\large\thepage}

\begin{center}
{\large\it La predicación de Juan el Bautista}
\end{center}

%\vspace{3mm}


\begin{tabbing}

{\changefont{cmss}{bx}{n} Entrada:\ \ \ \ \ }\= Toda la tierra espera (estr. 1 y 3)\footnotemark[1], Despertemos llega Cristo, \\ 
 \> Señor a ti clamamos (se puede usar para el encendido de la corona de Adviento). \\ \\

{\changefont{cmss}{bx}{n} Salmo:} \> 71 ant. 1 ``Tú eres, Señor, el único Rey...'' (estr. 1, 4, 6 u 8). \\
 \> (antífona de reemplazo: Sal 144 ant. 1 ``Te alabamos Señor...'') \\ \\

{\changefont{cmss}{bx}{n} Ofrendas:} \> Toda la tierra espera (si no se usa de entrada), Pan de vida y bebida de  luz, \\
\> Saber que vendrás\footnotemark[2], Te ofrecemos Padre nuestro (II).\\ \\

{\changefont{cmss}{bx}{n} Comunión:} \> Creo en ti Señor\footnotemark[3], Mensajero de la paz\footnotemark[3], Yo soy el camino.\\ \\

{\changefont{cmss}{bx}{n} Post-com.:} \> Adoremos a Dios (estr. 3), Mirarte sólo a ti, Nada te turbe, El Señor es mi fortaleza.\\ \\

{\changefont{cmss}{bx}{n} Salida:} \> Madre de los peregrinos.  \\  \\

\end{tabbing}

\vspace{-10mm}

\footnotetext[1]{El evangelio hace referencia a Is 40,3-5 (est. 1 y 3 de \textbf{Toda la tierra espera}). }
\footnotetext[2]{Ver inconvenientes detallados en el comentario a \textbf{Saber que vendrás}. }
\footnotetext[3]{Debido a los textos equivalentes en Lucas (Lc 3,3-9) del evangelio del día (Mt 3,1-12).}

%------------------------
\newpage
\setcounter{footnote}{0}
\thispagestyle{fancy}
\fancyhead{}
\fancyfoot{}
%\footskip=-1cm


\lhead{\changefont{cmss}{bx}{n} \small Revisión 2013}
\chead{\changefont{cmss}{bx}{n} \small Solemnidad del 8 de diciembre}
\rhead{\changefont{cmss}{bx}{n} \small Ciclo A}
\rfoot{\changefont{cmss}{bx}{n}\large\thepage}

\begin{center}
{\large\it Inmaculada Concepción de María}
\end{center}

%\vspace{3mm}


\begin{tabbing}

{\changefont{cmss}{bx}{n} Entrada:\ \ \ \ \ }\= Feliz de ti María, La Virgen María nos reúne. \\ \\

{\changefont{cmss}{bx}{n} Salmo:} \> 97 ant. 1 ``Cantemos al Señor un canto nuevo, aleluia...'' (estr. 1, 2 o 3). \\ 
\> (antífona de reemplazo: Sal 95 ``Cantemos al Señor un canto nuevo''  \\
\> del P. José Bevilacqua)\\ \\

{\changefont{cmss}{bx}{n} Ofrendas:} \> Coplas de Yaraví, Bendeciré al Señor.   \\ \\


{\changefont{cmss}{bx}{n} Comunión:} \>  Mi alma glorifica, Jesucristo danos de este pan\footnotemark[1].\\ \\

{\changefont{cmss}{bx}{n} Post-com.:} \> Bendita sea tu pureza, Quiero decir que sí.\\ \\

{\changefont{cmss}{bx}{n} Salida:} \> Toda de Dios, Los cielos, la tierra, El ángel vino de los cielos,\\
\> Oh Santísima, Oh María, Madre de nuestro pueblo. \\  \\

\end{tabbing}

\vspace{-10mm}

\footnotetext[1]{Porque en su última estrofa se refiere a María. }

%------------------------
\newpage
\setcounter{footnote}{0}
\thispagestyle{fancy}
\fancyhead{}
\fancyfoot{}
%\footskip=-1cm

\lhead{\changefont{cmss}{bx}{n} \small Revisión 2013}
\chead{\changefont{cmss}{bx}{n} \small Domingo III - Adviento}
\rhead{\changefont{cmss}{bx}{n} \small Ciclo A}
\rfoot{\changefont{cmss}{bx}{n}\large\thepage}

\begin{center}
{\large\it (Domingo de Gaudete) Testimonio de Jesús sobre Juan el Bautista }
\end{center}

\vspace{2mm}


\begin{tabbing}

{\changefont{cmss}{bx}{n} Entrada:\ \ \ \ \ }\= Que alegría (Sal 121)\footnotemark[1], Vienen con alegría\footnotemark[1], Señor a ti clamamos\footnotemark[1] (corona), \\
\> Toda la tierra espera\footnotemark[2], Despertemos llega Cristo. \\ \\

{\changefont{cmss}{bx}{n} Salmo:} \> 145 ant. 1 ``El Señor es fiel a su palabra...'' (estr 4, 5 o 6). \\ 
\> (antífona de reemplazo: Sal 84 ant. 1 ``Señor, revélanos tu amor...'') \\ \\

{\changefont{cmss}{bx}{n} Ofrendas:} \> Toda la tierra espera\footnotemark[2], Pan de vida y bebida de luz, Al altar del Señor.\\ \\

{\changefont{cmss}{bx}{n} Comunión:} \> Mensajero de la paz\footnotemark[3], Yo soy el camino, Creo en ti Señor\footnotemark[4], Bendeciré al Señor.\\ \\

{\changefont{cmss}{bx}{n} Post-com.:} \> Tan cerca de mí\footnotemark[4], El Señor es mi fortaleza.\\ \\

{\changefont{cmss}{bx}{n} Salida:} \> Salve María, Salve oh Reina, Mi camino eres tú\footnotemark[4].  \\  \\

\end{tabbing}

\vspace{-6mm}

\footnotetext[1]{Por ser domingo de \emph{Gaudete}. Sino, \emph{Señor a ti clamamos} debido a Is 35,4. }
\footnotetext[2]{Cfr. Sant 5,7 en la 2$^\circ$ lectura. }
\footnotetext[3]{Cfr. el paralelismo con Mt 11,10 en el evangelio del día. }
\footnotetext[4]{Porque se revela que Jesús es el Salvador.}

%------------------------
\newpage
\setcounter{footnote}{0}
\thispagestyle{fancy}
\fancyhead{}
\fancyfoot{}
%\footskip=-1cm

\lhead{\changefont{cmss}{bx}{n} \small Revisión 2013}
\chead{\changefont{cmss}{bx}{n} \small Domingo IV - Adviento}
\rhead{\changefont{cmss}{bx}{n} \small Ciclo A}
\rfoot{\changefont{cmss}{bx}{n}\large\thepage}

\begin{center}
{\large\it La concepción virginal y el nacimiento de Jesús }
\end{center}

\vspace{1mm}


\begin{tabbing}

{\changefont{cmss}{bx}{n} Entrada:\ \ \ \ \ }\= Toda la tierra espera (estr. 2)\footnotemark[1], Despertemos llega Cristo, Señor a ti clamamos.  \\ 
 \> (Señor a ti clamamos se puede usar para el encendido de la corona de Adviento) \\ \\

{\changefont{cmss}{bx}{n} Salmo:} \> 23 ant. 1 ``Abranse, puertas eternas, para que entre...'' (est. 1, 2 o 3). \\ 
\> (antífona de reemplazo: Sal 26 ant. 1 ``Cantaré y celebraré al Señor.'') \\ \\

{\changefont{cmss}{bx}{n} Ofrendas:} \> Saber que vendrás, Toda la tierra espera, Pan de vida y bebida de luz,\\
\>  Te ofrecemos Padre nuestro (II)\footnotemark[2].\\ \\

{\changefont{cmss}{bx}{n} Comunión:} \> Jesucristo danos de este pan\footnotemark[3], El viñador, Creo en ti Señor, Bendeciré al Señor.\\ \\

{\changefont{cmss}{bx}{n} Post-com.:} \> Adoremos a Dios (estr. 1), Alabe todo el mundo (de Taizé).\\ \\

{\changefont{cmss}{bx}{n} Salida:} \> Madre de nuetro pueblo (estr. 1-3), El ángel vino de los cielos\footnotemark[4].  \\  \\

\end{tabbing}

\vspace{-11mm}

\footnotetext[1]{Debido a que se refiere al texto de la 1$^\circ$ lectura Is 7,14. }
\footnotetext[2]{Ver comentario a \textbf{Te ofrecemos Padre nuestro (II)}. }
\footnotetext[3]{Porque en su última estrofa habla de María.}
\footnotetext[4]{Como el texto corresponde al \emph{Angelus}, debe cantarse completo. }

%------------------------
\newpage
\setcounter{footnote}{0}
\thispagestyle{fancy}
\fancyhead{}
\fancyfoot{}
%\footskip=-1cm

\lhead{\changefont{cmss}{bx}{n} \small Revisión 2013}
\chead{\changefont{cmss}{bx}{n} \small Solemnidad del 25 de diciembre - Natividad del Señor}
\rhead{\changefont{cmss}{bx}{n} \small Ciclo A}
\rfoot{\changefont{cmss}{bx}{n}\large\thepage}

\begin{center}
{\large\it Prólogo al evangelio según San Juan }
\end{center}

\vspace{1mm}


\begin{tabbing}

{\changefont{cmss}{bx}{n} Entrada:\ \ \ \ \ }\= Ha nacido el Rey del cielo\footnotemark[1], Gloria eterna\footnotemark[2], Mundo feliz\footnotemark[1], Pastores de la montaña\footnotemark[1], \\
\> Cristianos vayamos\footnotemark[1], Vamos pastorcitos\footnotemark[1], Ya llegó la Nochebuena\footnotemark[1].  \\  \\

{\changefont{cmss}{bx}{n} Salmo:} \> 97 con ant. del Sal 95  ``Hoy nos ha nacido un salvador...'' (est. 1, 5 o 6). \\ \\

{\changefont{cmss}{bx}{n} Ofrendas:} \> Cristianos vayamos\footnotemark[1], Vamos pastorcitos\footnotemark[1], Ya llegó la Nochebuena\footnotemark[1].\\ \\

{\changefont{cmss}{bx}{n} Comunión:} \> Pastores de la montaña\footnotemark[1], El pan de Belén\footnotemark[3], Noche de paz\footnotemark[3],\\
\> La peregrinación (A la huella)\footnotemark[3], Zamba de la Navidad. \\ \\

{\changefont{cmss}{bx}{n} Post-com.:} \> Noche anunciada\footnotemark[3], ?`Qué niño es éste?, Adoremos a Dios (estr. 1), \\
\> Alabe todo el mundo (de Taizé).\\ \\

{\changefont{cmss}{bx}{n} Salida:} \> Gloria eterna\footnotemark[2], Pastores de la montaña\footnotemark[1], Mundo feliz\footnotemark[1], Entonen tiernos cánticos.  \\  \\

\end{tabbing}

\vspace{-14mm}

\footnotetext[1]{Cfr. \emph{Ya llegó la Nochebuena} con Mt 1,25, \emph{Mundo feliz} con Sal 97, \emph{Pastores de la montaña} con Lc 2,8-16, \\ \emph{Vamos pastorcitos} con Lc 2,8-20 y \emph{Cristianos vayamos} con Lc 2,15-18. }
\footnotetext[2]{Ver comentario a \textbf{Gloria eterna}. Cfr. con Is 9,1-2 (1$^\circ$ lectura de la misa de la noche). }
\footnotetext[3]{Cfr. \emph{El pan de Belén} con Mt 1,23, \emph{La peregrinación} con Lc 2,6-7, \emph{Noche de paz} y \emph{Noche anunciada} con Lc 2,6-20.}

%------------------------
\newpage
\setcounter{footnote}{0}
\thispagestyle{fancy}
\fancyhead{}
\fancyfoot{}
%\footskip=-1cm

\lhead{\changefont{cmss}{bx}{n} \small Revisión 2013}
\chead{\changefont{cmss}{bx}{n} \small Domingo de la octava de Navidad}
\rhead{\changefont{cmss}{bx}{n} \small Ciclo A}
\rfoot{\changefont{cmss}{bx}{n}\large\thepage}

\begin{center}
{\large\it (La Sagrada Familia) El exilio de Jesús en Egipto }
\end{center}

\vspace{1mm}


\begin{tabbing}

{\changefont{cmss}{bx}{n} Entrada:\ \ \ \ \ }\= Ya llegó la Nochebuena (texto modificado), Gloria eterna, Mundo feliz, \\ 
\> Ha nacido el Rey del cielo, Pastores de la montaña, Cristianos vayamos,   \\ 
\> Vamos pastorcitos.\\ \\

{\changefont{cmss}{bx}{n} Salmo:} \> 127 ant. 1 ``!`Feliz quien ama al Señor...!'' (est. 1, 2). \\ \\

{\changefont{cmss}{bx}{n} Ofrendas:} \> Cristianos vayamos, Vamos pastorcitos, Ya llegó la Nochebuena.\\ \\

{\changefont{cmss}{bx}{n} Comunión:} \> La peregrinación (A la huella), Noche de paz, El pan de Belén,  \\
\>  Pastores de la montaña, Zamba de la Navidad. \\ \\

{\changefont{cmss}{bx}{n} Post-com.:} \> Noche anunciada, ?`Qué niño es éste?, Adoremos a Dios (estr. 3), \\
\> Alabe todo el mundo (de Taizé).\\ \\

{\changefont{cmss}{bx}{n} Salida:} \> Madre de nuestro pueblo (estr. 1,4,6)\footnotemark[1], Gloria eterna, Pastores de la montaña,\\
\>  Mundo feliz, Entonen tiernos cánticos.  \\  \\

\end{tabbing}

\vspace{-11mm}

\footnotetext[1]{Cfr. con Mt 2,13-15. }


%------------------------
\newpage
\setcounter{footnote}{0}
\thispagestyle{fancy}
\fancyhead{}
\fancyfoot{}
%\footskip=-1cm

\lhead{\changefont{cmss}{bx}{n} \small Revisión 2013}
\chead{\changefont{cmss}{bx}{n} \small Solemnidad del 1$^\circ$ de enero}
\rhead{\changefont{cmss}{bx}{n} \small Ciclo A}
\rfoot{\changefont{cmss}{bx}{n}\large\thepage}

\begin{center}
{\large\it Santa María Madre de Dios }
\end{center}

%\vspace{1mm}


\begin{tabbing}

{\changefont{cmss}{bx}{n} Entrada:\ \ \ \ \ }\= María Madre de Dios\footnotemark[1], Feliz de ti María (est. 1,2)\footnotemark[1], La Virgen María nos reúne\footnotemark[1], \\ 
\> Gloria eterna\footnotemark[2].\\ \\

{\changefont{cmss}{bx}{n} Salmo:} \> 66 ant. 1 ``!`A tí, Señor, te alabe la tierra...!'' (est. 1, 2). \\
\> (antífona de reemplazo: Sal 144 ant. 1 ``Te alabamos, Señor, ...'') \\ \\

{\changefont{cmss}{bx}{n} Ofrendas:} \> Bendeciré al Señor, Cristianos vayamos, Vamos pastorcitos.\\ \\

{\changefont{cmss}{bx}{n} Comunión:} \> Simple oración\footnotemark[3], La peregrinación (A la huella), El pan de Belén,  \\
\>  Pastores de la montaña. \\ \\

{\changefont{cmss}{bx}{n} Post-com.:} \> Quiero decir que sí\footnotemark[1], Noche anunciada, Adoremos a Dios (estr. 3), \\
\> Alabe todo el mundo (de Taizé).\\ \\

{\changefont{cmss}{bx}{n} Salida:} \> Salve María\footnotemark[1], Oh María\footnotemark[1], Junto a ti María\footnotemark[1],  \\
\> Madre de nuestro pueblo (estr. 1,4,5)\footnotemark[1], Gloria eterna\footnotemark[2].\\ \\

\end{tabbing}

\vspace{-14mm}
\footnotetext[1]{Por ser la solemnidad de Santa María Madre de Dios.}
\footnotetext[2]{Debido a que la antífona de entrada del día coincide con \emph{Gloria eterna} en que hacen referencia a Is 9,2. }
\footnotetext[3]{Debido a que es la Jornada mundial por la paz, en concordancia en el estribillo de \emph{Simple oración}. }

%------------------------
\newpage
\setcounter{footnote}{0}
\thispagestyle{fancy}
\fancyhead{}
\fancyfoot{}
%\footskip=-1cm

\lhead{\changefont{cmss}{bx}{n} \small Revisión 2013}
\chead{\changefont{cmss}{bx}{n} \small Domingo II - Navidad}
\rhead{\changefont{cmss}{bx}{n} \small Ciclo A}
\rfoot{\changefont{cmss}{bx}{n}\large\thepage}

\begin{center}
{\large\it Prólogo al evangelio según San Juan }
\end{center}

\vspace{1mm}


\begin{tabbing}

{\changefont{cmss}{bx}{n} Entrada\footnotemark[1]:\ \ \ \ \ }\= Ha nacido el Rey del cielo, Gloria eterna, Mundo feliz, Pastores de la montaña, \\
\> Cristianos vayamos, Vamos pastorcitos.  \\  \\

{\changefont{cmss}{bx}{n} Salmo:} \> 147 ant. 1 ``Glorifica al Señor Jerusalén...'' (est. 1, 2 o 4). \\ \\

{\changefont{cmss}{bx}{n} Ofrendas:} \> Vamos pastorcitos, Cristianos vayamos, Ya llegó la Nochebuena.\\ \\

{\changefont{cmss}{bx}{n} Comunión:} \> El pan de Belén, Pastores de la montaña, La peregrinación (A la huella). \\  \\

{\changefont{cmss}{bx}{n} Post-com.:} \> Adoremos a Dios (estr. 1), Alabe todo el mundo (de Taizé), Noche anunciada.\\ \\

{\changefont{cmss}{bx}{n} Salida:} \> Gloria eterna, Pastores de la montaña, Mundo feliz, Entonen tiernos cánticos, \\
\>  Oh Santísima.  \\  \\

\end{tabbing}

\vspace{-11mm}

\footnotetext[1]{Se repiten muchos de los cantos del día de Navidad. El evangelio del día es el mismo que el de Navidad. }

%------------------------
\newpage
\setcounter{footnote}{0}
\thispagestyle{fancy}
\fancyhead{}
\fancyfoot{}
%\footskip=-1cm

\lhead{\changefont{cmss}{bx}{n} \small Revisión 2013}
\chead{\changefont{cmss}{bx}{n} \small Solemnidad del 6 de enero - Navidad }
\rhead{\changefont{cmss}{bx}{n} \small Ciclo A}
\rfoot{\changefont{cmss}{bx}{n}\large\thepage}

\begin{center}
{\large\it (La Epifanía del Señor) La visita de los magos }
\end{center}

%\vspace{1mm}

\begin{tabbing}

{\changefont{cmss}{bx}{n} Entrada:\ \ \ \ \ }\= Mundo feliz, Ha nacido el Rey del cielo, Gloria eterna,    \\ 
\> Ya llegó la Nochebuena (texto modificado)\footnotemark[1], Cristianos vayamos\footnotemark[2].   \\ \\

{\changefont{cmss}{bx}{n} Salmo:} \> 71 ant. 1 ``Tú eres, Señor, el único Rey...'' (est. 1, 4 , 5 o 6). \\
\> (antífona de reemplazo: Sal 144 ant. 1 ``Te alabamos Señor...'') \\ \\

{\changefont{cmss}{bx}{n} Ofrendas:} \> Cristianos vayamos\footnotemark[2], Ya llegó la Nochebuena\footnotemark[1], Vamos pastorcitos.\\ \\

{\changefont{cmss}{bx}{n} Comunión:} \> Los reyes magos, El pan de Belén, La peregrinación (A la huella),    \\
\>  Pastores de la montaña, Zamba de la Navidad, Noche de paz. \\ \\

{\changefont{cmss}{bx}{n} Post-com.:} \> Adoremos a Dios (estr. 1), Alabe todo el mundo (de Taizé), ?`Qué niño es éste?,   \\
\> Noche anunciada.\\ \\

{\changefont{cmss}{bx}{n} Salida:} \> Gloria eterna, Pastores de la montaña, Mundo feliz, Entonen tiernos cánticos,  \\
\>  Madre de nuestro pueblo (estr. 1 y 4).  \\  \\

\end{tabbing}

\vspace{-11mm}

\footnotetext[1]{Con el texto modificado: ``Vamos todos a esperarlo'' por ``Vamos todos a adorarlo'', y ``va a nacer'' por ``ya nació''. }
\footnotetext[2]{Cfr. con Mt 2,2 y Mt 2,11. }


%------------------------
\newpage
\setcounter{footnote}{0}
\thispagestyle{fancy}
\fancyhead{}
\fancyfoot{}
%\footskip=-1cm

\lhead{\changefont{cmss}{bx}{n} \small Revisión 2013}
\chead{\changefont{cmss}{bx}{n} \small Domingo después del 6 de enero - Navidad }
\rhead{\changefont{cmss}{bx}{n} \small Ciclo A}
\rfoot{\changefont{cmss}{bx}{n}\large\thepage}

\begin{center}
{\large\it El bautismo de Jesús }
\end{center}

%\vspace{1mm}

\begin{tabbing}

{\changefont{cmss}{bx}{n} Entrada:\ \ \ \ \ }\= Gloria eterna\footnotemark[1],  Brilló la luz\footnotemark[2], Un sólo Señor (estr. 3)\footnotemark[3], Pueblo de Dios\footnotemark[1],    \\ 
\> Vine a alabar.   \\ \\

{\changefont{cmss}{bx}{n} Salmo:} \> 32 ant. 1 ``Cantemos todos al Señor, aleluia...'' (est. 1, 3 o 8). \\
\> (antífona de reemplazo: Sal 26 ant. 1 ``Cantaré y celebraré al Señor.'') \\ \\

{\changefont{cmss}{bx}{n} Ofrendas:} \> Padre nuestro recibid\footnotemark[1], Te ofrecemos oh Señor, Te presentamos, Bendito seas.\\ \\

{\changefont{cmss}{bx}{n} Comunión:} \> Brilló la luz\footnotemark[2], Yo soy el camino, Bendeciré al Señor, Como Cristo nos amó.  \\  \\

{\changefont{cmss}{bx}{n} Post-com.:} \> Tu fidelidad, Tan cerca de mí, Adoremos a Dios, Alabe todo el mundo (de Taizé).   \\ \\

{\changefont{cmss}{bx}{n} Salida:} \> Gloria eterna\footnotemark[1], Mi camino eres tú\footnotemark[1], Canción del testigo, Anunciaremos tu Reino. \\  \\

\end{tabbing}

\vspace{-11mm}

\footnotetext[1]{Cfr. con Sal 28,9. }
\footnotetext[2]{Cfr. con 1$^\circ$ lectura (Is 42,6-7). }
\footnotetext[3]{Debido a que habla de ``un solo Dios y Padre'', pero siempre que quede claro en la asamblea que el bautismo del Señor (bautismo de penitencia) no es igual a nuestro bautismo. }


%------------------------
\newpage
\setcounter{footnote}{0}
\thispagestyle{fancy}
\fancyhead{}
\fancyfoot{}
%\footskip=-1cm

\lhead{\changefont{cmss}{bx}{n} \small Revisión 2013}
\chead{\changefont{cmss}{bx}{n} \small Miércoles de Ceniza}
\rhead{\changefont{cmss}{bx}{n} \small Ciclo A}
\rfoot{\changefont{cmss}{bx}{n}\large\thepage}

\begin{center}
{\large\it La limosna, la oración, el ayuno }
\end{center}

\vspace{1mm}


\begin{tabbing}

{\changefont{cmss}{bx}{n} Entrada:\ \ \ \ \ }\= Dice el Señor conviértanse\footnotemark[1], Perdón Señor, Sí me levantaré.  \\ \\

{\changefont{cmss}{bx}{n} Salmo:} \> 50 ant. 1 ``Piedad, Señor, pecamos contra ti'' (est. 1, 2, 6, 7 u 8). \\ \\

{\changefont{cmss}{bx}{n} Cenizas:} \> Perdón oh Dios mío, Perdón Señor, Sí me levantaré, Vuelve a mí. \\ \\

{\changefont{cmss}{bx}{n} Ofrendas:} \> Recibe oh Dios el pan, Te ofrecemos Padre nuestro (vidala), \\
\>  Recibe oh Padre Santo, Coplas de Yaraví. \\ \\

{\changefont{cmss}{bx}{n} Comunión:} \> Oh buen Jesús, Sí me levantaré, Bienaventurados los pobres, \\
\>  Creo en ti Señor, Vuelve a mí.\\ \\

{\changefont{cmss}{bx}{n} Post-com.:} \> Vaso nuevo (El alfarero), Todos unidos, Nada te turbe (de Taizé).\\ \\

{\changefont{cmss}{bx}{n} Salida:} \> Soy peregrino, Virgen de la esperanza.  \\  \\

\end{tabbing}

\vspace{-11mm}

\footnotetext[1]{Aclamación al evangelio, pero que se puede usar de entrada (cfr. Joel 2, 12-18).  }

%------------------------
\newpage
\setcounter{footnote}{0}
\thispagestyle{fancy}
\fancyhead{}
\fancyfoot{}
%\footskip=-1cm

\lhead{\changefont{cmss}{bx}{n} \small Revisión 2013}
\chead{\changefont{cmss}{bx}{n} \small Domingo I - Cuaresma}
\rhead{\changefont{cmss}{bx}{n} \small Ciclo A}
\rfoot{\changefont{cmss}{bx}{n}\large\thepage}

\begin{center}
{\large\it Las tentaciones de Jesús en el desierto }
\end{center}

\vspace{1mm}


\begin{tabbing}

{\changefont{cmss}{bx}{n} Entrada:\ \ \ \ \ }\= Perdón Señor, Sí me levantaré\footnotemark[1], Perdón oh Dios mío (estr. 2), \\
\> Contritos nos postramos.   \\ \\

{\changefont{cmss}{bx}{n} Salmo:} \> 50 ant. 1 ``Piedad, Señor, pecamos contra ti'' (est. 1, 2, 6, 7 u 8). \\ \\

{\changefont{cmss}{bx}{n} Ofrendas:} \> Recibe oh Dios el pan, Te ofrecemos Padre nuestro (vidala),  \\
\>  Mira nuestra ofrenda, Recibe oh Padre Santo, Coplas de Yaraví. \\ \\

{\changefont{cmss}{bx}{n} Comunión:} \> Dios es fiel\footnotemark[2], Oh buen Jesús, Hambre de Dios, Sí me levantaré,   \\
\>  Creo en ti Señor, Bienaventurados los pobres, Vuelve a mí.\\ \\

{\changefont{cmss}{bx}{n} Post-com.:} \> Tu fidelidad\footnotemark[2], Nada te turbe (de Taizé).\\ \\

{\changefont{cmss}{bx}{n} Salida:} \> Santa María del Amén, Soy peregrino, Virgen de la esperanza.  \\  \\

\end{tabbing}

\vspace{-11mm}

\footnotetext[1]{Cfr. con el salmo del día (Sal 50,14 y 50,17).  }
\footnotetext[2]{Por el contexto de la obra mesiánica de Jesús en favor de la Alianza.  }

%------------------------
\newpage
\setcounter{footnote}{0}
\thispagestyle{fancy}
\fancyhead{}
\fancyfoot{}
%\footskip=-1cm

\lhead{\changefont{cmss}{bx}{n} \small Revisión 2013}
\chead{\changefont{cmss}{bx}{n} \small Domingo II - Cuaresma}
\rhead{\changefont{cmss}{bx}{n} \small Ciclo A}
\rfoot{\changefont{cmss}{bx}{n}\large\thepage}

\begin{center}
{\large\it La transfiguración de Jesús }
\end{center}

\vspace{1mm}


\begin{tabbing}

{\changefont{cmss}{bx}{n} Entrada:\ \ \ \ \ }\= Brilló la luz, Cruz de Cristo\footnotemark[1], Juntos como hermanos, Perdón Señor,   \\
\> Vuélvenos tu rostro (est. 1 y 3), Sí me levantaré (estr. 3).\\ \\

{\changefont{cmss}{bx}{n} Salmo:} \> 32 ant. 2 ``Que descienda Señor sobre nosotros...'' (est. 2, 7 u 8). \\ 
\> (antífona de reemplazo: Sal 26 ant. 2 ``El Señor es mi luz, mi salvación...'') \\ \\

{\changefont{cmss}{bx}{n} Ofrendas:} \> Pan de vida y bebida de luz, Este es nuestro pan, Recibe oh Dios el pan,  \\
\>  Te ofrecemos Padre nuestro (vidala). \\ \\

{\changefont{cmss}{bx}{n} Comunión:} \> En memoria tuya\footnotemark[2], Brilló la luz, Dios es fiel\footnotemark[2], Creo en ti Señor\footnotemark[3],   \\
\>   Vuelve a mí\footnotemark[2].\\ \\

{\changefont{cmss}{bx}{n} Post-com.:} \> Tu fidelidad\footnotemark[2], Mirarte sólo a ti.\\ \\

{\changefont{cmss}{bx}{n} Salida:} \> Madre de nuestro pueblo (estr. 9), Virgen de la esperanza.  \\  \\

\end{tabbing}

\vspace{-11mm}

\footnotetext[1]{Cfr. la presentación de la cruz con Mt 17,3 (Moisés y Elías hablando de la Pasión).  }
\footnotetext[2]{Por el contexto de la obra mesiánica de Jesús en favor de la Alianza.  }
\footnotetext[3]{Con el texto de \emph{Cantemos hermanos con amor}.  }

%------------------------
\newpage
\setcounter{footnote}{0}
\thispagestyle{fancy}
\fancyhead{}
\fancyfoot{}
%\footskip=-1cm

\lhead{\changefont{cmss}{bx}{n} \small Revisión 2013}
\chead{\changefont{cmss}{bx}{n} \small Domingo III - Cuaresma}
\rhead{\changefont{cmss}{bx}{n} \small Ciclo A}
\rfoot{\changefont{cmss}{bx}{n}\large\thepage}

\begin{center}
{\large\it El encuentro de Jesús con la samaritana}
\end{center}

\vspace{1mm}


\begin{tabbing}

{\changefont{cmss}{bx}{n} Entrada:\ \ \ \ \ }\= Sí me levantaré (estr. 4, 7, 8 o 12), Juntos como hermanos (estr. 1),  \\
\>  Cruz de Cristo, Perdón Señor.\\   \\

{\changefont{cmss}{bx}{n} Salmo:} \> 94 ant. 1 ``Adoremos al Señor, nuestro Dios.'' (est. 1, 2 o 4). \\ \\

{\changefont{cmss}{bx}{n} Ofrendas:} \> Recibe oh Dios eterno\footnotemark[1], Comienza el sacrificio\footnotemark[2], Este es nuestro pan, \\ 
\> Recibe oh Dios el pan.  \\ \\

{\changefont{cmss}{bx}{n} Comunión:} \> El pueblo de Dios\footnotemark[3], Hambre de Dios\footnotemark[3], Dios es fiel\footnotemark[3], Creo en ti Señor.  \\ \\

{\changefont{cmss}{bx}{n} Post-com.:} \> Adoremos a Dios, (silencio).\\ \\

{\changefont{cmss}{bx}{n} Salida:} \> Virgen de la esperanza (estr. 1,2,5), Santa María del Amén, Soy peregrino.  \\  \\

\end{tabbing}

\vspace{-11mm}

\footnotetext[1]{Este canto habla (en la 2$^\circ$ estrofa) del ``agua que se mezcla en la ofrenda''. Cfr. críticamente con Jn 4,5-15. }
\footnotetext[2]{Cfr. la 2$^\circ$ estrofa: ``la fe de nuestros padres, consérvanos, Señor'' con la 1$^\circ$ lectura (Ex 17,1-7). }
\footnotetext[3]{Cfr. con la 1$^\circ$ lectura (Ex 17,1-7).  }

%------------------------
\newpage
\setcounter{footnote}{0}
\thispagestyle{fancy}
\fancyhead{}
\fancyfoot{}
%\footskip=-1cm

\lhead{\changefont{cmss}{bx}{n} \small Revisión 2013}
\chead{\changefont{cmss}{bx}{n} \small Domingo IV - Cuaresma}
\rhead{\changefont{cmss}{bx}{n} \small Ciclo A}
\rfoot{\changefont{cmss}{bx}{n}\large\thepage}

\begin{center}
{\large\it (Domingo de Laetare) Curación de un ciego de nacimiento}
\end{center}

\vspace{1mm}


\begin{tabbing}

{\changefont{cmss}{bx}{n} Entrada:\ \ \ \ \ }\= Qué alegría, Vienen con alegría, Sí me levantaré (estr. 3, 4, 5, 7, 9 o 10),  \\
\>  Perdón Señor, Cruz de Cristo.\\   \\

{\changefont{cmss}{bx}{n} Salmo:} \> 22 ant. 1 ``El Señor es mi pastor...'' (estr. todo). \\ \\

{\changefont{cmss}{bx}{n} Ofrendas:} \> Pan de vida y bebida de luz\footnotemark[1], Al altar del Señor\footnotemark[2], Recibe oh Dios eterno.\\ \\ 

{\changefont{cmss}{bx}{n} Comunión:} \> Bienaventurados los pobres, En memoria tuya, Creo en ti Señor\footnotemark[3].  \\ \\

{\changefont{cmss}{bx}{n} Post-com.:} \> Adoremos a Dios, Creo en ti (estr. 2)\footnotemark[3], Tan cerca de mí, (silencio).\\ \\

{\changefont{cmss}{bx}{n} Salida:} \> Soy peregrino, En medio de los pueblos (estr. 1,2)\footnotemark[4].  \\  \\

\end{tabbing}

\vspace{-11mm}

\footnotetext[1]{Cfr. en la 2$^\circ$ estrofa ``es bebida de luz para poder guiar a los hombres'' con el evangelio (Jn 9,1-41). }
\footnotetext[2]{Por ser domingo de Laetare y debido al Sal 22,5.  }
\footnotetext[3]{Con el texto tradicional de \emph{Más cerca oh Dios de ti}.  }
\footnotetext[4]{Porque en su 3$^\circ$ estrofa habla de la luz de la Iglesia. Cfr. (Jn 9,1-41).  }

%------------------------
\newpage
\setcounter{footnote}{0}
\thispagestyle{fancy}
\fancyhead{}
\fancyfoot{}
%\footskip=-1cm

\lhead{\changefont{cmss}{bx}{n} \small Revisión 2013}
\chead{\changefont{cmss}{bx}{n} \small Domingo V - Cuaresma}
\rhead{\changefont{cmss}{bx}{n} \small Ciclo A}
\rfoot{\changefont{cmss}{bx}{n}\large\thepage}

\begin{center}
{\large\it La resurrección de Lázaro}
\end{center}

%\vspace{1mm}


\begin{tabbing}

{\changefont{cmss}{bx}{n} Entrada:\ \ \ \ \ }\= Sí me levantaré (estr. 5, 9 o 10)\footnotemark[1], Cruz de Cristo, Perdón Señor.\\   \\

{\changefont{cmss}{bx}{n} Salmo:} \> 129 ant. 1 ``Yo pongo mi esperanza en ti, Señor,...'' (estr. todo). \\
\> (antífona de reemplazo: Sal 24 ant. 1 ``A ti elevo mi alma, a ti, mi Dios y Señor'') \\ \\

{\changefont{cmss}{bx}{n} Ofrendas:} \> Sé como el grano de trigo, Zamba del grano de trigo, Entre tus manos, \\
\> Recibe oh Dios el pan.\\ \\ 

{\changefont{cmss}{bx}{n} Comunión:} \> Creo en ti Señor\footnotemark[2]\footnotemark[3], Bienaventurados los pobres\footnotemark[4], Antes de ser llevado a la muerte,\\
\> En memoria tuya.  \\ \\

{\changefont{cmss}{bx}{n} Post-com.:} \> Creo en ti (estr. 2)\footnotemark[3], Tu fidelidad, (silencio).\\ \\

{\changefont{cmss}{bx}{n} Salida:} \> Santa María del amén, Soy peregrino, Virgen de la esperanza (estr. 1,2,5),\\
\> Madre de nuestro pueblo (estr. 9).  \\  \\

\end{tabbing}

\vspace{-14mm}

\footnotetext[1]{Cfr. \emph{Sí me levantaré} (estr. 9) con Sal 129,6. }
\footnotetext[2]{Cfr. con el evangelio del día, especialmente con Jn 11,27 y Jn 11,40. }
\footnotetext[3]{Con el texto tradicional de \emph{Más cerca oh Dios de ti}.  }
\footnotetext[4]{Cfr. la 1$^\circ$ estrofa con Jn 11,33-35. }

%------------------------
\newpage
\setcounter{footnote}{0}
\thispagestyle{fancy}
\fancyhead{}
\fancyfoot{}
%\footskip=-1cm

\lhead{\changefont{cmss}{bx}{n} \small Revisión 2013}
\chead{\changefont{cmss}{bx}{n} \small Domingo de Ramos - Cuaresma}
\rhead{\changefont{cmss}{bx}{n} \small Ciclo A}
\rfoot{\changefont{cmss}{bx}{n}\large\thepage}

\begin{center}
{\large\it (Domingo de Pasión) La Pasión de Jesús}
\end{center}

%\vspace{1mm}


\begin{tabbing}

{\changefont{cmss}{bx}{n} Entrada:\ \ \ \ \ }\= Canta Jerusalén, Arriba nuestros ramos, Con ramos en las manos (Sal 23 ant. 3),\\
\> Bendito el que viene (Sal 46 ant. 2).\\   \\

{\changefont{cmss}{bx}{n} Salmo:} \> 21 ant. 1 ``Dios mío, no me abandones...'' (estr. 1, 3, 7, 8 o 9). \\ \\

{\changefont{cmss}{bx}{n} Ofrendas:} \> Este es nuestro pan\footnotemark[1], Te ofrecemos Padre nuestro (vidala)\footnotemark[1], Coplas de Yaraví, \\
\> Una espiga, Sé como el grano de trigo, Zamba del grano de trigo.\\ \\ 

{\changefont{cmss}{bx}{n} Comunión:} \> Rey de los reyes, En la postrera cena, Antes de ser llevado a la muerte,\\
\> En memoria tuya, Jesucristo danos de este pan, Queremos ser Señor\footnotemark[2].  \\ \\

{\changefont{cmss}{bx}{n} Post-com.:} \> Victoria tu reinarás, Creo en ti Señor\footnotemark[3], Tu fidelidad, (silencio).\\ \\

{\changefont{cmss}{bx}{n} Salida:} \> Virgen de la esperanza (estr. 1,2,5), Cristo Jesús (estr. 2,3).\\ \\

\end{tabbing}

\vspace{-14mm}

\footnotetext[1]{Cfr. 2$^\circ$ estrofa con Mt 26,26-28. }
\footnotetext[2]{Sólo en caso de necesidad. }
\footnotetext[3]{Con el texto tradicional de \emph{Más cerca oh Dios de ti}.  }







%------------------------
\newpage
\setcounter{footnote}{0}
\thispagestyle{fancy}
\fancyhead{}
\fancyfoot{}
%\footskip=-1cm

\lhead{\changefont{cmss}{bx}{n} \small Revisión 2013}
\chead{\changefont{cmss}{bx}{n} \small Jueves Santo - Cena del Señor}
\rhead{\changefont{cmss}{bx}{n} \small Ciclo A}
\rfoot{\changefont{cmss}{bx}{n}\large\thepage}

\begin{center}
{\large\it (Triduo pascual) El lavatorio de los pies}
\end{center}

%\vspace{1mm}


\begin{tabbing}

{\changefont{cmss}{bx}{n} Entrada\footnotemark[1]:\ \ \ \ \ }\= Me pongo en tus 
manos oh Señor, El Señor nos llama.\\   \\

{\changefont{cmss}{bx}{n} Salmo:} \> 115 ant. 1 ``?`Con qué pagaré al 
Señor...?'' (estr. 1 y 2). \\ \\

{\changefont{cmss}{bx}{n} Aclamación:} \> Si yo el maestro (dos 
veces). \\ \\

{\changefont{cmss}{bx}{n} Lavatorio:} \> Un mandamiento nuevo, Si yo el 
maestro (dos veces). \\ \\

{\changefont{cmss}{bx}{n} Ofrendas:} \> Los frutos de la tierra.\\ \\ 

{\changefont{cmss}{bx}{n} Comunión:} \> Antes de ser llevado a la muerte,$\,$No 
hay mayor amor,$\,$Dios me dio a mi hermano,\\
\> Memorial.  \\ \\

{\changefont{cmss}{bx}{n} Procesión:} \> Cantemos al amor 
de los amores, Yo soy el camino.\\ \\

{\changefont{cmss}{bx}{n} Adoración:} \> Tan sublime sacramento (Tantum 
ergo).\\ \\

\end{tabbing}

\vspace{-14mm}

\footnotetext[1]{Se canta el \emph{Gloria} (con sentimiento porque no se 
cantará más hasta la Vigilia pascual. }





%------------------------
\newpage
\setcounter{footnote}{0}
\thispagestyle{fancy}
\fancyhead{}
\fancyfoot{}
%\footskip=-1cm

\lhead{\changefont{cmss}{bx}{n} \small Revisión 2013}
\chead{\changefont{cmss}{bx}{n} \small Viernes Santo - Pasión del Señor}
\rhead{\changefont{cmss}{bx}{n} \small Ciclo A}
\rfoot{\changefont{cmss}{bx}{n}\large\thepage}

\begin{center}
{\large\it (Triduo pascual) La Pasión del Señor}
\end{center}

%\vspace{1mm}


\begin{tabbing}

{\changefont{cmss}{bx}{n} Entrada:\ \ \ \ \ }\= (en 
silencio; postración silenciosa).\\   \\

{\changefont{cmss}{bx}{n} Salmo:} \> 30 ant. 1 ``En tus manos 
Señor...'' (estr. 1, 2, 5 u 8). \\ \\

{\changefont{cmss}{bx}{n} Himno:} \> Jesús la imagen de Dios Padre. \\ \\

{\changefont{cmss}{bx}{n} Colecta:} \> Oh víctima inmolada. \\ \\

{\changefont{cmss}{bx}{n} Cruz:} \> antífona ``Te adoramos Cristo y 
te bendecimos...'' (se hace 3 veces). \\ \\

{\changefont{cmss}{bx}{n} Dolorosa:} \> Junto a la cruz, Madre de 
nuestro pueblo \\ \\

{\changefont{cmss}{bx}{n} Comunión:} \> Más cerca oh Dios,$\,$Perdón 
Señor misericordia,$\,$Salmo 50. \\ \\

{\changefont{cmss}{bx}{n} Ador. cruz:} \> Cruz de Cristo (Es la cruz), 
Coplas de soledad, Oh víctima inmolada, Salmo 41.\\ \\

\end{tabbing}

\vspace{-14mm}







%------------------------
\newpage
\setcounter{footnote}{0}
\thispagestyle{fancy}
\fancyhead{}
\fancyfoot{}
%\footskip=-1cm

\lhead{\changefont{cmss}{bx}{n} \small Revisión 2013}
\chead{\changefont{cmss}{bx}{n} \small Domingo de Pascua}
\rhead{\changefont{cmss}{bx}{n} \small Ciclo A}
\rfoot{\changefont{cmss}{bx}{n}\large\thepage}

\begin{center}
{\large\it El sepulcro vacío }
\end{center}

\vspace{1mm}


\begin{tabbing}

{\changefont{cmss}{bx}{n} Entrada:\ \ \ \ \ }\= Encendamos el cirio pascual, Esta es la luz de Cristo, Hoy la Iglesia victoriosa, \\
\> Que resuene por la tierra (si no se usa en otro momento).\\ \\

{\changefont{cmss}{bx}{n} Salmo:}\footnotemark[1] \> 117 ant. 2 ``Este es el día que hizo el Señor, aleluia,...'' (estr. 1, 7 u 8). \\ \\

{\changefont{cmss}{bx}{n} Aspersión:} \> Nueva vida, Esta es el agua pura, Tu agua bendita, Un solo Señor,\\ 
\>  Que resuene por la tierra. \\ \\

{\changefont{cmss}{bx}{n} Ofrendas:} \> Te ofrecemos oh Señor, Pan de vida y bebida de luz.\\ \\

{\changefont{cmss}{bx}{n} Comunión:} \> Gloria al Señor ha llegado la pascua, Que resuene por la tierra, \\
\> La gran noticia, Resucitó. \\ \\

{\changefont{cmss}{bx}{n} Post-com.:} \> Vive Jesús el Señor, Alabe todo el mundo (de Taizé).\\ \\

{\changefont{cmss}{bx}{n} Salida:} \> Alégrate María, Suenen campanas, Toda la tierra levante su voz.  \\  \\

\end{tabbing}

\vspace{-10mm}

\footnotetext[1]{Luego de la segunda lectura hay Secuencia pascual.}

%------------------------
\newpage
\setcounter{footnote}{0}
\thispagestyle{fancy}
\fancyhead{}
\fancyfoot{}
%\footskip=-1cm

\lhead{\changefont{cmss}{bx}{n} \small Revisión 2013}
\chead{\changefont{cmss}{bx}{n} \small Domingo II - Pascua}
\rhead{\changefont{cmss}{bx}{n} \small Ciclo A}
\rfoot{\changefont{cmss}{bx}{n}\large\thepage}

\vspace*{-10mm}

\begin{center}
{\large\it (Domingo de la Misericordia) La incredulidad de Tomás }
\end{center}

\vspace{-1mm}


\begin{tabbing}

{\changefont{cmss}{bx}{n} Entrada:\ \ \ \ \ }\= Resucitó, Hoy la Iglesia victoriosa, Encendamos el	 cirio pascual,  \\ 
 \> Esta es la luz de Cristo. \\ \\

{\changefont{cmss}{bx}{n} Salmo:} \> 117 ant. 1 ``Demos gracias al Señor porque es bueno...'' (estr. 1, 7 u 8). \\ 
\> (antífona de reemplazo: Sal 117 ant. 2 ``Este es el día que hizo el Señor...'') \\ \\

{\changefont{cmss}{bx}{n} Ofrendas:} \> Señor te ofrecemos\footnotemark[1], Pan de vida y bebida de luz, Te presentamos. \\ \\

{\changefont{cmss}{bx}{n} Comunión:} \> No hay mayor amor\footnotemark[2], Resucitó, Gloria al Señor ha llagado la pascua, \\
\> La gran noticia, Que resuene por la tierra. \\ \\

{\changefont{cmss}{bx}{n} Post-com.:} \> Tan cerca de mí\footnotemark[3], Adoremos a Dios (estr. 1)\footnotemark[3], Más cerca oh Dios (estr. 2)\\ 
\> Vive Jesús, Alabe todo el mundo, Gloria al Señor ha llagado la pascua. \\ \\

{\changefont{cmss}{bx}{n} Salida:} \> Toda la tierra levante su voz, Hoy la Iglesia victoriosa, Alégrate María, \\
\> Cantad a María, Suenen campanas. \\  \\

\end{tabbing}

\vspace{-12mm}

\footnotetext[1]{Es perfecto para la Divina Misericordia: ``?`quién podrá cantar tus misericordias...?''.  }
\footnotetext[2]{Cfr. la estrofa 3 con Jn 20,25-27, la estrofa 4 con Jn 20,19 y la estrofa 5 con Jn 20,21-23.}
\footnotetext[3]{Cfr. \textbf{Tan cerca de mí} con Jn 20,27-28 y \textbf{Adoremos a Dios} con Jn 20,28. }


%------------------------
\newpage
\setcounter{footnote}{0}
\thispagestyle{fancy}
\fancyhead{}
\fancyfoot{}
%\footskip=-1cm

\lhead{\changefont{cmss}{bx}{n} \small Revisión 2013}
\chead{\changefont{cmss}{bx}{n} \small Domingo III - Pascua}
\rhead{\changefont{cmss}{bx}{n} \small Ciclo A}
\rfoot{\changefont{cmss}{bx}{n}\large\thepage}

\vspace*{-10mm}

\begin{center}
{\large\it La aparición de Jesús a los discípulos de Emaús }
\end{center}

\vspace{-1mm}


\begin{tabbing}

{\changefont{cmss}{bx}{n} Entrada:\ \ \ \ \ }\= Un pueblo que camina (estr. 2)\footnotemark[1], Encendamos el cirio pascual,  \\ 
 \> Esta es la luz de Cristo (estr. 1 y 3), Vienen con alegría, Pueblo de Dios. \\ \\

{\changefont{cmss}{bx}{n} Salmo:} \> 15 ant. 1 ``Tú eres, Señor, mi herencia... '' (estr. 1, 2, 3 o 4). \\ \\

{\changefont{cmss}{bx}{n} Ofrendas:} \> Te ofrecemos oh Señor, Pan de vida y bebida de luz,  \\
\> Te ofrecemos Padre nuestro (vidala), Te presentamos. \\ \\

{\changefont{cmss}{bx}{n} Comunión:} \> Quedate con nosotros\footnotemark[1], Jesús eucaristía\footnotemark[2], La gran noticia, \\
\> Resucitó, Vayamos a la mesa.  \\ \\

{\changefont{cmss}{bx}{n} Post-com.:} \> Vive Jesús el Señor, El Señor es mi fortaleza.\\ \\ 

{\changefont{cmss}{bx}{n} Salida:} \> Suenen campanas, Toda la tierra levante su voz, Alégrate María, \\
\> Cantad a María. \\  \\

\end{tabbing}

\vspace{-10mm}

\footnotetext[1]{Porque se refiere al evangelio del día (Lc 24,13-29).  }
\footnotetext[2]{Cfr. el estribillo con Lc 24,13. }

%------------------------
\newpage
\setcounter{footnote}{0}
\thispagestyle{fancy}
\fancyhead{}
\fancyfoot{}
%\footskip=-1cm

\lhead{\changefont{cmss}{bx}{n} \small Revisión 2013}
\chead{\changefont{cmss}{bx}{n} \small Domingo IV - Pascua}
\rhead{\changefont{cmss}{bx}{n} \small Ciclo A}
\rfoot{\changefont{cmss}{bx}{n}\large\thepage}

\vspace*{-10mm}

\begin{center}
{\large\it (Domingo del Buen Pastor) El buen Pastor y la puerta }
\end{center}

\vspace{-1mm}


\begin{tabbing}

{\changefont{cmss}{bx}{n} Entrada:\ \ \ \ \ }\= El Señor nos llama (estr. 2)\footnotemark[1], Pueblo de Reyes (estr. 6)\footnotemark[1], Pueblo de Dios\footnotemark[1],   \\ 
\> Encendamos el cirio pascual,  Esta es la luz de Cristo.\\ \\ 

{\changefont{cmss}{bx}{n} Salmo\footnotemark[2]:} \> 22 ant. 1 ``El Señor es mi Pastor...'' (todo). \\  \\

{\changefont{cmss}{bx}{n} Ofrendas:} \> Te ofrecemos Padre nuestro (vidala)\footnotemark[3], Los frutos de la tierra\footnotemark[3],  \\
\> Pan de vida y bebida de luz, Te presentamos. \\ \\

{\changefont{cmss}{bx}{n} Comunión:} \> Yo soy el camino, Pueblo de Reyes, No hay mayor amor\footnotemark[3]. \\ \\

{\changefont{cmss}{bx}{n} Post-com.:} \> Cantemos hermanos (estr. 1), Vive Jesús, Alabe todo el mundo (de Taizé). \\ \\

{\changefont{cmss}{bx}{n} Salida:} \> Cantemos hermanos (todo), Suenen campanas, Alégrate María, \\
\> Vayan todos por el mundo (estr. 2). \\  \\

\end{tabbing}

\vspace{-14mm}

\footnotetext[1]{Cfr. con las lecturas del día, especialmente, Sal 22,1-6 (el rebaño de Dios) y Jn 10,3-5 (la voz del Pastor).}
\footnotetext[2]{Recordar que es conveniente que el Aleluia sea el de \emph{Yo soy el Maestro y el Pastor} de Néstor Gallego.}
\footnotetext[3]{Cfr. con la 1$^\circ$ lectura (Hech 2,36-41). }


%------------------------
\newpage
\setcounter{footnote}{0}
\thispagestyle{fancy}
\fancyhead{}
\fancyfoot{}
%\footskip=-1cm

\lhead{\changefont{cmss}{bx}{n} \small Revisión 2013}
\chead{\changefont{cmss}{bx}{n} \small Domingo V - Pascua}
\rhead{\changefont{cmss}{bx}{n} \small Ciclo A}
\rfoot{\changefont{cmss}{bx}{n}\large\thepage}

\vspace*{-10mm}

\begin{center}
{\large\it Jesús, camino hacia el Padre }
\end{center}

\vspace{-1mm}

\begin{tabbing}

{\changefont{cmss}{bx}{n} Entrada:\ \ \ \ \ }\= Pueblo de Reyes (estr. 5,8)\footnotemark[1], Pueblo de Dios\footnotemark[1], El Señor nos llama (estr. 3)\footnotemark[1],   \\ 
\> Que resuene por la tierra,  Vine a alabar.\\ \\ 

{\changefont{cmss}{bx}{n} Salmo:} \> 32 ant. 2 ``Que descienda Señor sobre nosotros...'' (estr. 1, 2 o 7). \\ 
\> (antífona de reemplazo: Sal 144 ant. 1 ``Te alabamos Señor...'') \\ \\

{\changefont{cmss}{bx}{n} Ofrendas:} \> Recibe oh Dios el pan\footnotemark[2], Los frutos de la tierra\footnotemark[2], Te ofrecemos oh Señor, \\
\> Te presentamos. \\ \\

{\changefont{cmss}{bx}{n} Comunión:} \> Yo soy el camino, Pueblo de Reyes, No hay mayor amor\footnotemark[3], \\
\> Cuerpo y Sangre de Jesús\footnotemark[1]. \\ \\

{\changefont{cmss}{bx}{n} Post-com.:} \> Cantemos hermanos (estr. 1), Vive Jesús, Alabe todo el mundo (de Taizé). \\ \\

{\changefont{cmss}{bx}{n} Salida:} \> Mi camino eres tu\footnotemark[3], Cantemos hermanos (todo), Suenen campanas, Alégrate María, \\
\> Cantad a María. \\  \\

\end{tabbing}

\vspace{-14mm}

\footnotetext[1]{Cfr. con las lecturas del día, especialmente, 1 Ped 2, 9-10 y Jn 14,8-9.}
\footnotetext[2]{Cfr. con la exhortación a ofrecer ``sacrificios espirituales'' de 1 Ped 2,5. }
\footnotetext[3]{Cfr. con el evangelio del día, epecialmente con Jn 14,3, Jn 14,6 y Jn 14,12. }

%------------------------
\newpage
\setcounter{footnote}{0}
\thispagestyle{fancy}
\fancyhead{}
\fancyfoot{}
%\footskip=-1cm

\lhead{\changefont{cmss}{bx}{n} \small Revisión 2013}
\chead{\changefont{cmss}{bx}{n} \small Domingo VI - Pascua}
\rhead{\changefont{cmss}{bx}{n} \small Ciclo A}
\rfoot{\changefont{cmss}{bx}{n}\large\thepage}

\vspace*{-10mm}

\begin{center}
{\large\it La promesa del Espíritu Santo }
\end{center}

\vspace{-1mm}

\begin{tabbing}

{\changefont{cmss}{bx}{n} Entrada:\ \ \ \ \ }\= Un solo Señor\footnotemark[1], Pueblo de Dios (estr. 1)\footnotemark[1],  Que resuene por la tierra,  \\ 
\>  Vine a alabar, Vienen con alegría. \\ \\

{\changefont{cmss}{bx}{n} Salmo:} \> 65 ant. 1 ``Todo el mundo cante la gloria...!'' (estr. 1, 2, 3 o 7).\\
\> (antífona de reemplazo: Sal 26 ant. 1 ``Cantaré y celebraré al Señor'') \\ \\ 

{\changefont{cmss}{bx}{n} Ofrendas:} \> Pan de vida y bebida de luz, Los frutos de la tierra, Padre nuestro recibid, \\
\> Te ofrecemos oh Señor, Te presentamos. \\ \\

{\changefont{cmss}{bx}{n} Comunión:} \> No hay mayor amor\footnotemark[2], Yo soy el camino, Queremos ser Señor\footnotemark[3],  \\
\> Ven hermano, Un mandamiento nuevo. \\ \\

{\changefont{cmss}{bx}{n} Post-com.:} \> Adoremos a Dios, Alabe todo el mundo (de Taizé), Vive Jesús. \\ \\

{\changefont{cmss}{bx}{n} Salida:} \> Cantad a María, Alégrate María, Que resuene por la tierra, \\
\> Cantemos hermanos, Mi camino eres Tú. \\  \\

\end{tabbing}

\vspace{-10mm}

\footnotetext[1]{Cfr. \emph{Un solo Señor} con la 1$^\circ$ lectura (Hech 8,5-8.14-17) y \emph{Pueblo de Dios} con Sal 65,2-3.}
\footnotetext[2]{Cfr. la 1$^\circ$ y 2$^\circ$ estrofa con Jn 14,18-19 y la 4$^\circ$ estrofa con Jn 14,16-17. }
\footnotetext[3]{Cfr. con la 2$^\circ$ lectura (1 Ped 3,15-18). }


%------------------------
\newpage
\setcounter{footnote}{0}
\thispagestyle{fancy}
\fancyhead{}
\fancyfoot{}
%\footskip=-1cm

\lhead{\changefont{cmss}{bx}{n} \small Revisión 2013}
\chead{\changefont{cmss}{bx}{n} \small Ascensión del Señor}
\rhead{\changefont{cmss}{bx}{n} \small Ciclo A}
\rfoot{\changefont{cmss}{bx}{n}\large\thepage}

\vspace*{-15mm}

\begin{center}
%{\large\it \'Ultimas instrucciones de Jesús y la ascensión }
\end{center}

\vspace{-6mm}

\begin{tabbing}

{\changefont{cmss}{bx}{n} Entrada:\ \ \ \ \ }\= Suenen cantos de alegría\footnotemark[1], Un solo Señor (estr. 1,3), Vienen con alegría,  \\ 
\>  Iglesia peregrina. \\ \\

{\changefont{cmss}{bx}{n} Salmo\footnotemark[2]:} \> 46 ant. 1 ``Cantemos al Señor, nuestro Dios, aleluia...'' (estr. 1, 3 o 4).\\
\> (antífona de reemplazo: Sal 147 ant. 1 ``Glorifica al Señor Jerusalén...'') \\ \\ 

{\changefont{cmss}{bx}{n} Ofrendas:} \> Nuestros dones\footnotemark[3], Bendeciré al Señor\footnotemark[3], Al altar del Señor, Te presentamos,  \\
\> Te ofrecemos oh Señor, Bendito seas. \\ \\

{\changefont{cmss}{bx}{n} Comunión:} \> Jesucristo danos de este pan\footnotemark[4], Oh buen Jesús\footnotemark[4], Bendeciré al Señor, \\
\> Yo soy el camino. \\ \\

{\changefont{cmss}{bx}{n} Post-com.:} \> Adoremos a Dios\footnotemark[5], Vive Jesús, Alabe todo el mundo (de Taizé). \\ \\

{\changefont{cmss}{bx}{n} Salida:} \> Suenen cantos de alegría\footnotemark[1], Canción del testigo\footnotemark[6], Vayan todos por el mundo\footnotemark[6],\\
\> Anunciaremos tu Reino, Cantemos hermanos. \\  \\

\end{tabbing}

\vspace{-12mm}

\footnotetext[1]{Cfr. \emph{Suenen cantos de alegría} con Hech 1,9-11, y \emph{Vienen con alegría} con Hech 1,8 y Mt 28,19-20.}
\footnotetext[2]{Se puede cantar el Aleluia del Padre Catena con la antífona ``Vayan por el mundo, aununcien mi Reino...''. }
\footnotetext[3]{Cfr. 3$^\circ$ estrofa de \emph{Nuestros dones} y \emph{Bendeciré al Señor} (Sal 33) con Sal 46,2. }
\footnotetext[4]{Cfr. \emph{Jesucristo danos de este pan} con Hech 1,11 y Mt 28,20, y \emph{Oh buen Jesús} con Heb. 10,12 (ant. comuni\'on). }
\footnotetext[5]{Cfr. \emph{Adoremos a Dios} con Mt 28,17, y \emph{Tu fidelidad} con Heb 10,23.}
\footnotetext[6]{Cfr. \emph{Canción del testigo} con Hech 1,8, y \emph{Vayan todos por el mundo} con Mt 28,19-20.}

%------------------------
\newpage
\setcounter{footnote}{0}
\thispagestyle{fancy}
\fancyhead{}
\fancyfoot{}
%\footskip=-1cm

\lhead{\changefont{cmss}{bx}{n} \small Revisión 2013}
\chead{\changefont{cmss}{bx}{n} \small Domingo de Pentecostés}
\rhead{\changefont{cmss}{bx}{n} \small Ciclo A}
\rfoot{\changefont{cmss}{bx}{n}\large\thepage}

\vspace{-1mm}

\begin{center}
{\large\it Apariciones de Jesús a los discípulos  }
\end{center}

\vspace{-1mm}


\begin{tabbing}

{\changefont{cmss}{bx}{n} Entrada:\ \ \ \ \ }\= Hoy tu Espíritu Señor\footnotemark[1], Espíritu divino, Pueblo de Reyes\footnotemark[2], Vine a alabar. \\ \\

{\changefont{cmss}{bx}{n} Salmo:}\footnotemark[3] \> 103 (estr. 1, 7 o 13) con música ant. del Sal 144 ``Envía tu Espíritu, Señor, ...''. \\ \\

{\changefont{cmss}{bx}{n} Ofrendas:} \> Ven de lo alto, Espíritu divino, Una espiga\footnotemark[4], Coplas de Yaraví, Te presentamos, \\
\> Bendito seas.\\ \\

{\changefont{cmss}{bx}{n} Comunión:} \> Soplo de Dios viviente, Envíanos Padre\footnotemark[5], Ven Espíritu Santo, Maranathá\footnotemark[5]. \\ \\

{\changefont{cmss}{bx}{n} Post-com.:} \> Ven oh Santo Espíritu\footnotemark[1] (de Taizé), Espíritu de Dios.\\ \\

{\changefont{cmss}{bx}{n} Salida:} \> Madre de nuestro pueblo\footnotemark[6] (estr.10), En medio de los pueblos\footnotemark[1], \\
\> Canción del misionero.  \\  \\

\end{tabbing}

\vspace{-15mm}

\footnotetext[1]{Cfr. con la la 1$^\circ$ lectura (Hech 2,1-11).}
\footnotetext[2]{Cfr. con Ex 19,6 que se lee como 2$^\circ$ lectura en la Vigilia de Pentecostés.}
\footnotetext[3]{Luego de la segunda lectura hay Secuencia de Pentecostés+Aleluia.}
\footnotetext[4]{Porque Pentecostés corresponde a la fiesta hebrea de las \emph{Siete Semanas}, o de las primicias (Lev 23,16).}
\footnotetext[5]{Cfr. \emph{Envíanos Padre} con Jn 20,22 y Hech 2,1-4, y \emph{Maranathá} con Hech 2,1-11 y 1 Cor 12,3-7.}
\footnotetext[6]{Cfr. con Hech 2,12-14.}

%------------------------
\newpage
\setcounter{footnote}{0}
\thispagestyle{fancy}
\fancyhead{}
\fancyfoot{}
%\footskip=-1cm

\lhead{\changefont{cmss}{bx}{n} \small Revisión 2013}
\chead{\changefont{cmss}{bx}{n} \small Domingo de Santísima Trinidad}
\rhead{\changefont{cmss}{bx}{n} \small Ciclo A}
\rfoot{\changefont{cmss}{bx}{n}\large\thepage}

\vspace{-1mm}

\begin{center}
{\large\it El diálogo de Jesús con Nicodemo}
\end{center}

\vspace{-1mm}


\begin{tabbing}

{\changefont{cmss}{bx}{n} Entrada:\ \ \ \ \ }\= Un solo Señor (estr. 1), El Señor nos llama (estr. 3), Pueblo de Dios,  \\ 
\> Vuélvenos tu rostro\footnotemark[1], Vine a alabar. \\ \\

{\changefont{cmss}{bx}{n} Salmo\footnotemark[2]:} \> Daniel 3,52-56 ant. 1 ``¡`A tí, eternamente gloria ...'' (todo, cantado de a pares). \\ 
\> (antífona de reemplazo: Sal 144 ant. 1 ``Te alabamos Señor...'') \\ \\ 

{\changefont{cmss}{bx}{n} Ofrendas:} \> Padre nuestro recibid\footnotemark[3], Ven de lo alto, Una espiga, Bendito seas, Te presentamos.\\ \\

{\changefont{cmss}{bx}{n} Comunión:} \> Cuerpo y Sangre de Jesús, Es mi Padre, Escondido, Yo soy el pan de vida,\\
\> Vayamos a la mesa, Este es mi Cuerpo. \\ \\

{\changefont{cmss}{bx}{n} Post-com.:} \> Adoremos a Dios (estr. 1), Tu fidelidad, Alabe todo el mundo (de Taizé).\\ \\

{\changefont{cmss}{bx}{n} Salida:} \> Mi camino eres tú\footnotemark[3], Vayan todos por el mundo\footnotemark[4], Canción del testigo, \\
\> Canción del misionero.  \\  \\

\end{tabbing}

\vspace{-15mm}

\footnotetext[1]{Se refiere a la Santísima Trinidad, pero muchas veces se lo asocia como canto de cuaresma.}
\footnotetext[2]{Se puede cantar el Aleluia del P. Catena con la antífona ``Dios es nuestro Padre, Jesús nuestro hermano...''.}
\footnotetext[3]{Porque glorifica a la Santísima Trinidad.}
\footnotetext[4]{Porque se refiere al evangelio de este día para el ciclo B (Mt 28,16-20).}

%------------------------
\newpage
\setcounter{footnote}{0}
\thispagestyle{fancy}
\fancyhead{}
\fancyfoot{}
%\footskip=-1cm

\lhead{\changefont{cmss}{bx}{n} \small Revisión 2013}
\chead{\changefont{cmss}{bx}{n} \small Domingo del Santísimo Cuerpo y Sangre de Cristo}
\rhead{\changefont{cmss}{bx}{n} \small Ciclo A}
\rfoot{\changefont{cmss}{bx}{n}\large\thepage}

\vspace{-5mm}

\begin{center}
{\large\it Discurso sobre el Pan de Vida}
\end{center}

\vspace{-5mm}


\begin{tabbing}

{\changefont{cmss}{bx}{n} Entrada:\ \ \ \ \ }\= El Señor nos llama (estr. 2), Iglesia peregrina, Somos la familia de Jesús. \\ \\

{\changefont{cmss}{bx}{n} Salmo:} \> 147 ant. 1 ``Glorifica al Señor, Jerusalén...?'' (estr. 1, 2 o 3). \\ \\

{\changefont{cmss}{bx}{n} Ofrendas:} \> Los frutos de la tierra, Recibe oh Dios eterno, Padre nuestro recibid, \\
\> Te presentamos, Bendito seas.\\ \\

{\changefont{cmss}{bx}{n} Comunión:} \> Panis angelicus, Yo soy el pan de vida\footnotemark[1], Es mi Padre\footnotemark[1], Este es mi cuerpo\footnotemark[1],\\
\>  En la postrera cena\footnotemark[1], Vayamos a la mesa\footnotemark[1], Cuerpo y Sangre de Jesús,  \\
\>  En memoria tuya, Escondido,  \\ \\

\vspace{-1mm}

{\changefont{cmss}{bx}{n} Post-com.:} \> Nuestro maná, Adoremos a Dios, Tantum Ergo (Tan sublime Sacramento), \\
\> Alabado sea el Santísimo, Te adoramos hostia divina.\\ \\

\vspace{-1mm}

{\changefont{cmss}{bx}{n} Salida:} \> Canción del testigo\footnotemark[2], En medio de los pueblos, Mi camino eres tú,   \\
\> Anunciaremos tu Reino, Vayan todos por el mundo.  \\  \\

\end{tabbing}

\vspace{-17mm}

\footnotetext[1]{Cfr. el discurso del Pan de Vida (Jn 6,51-58).}
\footnotetext[2]{Porque la brasa del ángel que rozó los labios de Isaías pre-figura la Eucaristía.}

%------------------------
\newpage
\setcounter{footnote}{0}
\thispagestyle{fancy}
\fancyhead{}
\fancyfoot{}
%\footskip=-1cm

\lhead{\changefont{cmss}{bx}{n} \small Revisión 2013}
\chead{\changefont{cmss}{bx}{n} \small Domingo II - Durante el año}
\rhead{\changefont{cmss}{bx}{n} \small Ciclo A}
\rfoot{\changefont{cmss}{bx}{n}\large\thepage}

\vspace{-1mm}

\begin{center}
{\large\it Jesús, el Cordero de Dios }
\end{center}

\vspace{-1mm}

\begin{tabbing}

{\changefont{cmss}{bx}{n} Entrada:\ \ \ \ \ }\= Pueblo de reyes, Vine a alabar, Un solo Señor (est. 3), Alabaré a mi Señor (est. 2).  \\ \\

{\changefont{cmss}{bx}{n} Salmo:} \> 26 ant. 1 ``Cantaré y celebraré al Señor'' (estr. 1 y 7).\\ \\

{\changefont{cmss}{bx}{n} Ofrendas:} \> Al altar del Señor\footnotemark[1], Comienza el sacrificio\footnotemark[1], Pan de vida y bebida de luz\footnotemark[2], \\
\> Padre nuestro recibid, Bendito seas. \\ \\

{\changefont{cmss}{bx}{n} Comunión:} \> Pueblo de reyes, Como Cristo nos amó, Yo soy el camino, Bendeciré al Señor.  \\ \\

{\changefont{cmss}{bx}{n} Post-com.:} \> El Señor es mi fortaleza (de Taizé)\footnotemark[3], Alabe todo el mundo (de Taizé), \\
\> Tan cerca de mí. \\ \\

{\changefont{cmss}{bx}{n} Salida:} \> Canción del testigo\footnotemark[4], Mi camino eres tú, Soy peregrino\footnotemark[4].\\  \\

\end{tabbing}

\vspace{-12mm}

\footnotetext[1]{Cfr. con la antífona de comunión del día (Sal 22,5). }
\footnotetext[2]{Cfr. la 2$^\circ$ estrofa con Is 49,6.}
\footnotetext[3]{Cfr. con Is 49,5.}
\footnotetext[4]{Cfr. la 3$^\circ$ estrofa de \emph{Canción del testigo} y la 2$^\circ$ estrofa de \emph{Soy peregrino} con la persona de Juan el Bautista \mbox{(Jn 1,29-34).}}


%------------------------
\newpage
\setcounter{footnote}{0}
\thispagestyle{fancy}
\fancyhead{}
\fancyfoot{}
%\footskip=-1cm

\lhead{\changefont{cmss}{bx}{n} \small Revisión 2013}
\chead{\changefont{cmss}{bx}{n} \small Domingo III - Durante el año}
\rhead{\changefont{cmss}{bx}{n} \small Ciclo A}
\rfoot{\changefont{cmss}{bx}{n}\large\thepage}

\vspace{-3mm}

\begin{center}
{\large\it El comienzo de la predicación de Jesús }
\end{center}

\vspace{-2mm}

\begin{tabbing}

{\changefont{cmss}{bx}{n} Entrada:\ \ \ \ \ }\= Pueblo de Dios\footnotemark[1], El Señor nos llama, Vine a alabar, Vienen con alegría.  \\ \\

{\changefont{cmss}{bx}{n} Salmo:} \> 26 ant. 2 ``El Señor es mi luz...'' (estr. 1, 3 o 7).\\ \\

{\changefont{cmss}{bx}{n} Ofrendas:} \> Pan de vida y bebida de luz\footnotemark[2], Recibe oh Dios eterno\footnotemark[2], Toma Señor nuestra vida\footnotemark[2], \\
\> Bendito seas. \\ \\

{\changefont{cmss}{bx}{n} Comunión:} \> Pescador de hombres\footnotemark[3], Yo soy el camino\footnotemark[3], Jesús te seguiré\footnotemark[3], Como Cristo nos amó. \\ \\

{\changefont{cmss}{bx}{n} Post-com.:} \> Mirarte sólo a ti Señor\footnotemark[4], El Señor es mi fortaleza (Taizé)\footnotemark[4], Bendigamos al Señor \\
\> (Cántico de caridad)\footnotemark[4], Si el mismo pan comimos\footnotemark[4]. \\ \\

{\changefont{cmss}{bx}{n} Salida:} \> En medio de los pueblos, Mi camino eres tú, Canción del misionero.\\  \\

\end{tabbing}

\vspace{-15mm}

\footnotetext[1]{Cfr. \emph{Pueblo de Dios} (est. 1) con la antífona de entrada del día (Sal 95 1.6). }
\footnotetext[2]{Cfr. \emph{Pan de vida y bebida de luz} (est. 1) y \emph{Toma Señor nuestra vida} (est. 3 y 4) con 1 Cor 1,10 y Mt 4,17; cfr. \emph{Recibe oh Dios eterno} (est. 1) con  1 Cor 1,10-12. }
\footnotetext[3]{Cfr. \emph{Pescador de hombres} con Mt 4,18-22; cfr. \emph{Yo soy el camino} (est. 1) con antífona de comunión del día \mbox{(Jn 8,12)}; cfr. \emph{Jesús te seguiré} (estribillo) con el relato equivalente de Mt 4,18-20 en (Jn 1,37-42) y las est. con Mt 4,23. }
\footnotetext[4]{Cfr. \emph{Miarate sólo a ti} y \emph{El Señor es mi fortaleza} con Sal 26,13-14; cfr. \emph{Bendigamos al Señor} y \emph{Si el mismo pan comimos} con 1 Cor 1,10.}


%------------------------
\newpage
\setcounter{footnote}{0}
\thispagestyle{fancy}
\fancyhead{}
\fancyfoot{}
%\footskip=-1cm

\lhead{\changefont{cmss}{bx}{n} \small Revisión 2013}
\chead{\changefont{cmss}{bx}{n} \small Domingo IV - Durante el año}
\rhead{\changefont{cmss}{bx}{n} \small Ciclo A}
\rfoot{\changefont{cmss}{bx}{n}\large\thepage}

\vspace{-3mm}

\begin{center}
{\large\it Las Bienaventuranzas }
\end{center}

\vspace{-2mm}

\begin{tabbing}

{\changefont{cmss}{bx}{n} Entrada:\ \ \ \ \ }\= Brilló la luz\footnotemark[1], El sermón de la montaña\footnotemark[1], Iglesia peregrina\footnotemark[1], \\
\> Juntos como hermanos (est. 2). \\ \\

{\changefont{cmss}{bx}{n} Salmo:} \> 145 ant. 1 ``El Señor es fiel a su palabra...'' (estr. 4, 5 o 6).\\ 
\> (antífona de reemplazo: Sal 127 ant. 1 ``Feliz quien ama al Señor...'') \\ \\

{\changefont{cmss}{bx}{n} Ofrendas:} \> Coplas de Yaravi, Te ofrecemos Padre nuestro (vidala), \\
\>  Pan de vida y bebida de luz, Te presentamos. \\ \\

{\changefont{cmss}{bx}{n} Comunión:} \> Queremos ser Señor\footnotemark[1], Brilló la luz\footnotemark[1], Bienaventurados los pobres\footnotemark[1], \\
\> Simple oración, Pescador de hombres\footnotemark[1].  \\ \\

{\changefont{cmss}{bx}{n} Post-com.:} \> Cuántas gracias te debemos, Aleluia Cristo vino con su paz, Vaso nuevo.  \\ \\

{\changefont{cmss}{bx}{n} Salida:} \> Simple oración, Anunciaremos tu reino, Mi camino eres tú, Vayan por el mundo. \\  \\

\end{tabbing}

\vspace{-15mm}

\footnotetext[1]{Cfr. \emph{Brilló la luz}, \emph{El sermón de la montaña}, Iglesia peregrina (estribillo), \emph{Queremos ser Señor} y \emph{Bienaventurados los pobres} con Mt 5,1-12 (este último se generalmente en cuaresma); cfr. \emph{Pescador de hombres} (est. 1) con \mbox{1 Cor 1,26-27}. }



%------------------------
\newpage
\setcounter{footnote}{0}
\thispagestyle{fancy}
\fancyhead{}
\fancyfoot{}
%\footskip=-1cm

\lhead{\changefont{cmss}{bx}{n} \small Revisión 2013}
\chead{\changefont{cmss}{bx}{n} \small Domingo V - Durante el año}
\rhead{\changefont{cmss}{bx}{n} \small Ciclo A}
\rfoot{\changefont{cmss}{bx}{n}\large\thepage}

\vspace*{-10mm}

\begin{center}
{\large\it La sal de la tierra y la luz del mundo }
\end{center}

\vspace{-3mm}

\begin{tabbing}

{\changefont{cmss}{bx}{n} Entrada:\ \ \ \ \ }\= Brilló la luz\footnotemark[1], El sermón de la montaña\footnotemark[1], Vienen con alegría\footnotemark[2], \\
\> Un pueblo que camina (est. 2), Juntos como hermanos (est. 2). \\ \\

{\changefont{cmss}{bx}{n} Salmo:} \> 23 ant. 2 ``!`Felices los que son fieles al Señor...!'' (estr. 1, 2 o 3).\\ 
\> (antífona de reemplazo: Sal 26 ant. 2 ``El Señor es mi luz...'') \\ \\

{\changefont{cmss}{bx}{n} Ofrendas:} \> Pan de vida y bebida de luz, Coplas de Yaravi, Ofrenda de amor (Por los niños),  \\
\>  Bendito seas, Te presentamos. \\ \\

{\changefont{cmss}{bx}{n} Comunión:} \> Yo soy el camino\footnotemark[2], Queremos ser Señor\footnotemark[1], Brilló la luz\footnotemark[1], Simple oración. \\ \\

{\changefont{cmss}{bx}{n} Post-com.:} \> Cántico de Caridad\footnotemark[2],$\,$Al atardecer de la vida\footnotemark[2],$\,$Adoremos a Dios\footnotemark[2],$\,$Mirarte sólo a ti.  \\ \\

{\changefont{cmss}{bx}{n} Salida:} \> Vayan todos por el mundo\footnotemark[3], Mi camino eres tú\footnotemark[3], Simple oración, \\
\> Madre de los peregrinos. \\  \\

\end{tabbing}

\vspace{-15mm}

\footnotetext[1]{Cfr. \emph{Brilló la luz}, \emph{Queremos ser Señor} y \emph{El sermón de la montaña} con ant. comunión del día (Mt 5,4.6). }
\footnotetext[2]{Cfr. \emph{Yo soy el camino} con  aclamación al evangelio Jn 8,12; cfr. \emph{Vienen con alegría} y \emph{Cántico de Caridad (Bendigamos al Señor)} con Is 58,7-10 y Mt 5,13-16; cfr. \emph{Al atardecer de la vida} (est 1) con Is 58,7-10; cfr. \emph{Adoremos a Dios} con la ant. de entrada Sal 94,6–7. }
\footnotetext[3]{Cfr. \emph{Vayan todos por el mundo} con  Mt 5,13-16; cfr. \emph{Mi camino eres tú} con aclamación al evangelio Jn 8,12. }



%------------------------
\newpage
\setcounter{footnote}{0}
\thispagestyle{fancy}
\fancyhead{}
\fancyfoot{}
%\footskip=-1cm

\lhead{\changefont{cmss}{bx}{n} \small Revisión 2013}
\chead{\changefont{cmss}{bx}{n} \small Domingo VI - Durante el año}
\rhead{\changefont{cmss}{bx}{n} \small Ciclo A}
\rfoot{\changefont{cmss}{bx}{n}\large\thepage}

\vspace*{-10mm}

\begin{center}
{\large\it Jesús y la Ley }
\end{center}

\vspace{-3mm}

\begin{tabbing}

{\changefont{cmss}{bx}{n} Entrada:\ \ \ \ \ }\= El sermón de la montaña\footnotemark[1],$\,$Soy peregrino\footnotemark[1],$\,$Que alegría,$\,$Juntos como hermanos (2). \\ \\

{\changefont{cmss}{bx}{n} Salmo:} \> 118 ant. 2 ``!`Felices los que escuchan...!'' (estr. 1, 2 o 3).\\ 
\> (antífona de reemplazo: Sal 18b ant. 1 ``Tu palabra, Señor,...'') \\ \\

{\changefont{cmss}{bx}{n} Ofrendas:} \> Al altar nos acercamos\footnotemark[2], Ofrenda de amor (Por los niños)\footnotemark[2], Los frutos de la tierra\footnotemark[2], \\
\>  Recibe oh Padre Santo, Recibe oh Dios el pan, Toma Señor nuestra vida. \\ \\

{\changefont{cmss}{bx}{n} Comunión:} \> Simple oración\footnotemark[3], La canción de la Alianza\footnotemark[3], Queremos ser Señor\footnotemark[3], \\
\>  Como Cristo nos amó, Si yo no tengo amor. \\ \\

{\changefont{cmss}{bx}{n} Post-com.:} \> Al atardecer de la vida\footnotemark[3], Si el mismo pan comimos, Vaso nuevo (El alfarero).  \\ \\

{\changefont{cmss}{bx}{n} Salida:} \> Vayan por el mundo, Anunciaremos tu reino, Simple oración\footnotemark[3], Soy peregrino\footnotemark[1]. \\  \\

\end{tabbing}

\vspace{-15mm}

\footnotetext[1]{Cfr. \emph{El sermón de la montaña} (est. 1) con aclamación al evangelio (Mt 11,25); cfr. \emph{Soy peregrino} (est 1) con Sal 118,1. Se acostumbra usar este canto de salida, pero es válido para procesión de entrada. }
\footnotetext[2]{Cfr. \emph{Al altar nos acercamos} (est. 3) y \emph{Ofrenda de amor} con Mt 5,23-24; cfr. \emph{Los frutos de la tierra} (estribillos) con Mt 5,19.  }
\footnotetext[3]{Cfr. \emph{Simple oración}, \emph{Queremos ser Señor}, \emph{La canción de la Alianza} y \emph{Al atardecer de la vida} con  Mt 5,19-26; cfr. \emph{Como Cristo nos amó} (est. 2) con antífona de comunión Jn 3,16. }


%------------------------
\newpage
\setcounter{footnote}{0}
\thispagestyle{fancy}
\fancyhead{}
\fancyfoot{}
%\footskip=-1cm

\lhead{\changefont{cmss}{bx}{n} \small Revisión 2013}
\chead{\changefont{cmss}{bx}{n} \small Domingo VII - Durante el año}
\rhead{\changefont{cmss}{bx}{n} \small Ciclo A}
\rfoot{\changefont{cmss}{bx}{n}\large\thepage}

\vspace*{-10mm}

\begin{center}
{\large\it La ley del talión y el amor a los enemigos }
\end{center}

\vspace{-3mm}

\begin{tabbing}

{\changefont{cmss}{bx}{n} Entrada:\ \ \ \ \ }\= El Señor nos llama, El sermón de la montaña\footnotemark[1], Vienen con alegría, \\
\> Iglesia peregrina de Dios. \\ \\

{\changefont{cmss}{bx}{n} Salmo:} \> 102 ant. 2 ``El amor del Señor permanece...'' (estr. 1, 2, 4, 5 o 6).\\ \\

{\changefont{cmss}{bx}{n} Ofrendas:} \> Una espiga\footnotemark[2], Pan de vida y bebida de luz\footnotemark[2], Bendeciré al Señor\footnotemark[2], Bendito seas. \\ \\

{\changefont{cmss}{bx}{n} Comunión:} \> Dios me dio a mi hermano\footnotemark[3], Si yo no tengo amor\footnotemark[3], Cántico de caridad , \\
\>  (Bendigamos al Señor)\footnotemark[3], La canción de la Alianza\footnotemark[3], Jesucristo danos de este pan. \\ \\

{\changefont{cmss}{bx}{n} Post-com.:} \> Al atardecer de la vida\footnotemark[1],$\,$Ubi caritas (Taizé),$\,$Si el mismo pan comimos,$\,$Vaso nuevo.  \\ \\

{\changefont{cmss}{bx}{n} Salida:} \> Simple oración\footnotemark[3], Anunciaremos tu reino, Madre de los peregrinos, \\
\> Santa María del camino, Canción del testigo. \\  \\

\end{tabbing}

\vspace{-15mm}

\footnotetext[1]{Cfr. \emph{El sermón de la montaña} (est. 2-4) y \emph{Al atardecer de la vida}  con Mt 5,43-45. }
\footnotetext[2]{Cfr. \emph{Una espiga} (est. 4) y \emph{Pan de vida y bebida de luz} (est. 1) con Lev 19,18 y Mt 5,43-45; cfr. \emph{Bendeciré al Señor} con Sal 102,1-2.  }
\footnotetext[3]{Cfr. \emph{Dios me dio a mi hermano}, \emph{Cántico de caridad} y \emph{Simple oración} con Lev 19,18 y Mt 5,43-48; cfr. \emph{Si yo no tengo amor} con Mt 5,43-45; cfr. \emph{La canción de la Alianza} (est. 1)  con Mt 5,43-44 y su paralelo en 1 Jn 4,21. }



%------------------------
\newpage
\setcounter{footnote}{0}
\thispagestyle{fancy}
\fancyhead{}
\fancyfoot{}
%\footskip=-1cm

\lhead{\changefont{cmss}{bx}{n} \small Revisión 2013}
\chead{\changefont{cmss}{bx}{n} \small Domingo VIII - Durante el año}
\rhead{\changefont{cmss}{bx}{n} \small Ciclo A}
\rfoot{\changefont{cmss}{bx}{n}\large\thepage}

\vspace*{-10mm}

\begin{center}
{\large\it La confianza en la Providencia }
\end{center}

\vspace{-3mm}

\begin{tabbing}

{\changefont{cmss}{bx}{n} Entrada:\ \ \ \ \ }\= Caminaré\footnotemark[1], Juntos como hermanos (est. 2)\footnotemark[1], El Señor nos llama\footnotemark[1], Vine a alabar. \\ \\

{\changefont{cmss}{bx}{n} Salmo\footnotemark[2]:} \> 30 ant. 1 ``En tus manos, Señor,...'' (estr. 1, 3, 8 o 9).\\ 
\> (antífona de reemplazo: Sal 24 ant. 1 ``A ti elevo mi alma,...'') \\ \\

{\changefont{cmss}{bx}{n} Ofrendas:} \> Mira nuestra ofrenda\footnotemark[2], Nuestros dones\footnotemark[2], Padre nuestro recibid\footnotemark[2],\\ 
\> Te ofrecemos Padre nuestro\footnotemark[2]. \\ \\

{\changefont{cmss}{bx}{n} Comunión:} \> Más cerca oh Dios\footnotemark[3], Es mi Padre\footnotemark[3], Yo soy el camino, Como Cristo nos amó. \\ \\

{\changefont{cmss}{bx}{n} Post-com.:} \> Nada te turbe (Taizé)\footnotemark[2], Cuántas gracias te debemos\footnotemark[2], Mirarte sólo a ti\footnotemark[2], \\
\> El Señor es mi fortaleza (Taizé)\footnotemark[3], Adoremos a Dios\footnotemark[2]. \\ \\ 

{\changefont{cmss}{bx}{n} Salida:} \> Mi camino eres tú\footnotemark[2], Vayan todos por el mundo, Anunciaremos tu reino,  \\
\> Madre de los peregrinos, Santa María del camino. \\  \\

\end{tabbing}

\vspace{-15mm}

\footnotetext[1]{Cfr. \emph{Caminaré} (est. 1-3) con antífona de entrada (Sal 17,19-20); cfr. \emph{Juntos como hermanos} (est. 2) con Sal 61,6-9 y Sal 17,19-20; cfr. \emph{El Señor nos llama} (est. 1) con Is 49,15. }
\footnotetext[2]{Cantar el Aleluia ``Busca primero el reino...''; cfr. todos estos cantos con Mt 6,25-34.  }
\footnotetext[3]{Cfr. \emph{Más cerca oh Dios} (est. 1-2) y \emph{Es mi Padre} (est. 1) con Mt 6,25-34; cfr. \emph{El Señor es mi fortaleza} con antífona de entrada (Sal 17,19-20). }



%------------------------
\newpage
\setcounter{footnote}{0}
\thispagestyle{fancy}
\fancyhead{}
\fancyfoot{}
%\footskip=-1cm

\lhead{\changefont{cmss}{bx}{n} \small Revisión 2013}
\chead{\changefont{cmss}{bx}{n} \small Domingo IX - Durante el año}
\rhead{\changefont{cmss}{bx}{n} \small Ciclo A}
\rfoot{\changefont{cmss}{bx}{n}\large\thepage}

\vspace*{-10mm}

\begin{center}
{\large\it La necesidad de practicar la Palabra de Dios }
\end{center}

\vspace{-3mm}

\begin{tabbing}

{\changefont{cmss}{bx}{n} Entrada:\ \ \ \ \ }\= Iglesia peregrina de Dios\footnotemark[1], Caminaré\footnotemark[1], Un pueblo que camina, \\
\> Juntos como hermanos (est. 2). \\ \\

{\changefont{cmss}{bx}{n} Salmo:} \> 30 ant. 1 ``En tus manos, Señor,...'' (estr. 1, 8 o 9).\\ 
\> (antífona de reemplazo: Sal 26 ant. 2 ``El Señor es mi luz,...'') \\ \\

{\changefont{cmss}{bx}{n} Ofrendas:} \> Toma Señor nuestra vida\footnotemark[2], Nuestros dones\footnotemark[2], Coplas de Yaraví, Bendeciré al Señor. \\ \\ 

{\changefont{cmss}{bx}{n} Comunión:} \> Queremos ser Señor\footnotemark[2], Yo soy el camino\footnotemark[2], Simple oración, Este es mi Cuerpo. \\ \\

{\changefont{cmss}{bx}{n} Post-com.:} \> Al atardecer de la vida, Mirarte sólo a ti, El Señor es mi fortaleza (Taizé), \\
\>  Tan cerca de mí. \\ \\ 

{\changefont{cmss}{bx}{n} Salida:} \> Anunciaremos tu reino, Soy peregrino, Mi camino eres tú, Simple oración.   \\ \\

\end{tabbing}

\vspace{-15mm}

\footnotetext[1]{Cfr. \emph{Iglesia peregrina} (est. 1) con Rom 3,23-25; cfr. \emph{Caminaré} (est. 1) con antífona de comunión (Sal 16,6.8). }
\footnotetext[2]{Cfr. \emph{Toma Señor nuestra vida} (est. 4) y \emph{Yo soy el camino} (est. 4) con aclamación al evangelio (Jn 15,5); \mbox{cfr. \emph{Nuestros dones}} con Mt 7,21-27; cfr. \emph{Queremos ser Señor} (estribillo y est. 3) con Rom 3,22-24.28. }



%------------------------
\newpage
\setcounter{footnote}{0}
\thispagestyle{fancy}
\fancyhead{}
\fancyfoot{}
%\footskip=-1cm

\lhead{\changefont{cmss}{bx}{n} \small Revisión 2013}
\chead{\changefont{cmss}{bx}{n} \small Domingo X - Durante el año}
\rhead{\changefont{cmss}{bx}{n} \small Ciclo A}
\rfoot{\changefont{cmss}{bx}{n}\large\thepage}

\vspace{-1mm}

\begin{center}
{\large\it El llamado de Mateo y la actitud de Jesús hacia los pecadores }
\end{center}

\vspace{-1mm}

\begin{tabbing}

{\changefont{cmss}{bx}{n} Entrada:\ \ \ \ \ }\= El sermón de la montaña, Contritos nos postramos, Juntos como hermanos (est.2),  \\ 
\>  Caminaré, Vine a alabar. \\ \\

{\changefont{cmss}{bx}{n} Salmo:} \> 18b ant. 1 ``Tu palabra, Señor, es la verdad...'' (estr. 1, 2, 3, 4, 5 o 7).\\ \\

{\changefont{cmss}{bx}{n} Ofrendas:} \> Recibe oh Dios el pan\footnotemark[2], Comienza el sacrificio\footnotemark[1], Pan de vida y bebida de luz, \\
\> Al altar del Señor, Coplas de Yaraví. \\ \\

{\changefont{cmss}{bx}{n} Comunión:} \> Pescador de hombres, Queremos ser Señor, Vuelve a mí\footnotemark[2], Simple oración, \\
\> La canción de la Alianza. \\ \\

{\changefont{cmss}{bx}{n} Post-com.:} \> Tan cerca de mí, Vaso nuevo (El alfarero), El Señor es mi fortaleza (de Taizé). \\ \\

{\changefont{cmss}{bx}{n} Salida:} \> Canción del testigo, Vayan todos por el mundo, Anunciaremos tu Reino,   \\
\>  Mi camino eres Tú, Quiero decir que sí. \\  \\

\end{tabbing}

\vspace{-12mm}

\footnotetext[1]{Cfr. la 2$^\circ$ estrofa con la 2$^\circ$ lectura (Rom 4,18,25). }
\footnotetext[2]{Cfr. Os. 6,6 y Mt 9,13. Ver, adem\'as, comentarios al canto \emph{Vuelve a mí}. Este canto se asocia al tiempo de cuaresma, pero la 1$^\circ$ lectura y el evangelio son una referencia muy directa del texto.}

%------------------------
\newpage
\setcounter{footnote}{0}
\thispagestyle{fancy}
\fancyhead{}
\fancyfoot{}
%\footskip=-1cm

\lhead{\changefont{cmss}{bx}{n} \small Revisión 2013}
\chead{\changefont{cmss}{bx}{n} \small Domingo XI - Durante el año}
\rhead{\changefont{cmss}{bx}{n} \small Ciclo A}
\rfoot{\changefont{cmss}{bx}{n}\large\thepage}

\vspace*{-10mm}

\begin{center}
{\large\it Institución de los Doce }
\end{center}

\vspace{-3mm}

\begin{tabbing}

{\changefont{cmss}{bx}{n} Entrada:\ \ \ \ \ }\= Pueblo de Dios\footnotemark[1], Pueblo de Reyes\footnotemark[1], Vienen con alegría\footnotemark[1], Un pueblo que camina. \\ \\

{\changefont{cmss}{bx}{n} Salmo:} \> 22 ant. 1 ``El Señor es mi pastor...'' (estr. 1 y 2).\\ \\

{\changefont{cmss}{bx}{n} Ofrendas:} \> Pan de vida y bebida de luz, Mira nuestra ofrenda, Padre nuestro recibid, \\
\>  Te presentamos, Bendito seas. \\ \\

{\changefont{cmss}{bx}{n} Comunión:} \> Mensajero de la paz\footnotemark[2], Yo soy el camino\footnotemark[2], Pescador de hombres,\\
\> El Señor de Galilea.\\ \\

{\changefont{cmss}{bx}{n} Post-com.:} \> Mirarte sólo a Tí\footnotemark[3], El Señor es mi fortaleza\footnotemark[3], Adoremos a Dios (estr. 3). \\ \\

{\changefont{cmss}{bx}{n} Salida:} \> Canción del misionero\footnotemark[4], Soy peregrino\footnotemark[4], Oh María\footnotemark[5], Santa María del camino.  \\ \\

\end{tabbing}

\vspace{-15mm}

\footnotetext[1]{Cfr. \emph{Pueblo de Reyes} con Ex 19,5-6, \emph{Pueblo de Dios} con Ex 19,5-6 y Sal 99,2-3, \emph{Vienen con alegría} con Mt 10,8.}
\footnotetext[2]{Cfr. \emph{Mensajero de la paz} con Mt 9,37 y Mt 10,6-8, \emph{Yo soy el camino} con Sal 99,3 y Mt 9,36.}
\footnotetext[3]{Cfr. con la antífona de entrada del día (Sal 26,8) y con Mt 9,36.}
\footnotetext[4]{Cfr. \emph{Canción del misionero} con Mt 10,6-8 y \emph{Soy peregrino} con Mt 10,7-8. }
\footnotetext[5]{Debido a que habla de la compasión y del consuelo, recordando Mt 9,36. }

%------------------------
\newpage
\setcounter{footnote}{0}
\thispagestyle{fancy}
\fancyhead{}
\fancyfoot{}
%\footskip=-1cm

\lhead{\changefont{cmss}{bx}{n} \small Revisión 2013}
\chead{\changefont{cmss}{bx}{n} \small Domingo XII - Durante el año}
\rhead{\changefont{cmss}{bx}{n} \small Ciclo A}
\rfoot{\changefont{cmss}{bx}{n}\large\thepage}

\vspace*{-10mm}

\begin{center}
{\large\it La valentía de los Apóstoles }
\end{center}

\vspace{-3mm}

\begin{tabbing}

{\changefont{cmss}{bx}{n} Entrada:\ \ \ \ \ }\= Juntos como hermanos (estr. 2),  Caminaré, Vienen con alegría, Vine a alabar.   \\ \\

{\changefont{cmss}{bx}{n} Salmo:} \> 24 ant. 1 ``A ti elevo mi alma...'' (estr. 1, 2 o 4).\\ \\

{\changefont{cmss}{bx}{n} Ofrendas:} \> Bendeciré al Señor\footnotemark[1], Ofrenda de amor (Por los niños)\footnotemark[1], \\
\> Te ofrecemos Padre nuestro\footnotemark[2], Recibe oh Dios eterno. \\  \\

{\changefont{cmss}{bx}{n} Comunión:} \>  Bendeciré al Señor\footnotemark[1], Creo en ti Señor (Más cerca oh Dios), Quédate con nosotros, \\
\> El Señor de Galilea. \\ \\

{\changefont{cmss}{bx}{n} Post-com.:} \> Nada te turbe (de Taizé), Mirarte sólo a ti, Tu fidelidad, \\
\> Si el mismo pan comimos (estribillo). \\ \\

{\changefont{cmss}{bx}{n} Salida:} \> Canción del testigo\footnotemark[3], Soy peregrino, Anunciaremos tu reino, \\
\> En medio de los pueblos. \\ \\

\end{tabbing}

\vspace{-15mm}

\footnotetext[1]{Cfr. \emph{Bendeciré al Señor} (Sal 33) con Sal 68,33 y \emph{Ofrenda de amor} con Jer 20,13.}
\footnotetext[2]{Versión de la misa nicarag\"uense (versión moderna). Cfr. con Jer 20,13. }
\footnotetext[3]{Cfr. la 1$^\circ$ estrofa con Jer 20,9 (que está en el contexto de la 1$^\circ$ lectura del día (Jer  20,10-13)).}

%------------------------
\newpage
\setcounter{footnote}{0}
\thispagestyle{fancy}
\fancyhead{}
\fancyfoot{}
%\footskip=-1cm

\lhead{\changefont{cmss}{bx}{n} \small Revisión 2013}
\chead{\changefont{cmss}{bx}{n} \small Domingo XIII - Durante el año}
\rhead{\changefont{cmss}{bx}{n} \small Ciclo A}
\rfoot{\changefont{cmss}{bx}{n}\large\thepage}

\vspace*{-10mm}

\begin{center}
{\large\it La manera de recibir a los Apóstoles }
\end{center}

\vspace{-3mm}

\begin{tabbing}

{\changefont{cmss}{bx}{n} Entrada:\ \ \ \ \ }\= Pueblo de reyes\footnotemark[1], Que alegría, Somos la familia de Jesús, Un pueblo que camina.   \\ \\

{\changefont{cmss}{bx}{n} Salmo:} \> 145 ant. 1 ``El Señor es fiel a su Palabra...'' (estr. 1, 3, 4 o 5).\\
\> (antífona de reemplazo: Sal 26 ant. 1 ``Cantaré y celebraré al Señor.'') \\ \\

{\changefont{cmss}{bx}{n} Ofrendas:} \> Señor te ofrecemos\footnotemark[1], Mira nuestra ofrenda, Los frutos de la tierra, Una espiga, \\
\> Toma Señor nuestra vida. \\  \\

{\changefont{cmss}{bx}{n} Comunión:} \>  Jesucristo danos de este pan\footnotemark[2], Creo en ti Señor (Más cerca oh Dios)\footnotemark[2],  \\
\> Como Cristo nos amó\footnotemark[2], Mensajero de la paz\footnotemark[2], Pueblo de reyes\footnotemark[1]. \\ \\

{\changefont{cmss}{bx}{n} Post-com.:} \> El Señor es mi fortaleza (de Taizé), Creo en ti Señor (Más cerca oh Dios),  \\
\> Cuantas gracias te debemos, Si el mismo pan comimos (estribillo). \\ \\

{\changefont{cmss}{bx}{n} Salida:} \> Canción del misionero, Mi camino eres tu, Vayan todos por el mundo,  \\
\> Anunciaremos tu reino. \\ \\

\end{tabbing}

\vspace{-15mm}

\footnotetext[1]{Cfr. \emph{Pueblo de reyes} con la aclamación al evangelio (1 Ped 2,9) y \emph{Señor te ofrecemos} con Sal 88,2-3.16-19.}
\footnotetext[3]{Cfr. la 1$^\circ$ y 2$^\circ$ estrofa de \emph{Jesucristo danos de este pan} con Rom 6,3-11, la  1$^\circ$ y 4$^\circ$ estrofa de \emph{Más cerca oh Dios de ti} con Mt 10,38, la  3$^\circ$ estrofa de \emph{Como Cristo nos amó} con Rom 6,4-11 y las estrofas 7 y 8 de \emph{Mensajero de la paz} con Mt 10,40. }


%------------------------
\newpage
\setcounter{footnote}{0}
\thispagestyle{fancy}
\fancyhead{}
\fancyfoot{}
%\footskip=-1cm

\lhead{\changefont{cmss}{bx}{n} \small Revisión 2013}
\chead{\changefont{cmss}{bx}{n} \small Domingo XIV - Durante el año}
\rhead{\changefont{cmss}{bx}{n} \small Ciclo A}
\rfoot{\changefont{cmss}{bx}{n}\large\thepage}

\vspace*{-10mm}

\begin{center}
{\large\it La revelación del evangelio a los humildes }
\end{center}

\vspace{-3mm}

\begin{tabbing}

{\changefont{cmss}{bx}{n} Entrada:\ \ \ \ \ }\= El sermón de la montaña\footnotemark[1], Un pueblo que camina (est. 3)\footnotemark[1], Vienen con alegría, \\
\> Vine a alabar.   \\ \\

{\changefont{cmss}{bx}{n} Salmo:} \> 144 ant. 1 ``Te alabamos Señor...'' (estr. 1, 4 o 5).\\ \\

{\changefont{cmss}{bx}{n} Ofrendas:} \> Recibe oh Dios eterno\footnotemark[2], Pan de vida y bebida de luz\footnotemark[2], Coplas de Yaraví\footnotemark[2], \\
\>   Bendeciré al Señor\footnotemark[2]. \\  \\

{\changefont{cmss}{bx}{n} Comunión:} \>  Panis angelicus\footnotemark[3], Bendeciré al Señor\footnotemark[2], Pescador de hombres\footnotemark[3], Yo soy el camino.\\  \\

{\changefont{cmss}{bx}{n} Post-com.:} \> Tan cerca e mí (estribillo), Cuantas gracias te debemos, Vaso nuevo (El alfarero)  \\
\>  Si el mismo pan comimos (estribillo). \\ \\

{\changefont{cmss}{bx}{n} Salida:} \> Junto a ti María\footnotemark[4], Madre de nuestro pueblo\footnotemark[4], Canto de María\footnotemark[4],  \\
\> Canción del misionero. \\   \\


\end{tabbing}

\vspace{-15mm}

\footnotetext[1]{Cfr. \emph{El sermón de la montaña} (est. 1) con Mt 11,25 y \emph{Un pueblo que camina} con (est. 3) con Mt 11,28-30.}
\footnotetext[2]{Cfr. \emph{Recibe oh Dios eterno} (est. 3) y \emph{Pan de vida y bebida de luz} (est. 3) con Mt 11,28-30. Cfr. \emph{Coplas de Yaraví} (est. 4) con Mt 11,25 y \emph{Bendeciré al Señor} (est. 1 y 3) con Mt 11,25 y con la antífona de comunión (Sal 33,9). }
\footnotetext[3]{Cfr. \emph{Panis angelicus} (latín o castellano) con Zac 9,9 y \emph{Pescador de hombres} (est. 1 y 3) con Mt 11,25 y Mt 11,28-30.}
\footnotetext[4]{Cfr. \emph{Junto a ti María} con Mt 11,25. Los demás cantos señalan a María como ejemplo de humildad cristiana.}


%------------------------
\newpage
\setcounter{footnote}{0}
\thispagestyle{fancy}
\fancyhead{}
\fancyfoot{}
%\footskip=-1cm

\lhead{\changefont{cmss}{bx}{n} \small Revisión 2013}
\chead{\changefont{cmss}{bx}{n} \small Domingo XV - Durante el año}
\rhead{\changefont{cmss}{bx}{n} \small Ciclo A}
\rfoot{\changefont{cmss}{bx}{n}\large\thepage}

\vspace*{-10mm}

\begin{center}
{\large\it La parábola del sembrador }
\end{center}

\vspace{-3mm}

\begin{tabbing}

{\changefont{cmss}{bx}{n} Entrada:\ \ \ \ \ }\= Iglesia peregrina\footnotemark[1], Pueblo de reyes\footnotemark[1], Vine a alabar, Juntos como hermanos (est. 2).\\ \\

{\changefont{cmss}{bx}{n} Salmo:} \> 64 con ant. del Sal 144 ``Te alabamos Señor...'' (estr. 7, 8, 9 o 10 del Sal 64).\\ 
\> (antífona de reemplazo: Sal 144 ant. 1 ``Te alabamos Señor...'') \\ \\

{\changefont{cmss}{bx}{n} Ofrendas:} \> Zamba del grano de trigo\footnotemark[2], Sé como el grano de trigo\footnotemark[2], Coplas de Yaraví\footnotemark[2],   \\
\>  Una espiga\footnotemark[2]. \\  \\

{\changefont{cmss}{bx}{n} Comunión:} \>  Coplas de Yaraví\footnotemark[2], Simple oración, Queremos ser Señor, Yo soy el camino\footnotemark[3].\\  \\

{\changefont{cmss}{bx}{n} Post-com.:} \> Vaso nuevo (El alfarero), Al atardecer de la vida, Todos unidos, Nuestro maná.  \\ \\

{\changefont{cmss}{bx}{n} Salida:} \> Vayan todos por el mundo\footnotemark[4], Salve oh Reina\footnotemark[4], Simple oración, Mi camino eres tú.   \\


\end{tabbing}

\vspace{-15mm}

\footnotetext[1]{Cfr. \emph{Iglesia peregrina} (estribillo) con Mt 13,8, y \emph{Pueblo de reyes} (estr. 1) con Is 55,10-11.}
\footnotetext[2]{Cfr. \emph{Sé como el grano de trigo} y \emph{Zamba del grano de trigo} con Is 55,10-11 y Mt 13,9. Cfr. \emph{Coplas de Yaraví} \mbox{(est. 1 y 3)} con Is 55,10-11 y Mt 13,1-9, respectivamente. Cfr. \emph{Una espiga} (est. 2) con Is 55,10-11, Rom 8,18-23 y \mbox{Mt 13,1-9}. }
\footnotetext[3]{Cfr. \emph{Yo soy el camino } con Is 55,10-11.}
\footnotetext[4]{Cfr. \emph{Vayan todos por el mundo} (est. 1) con Mt 13,1-9. Cfr. \emph{Salve oh Reina} (estr. 1) con Rom 8,22-23.}

%------------------------
\newpage
\setcounter{footnote}{0}
\thispagestyle{fancy}
\fancyhead{}
\fancyfoot{}
%\footskip=-1cm

\lhead{\changefont{cmss}{bx}{n} \small Revisión 2013}
    \chead{\changefont{cmss}{bx}{n} \small Domingo XVI - Durante el año}
\rhead{\changefont{cmss}{bx}{n} \small Ciclo A}
\rfoot{\changefont{cmss}{bx}{n}\large\thepage}

\vspace*{-10mm}

\begin{center}
{\large\it La parábola de la cizaña }
\end{center}

\vspace{-3mm}

\begin{tabbing}

{\changefont{cmss}{bx}{n} Entrada:\ \ \ \ \ }\= Iglesia peregrina\footnotemark[1], Vine a alabar\footnotemark[1], Juntos como hermanos (est. 2)\footnotemark[1], \\
\> Vienen con alegría.\\ \\

{\changefont{cmss}{bx}{n} Salmo:} \> 102 ant. 2 ``El amor del Señor...'' (estr. 1, 2, 4 o 5).\\ \\

{\changefont{cmss}{bx}{n} Ofrendas:} \> Los frutos de la tierra\footnotemark[2], Una espiga\footnotemark[2], Recibe oh Dios eterno\footnotemark[2], \\ 
\>Toma Señor nuestra vida\footnotemark[2],   Este es nuestro pan. \\  \\

{\changefont{cmss}{bx}{n} Comunión:} \>  Cuerpo y sangre de Jesús, Vayamos a la mesa, Este es mi cuerpo, \\ 
\> Yo soy el camino.\\  \\

{\changefont{cmss}{bx}{n} Post-com.:} \> Adoremos a Dios, El Señor es mi fortaleza, Espíritu de comunidad.  \\ \\

{\changefont{cmss}{bx}{n} Salida:} \> Vayan todos por el mundo\footnotemark[4], Anunciaremos tu reino, En medio de los pueblos, \\
\> Santa María del camino, Mi camino camino eres tú.  \\ \\


\end{tabbing}

\vspace{-12mm}

\footnotetext[1]{Cfr. \emph{Iglesia peregrina} (estribillo) Mt 13,24-32; cfr. \emph{Vine a alabar} con Sal 85,9; cfr. \emph{Juntos como hermanos} (est. 2) con Rom 8,27.}
\footnotetext[2]{Cfr. \emph{Los frutos de la tierra} y \emph{Una espiga} con Mt 13,24-30; cfr. Recibe oh Dios eterno (est. 1) con Sab 12,13.19; cfr. \emph{Toma Señor nuestra vida} (est. 2) con Mt 13,33. }
\footnotetext[4]{Cfr. \emph{Vayan todos por el mundo} (est. 1) con Mt 13,24-32.}

%------------------------
\newpage
\setcounter{footnote}{0}
\thispagestyle{fancy}
\fancyhead{}
\fancyfoot{}
%\footskip=-1cm

\lhead{\changefont{cmss}{bx}{n} \small Revisión 2013}
    \chead{\changefont{cmss}{bx}{n} \small Domingo XVII - Durante el año}
\rhead{\changefont{cmss}{bx}{n} \small Ciclo A}
\rfoot{\changefont{cmss}{bx}{n}\large\thepage}

\vspace*{-10mm}

\begin{center}
{\large\it Las parábolas del tesoro, la perla y la red }
\end{center}

\vspace{-3mm}

\begin{tabbing}

{\changefont{cmss}{bx}{n} Entrada:\ \ \ \ \ }\= El sermón de la montaña (est. 2 y 3)\footnotemark[1], Caminaré (est. 1)\footnotemark[1], Vine a alabar,  \\
\> Vienen con alegría.\\ \\

{\changefont{cmss}{bx}{n} Salmo\footnotemark[2]:} \> 18b ant. 1 ``Tu palabra, Señor,...'' (estr. 1, 4 o 6).\\ \\

{\changefont{cmss}{bx}{n} Ofrendas:} \> Pan de vida y bebida de luz,  Nuestros dones, Bendeciré al Señor, Te presentamos, \\ 
\> Bendito seas. \\  \\

{\changefont{cmss}{bx}{n} Comunión:} \> Jesucristo danos de este pan, Simple oración, Pescador de hombres,   \\ 
\> Este es mi cuerpo.\\  \\

{\changefont{cmss}{bx}{n} Post-com.:} \> Cuántas gracias te debemos, Si el mismo pan comimos, Mirarte sólo a ti, \\
\> Vaso nuevo.  \\ \\

{\changefont{cmss}{bx}{n} Salida:} \> Soy peregrino\footnotemark[3], Anunciaremos tu reino, Canción del misionero,  \\
\> Mi camino camino eres tú, En medio de los pueblos.  \\ \\


\end{tabbing}

\vspace{-12mm}

\footnotetext[1]{Cfr. \emph{El sermón de la montaña} (est. 2 y 3) con ant. de comunión Mt 5,7-8 y 1 Rey 3,10-12; cfr. \emph{Caminaré} (est. 1) con 1 Rey 3,5-12.}
\footnotetext[2]{Si se desea, se puede cantar el Aleluia ``Busca primero el reino...'' como aclamación al evangelio.}
\footnotetext[3]{Cfr. \emph{Soy peregrino} con Sal 118.}


%------------------------
\newpage
\setcounter{footnote}{0}
\thispagestyle{fancy}
\fancyhead{}
\fancyfoot{}
%\footskip=-1cm

\lhead{\changefont{cmss}{bx}{n} \small Revisión 2013}
    \chead{\changefont{cmss}{bx}{n} \small Domingo XVIII - Durante el año}
\rhead{\changefont{cmss}{bx}{n} \small Ciclo A}
\rfoot{\changefont{cmss}{bx}{n}\large\thepage}

\vspace*{-10mm}

\begin{center}
{\large\it La primera multiplicación de los panes }
\end{center}

\vspace{-3mm}

\begin{tabbing}

{\changefont{cmss}{bx}{n} Entrada:\ \ \ \ \ }\= Pueblo de Dios\footnotemark[1], El Señor nos llama, Qué alegría, Vienen con alegría.\\ \\

{\changefont{cmss}{bx}{n} Salmo:} \> 144 ant. 1 ``Te alabamos, Señor,...'' (estr. 4, 6 o 7).\\ \\

{\changefont{cmss}{bx}{n} Ofrendas:} \> Un niño se te acercó\footnotemark[2], Te ofrecemos oh Señor, Bendeciré al Señor, Bendito seas. \\  \\

{\changefont{cmss}{bx}{n} Comunión:} \> Como Cristo nos amó\footnotemark[3], Jesús Eucaristía\footnotemark[3], Jesucristo danos de este pan,    \\ 
\> Es mi Padre\footnotemark[3], Yo soy el camino\footnotemark[3], Jesús te seguiré\footnotemark[3].\\  \\

{\changefont{cmss}{bx}{n} Post-com.:} \> Nuestro maná\footnotemark[3], Cántico de Caridad (est. 1), Todos unidos, Adoremos a Dios.  \\ \\

{\changefont{cmss}{bx}{n} Salida:} \> Cantemos hermanos, Mi camino eres tu, En medio de los pueblos,   \\
\> Madre de los peregrinos, Oh María, Santa María del Camino.  \\ \\


\end{tabbing}

\vspace{-12mm}

\footnotetext[1]{Cfr. \emph{Pueblo de Dios} (est. 3) con Is 55,1-3 y Mt 14,13-21.}
\footnotetext[2]{Cfr. \emph{Un niño se te acercó} con Mt 14,13-21 y su paralelo Jn 6,1-13. }
\footnotetext[3]{Cfr. \emph{Como Cristo nos amó} (est. 4) con Rom 8,35-39: cfr. \emph{Jesús Eucaristía} (est. 1) con Mt 14,16; cfr. \emph{Es mi Padre} con Is 55,1-3 y Mt 14,13-21; cfr. \emph{Yo soy el camino} con ant. de comunión Jn 6,35; cfr. \emph{Jesús te seguiré} (est. 3) con \mbox{Mt 14,13-21}; cfr. \emph{Nuestro maná} con ant. de comunión Sal 16,20.}


%------------------------
\newpage
\setcounter{footnote}{0}
\thispagestyle{fancy}
\fancyhead{}
\fancyfoot{}
%\footskip=-1cm

\lhead{\changefont{cmss}{bx}{n} \small Revisión 2013}
    \chead{\changefont{cmss}{bx}{n} \small Domingo XIX - Durante el año}
\rhead{\changefont{cmss}{bx}{n} \small Ciclo A}
\rfoot{\changefont{cmss}{bx}{n}\large\thepage}

\vspace*{-10mm}

\begin{center}
{\large\it Jesús camina sobre el agua }
\end{center}

\vspace{-3mm}

\begin{tabbing}

{\changefont{cmss}{bx}{n} Entrada:\ \ \ \ \ }\= Iglesia peregrina\footnotemark[1], Caminaré\footnotemark[1], Qué alegría, Juntos como hermanos (est. 2), \\
\> Vine a alabar.\\ \\

{\changefont{cmss}{bx}{n} Salmo:} \> 84 ant. 1 ``!`Señor, revélanos tu amor...'' (estr. 4, 5 o 6).\\ \\

{\changefont{cmss}{bx}{n} Ofrendas:} \> Señor te ofrecemos\footnotemark[2], Padre nuestro recibid, Mira nuestra ofrenda, \\
\> Te ofrecemos Padre nuestro (vidala). \\  \\

{\changefont{cmss}{bx}{n} Comunión:} \> Más cerca oh Dios\footnotemark[3], El Señor de Galilea\footnotemark[2], Cuerpo y sangre de Jesús,    \\ 
\> Yo soy el camino\footnotemark[3].\\  \\

{\changefont{cmss}{bx}{n} Post-com.:} \> El Señor es mi fortaleza (Taize), Mirarte sólo a ti\footnotemark[4], Alabe todo el mundo (Taize), \\ 
\> Adoremos a Dios.  \\ \\

{\changefont{cmss}{bx}{n} Salida:} \> Soy peregrino\footnotemark[4], Canción del testigo, Mi camino eres tu,   \\
\> Salve oh Reina, Santa María del Camino.  \\ \\


\end{tabbing}

\vspace{-12mm}

\footnotetext[1]{Cfr. \emph{Iglesia peregrina} (est. 2) con Mt 14,22-33; cfr. \emph{Caminaré} (est. 1 y 2) con Mt 14,22-33. }
\footnotetext[2]{Cfr. \emph{Señor te ofrecemos} (est. 2 y 3) con Mt 14,22-33. }
\footnotetext[3]{Cfr. \emph{Más cerca oh Dios} (est. 2, 3 y 4) y \emph{El Señor de Galilea} (est. 1) con Mt 14,22-33.}
\footnotetext[4]{Cfr. \emph{Mirarte sólo a ti}, \emph{Soy peregrino} (estribillo) y \emph{Canción del testigo} (est. 1) con Mt 14,22-33.}

%------------------------
\newpage
\setcounter{footnote}{0}
\thispagestyle{fancy}
\fancyhead{}
\fancyfoot{}
%\footskip=-1cm

\lhead{\changefont{cmss}{bx}{n} \small Revisión 2013}
    \chead{\changefont{cmss}{bx}{n} \small Domingo XX - Durante el año}
\rhead{\changefont{cmss}{bx}{n} \small Ciclo A}
\rfoot{\changefont{cmss}{bx}{n}\large\thepage}

\vspace*{-10mm}

\begin{center}
{\large\it Curación de la hija de una cananea }
\end{center}

\vspace{-3mm}

\begin{tabbing}

{\changefont{cmss}{bx}{n} Entrada:\ \ \ \ \ }\= Pueblo de Dios\footnotemark[1], Pueblo de Reyes\footnotemark[1], Un pueblo que camina, \\
\> Somos la familia de Jesús, Vienen con alegría.\\ \\

{\changefont{cmss}{bx}{n} Salmo:} \> 66 ant. 1 ``!`A ti, Señor, te alabe la tierra...'' (estr. 1, 2 o 3).\\
\> (antífona de reemplazo: Sal 144 ant. 1 ``Te alabamos Señor...'') \\ \\

{\changefont{cmss}{bx}{n} Ofrendas:} \> Padre nuestro recibid\footnotemark[2], Te ofrecemos Padre nuestro (vidala)\footnotemark[2], Al altar del Señor\footnotemark[2],  \\
\> Recibe oh Dios eterno, Pan de vida y bebida de luz. \\  \\

{\changefont{cmss}{bx}{n} Comunión:} \> Cuerpo y sangre de Jesús\footnotemark[3], Escondido\footnotemark[3], Vayamos a la mesa, Es mi Padre,    \\ 
\> Bendeciré al Señor, Jesucristo danos de este pan.\\  \\

{\changefont{cmss}{bx}{n} Post-com.:} \> Cuántas gracias te debemos, Adoremos a Dios, El Señor es mi fortaleza (Taizé).  \\ \\

{\changefont{cmss}{bx}{n} Salida:} \> En medio de los pueblos\footnotemark[3], Oh María, Madre de los peregrinos, Mi camino eres tu.  \\ \\


\end{tabbing}

\vspace{-15mm}

\footnotetext[1]{Cfr. \emph{Pueblo de Dios} (estribillo y est. 1) con Sal 66,4-6; cfr. \emph{Pueblo de Reyes} (estribillo y est. 3) con Mt 15,22.26 y Rom 11,29-32. }
\footnotetext[2]{Cfr. \emph{Padre nuestro recibid} (est. 3) y \emph{Te ofrecemos Padre nuestro} (est. 1) con Mt 15,21-28; cfr. \emph{Al altar del Señor} (est. 3 con texto ``hoy nuestros ruegos, nuestro dolor...'') con Mt 15,21-28. }
\footnotetext[3]{Cfr. \emph{Cuerpo y sangre de Jesús} (est. 2) con Rom 11,13-15.29-32;  cfr. \emph{Escondido} (est. 5) y \emph{En medio de los pueblos} (estribillo) con Mt 15,21-28.}



%------------------------
\newpage
\setcounter{footnote}{0}
\thispagestyle{fancy}
\fancyhead{}
\fancyfoot{}
%\footskip=-1cm

\lhead{\changefont{cmss}{bx}{n} \small Revisión 2013}
    \chead{\changefont{cmss}{bx}{n} \small Domingo XXI - Durante el año}
\rhead{\changefont{cmss}{bx}{n} \small Ciclo A}
\rfoot{\changefont{cmss}{bx}{n}\large\thepage}

\vspace*{-10mm}

\begin{center}
{\large\it La profesión de fe de Pedro }
\end{center}

\vspace{-3mm}

\begin{tabbing}

{\changefont{cmss}{bx}{n} Entrada:\ \ \ \ \ }\= Un solo Señor\footnotemark[1],$\,$Caminaré (est. 1)\footnotemark[1],$\,$Vine a alabar\footnotemark[1],$\,$Juntos como hermanos (est. 2).\\ \\

{\changefont{cmss}{bx}{n} Salmo:} \> 137 ant. 1 ``Te doy gracias, Señor, por tu amor...'' (estr. 1, 2, 4 o 5).\\ 
\> (antífona de reemplazo: Sal 24 ant. 1 ``A ti elevo mi alma...'') \\ \\

{\changefont{cmss}{bx}{n} Ofrendas:} \> Los frutos de la tierra\footnotemark[2], Toma Señor nuestra vida\footnotemark[2], Padre nuestro recibid\footnotemark[2],  \\
\> Al altar del Señor\footnotemark[2], Este es nuestro pan, Bendito seas Señor. \\  \\

{\changefont{cmss}{bx}{n} Comunión:} \> Yo soy el pan de vida\footnotemark[3], Cuerpo y sangre de Jesús\footnotemark[3], Más cerca oh Dios,       \\ 
\> Oh buen Jesús, Es mi Padre, Bendeciré al Señor.\\  \\

{\changefont{cmss}{bx}{n} Post-com.:} \> Cuántas gracias te debemos, Adoremos a Dios, Alabe todo el mundo (Taizé).  \\ \\

{\changefont{cmss}{bx}{n} Salida:} \> En medio de los pueblos,$\,$Cantemos hermanos,$\,$Mi camino eres tu,$\,$Cantad a María.  \\ \\


\end{tabbing}

\vspace{-15mm}

\footnotetext[1]{Cfr. \emph{Un solo Señor} (estribillo) y \emph{Vine a alabar} (estribillo) con Mt 16,13-20; cfr. \emph{Caminaré} (estribillo y est. 1) con antífona de entrada del día (Sal 85,1.3). }
\footnotetext[2]{Cfr. \emph{Los frutos de la tierra}, \emph{Toma Señor nuestra vida} y \emph{Al altar del Señor} con antífona de comunión del día (Sal 103,13-15); cfr. \emph{Padre nuestro recibid} (est. 4) con Mt 16,16. }
\footnotetext[3]{Cfr. \emph{Yo soy el pan de vida} y \emph{Cuerpo y sangre de Jesús} con antífona de comunión del día (Jn 6,54). }





%------------------------
\newpage
\setcounter{footnote}{0}
\thispagestyle{fancy}
\fancyhead{}
\fancyfoot{}
%\footskip=-1cm

\lhead{\changefont{cmss}{bx}{n} \small Revisión 2013}
    \chead{\changefont{cmss}{bx}{n} \small Domingo XXII - Durante el año}
\rhead{\changefont{cmss}{bx}{n} \small Ciclo A}
\rfoot{\changefont{cmss}{bx}{n}\large\thepage}

\vspace*{-10mm}

\begin{center}
{\large\it El primer anuncio de la Pasión y las condiciones para seguir a Jesús }
\end{center}

\vspace{-3mm}

\begin{tabbing}

{\changefont{cmss}{bx}{n} Entrada:\ \ \ \ \ }\= El Sermón de la montaña\footnotemark[1], Qué alegría\footnotemark[1],$\,$Vine a alabar\footnotemark[1],$\,$Caminaré (est. 1)\footnotemark[1].\\ \\

{\changefont{cmss}{bx}{n} Salmo:} \> 62 ant. 1 ``Señor, mi Dios, te busco desde la aurora...'' (estr. 1, 2, 3).\\ 
\> (antífona de reemplazo: Sal 41 ant. 1 ``Mi alma tiene sed de Dios...'') \\ \\

{\changefont{cmss}{bx}{n} Ofrendas:}\footnotemark[2] \> Sé como el grano de trigo, Recibe oh Dios eterno, Te ofrecemos Padre nuestro$\,$(II),  \\
\>  Al altar del Señor, Recibe oh Dios el pan, Los frutos de la tierra. \\  \\

{\changefont{cmss}{bx}{n} Comunión:} \> Queremos ser Señor\footnotemark[3], Más cerca oh Dios\footnotemark[3], Quédate con nosotros\footnotemark[3],      \\ 
\> No hay mayor amor, Yo soy el camino. \\  \\

{\changefont{cmss}{bx}{n} Post-com.:} \> Cuántas gracias te debemos, Al atardecer de la vida, Nada te turbe (Taizé), \\
\> Si el mismo pan comimos.  \\ \\

{\changefont{cmss}{bx}{n} Salida:} \> Canción del testigo\footnotemark[3], En medio de los pueblos, Simple oración, Soy peregrino.  \\ \\


\end{tabbing}

\vspace{-15mm}

\footnotetext[1]{Cfr. \emph{El sermón de la montaña} (estribillo, est. 2 y 4) con Jer 20,8-9 y ant. de comunión Mt 5,9-10; cfr. \emph{Qué alegría} con Mt 16,21; cfr. \emph{Vine a alabar} (est. 1) con Jer 20,7-9; cfr. \emph{Caminaré} (estribillo y est. 1) con ant. entrada (Sal 85,3). }
\footnotetext[2]{Cfr. todos los cantos de ofrendas (menos \emph{Los frutos de la tierra}) con Rom 12,1. }
\footnotetext[3]{Cfr. \emph{Queremos ser Señor} y \emph{Más cerca oh Dios} (est. 1) con Mt 16,21-27; \emph{Quédate con nosotros} debe entenderse como las palabras de Pedro en Mt 16,22; cfr. \emph{Canción del testigo} (est. 1 y 3) con Jer 20,9. }



%------------------------
\newpage
\setcounter{footnote}{0}
\thispagestyle{fancy}
\fancyhead{}
\fancyfoot{}
%\footskip=-1cm

\lhead{\changefont{cmss}{bx}{n} \small Revisión 2013}
    \chead{\changefont{cmss}{bx}{n} \small Domingo XXIII - Durante el año}
\rhead{\changefont{cmss}{bx}{n} \small Ciclo A}
\rfoot{\changefont{cmss}{bx}{n}\large\thepage}

\vspace*{-10mm}

\begin{center}
{\large\it La corrección fraterna }
\end{center}

\vspace{-3mm}

\begin{tabbing}

{\changefont{cmss}{bx}{n} Entrada:\ \ \ \ \ }\= Juntos como hermanos\footnotemark[1], Iglesia peregrina \footnotemark[1], Somos la familia de Jesús, \\
\> Un pueblo que camina.\\ \\

{\changefont{cmss}{bx}{n} Salmo:} \> 94 ant. 1 ``Adoremos al Señor...'' (estr. 1 y 3).\\ \\

{\changefont{cmss}{bx}{n} Ofrendas:} \> Al altar nos acercamos\footnotemark[2], Pan de vida y bebida de luz\footnotemark[2], Una espiga\footnotemark[2],   \\
\>  Coplas de Yaraví, Recibe oh Dios eterno, Recibe oh Dios el pan. \\  \\

{\changefont{cmss}{bx}{n} Comunión:} \> Simple oración\footnotemark[3], Queremos ser Señor\footnotemark[3], La canción de la Alianza\footnotemark[3],      \\ 
\> Si yo no tengo amor\footnotemark[3], Jesucristo danos de este pan\footnotemark[3]. \\  \\

{\changefont{cmss}{bx}{n} Post-com.:} \> Donde hay amor y caridad (Taizé), Al atardecer de la vida,  \\
\> Tus misericordias cantaré (Taizé), Si el mismo pan comimos.  \\ \\

{\changefont{cmss}{bx}{n} Salida:} \> Cantemos hermanos, Canción del misionero, Salve oh Reina, Oh Santísima.  \\ \\


\end{tabbing}

\vspace{-15mm}

\footnotetext[1]{Cfr. \emph{Juntos como hermanos} (est. 1 y 2) con ant. comunión (Sal 41,2-3) y aclamación al evangelio (2 Cor 5,19). }
\footnotetext[2]{Cfr. \emph{Al altar nos acercamos} (est. 2 y 3), \emph{Pan de vida y bebida de luz} y \emph{Una espiga} (est. 4) con Mt 18,15-20.}
\footnotetext[3]{Todos estos cantos hablan del perdón mutuo en consonancia con Mt 18,15-20. }





%------------------------
\newpage
\setcounter{footnote}{0}
\thispagestyle{fancy}
\fancyhead{}
\fancyfoot{}
%\footskip=-1cm

\lhead{\changefont{cmss}{bx}{n} \small Revisión 2013}
    \chead{\changefont{cmss}{bx}{n} \small Domingo XXIV - Durante el año}
\rhead{\changefont{cmss}{bx}{n} \small Ciclo A}
\rfoot{\changefont{cmss}{bx}{n}\large\thepage}

\vspace*{-10mm}

\begin{center}
{\large\it La parábola del servidor despiadado }
\end{center}

\vspace{-3mm}

\begin{tabbing}

{\changefont{cmss}{bx}{n} Entrada:\ \ \ \ \ }\= Iglesia peregrina de Dios\footnotemark[1], Caminaré, Vienen con alegría. \\ \\

{\changefont{cmss}{bx}{n} Salmo:} \> 102 ant. 2 ``El amor del Señor...'' (estr. 1, 2, 4 o 5).\\ \\

{\changefont{cmss}{bx}{n} Ofrendas:} \> Pan de vida y bebida de luz\footnotemark[2], Al altar nos acercamos\footnotemark[2],  Entre tus manos\footnotemark[2],    \\
\> Te ofrecemos Padre nuestro (vidala)\footnotemark[2]. \\  \\

{\changefont{cmss}{bx}{n} Comunión:} \> Dios me dio a mi hermano\footnotemark[3],$\,$La canción de la Alianza\footnotemark[3],$\,$Cuerpo y Sangre de Jesús\footnotemark[3],      \\ 
\> Este es mi Cuerpo\footnotemark[3]. \\  \\

{\changefont{cmss}{bx}{n} Post-com.:} \> Si el mismo pan comimos\footnotemark[3], Donde hay amor y caridad (Taizé),  \\
\> Espíritu de comunidad, Cuántas gracias te debemos, Adoremos a Dios.  \\ \\

{\changefont{cmss}{bx}{n} Salida:} \> Cantemos hermanos, Madre de los peregrinos, Santa María del camino.  \\ \\


\end{tabbing}

\vspace{-15mm}

\footnotetext[1]{Cfr. \emph{Iglesia peregrina de Dios} (est. 1) con aclamación al evangelio (1 Cor 10,16). }
\footnotetext[2]{Cfr. \emph{Pan de vida y bebida de luz} con aclamación 1 Cor 10,16; cfr. \emph{Al altar nos acercamos} (est. 3) con Mt 18,21-35; \emph{Entre tus manos} tiene una estrofa adicional con el texto de Rom 14,8; cfr. \emph{Te ofrecemos Padre nuestro} (est. 3) con ant. de entrada (Ecli 36,18).}
\footnotetext[3]{Cfr. \emph{Dios me dio a mi hermano} con Mt 18,21-35 y con aclamación 1 Cor 10,16; cfr. \emph{La canción de la Alianza} con Mt 18,21-35; cfr. \emph{Cuerpo y Sangre de Jesús} (est. 4 y 5), \emph{Este es mi Cuerpo} y \emph{Si el mismo pan comimos} con aclamación 1 Cor 10,16. }


%------------------------
\newpage
\setcounter{footnote}{0}
\thispagestyle{fancy}
\fancyhead{}
\fancyfoot{}
%\footskip=-1cm

\lhead{\changefont{cmss}{bx}{n} \small Revisión 2013}
    \chead{\changefont{cmss}{bx}{n} \small Domingo XXV - Durante el año}
\rhead{\changefont{cmss}{bx}{n} \small Ciclo A}
\rfoot{\changefont{cmss}{bx}{n}\large\thepage}

\vspace*{-10mm}

\begin{center}
{\large\it La parábola de los obreros de última hora }
\end{center}

\vspace{-3mm}

\begin{tabbing}

{\changefont{cmss}{bx}{n} Entrada:\ \ \ \ \ }\= El Señor nos llama\footnotemark[1], Vine a alabar\footnotemark[1], Caminaré\footnotemark[1], Vienen con alegría.\\ \\

{\changefont{cmss}{bx}{n} Salmo:} \> 144 ant. 1 ``Te alabamos Señor...'' (estr. 1,4 o 3).\\ \\

{\changefont{cmss}{bx}{n} Ofrendas:} \> Recibe oh Dios eterno\footnotemark[2], Te ofrecemos Padre nuestro (vidala), Una espiga, \\
\>  Pan de vida y bebida de luz, Bendeciré al Señor, Te presentamos.  \\  \\

{\changefont{cmss}{bx}{n} Comunión:} \> Más cerca oh Dios\footnotemark[3], Yo soy el camino\footnotemark[3], Como Cristo nos amó, Vayamos a la mesa,    \\ 
\> Cuerpo y sangre de Jesús, Este es mi Cuerpo. \\  \\

{\changefont{cmss}{bx}{n} Post-com.:} \> Si el mismo pan comimos, Mirarte sólo a ti, Tus misericordias cantaré (Taizé).  \\
\> Cuántas gracias te debemos (Padre Bevilacqua).  \\ \\

{\changefont{cmss}{bx}{n} Salida:} \> Anunciaremos tu Reino, Canción del misionero\footnotemark[3], Vayan todos por el mundo, \\
\> Santa María del camino, Madre de los peregrinos.  \\ \\


\end{tabbing}

\vspace{-15mm}

\footnotetext[1]{Cfr. \emph{Caminaré} (est.) con ant. entrada; cfr. \emph{Vine a alabar} con aclamación al evangelio Hech 16,14; \emph{El Señor nos llama} tiene paralelos con esta parábola. }
\footnotetext[2]{Cfr. \emph{Recibe oh Dios eterno} (est. 2 y 3) con Fil 1,20-24 y Mt 19,30-20,16.}
\footnotetext[3]{Cfr. \emph{Más cerca oh Dios} (est 1 y 5) con Is 55,6, y (est 4) con ant. entrada del día; cfr. \emph{Yo soy el camino} (est. 3) con ant. de comunión Jn 10,14; cfr. \emph{Canción del misionero} con Fil 1,20-26. }




%------------------------
\newpage
\setcounter{footnote}{0}
\thispagestyle{fancy}
\fancyhead{}
\fancyfoot{}
%\footskip=-1cm

\lhead{\changefont{cmss}{bx}{n} \small Revisión 2013}
    \chead{\changefont{cmss}{bx}{n} \small Domingo XXVI - Durante el año}
\rhead{\changefont{cmss}{bx}{n} \small Ciclo A}
\rfoot{\changefont{cmss}{bx}{n}\large\thepage}

\vspace*{-10mm}

\begin{center}
{\large\it La parábola de los dos hijos }
\end{center}

\vspace{-3mm}

\begin{tabbing}

{\changefont{cmss}{bx}{n} Entrada:\ \ \ \ \ }\= Iglesia peregrina\footnotemark[1], Somos la familia de Jesús\footnotemark[1], Un pueblo que camina, Qué alegría.\\ \\

{\changefont{cmss}{bx}{n} Salmo:} \> 24 ant. 1 ``A tí elevo mi alma...'' (estr. 2,3 o 5).\\ \\

{\changefont{cmss}{bx}{n} Ofrendas:} \> Recibe oh Dios eterno\footnotemark[2], Toma Señor nuestra vida\footnotemark[2], Coplas de Yaraví,  \\
\> Pan de vida y bebida de luz, Bendeciré al Señor.  \\  \\

{\changefont{cmss}{bx}{n} Comunión:} \> Queremos ser Señor\footnotemark[3], Jesucristo danos de este pan\footnotemark[3], Escondido, Simple oración,    \\ 
\> La canción de la Alianza. \\  \\

{\changefont{cmss}{bx}{n} Post-com.:} \> Tu fidelidad, Tus misericordias cantaré (Taizé), Cuántas gracias te debemos,   \\
\> Mirarte sólo a ti, El alfarero (Vaso nuevo).  \\ \\

{\changefont{cmss}{bx}{n} Salida:} \> Canción del misionero, Anunciaremos tu Reino, Mi camino eres tu, Soy peregrino, \\
\> Simple oración.  \\ \\


\end{tabbing}

\vspace{-15mm}

\footnotetext[1]{Cfr. \emph{Iglesia peregrina} (est. 1) con Fil 2,2-11; cfr. \emph{Somos la familia de Jesús} (est. 1 y 2) con Mt 21,28-32. }
\footnotetext[2]{Cfr. \emph{Toma Señor nuestra vida} (est. 2 y 4) y \emph{Recibe oh Dios eterno} (est. 1) con Fil 2,2-4 y Mt 21,28-32.}
\footnotetext[3]{Cfr. \emph{Queremos ser Señor} (est 3) con Mt 21,32; cfr. \emph{Jesucristo danos de este pan} (estribillo y est. 2) Fil 2,2-11. }










%------------------------
\newpage
\setcounter{footnote}{0}
\thispagestyle{fancy}
\fancyhead{}
\fancyfoot{}
%\footskip=-1cm

\lhead{\changefont{cmss}{bx}{n} \small Revisión 2013}
    \chead{\changefont{cmss}{bx}{n} \small Domingo XXVII - Durante el año}
\rhead{\changefont{cmss}{bx}{n} \small Ciclo A}
\rfoot{\changefont{cmss}{bx}{n}\large\thepage}

\vspace*{-10mm}

\begin{center}
{\large\it La parábola de los viñadores homicidas }
\end{center}

\vspace{-3mm}

\begin{tabbing}

{\changefont{cmss}{bx}{n} Entrada:\ \ \ \ \ }\= Juntos como hermanos (est$\,$2)\footnotemark[1],$\,$Iglesia peregrina,$\,$Vienen con alegría,$\,$Vine a alabar.\\ \\

{\changefont{cmss}{bx}{n} Salmo:\footnotemark[1]} \> 79 ant. 3 ``!`Míranos, Señor, ven a salvarnos...!'' (estr. 3, 4, 5 o 6).\\ 
\> (antífona de reemplazo: Sal 84 ant. 1 ``Señor, revélanos tu amor,...'') \\ \\

{\changefont{cmss}{bx}{n} Ofrendas:} \> Te ofrecemos Padre nuestro (moderno)\footnotemark[2], Toma Señor nuestra vida\footnotemark[2],   \\
\> Recibe oh Dios eterno, Una espiga, Te presentamos. \\  \\

{\changefont{cmss}{bx}{n} Comunión:} \> El viñador\footnotemark[3], Yo soy el camino, Jesucristo danos de este pan,    \\ 
\> Cuerpo y Sangre de Jesús, Este es mi Cuerpo. \\  \\

{\changefont{cmss}{bx}{n} Post-com.:} \> Nada te turbe (Taizé)\footnotemark[4], Si el mismo pan comimos\footnotemark[4], Tus misericordias cantaré   \\
\>  (Taizé), Cuántas gracias te debemos, Mirarte sólo a ti, Nuestro maná.  \\ \\

{\changefont{cmss}{bx}{n} Salida:} \> En medio de los pueblos, Anunciaremos tu Reino,  Canción del testigo, \\
\> Salve María, Santa María del camino.  \\ \\


\end{tabbing}

\vspace{-15mm}

\footnotetext[1]{Cfr. \emph{Juntos como hermanos} (est. 2) con Fil 4,6-9; se puede cantar el Aleluya ``Busca primero el Reino...''. }
\footnotetext[2]{Cfr. \emph{Te ofrecemos Padre nuestro} (est. 3) con Is 5,1-7 y Sal 79; cfr. \emph{Toma Señor nuestra vida} con Is 5,1-7.}
\footnotetext[3]{Cfr. \emph{El viñador} con Is 5,1-7 y Mt 21,33-46. }
\footnotetext[4]{Cfr. \emph{Nada te turbe} con Fil 4,6;cfr. \emph{Si el mismo pan comimos} con ant. comunión del día 1 Cor 10,17. }







%------------------------
\newpage
\setcounter{footnote}{0}
\thispagestyle{fancy}
\fancyhead{}
\fancyfoot{}
%\footskip=-1cm

\lhead{\changefont{cmss}{bx}{n} \small Revisión 2013}
    \chead{\changefont{cmss}{bx}{n} \small Domingo XXVIII - Durante el año}
\rhead{\changefont{cmss}{bx}{n} \small Ciclo A}
\rfoot{\changefont{cmss}{bx}{n}\large\thepage}

\vspace*{-10mm}

\begin{center}
{\large\it La parábola del banquete nupcial }
\end{center}

\vspace{-3mm}

\begin{tabbing}

{\changefont{cmss}{bx}{n} Entrada:\ \ \ \ \ }\= El Señor nos llama\footnotemark[1], Somos la familia de Jesús, Qué alegría, Vine a alabar.\\ \\

{\changefont{cmss}{bx}{n} Salmo:} \> 22 ant. 1 ``El Señor es mi pastor...'' (estr. 1, 2, 3, o 4).\\ \\

{\changefont{cmss}{bx}{n} Ofrendas:} \> Al altar del Señor\footnotemark[2], Escondido\footnotemark[2], Bendeciré al Señor\footnotemark[2], Te ofrecemos Padre nuestro \\
\>  (vidala). \\  \\

{\changefont{cmss}{bx}{n} Comunión:} \>  Queremos ser Señor\footnotemark[3], Vayamos a la mesa\footnotemark[3], Como Cristo nos amó\footnotemark[3],      \\ 
\>  Bendeciré al Señor\footnotemark[2], Yo soy el camino. \\  \\

{\changefont{cmss}{bx}{n} Post-com.:} \> Tus misericordias cantaré (Taizé), Mirarte sólo a ti, Cuántas gracias te debemos,  \\
\>   Adoremos a Dios.  \\ \\

{\changefont{cmss}{bx}{n} Salida:} \> Mi camino eres tú, Vayan todos por el mundo, Cantad a María, Salve María,  \\
\> Santa María del camino, Salve oh Reina.  \\ \\


\end{tabbing}

\vspace{-15mm}

\footnotetext[1]{Cfr. \emph{El Señor nos llama} con Mt 22,1-14. }
\footnotetext[2]{Cfr. \emph{Al altar del Señor} (est. 1) con Sal 22,5; cfr. \emph{Escondido} (est. 3 y 5, y estribillo) con Mt 22,1-14; cfr. \emph{Bendeciré al Señor} con ant. de comunión del día (Sal 33,11).}
\footnotetext[3]{Cfr. \emph{Queremos ser Señor} y \emph{Vayamos a la mesa} con Mt 22,1-14; cfr. \emph{Como Cristo nos amó} (est. 4) con ant. de entrada del día (Sal 129,3-4). }





%------------------------
\newpage
\setcounter{footnote}{0}
\thispagestyle{fancy}
\fancyhead{}
\fancyfoot{}
%\footskip=-1cm

\lhead{\changefont{cmss}{bx}{n} \small Revisión 2013}
    \chead{\changefont{cmss}{bx}{n} \small Domingo XXIX - Durante el año}
\rhead{\changefont{cmss}{bx}{n} \small Ciclo A}
\rfoot{\changefont{cmss}{bx}{n}\large\thepage}

\vspace*{-10mm}

\begin{center}
{\large\it El impuesto debido a la autoridad }
\end{center}

\vspace{-3mm}

\begin{tabbing}

{\changefont{cmss}{bx}{n} Entrada:\ \ \ \ \ }\= Caminaré\footnotemark[1], Un solo Señor\footnotemark[1], Pueblo de Dios, Qué alegría.\\ \\

{\changefont{cmss}{bx}{n} Salmo:} \> 97 ant. 1 ``Cantemos al Señor un canto nuevo...'' (estr. 1, 2, 3, o 4).\\
\> (antífona de reemplazo: Sal 95 ant. 4 ``Cantemos al Señor un canto nuevo'') \\ \\

{\changefont{cmss}{bx}{n} Ofrendas:} \> Padre nuestro recibid\footnotemark[2], Mira nuestra ofrenda, Te presentamos, Bendito seas. \\ \\

{\changefont{cmss}{bx}{n} Comunión:} \>  Cuerpo y Sangre de Jesús\footnotemark[3], Bendeciré al Señor\footnotemark[3], Este es mi cuerpo\footnotemark[3], Escondido\footnotemark[3],    \\ 
\> Yo soy el camino. \\  \\

{\changefont{cmss}{bx}{n} Post-com.:} \> Alabe todo el mundo (Taizé), Adoremos a Dios, Mirarte sólo a ti, \\
\> Cuántas gracias te debemos.  \\ \\

{\changefont{cmss}{bx}{n} Salida:} \> Mi camino eres tú, En medio de los pueblos, Soy peregrino, Simple oración, \\
\> Canción del testigo, Anunciaremos tu reino.  \\ \\


\end{tabbing}

\vspace{-15mm}

\footnotetext[1]{Cfr. \emph{Caminaré} (est. 1) con antífona de entrada (Sal 16,6.8); \emph{Un solo Señor} es especialmente aplicable en la est. 3.  }
\footnotetext[2]{\emph{Padre nuestro recibid} tiene una doxología en la est. 4, muy apropiada para el día (Mt 22,15-21).}
\footnotetext[3]{\emph{Cuerpo y sangre de Jesús} habla de que somos el ``pueblo de Dios'', muy apropiado en este día (Mt 22,15-21). Todos los cantos de comunión son eucarísticos (siempre apropiados). }







%------------------------
\newpage
\setcounter{footnote}{0}
\thispagestyle{fancy}
\fancyhead{}
\fancyfoot{}
%\footskip=-1cm

\lhead{\changefont{cmss}{bx}{n} \small Revisión 2013}
    \chead{\changefont{cmss}{bx}{n} \small Domingo XXX - Durante el año}
\rhead{\changefont{cmss}{bx}{n} \small Ciclo A}
\rfoot{\changefont{cmss}{bx}{n}\large\thepage}

\vspace*{-10mm}

\begin{center}
{\large\it El mandamiento principal }
\end{center}

\vspace{-3mm}

\begin{tabbing}

{\changefont{cmss}{bx}{n} Entrada:\ \ \ \ \ }\= Pueblo de Dios\footnotemark[1], Caminaré\footnotemark[1], Alabaré (est. 2 y 3), Vienen con alegría, Qué alegría.\\ \\

{\changefont{cmss}{bx}{n} Salmo:} \> 17 ant. 1 ``!`Te amo, Señor, mi fuerza y mi refugio...'' (estr. 1, 2, o 7).\\
\> (antífona de reemplazo: Sal 15 ant. 1 ``Tú eres, Señor, mi herencia...'') \\ \\

{\changefont{cmss}{bx}{n} Ofrendas:} \> Señor te ofrecemos\footnotemark[1], Bendeciré al Señor\footnotemark[1], Pan de vida y bebida de luz,\\
\> Te presentamos, Bendito seas. \\ \\

{\changefont{cmss}{bx}{n} Comunión:} \>  Si yo no tengo amor\footnotemark[2], Como Cristo nos amó\footnotemark[2], Más cerca oh Dios\footnotemark[2],   \\ 
\> La canción de la Alianza\footnotemark[2]. \\  \\

{\changefont{cmss}{bx}{n} Post-com.:} \> Al atardecer de la vida, Donde hay amor y caridad (Taizé),\\
\>  El Señor es mi fortaleza (Taizé), Más cerca oh Dios, Mirarte sólo a ti.  \\ \\

{\changefont{cmss}{bx}{n} Salida:} \> Simple oración, Anunciaremos tu reino, Soy peregrino, Oh María, Salve María.  \\ \\


\end{tabbing}

\vspace{-15mm}

\footnotetext[1]{Cfr. \emph{Pueblo de Dios} (estribillo) y \emph{Bendeciré al Señor} con Sal 17,47; cfr. \emph{Caminaré} (est. 1) y \emph{Señor te ofrecemos} con Sal 17,2-4 y Ex 22,26.  }
\footnotetext[2]{Cfr. \emph{Si yo no tengo amor} con Mt 22,34-40; cfr. \emph{Como Cristo nos amó} con ant. comunión (Ef 5,2); cfr. \emph{La canción de la Alianza} con aclamación al evangelio (Jn 14,23); cfr. \emph{Más cerca oh Dios} con (est. 3 y 4) con Sal 17,2-4 y Mt 22,34-40. }



%------------------------
\newpage
\setcounter{footnote}{0}
\thispagestyle{fancy}
\fancyhead{}
\fancyfoot{}
%\footskip=-1cm

\lhead{\changefont{cmss}{bx}{n} \small Revisión 2013}
    \chead{\changefont{cmss}{bx}{n} \small Domingo XXXI - Durante el año}
\rhead{\changefont{cmss}{bx}{n} \small Ciclo A}
\rfoot{\changefont{cmss}{bx}{n}\large\thepage}

\vspace*{-10mm}

\begin{center}
{\large\it La hipocresía y la vanidad de los escribas y fariseos }
\end{center}

\vspace{-3mm}

\begin{tabbing}

{\changefont{cmss}{bx}{n} Entrada:\ \ \ \ \ }\= El sermón de la montaña\footnotemark[1], El Señor nos llama, Iglesia peregrina de Dios.\\ \\

{\changefont{cmss}{bx}{n} Salmo:} \> 24 ant. 1 ``A ti elevo mi alma...'' (estr. 1, 5 o 7 del salmo 33)\footnotemark[1].\\ \\

{\changefont{cmss}{bx}{n} Ofrendas:} \> Comienza el sacrificio\footnotemark[2], Pan de vida y bebida de luz\footnotemark[2], Coplas de Yaraví\footnotemark[2],  \\
\> Bendeciré al Señor\footnotemark[2]. \\ \\

{\changefont{cmss}{bx}{n} Comunión:} \>  Cuerpo y Sangre de Jesús\footnotemark[3], Este es mi Cuerpo\footnotemark[3], Vayamos a la mesa\footnotemark[3],  \\ 
\> Bendeciré al Señor\footnotemark[3]. \\  \\

{\changefont{cmss}{bx}{n} Post-com.:} \> Vaso nuevo (El alfarero), Cuántas gracias te debemos,  Mirarte sólo a ti, \\
\> Adoremos a Dios. \\ \\

{\changefont{cmss}{bx}{n} Salida:} \> Vayan por el mundo, Simple oración, Madre de los peregrinos\footnotemark[4],  \\
\> Madre de nuestro pueblo\footnotemark[4], Salve María.  \\ \\


\end{tabbing}

\vspace{-15mm}

\footnotetext[1]{Cfr. \emph{El sermón de la montaña} (est.1) con Mt 23,11; el salmo 24 y el 33 tienen la misma cantilación.  }
\footnotetext[2]{Cfr. \emph{Comienza el sacrificio} (est. 2) Mal 2,10; cfr. \emph{Pan de vida y bebida de luz} (est. 3) con Mt 23,1-12; cfr. \emph{Coplas de Yaraví} (est. 4) con Mt 23,11; cfr. \emph{Bendeciré al Señor} con Sal 33;  }
\footnotetext[3]{Todos estos cantos son eucarísticos y siempre válidos en el momento de la comunión.}
\footnotetext[4]{Ambos cantos hablan de la humildad de María; \emph{Madre de nuestro pueblo} se puede cantar con las estrofas 1 y 3.}



%------------------------
\newpage
\setcounter{footnote}{0}
\thispagestyle{fancy}
\fancyhead{}
\fancyfoot{}
%\footskip=-1cm

\lhead{\changefont{cmss}{bx}{n} \small Revisión 2013}
    \chead{\changefont{cmss}{bx}{n} \small Domingo XXXII - Durante el año}
\rhead{\changefont{cmss}{bx}{n} \small Ciclo A}
\rfoot{\changefont{cmss}{bx}{n}\large\thepage}

\vspace*{-10mm}

\begin{center}
{\large\it La parábola de las diez jóvenes del cortejo }
\end{center}

\vspace{-3mm}

\begin{tabbing}

{\changefont{cmss}{bx}{n} Entrada:\ \ \ \ \ }\= El Señor nos llama\footnotemark[1], Alabaré, Caminaré\footnotemark[1], Qué alegría.\\ \\

{\changefont{cmss}{bx}{n} Salmo:} \> 62 ant. 1 ``!`Señor mi Dios, te busco desde la aurora...'' (estr. 1, 2 o 3).\\
\> (antífona de reemplazo: Sal 41 ant. 1 ``Mi alma tiene sed de Dios...'') \\ \\

{\changefont{cmss}{bx}{n} Ofrendas:} \> Al altar del Señor\footnotemark[2],$\,$Los frutos de la tierra,$\,$Bendeciré al Señor,$\,$Bendito seas Señor. \\  \\

{\changefont{cmss}{bx}{n} Comunión:} \>  Yo soy el pan de vida\footnotemark[2], Quédate con nosotros\footnotemark[2], Vayamos a la mesa,  \\ 
\> Cuerpo y Sangre de Jesús, Este es mi Cuerpo. \\  \\

{\changefont{cmss}{bx}{n} Post-com.:} \>  Mirarte sólo a ti, Adoremos a Dios, Cuántas gracias te debemos. \\ \\

{\changefont{cmss}{bx}{n} Salida:} \> Soy peregrino, Mi camino eres tú, Oh Santísima, Santa María del Camino. \\ \\


\end{tabbing}

\vspace{-15mm}

\footnotetext[1]{\emph{El Señor nos llama} está basado en un contexto de banquete de bodas, al igual que Mt 25,1-13; cfr. \emph{Caminaré} \mbox{(est. 1)} con ant. de entrada (Sal 87,3).  }
\footnotetext[2]{Cfr. \emph{Al altar del Señor} (est. 1 y 2) con Sal 62,6 y Mt 25,1-13; \emph{Yo soy el pan de vida} es muy apropiado porque es eucarístco y lo vincula a la resurrección (cfr. con 1 Tes 4,13-18 y Mt 25,1-13); cfr. \emph{Quédate con nosotros} (estribillo y est. 4) con Mt 25,1-13.  }



%------------------------
\newpage
\setcounter{footnote}{0}
\thispagestyle{fancy}
\fancyhead{}
\fancyfoot{}
%\footskip=-1cm

\lhead{\changefont{cmss}{bx}{n} \small Revisión 2013}
    \chead{\changefont{cmss}{bx}{n} \small Domingo XXXIII - Durante el año}
\rhead{\changefont{cmss}{bx}{n} \small Ciclo A}
\rfoot{\changefont{cmss}{bx}{n}\large\thepage}

\vspace*{-10mm}

\begin{center}
{\large\it La parábola de los talentos }
\end{center}

\vspace{-3mm}

\begin{tabbing}

{\changefont{cmss}{bx}{n} Entrada:\ \ \ \ \ }\= Vienen con alegría\footnotemark[1], Iglesia peregrina\footnotemark[1], Un pueblo que camina\footnotemark[1].\\ \\

{\changefont{cmss}{bx}{n} Salmo:} \> 127 ant. 1 ``!`Feliz quien ama al Señor...'' (estr. 1 y 2).\\ \\

{\changefont{cmss}{bx}{n} Ofrendas:} \> Toma Señor nuestra vida\footnotemark[2], Este es nuestro pan\footnotemark[1], Los frutos de la tierra\footnotemark[1],  \\
\> Una espiga. \\ \\

{\changefont{cmss}{bx}{n} Comunión:} \>  Más cerca oh Dios\footnotemark[2], Escondido\footnotemark[2], Canción del misionero\footnotemark[2],  Jesucristo danos \\ 
\>  de este pan\footnotemark[2], Este es mi cuerpo. \\  \\

{\changefont{cmss}{bx}{n} Post-com.:} \> El Señor es mi fortaleza (Taizé), Mirarte sólo a ti, Cuántas gracias te debemos, \\
\>  Adoremos a Dios.  \\ \\

{\changefont{cmss}{bx}{n} Salida:} \> Santa María del camino, Oh Santísima, Oh María, Salve María, Cantad a María.  \\ \\


\end{tabbing}

\vspace{-15mm}

\footnotetext[1]{Los cantos elegidos vienen bien para el contexto escatológico del evangelio (Mt 25,14-30).  }
\footnotetext[2]{Cfr. \emph{Toma Señor nuestra vida} con aclamación al evangelio (Jn 15,4-5); cfr. \emph{Más cerca oh Dios} (est. 1 y 4) con ant. comunión (Sal 72,28); cfr. \emph{Escondido} con Mt 25,14-30; cfr. \emph{Canción del misionero} con Mt 25,14-30. Este canto puede hacerse durante la comunión para dejar paso a los cantos marianos en la salida; \emph{Jesucristo danos de este pan} tiene texto eucarístico y además recuerda a María (en el mes de María). }



%------------------------
\newpage
\setcounter{footnote}{0}
\thispagestyle{fancy}
\fancyhead{}
\fancyfoot{}
%\footskip=-1cm

\lhead{\changefont{cmss}{bx}{n} \small Revisión 2013}
\chead{\changefont{cmss}{bx}{n} \small Domingo XXXIV - Durante el año}
\rhead{\changefont{cmss}{bx}{n} \small Ciclo A}
\rfoot{\changefont{cmss}{bx}{n}\large\thepage}

\vspace*{-11mm}

\begin{center}
{\large\it (Domingo de Cristo Rey) El juicio final }
\end{center}

\vspace{-3mm}

\begin{tabbing}

{\changefont{cmss}{bx}{n} Entrada:\ \ \ \ \ }\= Pueblo de reyes\footnotemark[1], Alabaré\footnotemark[1], El Señor nos llama\footnotemark[1]. \\ \\


{\changefont{cmss}{bx}{n} Salmo:} \> 22 ant. 1 ``!`El Señor es mi pastor...'' (todas).\\ \\

{\changefont{cmss}{bx}{n} Ofrendas:} \> Al altar del Señor\footnotemark[2], Padre nuestro recibid\footnotemark[2], Te ofrecemos oh Señor, Bendito seas.\\ \\

{\changefont{cmss}{bx}{n} Comunión:} \> Rey de los reyes\footnotemark[3], Cuerpo y sangre de Jesús\footnotemark[3], Yo soy el pan de vida,  \\
\> Yo soy el camino, Vayamos a la mesa. \\  \\ 


{\changefont{cmss}{bx}{n} Post-com.:} \> Al atardecer de la vida\footnotemark[4], Alabe todo el mundo (de Taizé), Adoremos a Dios.\\ \\ 

{\changefont{cmss}{bx}{n} Salida:} \> Christus vincit, Oh María, Oh Santísima, Madre de los peregrinos, \\
\> Madre de nuestro pueblo, Mi camino eres tu. \\ \\

\end{tabbing}

\vspace{-15mm}

\footnotetext[1]{Cfr. \emph{Pueblo de reyes} (estribillo) con antífona de entrada Apoc 1,6, est. 3 con aclamación al evangelio (Mc 11,9-10); cfr. \emph{Alabaré} con antifona de entrada Apoc 5,12; cfr. \emph{El Señor nos llama}, que habla del banquete de bodas, con  Mt 25,31-46. }
\footnotetext[2]{Cfr. \emph{Al altar del Señor} con Sal 22,5; \emph{Padre nuestro recibid} (estr. 4) con aclamación al evangelio Mc 11,9-10. }
\footnotetext[3]{Cfr. \emph{Rey de los reyes} con la antífona de entrada Apoc 1,6 y 1 Cor 15,25; cfr. \emph{Cuerpo y sangre de Jesús} (est. 3) con la antífona de entrada Apoc 1,6. }
\footnotetext[4]{Cfr. \emph{Al atardecer de la vida} con Mt 25,31-46; \emph{Christus vincit} (Cristo vence) es muy tradicional para este domingo.}

















%------------------------
\newpage
\setcounter{footnote}{0}
\thispagestyle{fancy}
\fancyhead{}
\fancyfoot{}
%\footskip=-1cm

\lhead{\changefont{cmss}{bx}{n} \small Revisión 2013}
\chead{\changefont{cmss}{bx}{n} \small Fiesta del 2 de febrero - Presentación del Señor}
\rhead{\changefont{cmss}{bx}{n} \small Ciclo A}
\rfoot{\changefont{cmss}{bx}{n}\large\thepage}

\vspace*{-11mm}

\begin{center}
{\large\it La presentación de Jesús en el Templo }
\end{center}

\vspace{-3mm}

\begin{tabbing}

{\changefont{cmss}{bx}{n} Entrada:\ \ \ \ \ }\= Sal 23 (ant. 1)\footnotemark[1], Pueblo de reyes\footnotemark[1], Pueblo de Dios, Alabaré (est.  2 o 3). \\ \\


{\changefont{cmss}{bx}{n} Salmo:} \> 23 ant. 2 ``Felices los que son fieles al Señor...'' (estr. 1, 2 o 3). \\ 
\> (antífona de reemplazo: Sal 147 ant. 1 ``Glorifica al Señor Jerusalén...'') \\ \\

{\changefont{cmss}{bx}{n} Ofrendas:} \> Recibe oh Dios eterno\footnotemark[2], Toma Señor nuestra vida\footnotemark[2], Te ofrecemos Padre nuestro  \\
\> (vidala), Padre nuestro recibid. \\ \\ 

{\changefont{cmss}{bx}{n} Comunión:} \> Más cerca oh Dios\footnotemark[3], Pueblo de reyes\footnotemark[1], Como Cristo nos amó\footnotemark[3], Este es mi cuerpo, \\
\>  Bendeciré al Señor. \\  \\

{\changefont{cmss}{bx}{n} Post-com.:} \> Aleluia Cristo vino con su paz, Adoremos a Dios, Alabe todo el mundo (Taizé).\\ \\

{\changefont{cmss}{bx}{n} Salida:} \> Madre de nuestro pueblo (est. 5)\footnotemark[4], Canción del testigo\footnotemark[4], Soy peregrino. \\ \\


\end{tabbing}

\vspace{-15mm}

\footnotetext[1]{Se ingresa en procesión con las candelas encendidas. Si se canta \emph{Pueblo de reyes}, hacer especialmente est. 1, 2  y 3.  }
\footnotetext[2]{Cfr. \emph{Recibe oh Dios eterno} (est. 1) y \emph{Toma Señor nuestra vida} con Mal 3,3-4.}
\footnotetext[3]{Cfr. \emph{Más cerca oh Dios (Creo en ti Señor)} (est. 3) con el cánto de Simeón (Lc 2,29-32); cfr. \emph{Como Cristo nos amó} (est. 2) con  Heb. 2,17-18.  }
\footnotetext[4]{Cfr. \emph{Madre de nuestro pueblo} (est. 5) con Lc 2,22-40); el protagonista  de \emph{Canción del testigo} puede ser identificado, en cierto aspecto, con Simeón (Lc 2,22-40).}


%------------------------
\newpage
\setcounter{footnote}{0}
\thispagestyle{fancy}
\fancyhead{}
\fancyfoot{}
%\footskip=-1cm

\lhead{\changefont{cmss}{bx}{n} \small Revisión 2013}
\chead{\changefont{cmss}{bx}{n} \small Solemnidad del 8 de mayo - Nuestra Señora de Luján}
\rhead{\changefont{cmss}{bx}{n} \small Ciclo A}
\rfoot{\changefont{cmss}{bx}{n}\large\thepage}

\vspace*{-11mm}

\begin{center}
{\large\it Jesús y su madre }
\end{center}

\vspace{-3mm}

\begin{tabbing}

{\changefont{cmss}{bx}{n} Entrada:\ \ \ \ \ }\= Somos un pueblo que camina, Madre de los peregrinos, Pueblo de Dios peregrino\footnotemark[1],\\
\> La Virgen María nos reúne, Feliz de ti María, Iglesia peregrina de Dios.\\ \\  


{\changefont{cmss}{bx}{n} Salmo:} \> Magnificat. ``El Señor hizo en mí maravillas...'' (todo) \\ \\

{\changefont{cmss}{bx}{n} Ofrendas:} \> Bendeciré al Señor\footnotemark[2], Te ofrecemos oh Señor, Te presentamos, Bendito seas. \\ \\ 

{\changefont{cmss}{bx}{n} Comunión:} \> Ave Verum, Jesucristo danos de este pan\footnotemark[3], Bendeciré al Señor\footnotemark[2], Este es mi cuerpo, \\
\> Mi alma glorifica\footnotemark[3]. \\  \\

{\changefont{cmss}{bx}{n} Post-com.:} \> Quiero decir que sí, Bendita sea tu pureza, Bajo tu amparo (P. Bevilacqua).\\ \\

{\changefont{cmss}{bx}{n} Salida:} \> Madre de los peregrinos, Madre de nuestro pueblo, Oh María, Cantad a María,\\
\> Canto de María, Santa María del camino.\\ \\

\end{tabbing}

\vspace{-15mm}

\footnotetext[1]{Canto reciente compuesto por Matías Sagreras y grabado por el Grupo de Música Litúrgica (P. Esteban Sacchi). }
\footnotetext[2]{Cfr. \emph{Bendeciré al Señor} con Lc 1,46-55.}
\footnotetext[3]{\emph{Jesucristo danos de este pan} menciona a María en su est. 4; cfr. \emph{Mi alma glorifica} con Lc 1,46-55.  }


%------------------------
\newpage
\setcounter{footnote}{0}
\thispagestyle{fancy}
\fancyhead{}
\fancyfoot{}
%\footskip=-1cm

\lhead{\changefont{cmss}{bx}{n} \small Revisión 2013}
\chead{\changefont{cmss}{bx}{n} \small Solemnidad del 29 de junio - San Pedro y San Pablo}
\rhead{\changefont{cmss}{bx}{n} \small Ciclo A}
\rfoot{\changefont{cmss}{bx}{n}\large\thepage}

\vspace*{-11mm}

\begin{center}
{\large\it La profesión de fe de Pedro}
\end{center}

\vspace{-3mm}

\begin{tabbing}

{\changefont{cmss}{bx}{n} Entrada:\ \ \ \ \ }\= Cante la Iglesia\footnotemark[1], Iglesia peregrina, Un pueblo que camina, Vienen con alegría.\\ \\  


{\changefont{cmss}{bx}{n} Salmo:} \> 33 ant. 1 ``Vayamos a gustar la bondad del Señor'' (1, 2, 3 o 4) \\ \\

{\changefont{cmss}{bx}{n} Ofrendas:} \> Pan de vida y bebida de luz,$\,$Recibe oh Dios el pan\footnotemark[2],$\,$Te presentamos,$\,$Bendito seas. \\ \\ 

{\changefont{cmss}{bx}{n} Comunión:} \> Más cerca oh Dios\footnotemark[3], Pescador de hombres\footnotemark[3], Bendeciré al Señor\footnotemark[3],\\ \> El Señor de Galilea. \\  \\

{\changefont{cmss}{bx}{n} Post-com.:} \> Si el mismo pan comimos, El Señor es mi fortaleza (Taizé), Mirarte sólo a ti\footnotemark[4].\\ \\

{\changefont{cmss}{bx}{n} Salida:} \> En medio de los pueblos, Mi camino eres tú\footnotemark[4], Canción del testigo, \\ 
\> Vayan todos por el mundo. \\ \\

\end{tabbing}

\vspace{-15mm}

\footnotetext[1]{Canto para el día de \emph{todos} los Santos, en particular para San Pedro y San Pablo. }
\footnotetext[2]{\emph{Recibe oh Dios el pan} (est. 3) pide por los difuntos.  }
\footnotetext[3]{Cfr. \emph{Más cerca oh Dios} con Jn 21,18-19 (martirio de Pedro); cfr. \emph{Pescador de hombres} (est. 1 y 2) con Jn 21,19 y Hech 3,6; cfr. \emph{Bendeciré al Señor} con Sal 33,2-9.}
\footnotetext[4]{Cfr. \emph{Mirarte sólo a ti} y \emph{Mi camino eres tú} son perfectamente aplicables a la vida de los apóstoles Pedro y Pablo.}

%------------------------
\newpage
\setcounter{footnote}{0}
\thispagestyle{fancy}
\fancyhead{}
\fancyfoot{}
%\footskip=-1cm

\lhead{\changefont{cmss}{bx}{n} \small Revisión 2013}
\chead{\changefont{cmss}{bx}{n} \small Solemnidad del 15 de agosto - Asunción de la Virgen María}
\rhead{\changefont{cmss}{bx}{n} \small Ciclo A}
\rfoot{\changefont{cmss}{bx}{n}\large\thepage}

\vspace*{-11mm}

\begin{center}
{\large\it El canto de la Virgen María }
\end{center}

\vspace{-3mm}

\begin{tabbing}

{\changefont{cmss}{bx}{n} Entrada:\ \ \ \ \ }\= Feliz de ti María\footnotemark[1], La Virgen María nos reúne, Que alegría.\\ \\  


{\changefont{cmss}{bx}{n} Salmo:} \> 44 con ant. ``A tu diestra, Señor, resplandece la Reina'' (música de Sal 30 ant. 1) \\ 
\> (estr. 5 y 7 del Sal 44) \\ \\

{\changefont{cmss}{bx}{n} Ofrendas:} \> Bendeciré al Señor\footnotemark[2], Pan de vida y bebida de luz\footnotemark[2], Te presentamos, Bendito seas. \\ \\ 

{\changefont{cmss}{bx}{n} Comunión:} \> Bendeciré al Señor\footnotemark[2],$\,$Jesucristo danos de este pan\footnotemark[3],$\,$Mi alma glorifica\footnotemark[3],$\,$Es mi Padre. \\  \\

{\changefont{cmss}{bx}{n} Post-com.:} \> Quiero decir que sí, Bendita sea tu pureza.\\ \\

{\changefont{cmss}{bx}{n} Salida:} \> Un día la veré\footnotemark[4], Cantad a María, Los cielos la tierra\footnotemark[4], Canto de María.\\ \\

\end{tabbing}

\vspace{-15mm}

\footnotetext[1]{Cfr. \emph{Feliz de ti María} (est. 5) con la antífona de entrada Apoc 12,1 y con evangelio de la Vigilia Lc 11,27-28. }
\footnotetext[2]{Cfr. \emph{Bendeciré al Señor} con Lc 1,46-55; \emph{Pan de vida y bebida de luz} con 1 Cor 15,20-27.}
\footnotetext[3]{\emph{Jesucristo danos de este pan} menciona a María en su est. 4; cfr. \emph{Mi alma glorifica} con Lc 1,46-55; cfr. \emph{Es mi Padre} (est. 3) con 1 Cor 15,20-27.  }
\footnotetext[4]{ Cfr. \emph{Un día la veré} (est. 1) con Sal 44,18; \emph{Los cielos, la tierra} (est. 2) con Lc 1,39-43. }


%------------------------
\newpage
\setcounter{footnote}{0}
\thispagestyle{fancy}
\fancyhead{}
\fancyfoot{}
%\footskip=-1cm

\lhead{\changefont{cmss}{bx}{n} \small Revisión 2013}
\chead{\changefont{cmss}{bx}{n} \small Fiesta del 14 de septiembre - Exaltación de la Santa Cruz}
\rhead{\changefont{cmss}{bx}{n} \small Ciclo A}
\rfoot{\changefont{cmss}{bx}{n}\large\thepage}

\vspace*{-11mm}

\begin{center}
{\large\it El diálogo de Jesús con Nicodemo }
\end{center}

\vspace{-3mm}

\begin{tabbing}

{\changefont{cmss}{bx}{n} Entrada:\ \ \ \ \ }\=  Cruz de Cristo\footnotemark[1], Somos la familia de Jesús\footnotemark[1], Juntos como hermanos (est. 1), \\
\> Caminaré (est. 1). \\ \\


{\changefont{cmss}{bx}{n} Salmo:} \> 17 ant. 1 ``Te amo, Señor, mi fuerza...'' (estr. 1, 2 o 4). \\ 
\> (antífona de reemplazo: Sal 29 ant. 1 ``Te glorifico, Señor, ...'' del P. Bevilacqua) \\ \\

{\changefont{cmss}{bx}{n} Ofrendas:} \> Este es nuestro pan\footnotemark[1], Te ofrecemos Padre nuestro (nuevo)\footnotemark[1], Mira nuestra ofrenda,   \\
\> Bendito seas, Te presentamos. \\ \\ 

{\changefont{cmss}{bx}{n} Comunión:} \> Más cerca oh Dios\footnotemark[2], En memoria tuya\footnotemark[2], Jesús la imagen de Dios Padre\footnotemark[3], \\
\> En la postrera cena,  Como Cristo nos amó\footnotemark[2], Jesucristo danos de este pan.  \\  \\

{\changefont{cmss}{bx}{n} Post-com.:} \> Tu fidelidad, Adoremos a Dios, Mirarte solo a ti.\\ \\

{\changefont{cmss}{bx}{n} Salida:} \> Madre de nuestro pueblo (est. 1 y 9)\footnotemark[1], En medio de los pueblos, Mi camino eres tú \\
\> Santa María del camino, Salve oh Reina. \\ \\


\end{tabbing}

\vspace{-15mm}

\footnotetext[1]{\emph{Cruz de Cristo} se acostumbra cantar en cuaresma, pero su texto es perfecto para este día. Los demás cantos también hacen referencia directa a la cruz del Señor.   }
\footnotetext[2]{Cfr. \emph{Más cerca oh Dios} (est. 1 y 5) y \emph{Como Cristo nos amó} (est. 3) con Jn 3,13-17; cfr. \emph{Jesús la imagen de Dios Padre} con Fil 2,6-11; cfr. \emph{En memoria tuya} (est. 5 y 6) con Fil. 2,6-11. }




%------------------------
\newpage
\setcounter{footnote}{0}
\thispagestyle{fancy}
\fancyhead{}
\fancyfoot{}
%\footskip=-1cm

\lhead{\changefont{cmss}{bx}{n} \small Revisión 2013}
\chead{\changefont{cmss}{bx}{n} \small 1 de noviembre - Solemnidad de todos 
los santos}
\rhead{\changefont{cmss}{bx}{n} \small Ciclo A}
\rfoot{\changefont{cmss}{bx}{n}\large\thepage}

\vspace*{-11mm}

\begin{center}
{\large\it Las Bienaventuranzas }
\end{center}

\vspace{-3mm}

\begin{tabbing}

{\changefont{cmss}{bx}{n} Entrada:\ \ \ \ \ }\=  Cante la 
Iglesia\footnotemark[1], Brilló la luz\footnotemark[1], 
Alabaré\footnotemark[1], Iglesia peregrina. \\ \\


{\changefont{cmss}{bx}{n} Salmo:} \> 23 ant. 2 ``Felices los que son 
fieles...'' (estr. 1, 2 o 3). \\ 
\> (antífona de reemplazo: Sal 24 ant. 1 ``A ti elevo mi alma, ...'') \\ \\

{\changefont{cmss}{bx}{n} Ofrendas:} \> Recibe oh Dios el pan\footnotemark[2], 
Bendeciré al Señor, Te presentamos. \\  \\

{\changefont{cmss}{bx}{n} Comunión:} \> Brilló la luz\footnotemark[1], 
Cuerpo y Sangre de Jesús, Vayamos a la mesa, Es mi Padre.\\ \\

{\changefont{cmss}{bx}{n} Post-com.:} \> La misericordia del Señor (Taizé), Al 
atardecer de la vida, Adoremos a Dios.\\ \\

{\changefont{cmss}{bx}{n} Salida:} \> Cante la 
Iglesia\footnotemark[1], Anunciaremos tu Reino, Canción del testigo, \\
\> Salve oh Reina\footnotemark[3], En medio de los pueblos. \\ \\


\end{tabbing}

\vspace{-15mm}

\footnotetext[1]{\emph{Cante la Iglesia} es un canto especial para este día; 
cfr. \emph{Brilló la luz} con Mt 5,1-12; cfr. \emph{Alabaré} (est. 1) con 
\mbox{Apoc 72-14}. }
\footnotetext[2]{ \emph{Recibe oh Dios el pan} (est. 3) pide por los difuntos. }
\footnotetext[3]{Cfr. \emph{Salve oh Reina} (est. 2) habla del 
final de este ``destierro''.}



%------------------------
\newpage
\setcounter{footnote}{0}
\thispagestyle{fancy}
\fancyhead{}
\fancyfoot{}
%\footskip=-1cm

\lhead{\changefont{cmss}{bx}{n} \small Revisión 2013}
\chead{\changefont{cmss}{bx}{n} \small 2 de noviembre - Conmemoración de todos los fieles difuntos}
\rhead{\changefont{cmss}{bx}{n} \small Ciclo A}
\rfoot{\changefont{cmss}{bx}{n}\large\thepage}

\vspace*{-11mm}

\begin{center}
{\large\it El anuncio de la resurrección }
\end{center}

\vspace{-3mm}

\begin{tabbing}

{\changefont{cmss}{bx}{n} Entrada:\ \ \ \ \ }\=  Hacia ti morada santa\footnotemark[1], Brilló la luz\footnotemark[1], Sal 129 (ant. 1), Sal 83 (ant. 1), \\
\> Juntos como hermanos (est. 2). \\ \\


{\changefont{cmss}{bx}{n} Salmo:} \> 26 ant. 2 ``El Señor es mi luz, mi salvación...'' (estr. 1, 3, 5 o 7). \\ \\

{\changefont{cmss}{bx}{n} Ofrendas:} \> Recibe oh Dios el pan\footnotemark[2], Sé como el grano de trigo\footnotemark[2], Mira nuestra ofrenda. \\  \\

{\changefont{cmss}{bx}{n} Comunión:} \> Yo soy el pan de vida\footnotemark[3], Más cerca oh Dios\footnotemark[3], Brilló la luz\footnotemark[1], Es mi Padre,\\
\> Vayamos a la mesa. \\  \\

{\changefont{cmss}{bx}{n} Post-com.:} \> Mirarte sólo a ti\footnotemark[4], La misericordia del Señor (Taizé), Nada te turbe (Taizé).\\ \\

{\changefont{cmss}{bx}{n} Salida:} \> Soy peregrino\footnotemark[4], Salve oh Reina\footnotemark[4], En medio de los pueblos. \\ \\


\end{tabbing}

\vspace{-15mm}

\footnotetext[1]{\emph{Hacia ti morada santa} se acostumbra para misas por los difuntos; \emph{Brilló la luz} es muy apropiada por las bienaventuranzas. }
\footnotetext[2]{ \emph{Recibe oh Dios el pan} (est. 3) pide por los difuntos; \emph{Sé como el grano de trigo} (est. 4) habla de la vuelta al Padre. }
\footnotetext[3]{Cfr. \emph{Yo soy el pan de vida} con ant. comunión (Jn 11,25-26); \emph{Más cerca oh Dios} (est. 3) pide morar cerca del Señor.  }
\footnotetext[4]{Cfr. \emph{Mirarte sólo a ti} con Sal 26,8.13; cfr. \emph{Soy peregrino} (est. 4) con Apoc 21,2-5; \emph{Salve oh Reina} (est. 2) habla del final de este ``destierro''.}



%------------------------
\newpage
\setcounter{footnote}{0}
\thispagestyle{fancy}
\fancyhead{}
\fancyfoot{}
%\footskip=-1cm

\lhead{\changefont{cmss}{bx}{n} \small Revisión 2013}
\chead{\changefont{cmss}{bx}{n} \small Fiesta del 9 de noviembre - Dedicación de San Juan de Letrán}
\rhead{\changefont{cmss}{bx}{n} \small Ciclo A}
\rfoot{\changefont{cmss}{bx}{n}\large\thepage}

\vspace*{-11mm}

\begin{center}
{\large\it La expulsión de los vendedores del Templo }
\end{center}

\vspace{-3mm}

\begin{tabbing}

{\changefont{cmss}{bx}{n} Entrada:\ \ \ \ \ }\=  Qué alegría\footnotemark[1], Pueblo de Dios\footnotemark[1], Pueblo de Reyes\footnotemark[1], Iglesia peregrina de Dios. \\ \\


{\changefont{cmss}{bx}{n} Salmo:} \> 94 ant. 1 ``Adoremos al Señor...'' (estr. 1 y 3). \\ \\

{\changefont{cmss}{bx}{n} Ofrendas:} \> Al altar del Señor, Te ofrecemos oh Señor, Te presentamos, Bendeciré al Señor. \\  \\

{\changefont{cmss}{bx}{n} Comunión:} \> Vayamos a la mesa\footnotemark[2], Cuerpo y sangre de Jesús\footnotemark[2], Este es mi cuerpo\footnotemark[2], \\
\> Yo soy el camino\footnotemark[2]. \\ \\

{\changefont{cmss}{bx}{n} Post-com.:} \> Adoremos a Dios, Alabe todo el mundo (Taizé), Cuantas gracias te debemos, \\
\> Si el mismo pan comimos.\\ \\

{\changefont{cmss}{bx}{n} Salida:} \> Soy peregrino\footnotemark[3], Madre de los peregrinos, Madre de nuestro pueblo, \\
\> Salve María,  Santa María del camino. \\ \\


\end{tabbing}

\vspace{-15mm}

\footnotetext[1]{Cfr. \emph{Qué alegría} con ant. de entrada (Apoc 21,2); cfr. \emph{Pueblo de Dios} (estribillo) con Sal 45,9; cfr. \emph{Pueblo de reyes} (est. 2 y 5) con ant. de entrada (Apoc 21,2) y 1 Cor 3,11. }
\footnotetext[2]{ Todos estos son cantos eucarísticos. Es bueno aprovechar estas fiestas, que no tienen canto específico, para cantarlos.  }
\footnotetext[3]{Cfr. \emph{Soy peregrino} con (est. 4) con Apoc 21,2-5; el resto de los cantos son marianos por el mes de María.  }



%------------------------
\newpage
\setcounter{footnote}{0}
\thispagestyle{fancy}
\fancyhead{}
\fancyfoot{}
%\footskip=-1cm

\lhead{\changefont{cmss}{bx}{n} \small Revisión 2013}
\chead{\changefont{cmss}{bx}{n} \small Memoria del 11 de noviembre - San Martín 
de Tours}
\rhead{\changefont{cmss}{bx}{n} \small Ciclo A}
\rfoot{\changefont{cmss}{bx}{n}\large\thepage}

\vspace*{-11mm}

\begin{center}
{\large\it El juicio final }
\end{center}

\vspace{-3mm}

\begin{tabbing}

{\changefont{cmss}{bx}{n} Entrada:\ \ \ \ \ }\=  El sermón de 
la montaña\footnotemark[1], Brilló la luz\footnotemark[1], El Señor nos llama, 
Vine a alabar. \\ \\


{\changefont{cmss}{bx}{n} Salmo:} \> 118 ant. 2 ``Felices los que escuchan...'' 
(estr. 1 y 2). \\ 
\> (antífona de reemplazo: Sal 102 ant. 2 ``El amor del Señor, ...'') \\ \\

{\changefont{cmss}{bx}{n} Ofrendas:} \> Mira nuestra ofrenda, Este es nuestro 
pan, Te presentamos, Bendeciré al Señor. \\  \\

{\changefont{cmss}{bx}{n} Comunión:} \> Queremos ser Señor\footnotemark[2], 
Yo soy el camino, Creo en ti (Más cerca oh Dios). \\ \\

{\changefont{cmss}{bx}{n} Post-com.:} \> Al atardecer de la vida, Nada te turbe 
(Taizé), Mirarte sólo a ti, \\
\> Si el mismo pan comimos.\\ \\

{\changefont{cmss}{bx}{n} Salida:} \> Santa María del camino, Madre de 
los peregrinos, Madre de nuestro pueblo. \\ \\


\end{tabbing}

\vspace{-15mm}

\footnotetext[1]{\emph{El sermón de la montaña} y \emph{Brilló la luz} 
corresponden a las bienaventuranzas, muy apropiado a la vida de San Martín de 
Tours (y al juicio final). }
\footnotetext[2]{ \emph{Queremos ser Señor} está en línea con la vida del 
santo al proclamar ``queremos ser Señor servidores de verdad''; cfr. \emph{Al 
altardecer de la vida} (estr. 1) con Mt 25,31-40.  }











%------------------------
\newpage
\thispagestyle{empty}

%\begin{center}
%{\changefont{cmss}{bx}{n} \Huge VADEMECUM}\\
%\vspace{6mm}
%{\changefont{cmss}{bx}{n} \Huge de cantos litúrgicos}
%\end{center}

\vspace*{30mm}

\begin{center}
{\changefont{cmss}{bx}{n} \Huge CICLO B  (Marcos)}

\end{center}

%------------------------
\newpage
\thispagestyle{empty}

%------------------------
\newpage
\setcounter{footnote}{0}
\thispagestyle{fancy}
\fancyhead{}
\fancyfoot{}
%\footskip=-1cm

\lhead{\changefont{cmss}{bx}{n} \small Revisión 2013}
\chead{\changefont{cmss}{bx}{n} \small Domingo I - Adviento}
\rhead{\changefont{cmss}{bx}{n} \small Ciclo B}
\rfoot{\changefont{cmss}{bx}{n}\large\thepage}

\begin{center}
{\large\it Exhortación a la vigilancia y a la fidelidad}
\end{center}

\vspace{3mm}

\begin{tabbing}

{\changefont{cmss}{bx}{n} Entrada:\ \ \ \ \ }\= Despertemos llega Cristo\footnotemark[1], Juntos como hermanos (estr. 2)\footnotemark[1], Toda la tierra espera, \\ 
 \> Señor a ti clamamos (se puede usar para el encendido de la corona de Adviento). \\ \\

{\changefont{cmss}{bx}{n} Salmo:} \> 79 ant. 3 ``Míranos, Señor, ven a salvarnos...'' (estr. 1, 6 o 7). \\ 
\> (antífona de reemplazo: Sal 24 ant. 1 ``A ti elevo mi alma...'') \\ \\

{\changefont{cmss}{bx}{n} Ofrendas:} \> Este es nuestro pan\footnotemark[2], Saber que vendrás, Toda la tierra espera.\\ \\

{\changefont{cmss}{bx}{n} Comunión:} \> El viñador\footnotemark[3], Yo soy el camino, Bendeciré al Señor.\\ \\

{\changefont{cmss}{bx}{n} Post-com.:} \> Vaso nuevo\footnotemark[4], Si rasgaras los cielos\footnotemark[4], Al atardecer de la vida.\\ \\

{\changefont{cmss}{bx}{n} Salida:} \> Santa María del camino.  \\  \\

\end{tabbing}

\vspace{-10mm}

\footnotetext[1]{Ver comentario a \textbf{Juntos como hermanos}. }
\footnotetext[2]{Porque en la estr. 3 habla de ``nuestra libertad'', apropiado para este tiempo de espera.}
\footnotetext[3]{Porque hace referencia al salmo 79, que dice ``ven a visitar tu vid''.}
\footnotetext[4]{Se hace econ de la 1$^\circ$ lectura (Is 64,7, Is 64,4 e Is 63,19, respectivamente).}

%------------------------
\newpage
\setcounter{footnote}{0}
\thispagestyle{fancy}
\fancyhead{}
\fancyfoot{}
%\footskip=-1cm


\lhead{\changefont{cmss}{bx}{n} \small Revisión 2013}
\chead{\changefont{cmss}{bx}{n} \small Domingo II - Adviento}
\rhead{\changefont{cmss}{bx}{n} \small Ciclo B}
\rfoot{\changefont{cmss}{bx}{n}\large\thepage}

\begin{center}
{\large\it La predicación de Juan el Bautista}
\end{center}

%\vspace{3mm}


\begin{tabbing}

{\changefont{cmss}{bx}{n} Entrada:\ \ \ \ \ }\= Toda la tierra espera (estr. 1 y 3)\footnotemark[1], Despertemos llega Cristo, \\ 
 \> Señor a ti clamamos (se puede usar para el encendido de la corona de Adviento). \\ \\

{\changefont{cmss}{bx}{n} Salmo:} \> 84 ant. 1 ``Señor, revélanos tu amor...'' (estr. 4, 5 o 6). \\ \\

{\changefont{cmss}{bx}{n} Ofrendas:} \> Te ofrecemos Padre nuestro (II)\footnotemark[2], Pan de vida y bebida de  luz, Toda la tierra espera,   \\
\> Saber que vendrás\footnotemark[3], .\\ \\

{\changefont{cmss}{bx}{n} Comunión:} \>  Mensajero de la paz\footnotemark[4], Creo en ti Señor\footnotemark[4], Yo soy el camino.\\ \\

{\changefont{cmss}{bx}{n} Post-com.:} \> Adoremos a Dios (estr. 3), Mirarte sólo a ti, Nada te turbe, El Señor es mi fortaleza.\\ \\

{\changefont{cmss}{bx}{n} Salida:} \> Madre de los peregrinos, Oh Santísima.  \\  \\

\end{tabbing}

\vspace{-10mm}

\footnotetext[1]{El evangelio hace referencia a Is 40,3-5 (est. 1 y 3 de \textbf{Toda la tierra espera}). }
\footnotetext[2]{Hace referencia a ``la naturaleza entera, anhelando libertad'', apropiado para este tiempo litúrgico. }
\footnotetext[3]{Ver inconvenientes detallados en el comentario a \textbf{Saber que vendrás}. }
\footnotetext[4]{Debido a los textos equivalentes en Lucas (Lc 3,3-9) del evangelio del día (Mc 1,1-8).}

%------------------------
\newpage
\setcounter{footnote}{0}
\thispagestyle{fancy}
\fancyhead{}
\fancyfoot{}
%\footskip=-1cm


\lhead{\changefont{cmss}{bx}{n} \small Revisión 2013}
\chead{\changefont{cmss}{bx}{n} \small Solemnidad del 8 de diciembre}
\rhead{\changefont{cmss}{bx}{n} \small Ciclo B}
\rfoot{\changefont{cmss}{bx}{n}\large\thepage}

\begin{center}
{\large\it Inmaculada Concepción de María}
\end{center}

%\vspace{3mm}


\begin{tabbing}

{\changefont{cmss}{bx}{n} Entrada:\ \ \ \ \ }\= Feliz de ti María, La Virgen María nos reúne. \\ \\

{\changefont{cmss}{bx}{n} Salmo:} \> 97 ant. 1 ``Cantemos al Señor un canto nuevo, aleluia...'' (estr. 1, 2 o 3). \\ 
\> (antífona de reemplazo: Sal 95 ``Cantemos al Señor un canto nuevo''  \\
\> del P. José Bevilacqua)\\ \\

{\changefont{cmss}{bx}{n} Ofrendas:} \> Coplas de Yaraví, Bendeciré al Señor.   \\ \\


{\changefont{cmss}{bx}{n} Comunión:} \>  Mi alma glorifica, Jesucristo danos de este pan\footnotemark[1].\\ \\

{\changefont{cmss}{bx}{n} Post-com.:} \> Bendita sea tu pureza, Quiero decir que sí.\\ \\

{\changefont{cmss}{bx}{n} Salida:} \> Toda de Dios, Los cielos, la tierra, El ángel vino de los cielos,\\
\> Oh Santísima, Oh María, Madre de nuestro pueblo. \\  \\

\end{tabbing}

\vspace{-10mm}

\footnotetext[1]{Porque en su última estrofa se refiere a María. }

%------------------------
\newpage
\setcounter{footnote}{0}
\thispagestyle{fancy}
\fancyhead{}
\fancyfoot{}
%\footskip=-1cm

\lhead{\changefont{cmss}{bx}{n} \small Revisión 2013}
\chead{\changefont{cmss}{bx}{n} \small Domingo III - Adviento}
\rhead{\changefont{cmss}{bx}{n} \small Ciclo B}
\rfoot{\changefont{cmss}{bx}{n}\large\thepage}

\begin{center}
{\large\it (Domingo de Gaudete) El testimonio de Juan el Bautista }
\end{center}

\vspace{2mm}


\begin{tabbing}

{\changefont{cmss}{bx}{n} Entrada:\ \ \ \ \ }\= Que alegría (Sal 121)\footnotemark[1], Vienen con alegría\footnotemark[1], Señor a ti clamamos\footnotemark[1] (corona), \\
\> Toda la tierra espera\footnotemark[2], Despertemos llega Cristo. \\ \\

{\changefont{cmss}{bx}{n} Salmo:} \> Magníficat ``El Señor hizo en mí maravillas...'' (todas las estrofas). \\ 
\> (antífona de reemplazo: Sal 95 ``Cantemos al Señor un canto nuevo'' \\
\> del P. José Bevilacqua) \\ \\

{\changefont{cmss}{bx}{n} Ofrendas:} \> Al altar del Señor, Pan de vida y bebida de luz, Toda la tierra espera, \\
\> Te presentamos.\\ \\

{\changefont{cmss}{bx}{n} Comunión:} \> Yo soy el camino, Creo en ti Señor\footnotemark[2], Bendeciré al Señor.\\ \\

{\changefont{cmss}{bx}{n} Post-com.:} \> Tu fidelidad\footnotemark[3], Tan cerca de mí\footnotemark[2], El Señor es mi fortaleza.\\ \\

{\changefont{cmss}{bx}{n} Salida:} \> Salve María, Salve oh Reina, Oh Santísima.  \\  \\

\end{tabbing}

\vspace{-12mm}

\footnotetext[1]{Por ser domingo de \emph{Gaudete}. Sino, \emph{Señor a ti clamamos} debido a Is 35,4. }
\footnotetext[2]{Porque se revela que Jesús es el Salvador.}
\footnotetext[3]{Debido al texto de la 2$^\circ$ lectura (1 Tes 5,24).}

%------------------------
\newpage
\setcounter{footnote}{0}
\thispagestyle{fancy}
\fancyhead{}
\fancyfoot{}
%\footskip=-1cm

\lhead{\changefont{cmss}{bx}{n} \small Revisión 2013}
\chead{\changefont{cmss}{bx}{n} \small Domingo IV - Adviento}
\rhead{\changefont{cmss}{bx}{n} \small Ciclo B}
\rfoot{\changefont{cmss}{bx}{n}\large\thepage}

\begin{center}
{\large\it El anuncio del nacimiento de Jesús }
\end{center}

\vspace{1mm}


\begin{tabbing}

{\changefont{cmss}{bx}{n} Entrada:\ \ \ \ \ }\= Despertemos llega Cristo, Señor a ti clamamos (opcionalente, para la corona),  \\ 
 \> Toda la tierra espera (estr. 2). \\ \\

{\changefont{cmss}{bx}{n} Salmo:} \> 145 ant. 1 ``El Señor es fiel a su palabra...'' (est. 4, 5 o 6). \\ 
\> (antífona de reemplazo: Sal 26 ant. 1 ``Cantaré y celebraré al Señor.'') \\ \\

{\changefont{cmss}{bx}{n} Ofrendas:} \> Señor te ofrecemos\footnotemark[1], Toda la tierra espera, Saber que vendrás.\\ \\

{\changefont{cmss}{bx}{n} Comunión:} \> Mi alma glorifica, Jesucristo danos de este pan\footnotemark[2], Bendeciré al Señor, \\
\> Creo en ti Señor. \\ \\

{\changefont{cmss}{bx}{n} Post-com.:} \> Adoremos a Dios (estr. 1), Alabe todo el mundo (de Taizé).\\ \\

{\changefont{cmss}{bx}{n} Salida:} \> Los cielos la tierra\footnotemark[3], El ángel vino de los cielos\footnotemark[3].  \\  \\

\end{tabbing}

\vspace{-11mm}

\footnotetext[1]{Debido a que se refiere al texto del salmo 88,2 (cfr. vv. 3, 18, 19, 36). }
\footnotetext[2]{Porque en su última estrofa habla de María.}
\footnotetext[3]{\emph{Los cielos, la tierra} se refiere al saludo del ángel. \emph{El ángel vino de los cielos} corresponde al \emph{Angelus}. }

%------------------------
\newpage
\setcounter{footnote}{0}
\thispagestyle{fancy}
\fancyhead{}
\fancyfoot{}
%\footskip=-1cm

\lhead{\changefont{cmss}{bx}{n} \small Revisión 2013}
\chead{\changefont{cmss}{bx}{n} \small Solemnidad del 25 de diciembre - Natividad del Señor}
\rhead{\changefont{cmss}{bx}{n} \small Ciclo B}
\rfoot{\changefont{cmss}{bx}{n}\large\thepage}

\begin{center}
{\large\it Prólogo al evangelio según San Juan }
\end{center}

\vspace{1mm}


\begin{tabbing}

{\changefont{cmss}{bx}{n} Entrada:\ \ \ \ \ }\= Ha nacido el Rey del cielo\footnotemark[1], Gloria eterna\footnotemark[2], Mundo feliz\footnotemark[1], Pastores de la montaña\footnotemark[1], \\
\> Cristianos vayamos\footnotemark[1], Vamos pastorcitos\footnotemark[1], Ya llegó la Nochebuena\footnotemark[1].  \\  \\

{\changefont{cmss}{bx}{n} Salmo:} \> 97 con ant. del Sal 95  ``Hoy nos ha nacido un salvador...'' (est. 1, 5 o 6). \\ \\

{\changefont{cmss}{bx}{n} Ofrendas:} \> Cristianos vayamos\footnotemark[1], Vamos pastorcitos\footnotemark[1], Ya llegó la Nochebuena\footnotemark[1].\\ \\

{\changefont{cmss}{bx}{n} Comunión:} \> Pastores de la montaña\footnotemark[1], El pan de Belén\footnotemark[3], Noche de paz\footnotemark[3],\\
\> La peregrinación (A la huella)\footnotemark[3], Zamba de la Navidad. \\ \\

{\changefont{cmss}{bx}{n} Post-com.:} \> Noche anunciada\footnotemark[3], ?`Qué niño es éste?, Adoremos a Dios (estr. 1), \\
\> Alabe todo el mundo (de Taizé).\\ \\

{\changefont{cmss}{bx}{n} Salida:} \> Gloria eterna\footnotemark[2], Pastores de la montaña\footnotemark[1], Mundo feliz\footnotemark[1], Entonen tiernos cánticos.  \\  \\

\end{tabbing}

\vspace{-14mm}

\footnotetext[1]{Cfr. \emph{Ya llegó la Nochebuena} con Mt 1,25, \emph{Mundo feliz} con Sal 97, \emph{Pastores de la montaña} con Lc 2,8-16, \\ \emph{Vamos pastorcitos} con Lc 2,8-20 y \emph{Cristianos vayamos} con Lc 2,15-18. }
\footnotetext[2]{Ver comentario a \textbf{Gloria eterna}. Cfr. con Is 9,1-2 (1$^\circ$ lectura de la misa de la noche). }
\footnotetext[3]{Cfr. \emph{El pan de Belén} con Mt 1,23, \emph{La peregrinación} con Lc 2,6-7, \emph{Noche de paz} y \emph{Noche anunciada} con Lc 2,6-20.}

%------------------------
\newpage
\setcounter{footnote}{0}
\thispagestyle{fancy}
\fancyhead{}
\fancyfoot{}
%\footskip=-1cm

\lhead{\changefont{cmss}{bx}{n} \small Revisión 2013}
\chead{\changefont{cmss}{bx}{n} \small Domingo de la octava de Navidad}
\rhead{\changefont{cmss}{bx}{n} \small Ciclo B}
\rfoot{\changefont{cmss}{bx}{n}\large\thepage}

\begin{center}
{\large\it (La Sagrada Familia) La presentación de Jesús en el Templo }
\end{center}

\vspace{1mm}


\begin{tabbing}

{\changefont{cmss}{bx}{n} Entrada:\ \ \ \ \ }\= Ya llegó la Nochebuena (texto modificado), Gloria eterna, Mundo feliz, \\ 
\> Ha nacido el Rey del cielo, Pastores de la montaña, Cristianos vayamos,   \\ 
\> Vamos pastorcitos.\\ \\

{\changefont{cmss}{bx}{n} Salmo:} \> 127 ant. 1 ``!`Feliz quien ama al Señor...!'' (est. 1, 2). \\ \\

{\changefont{cmss}{bx}{n} Ofrendas:} \> Cristianos vayamos, Vamos pastorcitos, Ya llegó la Nochebuena.\\ \\

{\changefont{cmss}{bx}{n} Comunión:} \> La peregrinación (A la huella)\footnotemark[1], Noche de paz, El pan de Belén,  \\
\>  Pastores de la montaña, Zamba de la Navidad. \\ \\

{\changefont{cmss}{bx}{n} Post-com.:} \> Noche anunciada, ?`Qué niño es éste?, Adoremos a Dios (estr. 3), \\
\> Alabe todo el mundo (de Taizé).\\ \\

{\changefont{cmss}{bx}{n} Salida:} \> Madre de nuestro pueblo (estr. 1,4,5)\footnotemark[2], Gloria eterna, Pastores de la montaña,\\
\>  Mundo feliz, Entonen tiernos cánticos.  \\  \\

\end{tabbing}

\vspace{-12mm}

\footnotetext[1]{Por tratarse de la Fiesta de La Sagrada Familia. }
\footnotetext[2]{Cfr. con Lc 2,22-36. }

%------------------------
\newpage
\setcounter{footnote}{0}
\thispagestyle{fancy}
\fancyhead{}
\fancyfoot{}
%\footskip=-1cm

\lhead{\changefont{cmss}{bx}{n} \small Revisión 2013}
\chead{\changefont{cmss}{bx}{n} \small Solemnidad del 1$^\circ$ de enero}
\rhead{\changefont{cmss}{bx}{n} \small Ciclo B}
\rfoot{\changefont{cmss}{bx}{n}\large\thepage}

\begin{center}
{\large\it Santa María Madre de Dios }
\end{center}

%\vspace{1mm}


\begin{tabbing}

{\changefont{cmss}{bx}{n} Entrada:\ \ \ \ \ }\= María Madre de Dios\footnotemark[1], Feliz de ti María (est. 1,2)\footnotemark[1], La Virgen María nos reúne\footnotemark[1], \\ 
\> Gloria eterna\footnotemark[2].\\ \\

{\changefont{cmss}{bx}{n} Salmo:} \> 66 ant. 1 ``!`A tí, Señor, te alabe la tierra...!'' (est. 1, 2). \\
\> (antífona de reemplazo: Sal 144 ant. 1 ``Te alabamos, Señor, ...'') \\ \\

{\changefont{cmss}{bx}{n} Ofrendas:} \> Bendeciré al Señor, Cristianos vayamos, Vamos pastorcitos.\\ \\

{\changefont{cmss}{bx}{n} Comunión:} \> Simple oración\footnotemark[3], La peregrinación (A la huella), El pan de Belén,  \\
\>  Pastores de la montaña. \\ \\

{\changefont{cmss}{bx}{n} Post-com.:} \> Quiero decir que sí\footnotemark[1], Noche anunciada, Adoremos a Dios (estr. 3), \\
\> Alabe todo el mundo (de Taizé).\\ \\

{\changefont{cmss}{bx}{n} Salida:} \> Salve María\footnotemark[1], Oh María\footnotemark[1], Junto a ti María\footnotemark[1],  \\
\> Madre de nuestro pueblo (estr. 1,4,5)\footnotemark[1], Gloria eterna\footnotemark[2].\\ \\

\end{tabbing}

\vspace{-14mm}
\footnotetext[1]{Por ser la solemnidad de Santa María Madre de Dios.}
\footnotetext[2]{Debido a que la antífona de entrada del día coincide con \emph{Gloria eterna} en que hacen referencia a Is 9,2. }
\footnotetext[3]{Debido a que es la Jornada mundial por la paz, en concordancia en el estribillo de \emph{Simple oración}. }

%------------------------
\newpage
\setcounter{footnote}{0}
\thispagestyle{fancy}
\fancyhead{}
\fancyfoot{}
%\footskip=-1cm

\lhead{\changefont{cmss}{bx}{n} \small Revisión 2013}
\chead{\changefont{cmss}{bx}{n} \small Domingo II - Navidad}
\rhead{\changefont{cmss}{bx}{n} \small Ciclo B}
\rfoot{\changefont{cmss}{bx}{n}\large\thepage}

\begin{center}
{\large\it Prólogo al evangelio según San Juan }
\end{center}

\vspace{1mm}


\begin{tabbing}

{\changefont{cmss}{bx}{n} Entrada\footnotemark[1]:\ \ \ \ \ }\= Ha nacido el Rey del cielo, Gloria eterna, Mundo feliz, Pastores de la montaña, \\
\> Cristianos vayamos, Vamos pastorcitos.  \\  \\

{\changefont{cmss}{bx}{n} Salmo:} \> 147 ant. 1 ``Glorifica al Señor Jerusalén...'' (est. 1, 2 o 4). \\ \\

{\changefont{cmss}{bx}{n} Ofrendas:} \> Vamos pastorcitos, Cristianos vayamos, Ya llegó la Nochebuena.\\ \\

{\changefont{cmss}{bx}{n} Comunión:} \> El pan de Belén, Pastores de la montaña, La peregrinación (A la huella). \\  \\

{\changefont{cmss}{bx}{n} Post-com.:} \> Adoremos a Dios (estr. 1), Alabe todo el mundo (de Taizé), Noche anunciada.\\ \\

{\changefont{cmss}{bx}{n} Salida:} \> Gloria eterna, Pastores de la montaña, Mundo feliz, Entonen tiernos cánticos, \\
\>  Oh Santísima.  \\  \\

\end{tabbing}

\vspace{-11mm}

\footnotetext[1]{Se repiten muchos de los cantos del día de Navidad. El evangelio del día es el mismo que el de Navidad. }

%------------------------
\newpage
\setcounter{footnote}{0}
\thispagestyle{fancy}
\fancyhead{}
\fancyfoot{}
%\footskip=-1cm

\lhead{\changefont{cmss}{bx}{n} \small Revisión 2013}
\chead{\changefont{cmss}{bx}{n} \small Solemnidad del 6 de enero - Navidad }
\rhead{\changefont{cmss}{bx}{n} \small Ciclo B}
\rfoot{\changefont{cmss}{bx}{n}\large\thepage}

\begin{center}
{\large\it (La Epifanía del Señor) La visita de los magos }
\end{center}

%\vspace{1mm}

\begin{tabbing}

{\changefont{cmss}{bx}{n} Entrada:\ \ \ \ \ }\= Mundo feliz, Ha nacido el Rey del cielo, Gloria eterna,    \\ 
\> Ya llegó la Nochebuena (texto modificado)\footnotemark[1], Cristianos vayamos\footnotemark[2].   \\ \\

{\changefont{cmss}{bx}{n} Salmo:} \> 71 ant. 1 ``Tú eres, Señor, el único Rey...'' (est. 1, 4 , 5 o 6). \\
\> (antífona de reemplazo: Sal 144 ant. 1 ``Te alabamos Señor...'') \\ \\

{\changefont{cmss}{bx}{n} Ofrendas:} \> Cristianos vayamos\footnotemark[2], Ya llegó la Nochebuena\footnotemark[1], Vamos pastorcitos.\\ \\

{\changefont{cmss}{bx}{n} Comunión:} \> Los reyes magos, El pan de Belén, La peregrinación (A la huella),    \\
\>  Pastores de la montaña, Zamba de la Navidad, Noche de paz. \\ \\

{\changefont{cmss}{bx}{n} Post-com.:} \> Adoremos a Dios (estr. 1), Alabe todo el mundo (de Taizé), ?`Qué niño es éste?,   \\
\> Noche anunciada.\\ \\

{\changefont{cmss}{bx}{n} Salida:} \> Gloria eterna, Pastores de la montaña, Mundo feliz, Entonen tiernos cánticos,  \\
\>  Madre de nuestro pueblo (estr. 1 y 4).  \\  \\

\end{tabbing}

\vspace{-11mm}

\footnotetext[1]{Con el texto modificado: ``Vamos todos a esperarlo'' por ``Vamos todos a adorarlo'', y ``va a nacer'' por ``ya nació''. }
\footnotetext[2]{Cfr. con Mt 2,2 y Mt 2,11. }

%------------------------
\newpage
\setcounter{footnote}{0}
\thispagestyle{fancy}
\fancyhead{}
\fancyfoot{}
%\footskip=-1cm

\lhead{\changefont{cmss}{bx}{n} \small Revisión 2013}
\chead{\changefont{cmss}{bx}{n} \small Domingo después del 6 de enero - Navidad }
\rhead{\changefont{cmss}{bx}{n} \small Ciclo B}
\rfoot{\changefont{cmss}{bx}{n}\large\thepage}

\begin{center}
{\large\it El bautismo de Jesús }
\end{center}

%\vspace{1mm}

\begin{tabbing}

{\changefont{cmss}{bx}{n} Entrada:\ \ \ \ \ }\= Gloria eterna,  Brilló la luz, Un sólo Señor (estr. 3)\footnotemark[1], Pueblo de Dios,    \\ 
\> Vine a alabar.   \\ \\

{\changefont{cmss}{bx}{n} Salmo:} \> 32 ant. 1 ``Cantemos todos al Señor, aleluia...'' (est. 1, 3 o 8). \\
\> (antífona de reemplazo: Sal 26 ant. 1 ``Cantaré y celebraré al Señor.'') \\ \\

{\changefont{cmss}{bx}{n} Ofrendas:} \> Padre nuestro recibid, Te ofrecemos oh Señor, Te presentamos, Bendito seas.\\ \\

{\changefont{cmss}{bx}{n} Comunión:} \> Brilló la luz, Yo soy el camino, Bendeciré al Señor, Como Cristo nos amó.  \\  \\

{\changefont{cmss}{bx}{n} Post-com.:} \> Creo en Tí Señor (estr. 2)\footnotemark[2], Tu fidelidad, Adoremos a Dios, \\
\> Alabe todo el mundo (de Taizé).   \\ \\

{\changefont{cmss}{bx}{n} Salida:} \> Gloria eterna, Mi camino eres tú, Canción del testigo, Anunciaremos tu Reino. \\  \\

\end{tabbing}

\vspace{-11mm}

\footnotetext[1]{Debido a que habla de ``un solo Dios y Padre'', pero siempre que quede claro en la asamblea que el bautismo del Señor (bautismo de penitencia) no es igual a nuestro bautismo. }
\footnotetext[2]{Con el texto de \emph{Más cerca oh Dios de Ti}. }

%------------------------
\newpage
\setcounter{footnote}{0}
\thispagestyle{fancy}
\fancyhead{}
\fancyfoot{}
%\footskip=-1cm

\lhead{\changefont{cmss}{bx}{n} \small Revisión 2013}
\chead{\changefont{cmss}{bx}{n} \small Miércoles de Ceniza}
\rhead{\changefont{cmss}{bx}{n} \small Ciclo B}
\rfoot{\changefont{cmss}{bx}{n}\large\thepage}

\begin{center}
{\large\it La limosna, la oración, el ayuno }
\end{center}

\vspace{1mm}


\begin{tabbing}

{\changefont{cmss}{bx}{n} Entrada:\ \ \ \ \ }\= Dice el Señor conviértanse\footnotemark[1], Perdón Señor, Sí me levantaré.  \\ \\

{\changefont{cmss}{bx}{n} Salmo:} \> 50 ant. 1 ``Piedad, Señor, pecamos contra ti'' (est. 1, 2, 6, 7 u 8). \\ \\

{\changefont{cmss}{bx}{n} Cenizas:} \> Perdón oh Dios mío, Perdón Señor, Sí me levantaré, Vuelve a mí. \\ \\

{\changefont{cmss}{bx}{n} Ofrendas:} \> Recibe oh Dios el pan, Te ofrecemos Padre nuestro (vidala), \\
\>  Recibe oh Padre Santo, Coplas de Yaraví. \\ \\

{\changefont{cmss}{bx}{n} Comunión:} \> Oh buen Jesús, Sí me levantaré, Bienaventurados los pobres, \\
\>  Creo en ti Señor, Vuelve a mí.\\ \\

{\changefont{cmss}{bx}{n} Post-com.:} \> Vaso nuevo (El alfarero), Todos unidos, Nada te turbe (de Taizé).\\ \\

{\changefont{cmss}{bx}{n} Salida:} \> Soy peregrino, Virgen de la esperanza.  \\  \\

\end{tabbing}

\vspace{-11mm}

\footnotetext[1]{Aclamación al evangelio, pero que se puede usar de entrada (cfr. Joel 2, 12-18).  }

%------------------------
\newpage
\setcounter{footnote}{0}
\thispagestyle{fancy}
\fancyhead{}
\fancyfoot{}
%\footskip=-1cm

\lhead{\changefont{cmss}{bx}{n} \small Revisión 2013}
\chead{\changefont{cmss}{bx}{n} \small Domingo I - Cuaresma}
\rhead{\changefont{cmss}{bx}{n} \small Ciclo B}
\rfoot{\changefont{cmss}{bx}{n}\large\thepage}

\begin{center}
{\large\it Las tentaciones de Jesús en el desierto }
\end{center}

\vspace{1mm}


\begin{tabbing}

{\changefont{cmss}{bx}{n} Entrada:\ \ \ \ \ }\= Cruz de Cristo\footnotemark[1], Perdón Señor, Sí me levantaré, Perdón oh Dios mío.\\ \\

{\changefont{cmss}{bx}{n} Salmo:} \> 24 ant. 2 ``Muéstrame Señor tu camino y guíame por él'' (estr. 2, 3 o 5) \\ 
\> (antífona de reemplazo: Sal 24 ant. 1 ``A ti elevo mi alma...'') \\ \\

{\changefont{cmss}{bx}{n} Ofrendas:} \> Recibe oh Dios el pan, Te ofrecemos Padre nuestro (vidala),  \\
\>  Mira nuestra ofrenda, Recibe oh Padre Santo, Coplas de Yaraví. \\ \\

{\changefont{cmss}{bx}{n} Comunión:} \> Cruz de Cristo\footnotemark[1], Dios es fiel\footnotemark[2], Oh buen Jesús, Creo en ti Señor,  \\
\> Vuelve a mí, Bienaventurados los pobres, Hambre de Dios. \\ \\

{\changefont{cmss}{bx}{n} Post-com.:} \> Tu fidelidad\footnotemark[2], Nada te turbe (de Taizé).\\ \\

{\changefont{cmss}{bx}{n} Salida:} \> Santa María del Amén, Soy peregrino, Virgen de la esperanza.  \\  \\

\end{tabbing}

\vspace{-11mm}

\footnotetext[1]{Cfr. estrofas 1 a 3 con la 2$^\circ$ lectura 1 Ped 3,18-22, 4$^\circ$ estrofa con 1$^\circ$ lectura (Gen 8,13).  }
\footnotetext[2]{Por el contexto de la obra mesiánica de Jesús en favor de la Alianza.  }

%------------------------
\newpage
\setcounter{footnote}{0}
\thispagestyle{fancy}
\fancyhead{}
\fancyfoot{}
%\footskip=-1cm

\lhead{\changefont{cmss}{bx}{n} \small Revisión 2013}
\chead{\changefont{cmss}{bx}{n} \small Domingo II - Cuaresma}
\rhead{\changefont{cmss}{bx}{n} \small Ciclo B}
\rfoot{\changefont{cmss}{bx}{n}\large\thepage}

\begin{center}
{\large\it La transfiguración de Jesús }
\end{center}

%\vspace{1mm}


\begin{tabbing}

{\changefont{cmss}{bx}{n} Entrada:\ \ \ \ \ }\= Brilló la luz, Cruz de Cristo\footnotemark[1], Juntos como hermanos, Perdón Señor,   \\
\> Vuélvenos tu rostro (est. 1 y 3), Sí me levantaré (estr. 3).\\ \\

{\changefont{cmss}{bx}{n} Salmo:} \> 115 ant. 1``?`con qué pagaré al Señor...?'' (estrofas 1,2)  \\ 
\> (antífona de reemplazo: Sal 26 ant. 2 ``El Señor es mi luz, mi salvación...'') \\ \\

{\changefont{cmss}{bx}{n} Ofrendas:} \> Pan de vida y bebida de luz, Este es nuestro pan, Recibe oh Dios el pan,  \\
\>  Te ofrecemos Padre nuestro (vidala). \\ \\

{\changefont{cmss}{bx}{n} Comunión:} \> En memoria tuya\footnotemark[2], Brilló la luz, Dios es fiel\footnotemark[2], Creo en ti Señor\footnotemark[3],   \\
\>   Vuelve a mí\footnotemark[2].\\ \\

{\changefont{cmss}{bx}{n} Post-com.:} \> Tu fidelidad\footnotemark[2], Más cerca oh Dios de ti (estr. 2)\footnotemark[4].\\ \\

{\changefont{cmss}{bx}{n} Salida:} \> Madre de nuestro pueblo (estr. 9), Virgen de la esperanza.  \\  \\

\end{tabbing}

\vspace{-13mm}

\footnotetext[1]{Cfr. la presentación de la cruz con Mc 9,4 (Moisés y Elías hablando de la Pasión).  }
\footnotetext[2]{Por el contexto de la obra mesiánica de Jesús en favor de la Alianza.  }
\footnotetext[3]{Con el texto de \emph{Cantemos hermanos con amor}.  }
\footnotetext[4]{Texto tradicional de \emph{Creo en ti Señor}.  }

%------------------------
\newpage
\setcounter{footnote}{0}
\thispagestyle{fancy}
\fancyhead{}
\fancyfoot{}
%\footskip=-1cm

\lhead{\changefont{cmss}{bx}{n} \small Revisión 2013}
\chead{\changefont{cmss}{bx}{n} \small Domingo III - Cuaresma}
\rhead{\changefont{cmss}{bx}{n} \small Ciclo B}
\rfoot{\changefont{cmss}{bx}{n}\large\thepage}

\begin{center}
{\large\it Expulsión de los vendedores del templo}
\end{center}

\vspace{1mm}


\begin{tabbing}

{\changefont{cmss}{bx}{n} Entrada:\ \ \ \ \ }\= Sí me levantaré (estr. 1, 3, 10 o 11), Cruz de Cristo\footnotemark[1],   \\
\> Juntos como hermanos (estr. 1), Perdón Señor.\\   \\

{\changefont{cmss}{bx}{n} Salmo:} \> 18b ant. 1 ``Tu Palabra, Señor, es la verdad...'' (estr. 1 y 2). \\ \\

{\changefont{cmss}{bx}{n} Ofrendas:} \> Recibe oh Dios el pan, Comienza el sacrificio\footnotemark[2], Este es nuestro pan.\\ \\ 

{\changefont{cmss}{bx}{n} Comunión:} \> Cruz de Cristo\footnotemark[1], El pueblo de Dios\footnotemark[3], Hambre de Dios\footnotemark[3], Dios es fiel\footnotemark[3],\\
\>  Creo en ti Señor.  \\ \\

{\changefont{cmss}{bx}{n} Post-com.:} \> Creo en ti Señor, (silencio).\\ \\

{\changefont{cmss}{bx}{n} Salida:} \> Santa María del Amén, Soy peregrino.  \\  \\

\end{tabbing}

\vspace{-11mm}

\footnotetext[1]{Cfr. con la 2$^\circ$ lectura (1 Cor 1,22-25). }
\footnotetext[2]{Cfr. la 2$^\circ$ estrofa: ``la fe de nuestros padres, consérvanos, Señor'' con la 1$^\circ$ lectura (Ex 20,1-17). }
\footnotetext[3]{Cfr. con la 1$^\circ$ lectura (Ex 20,1-17).  }

%------------------------
\newpage
\setcounter{footnote}{0}
\thispagestyle{fancy}
\fancyhead{}
\fancyfoot{}
%\footskip=-1cm

\lhead{\changefont{cmss}{bx}{n} \small Revisión 2013}
\chead{\changefont{cmss}{bx}{n} \small Domingo IV - Cuaresma}
\rhead{\changefont{cmss}{bx}{n} \small Ciclo B}
\rfoot{\changefont{cmss}{bx}{n}\large\thepage}

\begin{center}
{\large\it (Domingo de Laetare) El diálogo de Jesús con Nicodemo}
\end{center}

\vspace{1mm}


\begin{tabbing}

{\changefont{cmss}{bx}{n} Entrada:\ \ \ \ \ }\= Qué alegría\footnotemark[1], Vienen con alegría\footnotemark[1], Cruz de Cristo\footnotemark[2], Perdón Señor\footnotemark[2].\\  \\

{\changefont{cmss}{bx}{n} Salmo:} \> 136 ant. 1 ``Señor, soy peregrino en esta tierra...'' (estr. 1, 2 o 3). \\
\>  (antífona de reemplazo: Sal 84 ant. 1 ``Señor revélanos tu amor...'') \\ \\

{\changefont{cmss}{bx}{n} Ofrendas:} \> Pan de vida y bebida de luz, Al altar del Señor\footnotemark[1], Recibe oh Dios eterno.\\ \\ 

{\changefont{cmss}{bx}{n} Comunión:} \> Bienaventurados los pobres, En memoria tuya, Creo en ti Señor\footnotemark[3].  \\ \\

{\changefont{cmss}{bx}{n} Post-com.:} \> Adoremos a Dios, Creo en ti (estr. 2)\footnotemark[3], (silencio).\\ \\

{\changefont{cmss}{bx}{n} Salida:} \> Soy peregrino\footnotemark[4], En medio de los pueblos (estr. 1,2)\footnotemark[4].  \\  \\

\end{tabbing}

\vspace{-11mm}

\footnotetext[1]{Por ser domingo de Laetare.  }
\footnotetext[2]{Cfr. \emph{Cruz de Cristo} con Jn 3,14-17 y \emph{Perdón Señor} (estr. 3) con Jn 3,16. }
\footnotetext[3]{Con el texto tradicional de \emph{Más cerca oh Dios de ti}.  }
\footnotetext[4]{Cfr. \emph{Soy peregrino} (estribillo) con 2 Cro 36,23 y  Sal 136,1.5. Cfr. \emph{En medio de los pueblos} con 2 Cro 36,14-23.  }

%------------------------
\newpage
\setcounter{footnote}{0}
\thispagestyle{fancy}
\fancyhead{}
\fancyfoot{}
%\footskip=-1cm

\lhead{\changefont{cmss}{bx}{n} \small Revisión 2013}
\chead{\changefont{cmss}{bx}{n} \small Domingo V - Cuaresma}
\rhead{\changefont{cmss}{bx}{n} \small Ciclo B}
\rfoot{\changefont{cmss}{bx}{n}\large\thepage}

\begin{center}
{\large\it La glorificación de Jesús  por medio de la muerte}
\end{center}

\vspace{1mm}


\begin{tabbing}

{\changefont{cmss}{bx}{n} Entrada:\ \ \ \ \ }\= Cruz de Cristo\footnotemark[1], Sí me levantaré (estr. 7)\footnotemark[1], Perdón Señor.\\   \\

{\changefont{cmss}{bx}{n} Salmo:} \> 50 ant. 1 ``Pieded, Señor, pecamos contra ti'' (estr. 1,7 o 9). \\ \\

{\changefont{cmss}{bx}{n} Ofrendas\footnotemark[2]:} \> Sé como el grano de trigo, Zamba del grano de trigo, Entre tus manos,\\
\> Una espiga, Comienza el sacrificio. \\ \\ 

{\changefont{cmss}{bx}{n} Comunión:} \> Antes de ser llevado a la muerte\footnotemark[3], Cruz de Cristo, Sí me levantaré, \\
\> En memoria tuya.  \\ \\

{\changefont{cmss}{bx}{n} Post-com.:} \>  Creo en ti (estr. 2)\footnotemark[4], Tu fidelidad, (silencio).\\ \\

{\changefont{cmss}{bx}{n} Salida:} \> Virgen de la esperanza (estr. 1,2,5), Santa María del amén, Soy peregrino, \\
\> Madre de nuestro pueblo (estr. 9).  \\  \\

\end{tabbing}

\vspace{-10mm}

\footnotetext[1]{Cfr. \emph{Cruz de Cristo} con el evangelio del día (Jn 12,20-33) y \emph{Sí me levantaré} (estr. 7) con Sal 50,3. }
\footnotetext[2]{Todos estos cantos son equivalentes con respecto a que hacen referencia al evangelio del día (Jn 12,20-33).  }
\footnotetext[3]{Cfr. la 4$^\circ$ estrofa con Jn 12,24.}
\footnotetext[4]{Con el texto tradicional de \emph{Más cerca oh Dios de ti}.  }

%------------------------
\newpage
\setcounter{footnote}{0}
\thispagestyle{fancy}
\fancyhead{}
\fancyfoot{}
%\footskip=-1cm

\lhead{\changefont{cmss}{bx}{n} \small Revisión 2013}
\chead{\changefont{cmss}{bx}{n} \small Domingo de Ramos - Cuaresma}
\rhead{\changefont{cmss}{bx}{n} \small Ciclo B}
\rfoot{\changefont{cmss}{bx}{n}\large\thepage}

\begin{center}
{\large\it (Domingo de Pasión) La Pasión de Jesús}
\end{center}

%\vspace{1mm}


\begin{tabbing}

{\changefont{cmss}{bx}{n} Entrada:\ \ \ \ \ }\= Canta Jerusalén, Arriba nuestros ramos, Con ramos en las manos (Sal 23 ant. 3),\\
\> Bendito el que viene (Sal 46 ant. 2).\\   \\

{\changefont{cmss}{bx}{n} Salmo:} \> 21 ant. 1 ``Dios mío, no me abandones...'' (estr. 1, 3, 7, 8 o 9). \\ \\

{\changefont{cmss}{bx}{n} Ofrendas:} \> Este es nuestro pan\footnotemark[1], Te ofrecemos Padre nuestro (vidala)\footnotemark[1], Coplas de Yaraví, \\
\> Una espiga, Sé como el grano de trigo, Zamba del grano de trigo.\\ \\ 

{\changefont{cmss}{bx}{n} Comunión:} \> Rey de los reyes, En la postrera cena, Antes de ser llevado a la muerte,\\
\> En memoria tuya, Jesucristo danos de este pan, Queremos ser Señor\footnotemark[2].  \\ \\

{\changefont{cmss}{bx}{n} Post-com.:} \> Victoria tu reinarás, Creo en ti Señor\footnotemark[3], Tu fidelidad, (silencio).\\ \\

{\changefont{cmss}{bx}{n} Salida:} \> Virgen de la esperanza (estr. 1,2,5), Cristo Jesús (estr. 2,3).\\ \\

\end{tabbing}

\vspace{-14mm}

\footnotetext[1]{Cfr. 2$^\circ$ estrofa con Mc 14,22-24. }
\footnotetext[2]{Sólo en caso de necesidad. }
\footnotetext[3]{Con el texto tradicional de \emph{Más cerca oh Dios de ti}.  }




%------------------------
\newpage
\setcounter{footnote}{0}
\thispagestyle{fancy}
\fancyhead{}
\fancyfoot{}
%\footskip=-1cm

\lhead{\changefont{cmss}{bx}{n} \small Revisión 2013}
\chead{\changefont{cmss}{bx}{n} \small Jueves Santo - Cena del Señor}
\rhead{\changefont{cmss}{bx}{n} \small Ciclo B}
\rfoot{\changefont{cmss}{bx}{n}\large\thepage}

\begin{center}
{\large\it (Triduo pascual) El lavatorio de los pies}
\end{center}

%\vspace{1mm}


\begin{tabbing}

{\changefont{cmss}{bx}{n} Entrada\footnotemark[1]:\ \ \ \ \ }\= Me pongo en tus 
manos oh Señor, El Señor nos llama.\\   \\

{\changefont{cmss}{bx}{n} Salmo:} \> 115 ant. 1 ``?`Con qué pagaré al 
Señor...?'' (estr. 1 y 2). \\ \\

{\changefont{cmss}{bx}{n} Aclamación:} \> Si yo el maestro (dos 
veces). \\ \\

{\changefont{cmss}{bx}{n} Lavatorio:} \> Un mandamiento nuevo, Si yo el 
maestro (dos veces). \\ \\

{\changefont{cmss}{bx}{n} Ofrendas:} \> Los frutos de la tierra.\\ \\ 

{\changefont{cmss}{bx}{n} Comunión:} \> Antes de ser llevado a la muerte,$\,$No 
hay mayor amor,$\,$Dios me dio a mi hermano,\\
\> Memorial.  \\ \\

{\changefont{cmss}{bx}{n} Procesión:} \> Cantemos al amor 
de los amores, Yo soy el camino.\\ \\

{\changefont{cmss}{bx}{n} Adoración:} \> Tan sublime sacramento (Tantum 
ergo).\\ \\

\end{tabbing}

\vspace{-14mm}

\footnotetext[1]{Se canta el \emph{Gloria} (con sentimiento porque no se 
cantará más hasta la Vigilia pascual. }





%------------------------
\newpage
\setcounter{footnote}{0}
\thispagestyle{fancy}
\fancyhead{}
\fancyfoot{}
%\footskip=-1cm

\lhead{\changefont{cmss}{bx}{n} \small Revisión 2013}
\chead{\changefont{cmss}{bx}{n} \small Viernes Santo - Pasión del Señor}
\rhead{\changefont{cmss}{bx}{n} \small Ciclo B}
\rfoot{\changefont{cmss}{bx}{n}\large\thepage}

\begin{center}
{\large\it (Triduo pascual) La Pasión del Señor}
\end{center}

%\vspace{1mm}


\begin{tabbing}

{\changefont{cmss}{bx}{n} Entrada:\ \ \ \ \ }\= (en 
silencio; postración silenciosa).\\   \\

{\changefont{cmss}{bx}{n} Salmo:} \> 30 ant. 1 ``En tus manos 
Señor...'' (estr. 1, 2, 5 u 8). \\ \\

{\changefont{cmss}{bx}{n} Himno:} \> Jesús la imagen de Dios Padre. \\ \\

{\changefont{cmss}{bx}{n} Colecta:} \> Oh víctima inmolada. \\ \\

{\changefont{cmss}{bx}{n} Cruz:} \> antífona ``Te adoramos Cristo y 
te bendecimos...'' (se hace 3 veces). \\ \\

{\changefont{cmss}{bx}{n} Dolorosa:} \> Junto a la cruz, Madre de 
nuestro pueblo \\ \\

{\changefont{cmss}{bx}{n} Comunión:} \> Más cerca oh Dios,$\,$Perdón 
Señor misericordia,$\,$Salmo 50. \\ \\

{\changefont{cmss}{bx}{n} Ador. cruz:} \> Cruz de Cristo (Es la cruz), 
Coplas de soledad, Oh víctima inmolada, Salmo 41.\\ \\

\end{tabbing}

\vspace{-14mm}






%------------------------
\newpage
\setcounter{footnote}{0}
\thispagestyle{fancy}
\fancyhead{}
\fancyfoot{}
%\footskip=-1cm

\lhead{\changefont{cmss}{bx}{n} \small Revisión 2013}
\chead{\changefont{cmss}{bx}{n} \small Domingo de Pascua}
\rhead{\changefont{cmss}{bx}{n} \small Ciclo B}
\rfoot{\changefont{cmss}{bx}{n}\large\thepage}

\begin{center}
{\large\it El sepulcro vacío }
\end{center}

\vspace{1mm}


\begin{tabbing}

{\changefont{cmss}{bx}{n} Entrada:\ \ \ \ \ }\= Encendamos el cirio pascual, Esta es la luz de Cristo, Hoy la Iglesia victoriosa, \\
\> Que resuene por la tierra (si no se usa en otro momento).\\ \\

{\changefont{cmss}{bx}{n} Salmo:}\footnotemark[1] \> 117 ant. 2 ``Este es el día que hizo el Señor, aleluia,...'' (estr. 1, 7 u 8). \\ \\

{\changefont{cmss}{bx}{n} Aspersión:} \> Nueva vida, Esta es el agua pura, Tu agua bendita, Un solo Señor,\\ 
\>  Que resuene por la tierra. \\ \\

{\changefont{cmss}{bx}{n} Ofrendas:} \> Te ofrecemos oh Señor, Pan de vida y bebida de luz.\\ \\

{\changefont{cmss}{bx}{n} Comunión:} \> Gloria al Señor ha llegado la pascua, Que resuene por la tierra, \\
\> La gran noticia, Resucitó. \\ \\

{\changefont{cmss}{bx}{n} Post-com.:} \> Vive Jesús el Señor, Alabe todo el mundo (de Taizé).\\ \\

{\changefont{cmss}{bx}{n} Salida:} \> Alégrate María, Suenen campanas, Toda la tierra levante su voz.  \\  \\

\end{tabbing}

\vspace{-10mm}

\footnotetext[1]{Luego de la segunda lectura hay Secuencia pascual.}

%------------------------
\newpage
\setcounter{footnote}{0}
\thispagestyle{fancy}
\fancyhead{}
\fancyfoot{}
%\footskip=-1cm

\lhead{\changefont{cmss}{bx}{n} \small Revisión 2013}
\chead{\changefont{cmss}{bx}{n} \small Domingo II - Pascua}
\rhead{\changefont{cmss}{bx}{n} \small Ciclo B}
\rfoot{\changefont{cmss}{bx}{n}\large\thepage}

\vspace*{-10mm}

\begin{center}
{\large\it (Domingo de la Misericordia) La incredulidad de Tomás }
\end{center}

\vspace{-1mm}


\begin{tabbing}

{\changefont{cmss}{bx}{n} Entrada:\ \ \ \ \ }\= Resucitó, Hoy la Iglesia victoriosa, Encendamos el	 cirio pascual,  \\ 
 \> Esta es la luz de Cristo. \\ \\

{\changefont{cmss}{bx}{n} Salmo:} \> 117 ant. 1 ``Demos gracias al Señor porque es bueno...'' (estr. 1, 7 u 8). \\ 
\> (antífona de reemplazo: Sal 117 ant. 2 ``Este es el día que hizo el Señor...'') \\ \\

{\changefont{cmss}{bx}{n} Ofrendas:} \> Señor te ofrecemos\footnotemark[1], Pan de vida y bebida de luz, Te presentamos. \\ \\

{\changefont{cmss}{bx}{n} Comunión:} \> No hay mayor amor\footnotemark[2], Resucitó, Gloria al Señor ha llagado la pascua, \\
\> La gran noticia, Que resuene por la tierra. \\ \\

{\changefont{cmss}{bx}{n} Post-com.:} \> Tan cerca de mí\footnotemark[3], Adoremos a Dios (estr. 1)\footnotemark[3], Más cerca oh Dios (estr. 2)\\ 
\> Vive Jesús, Alabe todo el mundo, Gloria al Señor ha llagado la pascua. \\ \\

{\changefont{cmss}{bx}{n} Salida:} \> Toda la tierra levante su voz, Hoy la Iglesia victoriosa, Alégrate María, \\
\> Cantad a María, Suenen campanas. \\  \\

\end{tabbing}

\vspace{-12mm}

\footnotetext[1]{Es perfecto para la Divina Misericordia: ``?`quién podrá cantar tus misericordias...?''.  }
\footnotetext[2]{Cfr. la estrofa 3 con Jn 20,25-27, la estrofa 4 con Jn 20,19 y la estrofa 5 con Jn 20,21-23.}
\footnotetext[3]{Cfr. \textbf{Tan cerca de mí} con Jn 20,27-28 y \textbf{Adoremos a Dios} con Jn 20,28. }

%------------------------
\newpage
\setcounter{footnote}{0}
\thispagestyle{fancy}
\fancyhead{}
\fancyfoot{}
%\footskip=-1cm

\lhead{\changefont{cmss}{bx}{n} \small Revisión 2013}
\chead{\changefont{cmss}{bx}{n} \small Domingo III - Pascua}
\rhead{\changefont{cmss}{bx}{n} \small Ciclo B}
\rfoot{\changefont{cmss}{bx}{n}\large\thepage}

\vspace*{-10mm}

\begin{center}
{\large\it La aparición de Jesús a los Apóstoles }
\end{center}

\vspace{-1mm}


\begin{tabbing}

{\changefont{cmss}{bx}{n} Entrada:\ \ \ \ \ }\= Encendamos el cirio pascual, Esta es la luz de Cristo, Pueblo de Dios,  \\ 
 \> Vienen con alegría. \\ \\

{\changefont{cmss}{bx}{n} Salmo:} \> 65 ant. 1 ``Todo el mundo cante la gloria... '' (estr. 1, 5 o 7).\\
\>  (antífona de reemplazo: Sal 26 ant. 2 ``El Señor es mi luz , mi salvación...'') \\ \\

{\changefont{cmss}{bx}{n} Ofrendas:} \> Pan de vida y bebida de luz, Te ofrecemos oh Señor, Te presentamos.\\ \\

{\changefont{cmss}{bx}{n} Comunión:} \> Quedate con nosotros\footnotemark[1], La gran noticia, Resucitó, Vayamos a la mesa.  \\ \\

{\changefont{cmss}{bx}{n} Post-com.:} \> Vive Jesús el Señor, El Señor es mi fortaleza, Cuántas gracias te debemos, \\
\> Lauda Jerusalem. \\ \\ 

{\changefont{cmss}{bx}{n} Salida:} \> Suenen campanas, Toda la tierra levante su voz, Que resuene por la tierra, \\
\> Alégrate María, Cantad a María. \\ \\

\end{tabbing}

\vspace{-10mm}

\footnotetext[1]{Porque el evangelio del día es la continuación del relato de Emaús (Lc 24,13-35).  }

%------------------------
\newpage
\setcounter{footnote}{0}
\thispagestyle{fancy}
\fancyhead{}
\fancyfoot{}
%\footskip=-1cm

\lhead{\changefont{cmss}{bx}{n} \small Revisión 2013}
\chead{\changefont{cmss}{bx}{n} \small Domingo IV - Pascua}
\rhead{\changefont{cmss}{bx}{n} \small Ciclo B}
\rfoot{\changefont{cmss}{bx}{n}\large\thepage}

\vspace*{-10mm}

\begin{center}
{\large\it (Domingo del Buen Pastor) El buen Pastor y el asalariado }
\end{center}

\vspace{-1mm}


\begin{tabbing}

{\changefont{cmss}{bx}{n} Entrada\footnotemark[1]:\ \ \ \ \ }\= El Señor nos llama (estr. 2), Pueblo de Reyes (estr. 6), Pueblo de Dios,   \\ 
\> Encendamos el cirio pascual,  Esta es la luz de Cristo, Que resuene por la tierra. \\ \\ 

{\changefont{cmss}{bx}{n} Salmo\footnotemark[2]:} \> 117 ant. 1 ``Demos gracias al Señor...'' (estr. 3, 7, o 8). \\ 
\> (antífona de reemplazo: Sal 22 ant. 1 ``El Señor es mi Pastor...'') \\ \\

{\changefont{cmss}{bx}{n} Ofrendas:} \>  Señor te ofrecemos\footnotemark[3], Los frutos de la tierra\footnotemark[3],  \\
\>  Te ofrecemos Padre nuestro (vidala)\footnotemark[3], Te presentamos. \\ \\

{\changefont{cmss}{bx}{n} Comunión:} \> Yo soy el camino, Pueblo de Reyes, No hay mayor amor. \\ \\

{\changefont{cmss}{bx}{n} Post-com.:} \> Cantemos hermanos (estr. 1), Vive Jesús, Alabe todo el mundo (de Taizé). \\ \\

{\changefont{cmss}{bx}{n} Salida:} \> Cantemos hermanos (todo), Suenen campanas, Alégrate María, \\
\> Toda la tierra levante su voz, Vayan todos por el mundo (estr. 2). \\  \\

\end{tabbing}

\vspace{-14mm}

\footnotetext[1]{Cfr. todos estos cantos con las lecturas del día, especialmente: Sal 117,28-29 (el pueblo de Dios) \\ y Jn 10,16 (la voz del Pastor).}
\footnotetext[2]{Recordar que es conveniente que el Aleluia sea el de \emph{Yo soy el Maestro y el Pastor} de Néstor Gallego.}
\footnotetext[3]{Cfr. \emph{Señor te ofrecemos} con Jn 10,11 y Jn 10,17-18, y los demás cantos con las lecturas Hech 4,8-12 y 1 Jn 3, 1-2. }

%------------------------
\newpage
\setcounter{footnote}{0}
\thispagestyle{fancy}
\fancyhead{}
\fancyfoot{}
%\footskip=-1cm

\lhead{\changefont{cmss}{bx}{n} \small Revisión 2013}
\chead{\changefont{cmss}{bx}{n} \small Domingo V - Pascua}
\rhead{\changefont{cmss}{bx}{n} \small Ciclo B}
\rfoot{\changefont{cmss}{bx}{n}\large\thepage}

\vspace*{-10mm}

\begin{center}
{\large\it Jesús, la verdadera vid }
\end{center}

\vspace{-1mm}

\begin{tabbing}

{\changefont{cmss}{bx}{n} Entrada:\ \ \ \ \ }\= Un solo Señor (estr. 2)\footnotemark[1], Vine a alabar\footnotemark[1], Que resuene por la tierra,   \\ 
\> Toda la tierra levante su voz. \\ \\

{\changefont{cmss}{bx}{n} Salmo:} \> 65 ant. 1 ``Todo el mundo cante la gloria...'' (estr. 2 y 4). \\ 
\> (antífona de reemplazo: Sal 26 ant. 1 ``Cantaré y celebraré al Señor'') \\ \\

{\changefont{cmss}{bx}{n} Ofrendas:} \> Recibe oh Dios eterno, Te presentamos, Te ofrecemos oh Señor. \\ \\

{\changefont{cmss}{bx}{n} Comunión:} \> Yo soy el camino, Como Cristo nos amó\footnotemark[2], Cuerpo y Sangre de Jesús\footnotemark[1], \\
\> Un mandamiento nuevo\footnotemark[3]. \\ \\

{\changefont{cmss}{bx}{n} Post-com.:} \> Cantemos hermanos (estr. 1), Vive Jesús, Alabe todo el mundo (de Taizé). \\ \\

{\changefont{cmss}{bx}{n} Salida:} \> Cantemos hermanos (todo), Suenen campanas, Alégrate María, Cantad a María. \\  \\

\end{tabbing}

\vspace{-14mm}

\footnotetext[1]{Cfr. con las lecturas del día, especialmente, Sal 22,27 (\emph{Vine a alabar}), Jn 15,4 y Jn 15,8.}
\footnotetext[2]{Cfr. la afirmación ``nada puede separarnos de su amor'' de este canto con 1 Jn 3,24. }
\footnotetext[3]{Cfr. con 1 Jn 3,18 y 1 Jn 3,23-24. }

%------------------------
\newpage
\setcounter{footnote}{0}
\thispagestyle{fancy}
\fancyhead{}
\fancyfoot{}
%\footskip=-1cm

\lhead{\changefont{cmss}{bx}{n} \small Revisión 2013}
\chead{\changefont{cmss}{bx}{n} \small Domingo VI - Pascua}
\rhead{\changefont{cmss}{bx}{n} \small Ciclo B}
\rfoot{\changefont{cmss}{bx}{n}\large\thepage}

\vspace*{-10mm}

\begin{center}
{\large\it El mandamiento del amor }
\end{center}

\vspace{-1mm}

\begin{tabbing}

{\changefont{cmss}{bx}{n} Entrada:\ \ \ \ \ }\= Pueblo de Dios (estr. 1)\footnotemark[1], Un solo Señor\footnotemark[1], Que resuene por la tierra,  \\ 
\>  Vine a alabar. \\ \\

{\changefont{cmss}{bx}{n} Salmo:} \> 97 ant. 2 ``El Señor ha triunfado, aleluia...'' (estr. 1, 2 o 3).\\
\> (antífona de reemplazo: Sal 95 ``Cantemos al Señor un canto nuevo'' - Bevilacqua) \\ \\ 

{\changefont{cmss}{bx}{n} Ofrendas:} \> Pan de vida y bebida de luz, Los frutos de la tierra, Toma Señor nuestra vida\footnotemark[2], \\
\> Te ofrecemos oh Señor, Te presentamos. \\ \\

{\changefont{cmss}{bx}{n} Comunión:} \> No hay mayor amor\footnotemark[3], La canción de la Alianza\footnotemark[3], Dios me dio a mi hermano,  \\
\> Ven hermano, Un mandamiento nuevo, Yo soy el camino. \\ \\

{\changefont{cmss}{bx}{n} Post-com.:} \> Ubi caritas (de Taizé), Vive Jesús, Tu fidelidad, Alabe todo el mundo (de Taizé). \\ \\

{\changefont{cmss}{bx}{n} Salida:} \> Cantad a María, Alégrate María, Que resuene por la tierra, \\
\> Cantemos hermanos, Mi camino eres Tú. \\  \\

\end{tabbing}

\vspace{-10mm}

\footnotetext[1]{Cfr. \emph{Pueblo de Dios} con Sal 97,1 y Jn 15,11.16, y \emph{Un solo Señor} con la 1$^\circ$ lectura (Hech 10,34-35.47-48).}
\footnotetext[2]{Cfr. la 3$^\circ$ estrofa con Jn 15,9.}
\footnotetext[3]{Cfr. estribillo de \emph{No hay mayor amor} con Jn 15,13, y \emph{La canción de la Alianza} con la 2$^\circ$ lectura (1 Jn 4,7-10). }

%------------------------
\newpage
\setcounter{footnote}{0}
\thispagestyle{fancy}
\fancyhead{}
\fancyfoot{}
%\footskip=-1cm

\lhead{\changefont{cmss}{bx}{n} \small Revisión 2013}
\chead{\changefont{cmss}{bx}{n} \small Ascensión del Señor}
\rhead{\changefont{cmss}{bx}{n} \small Ciclo B}
\rfoot{\changefont{cmss}{bx}{n}\large\thepage}

\vspace*{-15mm}

\begin{center}
%{\large\it \'Ultimas instrucciones de Jesús y la ascensión }
\end{center}

\vspace{-6mm}

\begin{tabbing}

{\changefont{cmss}{bx}{n} Entrada:\ \ \ \ \ }\= Suenen cantos de alegría\footnotemark[1], Un solo Señor (estr. 1,3)\footnotemark[1], Vienen con alegría,  \\ 
\>  Iglesia peregrina. \\ \\

{\changefont{cmss}{bx}{n} Salmo\footnotemark[2]:} \> 46 ant. 1 ``Cantemos al Señor, nuestro Dios, aleluia...'' (estr. 1, 3 o 4).\\
\> (antífona de reemplazo: Sal 147 ant. 1 ``Glorifica al Señor Jerusalén...'') \\ \\ 

{\changefont{cmss}{bx}{n} Ofrendas:} \> Nuestros dones\footnotemark[3], Bendeciré al Señor\footnotemark[3], Al altar del Señor, Te presentamos,  \\
\> Te ofrecemos oh Señor, Bendito seas. \\ \\

{\changefont{cmss}{bx}{n} Comunión:} \> Jesucristo danos de este pan\footnotemark[4], Oh buen Jesús\footnotemark[4], Bendeciré al Señor, \\
\> Yo soy el camino. \\ \\

{\changefont{cmss}{bx}{n} Post-com.:} \> Adoremos a Dios, Vive Jesús, Alabe todo el mundo (de Taizé). \\ \\

{\changefont{cmss}{bx}{n} Salida:} \> Suenen cantos de alegría\footnotemark[1], Canción del testigo\footnotemark[5], Vayan todos por el mundo\footnotemark[5],\\
\> Anunciaremos tu Reino, Cantemos hermanos. \\  \\

\end{tabbing}

\vspace{-12mm}

\footnotetext[1]{Cfr. \emph{Suenen cantos de alegría} con Hech 1,9-11, y \emph{Un solo Señor} con Ef 4,4-6 (2$^\circ$ lectura).}
\footnotetext[2]{Se puede cantar el Aleluia del Padre Catena con la antífona ``Vayan por el mundo, aununcien mi Reino...''. }
\footnotetext[3]{Cfr. 3$^\circ$ estrofa de \emph{Nuestros dones} y \emph{Bendeciré al Señor} (Sal 33) con Sal 46,2. }
\footnotetext[4]{Cfr. \emph{Jesucristo danos de este pan} con Hech 1,11 y \emph{Oh buen Jesús} con Heb. 10,12 (ant. comunión). }
\footnotetext[5]{Cfr. \emph{Canción del testigo} con Hech 1,8, y \emph{Vayan todos por el mundo} con Mc 16,15.}

%------------------------
\newpage
\setcounter{footnote}{0}
\thispagestyle{fancy}
\fancyhead{}
\fancyfoot{}
%\footskip=-1cm

\lhead{\changefont{cmss}{bx}{n} \small Revisión 2013}
\chead{\changefont{cmss}{bx}{n} \small Domingo de Pentecostés}
\rhead{\changefont{cmss}{bx}{n} \small Ciclo B}
\rfoot{\changefont{cmss}{bx}{n}\large\thepage}

\vspace{-1mm}

\begin{center}
{\large\it Apariciones de Jesús a los discípulos  }
\end{center}

\vspace{-1mm}


\begin{tabbing}

{\changefont{cmss}{bx}{n} Entrada:\ \ \ \ \ }\= Hoy tu Espíritu Señor\footnotemark[1], Espíritu divino, Pueblo de Reyes\footnotemark[2], Vine a alabar. \\ \\

{\changefont{cmss}{bx}{n} Salmo:}\footnotemark[3] \> 103 (estr. 1, 7 o 13) con música ant. del Sal 144 ``Envía tu Espíritu, Señor, ...''. \\ \\

{\changefont{cmss}{bx}{n} Ofrendas:} \> Ven de lo alto, Espíritu divino, Una espiga\footnotemark[4], Coplas de Yaraví, Te presentamos, \\
\> Bendito seas.\\ \\

{\changefont{cmss}{bx}{n} Comunión:} \> Soplo de Dios viviente, Envíanos Padre\footnotemark[5], Ven Espíritu Santo, Maranathá\footnotemark[5]. \\ \\

{\changefont{cmss}{bx}{n} Post-com.:} \> Ven oh Santo Espíritu\footnotemark[1] (de Taizé), Espíritu de Dios.\\ \\

{\changefont{cmss}{bx}{n} Salida:} \> Madre de nuestro pueblo\footnotemark[6] (estr.10), En medio de los pueblos\footnotemark[1], \\
\> Canción del misionero.  \\  \\

\end{tabbing}

\vspace{-15mm}

\footnotetext[1]{Cfr. con la la 1$^\circ$ lectura (Hech 2,1-11).}
\footnotetext[2]{Cfr. con Ex 19,6 que se lee como 2$^\circ$ lectura en la Vigilia de Pentecostés.}
\footnotetext[3]{Luego de la segunda lectura hay Secuencia de Pentecostés+Aleluia.}
\footnotetext[4]{Porque Pentecostés corresponde a la fiesta hebrea de las \emph{Siete Semanas}, o de las primicias (Lev 23,16).}
\footnotetext[5]{Cfr. \emph{Envíanos Padre} con Jn 20,22 y Hech 2,1-4, y \emph{Maranathá} con Hech 2,1-11 y 1 Cor 12,3-7.}
\footnotetext[6]{Cfr. con Hech 2,12-14.}

%------------------------
\newpage
\setcounter{footnote}{0}
\thispagestyle{fancy}
\fancyhead{}
\fancyfoot{}
%\footskip=-1cm

\lhead{\changefont{cmss}{bx}{n} \small Revisión 2013}
\chead{\changefont{cmss}{bx}{n} \small Domingo de Santísima Trinidad}
\rhead{\changefont{cmss}{bx}{n} \small Ciclo B}
\rfoot{\changefont{cmss}{bx}{n}\large\thepage}

\vspace{-1mm}

\begin{center}
{\large\it La misión universal de los Apóstoles}
\end{center}

\vspace{-1mm}


\begin{tabbing}

{\changefont{cmss}{bx}{n} Entrada:\ \ \ \ \ }\= Un solo Señor (estr. 1), El Señor nos llama (estr. 3), Pueblo de Dios,  \\ 
\> Vuélvenos tu rostro\footnotemark[1], Vine a alabar. \\ \\

{\changefont{cmss}{bx}{n} Salmo\footnotemark[2]:} \> 32 ant. 1 ``Cantemos todos al Señor...'' (estr. 2, 7 u 8). \\ 
\> (antífona de reemplazo: Sal 144 ant. 1 ``Te alabamos Señor...'') \\ \\ 

{\changefont{cmss}{bx}{n} Ofrendas:} \> Padre nuestro recibid\footnotemark[3], Ven de lo alto, Una espiga, Te presentamos, Bendito seas.\\ \\

{\changefont{cmss}{bx}{n} Comunión:} \> Cuerpo y Sangre de Jesús, Es mi Padre, Escondido, Yo soy el pan de vida,\\
\> Vayamos a la mesa, Este es mi Cuerpo. \\ \\

{\changefont{cmss}{bx}{n} Post-com.:} \> Adoremos a Dios (estr. 1), Tu fidelidad, Alabe todo el mundo (de Taizé).\\ \\

{\changefont{cmss}{bx}{n} Salida:} \> Mi camino eres tú\footnotemark[3], Vayan todos por el mundo\footnotemark[4], Canción del testigo, \\
\> Canción del misionero.  \\  \\

\end{tabbing}

\vspace{-15mm}

\footnotetext[1]{Se refiere a la Santísima Trinidad, pero muchas veces se lo asocia como canto de cuaresma.}
\footnotetext[2]{Se puede cantar el Aleluia del P. Catena con la antífona ``Dios es nuestro Padre, Jesús nuestro hermano...''.}
\footnotetext[3]{Porque glorifica a la Santísima Trinidad.}
\footnotetext[4]{Cfr. con el evangelio del día (Mt 28,16-20).}

%------------------------
\newpage
\setcounter{footnote}{0}
\thispagestyle{fancy}
\fancyhead{}
\fancyfoot{}
%\footskip=-1cm

\lhead{\changefont{cmss}{bx}{n} \small Revisión 2013}
\chead{\changefont{cmss}{bx}{n} \small Domingo del Santísimo Cuerpo y Sangre de Cristo}
\rhead{\changefont{cmss}{bx}{n} \small Ciclo B}
\rfoot{\changefont{cmss}{bx}{n}\large\thepage}

\vspace{-5mm}

\begin{center}
{\large\it La institución de la Eucaristía}
\end{center}

\vspace{-5mm}


\begin{tabbing}

{\changefont{cmss}{bx}{n} Entrada:\ \ \ \ \ }\= El Señor nos llama (estr. 2), Iglesia peregrina, Somos la familia de Jesús. \\ \\

{\changefont{cmss}{bx}{n} Salmo:} \> 115 ant. 1 ``?`Con qué pagaré al Señor...?'' (estr. 1 y 2). \\ 
\> (antífona de reemplazo: Sal 147 ant. 1 ``Glorifica al Señor, Jerusalén...'') \\ \\ 

{\changefont{cmss}{bx}{n} Ofrendas:} \> Los frutos de la tierra, Recibe oh Dios eterno, Padre nuestro recibid, \\
\> Te presentamos, Bendito seas.\\ \\

{\changefont{cmss}{bx}{n} Comunión:} \> Panis angelicus, Cuerpo y Sangre de Jesús, En memoria tuya, Es mi Padre\footnotemark[1],\\
\> Yo soy el pan de vida\footnotemark[1], Escondido, Este es mi cuerpo\footnotemark[1],En la postrera cena\footnotemark[1], \\
\>  Vayamos a la mesa\footnotemark[1]. \\ \\

\vspace{-1mm}

{\changefont{cmss}{bx}{n} Post-com.:} \> Nuestro maná, Adoremos a Dios, Tantum Ergo (Tan sublime Sacramento), \\
\> Alabado sea el Santísimo, Te adoramos hostia divina.\\ \\

\vspace{-1mm}

{\changefont{cmss}{bx}{n} Salida:} \> Canción del testigo\footnotemark[2], En medio de los pueblos, Mi camino eres tú,   \\
\> Anunciaremos tu Reino, Vayan todos por el mundo.  \\  \\

\end{tabbing}

\vspace{-17mm}

\footnotetext[1]{Cfr. con la antífona del Aleluia (Jn 6,51).}
\footnotetext[2]{Porque la brasa del ángel que rozó los labios de Isaías pre-figura la Eucaristía.}


%------------------------
\newpage
\setcounter{footnote}{0}
\thispagestyle{fancy}
\fancyhead{}
\fancyfoot{}
%\footskip=-1cm

\lhead{\changefont{cmss}{bx}{n} \small Revisión 2013}
\chead{\changefont{cmss}{bx}{n} \small Domingo II - Durante el año}
\rhead{\changefont{cmss}{bx}{n} \small Ciclo B}
\rfoot{\changefont{cmss}{bx}{n}\large\thepage}

\vspace{-1mm}

\begin{center}
{\large\it Los primeros discípulos de Jesús }
\end{center}

\vspace{-1mm}

\begin{tabbing}

{\changefont{cmss}{bx}{n} Entrada:\ \ \ \ \ }\= Callemos hermanos\footnotemark[1], El Señor nos llama, Vine a alabar, Un solo Señor (est. 3).  \\ \\

{\changefont{cmss}{bx}{n} Salmo:} \> 26 ant. 2 ``El Señor es mi luz, mi salvación...'' (estr. 1 y 7).\\ \\

{\changefont{cmss}{bx}{n} Ofrendas:} \> Al altar del Señor\footnotemark[2], Comienza el sacrificio\footnotemark[2], Te ofrecemos Padre nuestro (vidala),  \\
\> Te presentamos, Bendito seas. \\ \\

{\changefont{cmss}{bx}{n} Comunión:} \> Quedate con nosotros, Pescador de hombres, Yo soy el camino, Bendeciré al Señor.  \\ \\

{\changefont{cmss}{bx}{n} Post-com.:} \> Mirarte sólo a ti, El Señor es mi fortaleza (de Taizé), Cuántas gracias te debemos, \\
\> Nada te turbe (de Taizé). \\ \\

{\changefont{cmss}{bx}{n} Salida:} \> Canción del testigo\footnotemark[4], Mi camino eres tú, Soy peregrino\footnotemark[4], Santa María del camino.\\  \\

\end{tabbing}

\vspace{-12mm}

\footnotetext[1]{Cfr. la 1$^\circ$ estrofa con 1 Sam 3,9-10. Este canto es apto para abrir la liturgia de la Palabra, y se lo puede adaptar como canto de entrada en la celebración eucarística. }
\footnotetext[2]{Cfr. con la antífona de comunión del día (Sal 22,5). }
\footnotetext[4]{Cfr. la 3$^\circ$ estrofa de \emph{Canción del testigo} y la 2$^\circ$ estrofa de \emph{Soy peregrino} con la persona de Juan el Bautista.}

%------------------------
\newpage
\setcounter{footnote}{0}
\thispagestyle{fancy}
\fancyhead{}
\fancyfoot{}
%\footskip=-1cm

\lhead{\changefont{cmss}{bx}{n} \small Revisión 2013}
\chead{\changefont{cmss}{bx}{n} \small Domingo III - Durante el año}
\rhead{\changefont{cmss}{bx}{n} \small Ciclo B}
\rfoot{\changefont{cmss}{bx}{n}\large\thepage}

\vspace{-1mm}

\begin{center}
{\large\it Comienzo de la predicación de Jesús }
\end{center}

\vspace{-1mm}

\begin{tabbing}

{\changefont{cmss}{bx}{n} Entrada:\ \ \ \ \ }\= Pueblo de Dios\footnotemark[1], Vienen con alegría\footnotemark[1],  El Señor nos llama, Vine a alabar.  \\ \\

{\changefont{cmss}{bx}{n} Salmo:} \> 24 ant. 2 ``Muéstrame, Señor, tu camino ...'' (estr. 2, 3 o 4).\\ 
\> (antífona de reemplazo: Sal 84 ant. 1 ``Señor, revélanos tu amor,...'') \\ \\ 

{\changefont{cmss}{bx}{n} Ofrendas:} \> Recibe oh Dios el Pan\footnotemark[2], Al altar nos acercamos\footnotemark[2],   \\
\> Te ofrecemos Padre nuestro (vidala)\footnotemark[2], Bendito seas. \\ \\

{\changefont{cmss}{bx}{n} Comunión:} \> Pescador de hombres\footnotemark[3], Yo soy el camino\footnotemark[3], Jesús te seguiré\footnotemark[3], Bendeciré al Señor.  \\ \\

{\changefont{cmss}{bx}{n} Post-com.:} \> Mirarte sólo a ti\footnotemark[3], Vaso nuevo (El alfarero), Cuántas gracias te debemos. \\ \\

{\changefont{cmss}{bx}{n} Salida:} \> Vayan todos por el mundo\footnotemark[4], Soy peregrino, Anunciaremos tu Reino.\\  \\

\end{tabbing}

\vspace{-12mm}

\footnotetext[1]{Cfr. \emph{Vienen con alegría} (est. 1) con Jon 3,4 y Mc 1,14; cfr. \emph{Pueblo de Dios} (estribillo y est.1) con ant. de entrada (Sal 95,1.6) y Mc 1,14. }
\footnotetext[2]{Cfr. estos cantos con Jon 3,1-5.10. }
\footnotetext[3]{Cfr. \emph{Pescador de hombres} con Mc 1,17; cfr. \emph{Yo soy el camino} (est. 1) con antífona de comunión Jn 8,12; cfr. \emph{Jesús te seguiré} con Mc 1,14-20; cfr. \emph{Mirarte sólo a ti} con ant. de comunión (Sal 33,6). }
\footnotetext[4]{Cfr. \emph{Vayan todos por el mundo} (est. 2) con Mc 1,14-20 cuando dice ``son los amigos que quise elegir.}

%------------------------
\newpage
\setcounter{footnote}{0}
\thispagestyle{fancy}
\fancyhead{}
\fancyfoot{}
%\footskip=-1cm

\lhead{\changefont{cmss}{bx}{n} \small Revisión 2013}
\chead{\changefont{cmss}{bx}{n} \small Domingo IV - Durante el año}
\rhead{\changefont{cmss}{bx}{n} \small Ciclo B}
\rfoot{\changefont{cmss}{bx}{n}\large\thepage}

\vspace{-2mm}

\begin{center}
{\large\it Enseñanza de Jesús en la sinagoga de Cafarnaún}
\end{center}

\vspace{-2mm}

\begin{tabbing}

{\changefont{cmss}{bx}{n} Entrada:\ \ \ \ \ }\= El sermón de la montaña\footnotemark[1], Las Bienaventuranzas (Brilló la luz)\footnotemark[1], Pueblo de Dios\footnotemark[1],\\
\> Vine a alabar, Vienen con alegría.  \\ \\

{\changefont{cmss}{bx}{n} Salmo:} \> 94 ant. 1 ``Adoremos al Señor ...'' (estr. 1, 3 o 4).\\ \\

{\changefont{cmss}{bx}{n} Ofrendas:} \> Te ofrecemos Padre nuestro (vidala), Toma Señor nuestra vida,    \\
\> Padre nuestro recibid, Recibe oh Dios eterno, Te presentamos. \\ \\

{\changefont{cmss}{bx}{n} Comunión:} \> Jesús te seguiré\footnotemark[1], Vayamos a la mesa\footnotemark[1], El Señor de Galilea, Pescador de hombres.  \\ \\

{\changefont{cmss}{bx}{n} Post-com.:} \> Adoremos al Señor, Mirarte sólo a ti, Tu fidelidad. \\ \\

{\changefont{cmss}{bx}{n} Salida:} \> Jesús te seguiré\footnotemark[1], Madre de nuestro pueblo\footnotemark[2], Anunciaremos tu Reino, \\ 
\> Santa María del camino.\\  \\

\end{tabbing}

\vspace{-15mm}

\footnotetext[1]{Cfr. \emph{El sermón de la montaña} (est. 2 y 4) y \emph{Las Bienaventuranzas (Brilló la luz)} (est. 1 y 4) con ant. de comunión del día (Mt 5,3.5); \emph{Pueblo de Dios} (estribillo) con Sal 94,1; cfr. \emph{Jesús te seguiré} (est. 1 y 3) y \emph{Vayamos a la mesa} (est. 2 y 3) con Mc 1,21-28. }
\footnotetext[2]{Cfr. \emph{Madre de nuestro pueblo} (est. 1 y 8) porque habla de Jn 2,1-11, que tiene puntos de contacto con Mc 1,21-28. Si este domingo está cerca del 2 de febrero, cantar estrofa 5.}

%------------------------
\newpage
\setcounter{footnote}{0}
\thispagestyle{fancy}
\fancyhead{}
\fancyfoot{}
%\footskip=-1cm

\lhead{\changefont{cmss}{bx}{n} \small Revisión 2013}
\chead{\changefont{cmss}{bx}{n} \small Domingo V - Durante el año}
\rhead{\changefont{cmss}{bx}{n} \small Ciclo B}
\rfoot{\changefont{cmss}{bx}{n}\large\thepage}

\vspace*{-10mm}

\begin{center}
{\large\it Diversas curaciones y la misión de Jesús}
\end{center}

\vspace{-3mm}

\begin{tabbing}

{\changefont{cmss}{bx}{n} Entrada:\ \ \ \ \ }\=  Vienen con alegría\footnotemark[1], El sermón de la montaña\footnotemark[1], Brilló la luz\footnotemark[1], Pueblo de Dios\footnotemark[1], \\
\> Juntos como hermanos (est. 2).  \\ \\

{\changefont{cmss}{bx}{n} Salmo:} \> 89 ant. 1 ``Nuestra vida, Señor pasa como un soplo ...'' (estr. 1 y 3 del salmo 146).\\ 
\> (antífona de reemplazo: Sal 144 ant. 1 ``Te alabamos Señor...'') \\ \\ 

{\changefont{cmss}{bx}{n} Ofrendas:} \> Este es nuestro pan\footnotemark[2], Señor te ofrecemos\footnotemark[2], Bendeciré al Señor\footnotemark[2], \\
\>  Te ofrecemos Padre nuestro (moderno), Te presentamos. \\ \\

{\changefont{cmss}{bx}{n} Comunión:} \> Queremos ser Señor, Vayamos a la mesa, Jesús te seguiré, El Señor de Galilea.  \\ \\

{\changefont{cmss}{bx}{n} Post-com.:} \> Tu fidelidad, Tan cerca de mí, Las misericordias del Señor (Taizé). \\ \\

{\changefont{cmss}{bx}{n} Salida:} \> Jesús te seguiré, Madre de nuestro pueblo\footnotemark[2], Canción del testigo, \\ 
\>  Anunciaremos tu Reino, Madre de los peregrinos.\\  \\

\end{tabbing}

\vspace{-15mm}

\footnotetext[1]{Cfr. \emph{Vienen con alegría} (est. 1) con la angustia y ansiedad de Job (Job 7,1-7); cfr. \emph{El sermón de la montaña} \mbox{(est. 1 y 2)} y \emph{Brilló la luz} (est. 1-3) con ant. de comunión del día (Mt 5,4.6); cfr. \emph{Pueblo de Dios} (estribillo) y \emph{Bendeciré al Señor} con ant. de comunión (Sal 106,8-9). }
\footnotetext[2]{Cfr. \emph{Este es nuestro pan} (est. 2) con aclamación al evangelio (Mt 8,17); cfr. \emph{Señor te ofrecemos} con salmo y ant. de comunión del día (Sal 146,1-6 y Sal 106,8-9); si este domingo está cerca del 2 de febrero, cantar \emph{Madre de nuestro pueblo} (est. 5). }


%------------------------
\newpage
\setcounter{footnote}{0}
\thispagestyle{fancy}
\fancyhead{}
\fancyfoot{}
%\footskip=-1cm

\lhead{\changefont{cmss}{bx}{n} \small Revisión 2013}
\chead{\changefont{cmss}{bx}{n} \small Domingo VI - Durante el año}
\rhead{\changefont{cmss}{bx}{n} \small Ciclo B}
\rfoot{\changefont{cmss}{bx}{n}\large\thepage}

\vspace*{-10mm}

\begin{center}
{\large\it Curación de un leproso}
\end{center}

\vspace{-3mm}

\begin{tabbing}

{\changefont{cmss}{bx}{n} Entrada:\ \ \ \ \ }\=  Pueblo de Dios\footnotemark[1], Me pongo en tus manos (estribillo)\footnotemark[1], El sermón de la montaña,\\
\> Brilló la luz (Las bienaventuranzas). \\ \\


{\changefont{cmss}{bx}{n} Salmo:} \> 31 ant. 2 ``Padre, tu perdonas mi culpa ...'' (estr. 1, 2 o 5).\\ 
\> (antífona de reemplazo: Sal 29 ant. 1 ``Te glorifico Señor...'') \\ \\ 

{\changefont{cmss}{bx}{n} Ofrendas:} \> Padre nuestro recibid\footnotemark[2],  Mira nuestra ofrenda, Recibe oh Dios el pan. \\ \\


{\changefont{cmss}{bx}{n} Comunión:} \> Jesús te seguiré\footnotemark[3], Oh buen Jesús\footnotemark[3], El Señor de Galilea, Vayamos a la mesa.  \\ \\

{\changefont{cmss}{bx}{n} Post-com.:} \> El Señor es mi fortaleza (Taizé)\footnotemark[3], Tan cerca de mí, \\
\> Las misericordias del Señor (Taizé). \\ \\

{\changefont{cmss}{bx}{n} Salida:} \> Jesús te seguiré\footnotemark[3], Mi camino eres tu, Quiero decir que sí, Santa María del camino. \\ \\

\end{tabbing}

\vspace{-13mm}

\footnotetext[1]{Cfr. \emph{Pueblo de Dios} (estribillo) Mc. 1,45; \emph{Me pongo en tus manos} se acostumbra cantar en cuaresma o semana santa. Sin embargo, su texto (estribillo) es muy correcto respecto de Mc. 1,40 y con ant. de entrada del día (Sal 30,3-4). }
\footnotetext[2]{Cfr. \emph{Padre nuestro recibid} (est. 3) con actitud del leproso (Mc. 1,40-45). }
\footnotetext[3]{Cfr. \emph{Jesús te seguiré} (est. 3) con Mc. 1,40-45; \emph{Oh buen Jesús} es un típico canto pentitencial en consonancia con Mc. 1,40-45; cfr. \emph{El Señor es mi fortaleza} con Mc. 1,40-45 y ant. de entrada (Sal 30,3-4).}



%------------------------
\newpage
\setcounter{footnote}{0}
\thispagestyle{fancy}
\fancyhead{}
\fancyfoot{}
%\footskip=-1cm

\lhead{\changefont{cmss}{bx}{n} \small Revisión 2013}
  \chead{\changefont{cmss}{bx}{n} \small Domingo VII - Durante el año}
\rhead{\changefont{cmss}{bx}{n} \small Ciclo B}
\rfoot{\changefont{cmss}{bx}{n}\large\thepage}

\vspace*{-10mm}

\begin{center}
{\large\it Curación de un paralítico}
\end{center}

\vspace{-3mm}

\begin{tabbing}

{\changefont{cmss}{bx}{n} Entrada:\ \ \ \ \ }\=  Caminaré\footnotemark[1], Pueblo de Dios\footnotemark[1], Vine a alabar\footnotemark[1], Vienen con alegría. \\ \\


{\changefont{cmss}{bx}{n} Salmo:} \> 24 ant. 1 ``A tí elevo mi alma, ...'' (estr. 2, 3 o 5). \\ \\ 

{\changefont{cmss}{bx}{n} Ofrendas:} \> Comienza el sacrificio,$\,$Te ofrecemos Padre nuestro (vidala),$\,$Recibe oh Dios el pan, \\
\> Bendito seas Señor. \\ \\


{\changefont{cmss}{bx}{n} Comunión:} \> Cuerpo y sangre de Jesús\footnotemark[2], Quédate con nosotros\footnotemark[2], El Señor de Galilea, \\
\> Vayamos a la mesa.  \\ \\

{\changefont{cmss}{bx}{n} Post-com.:} \> Cuántas gracias te debemos (P. Bevilacqua), El Señor es mi fortaleza (Taizé),  \\
\> Las misericordias del Señor (Taizé). \\ \\

{\changefont{cmss}{bx}{n} Salida:} \> Oh María, Jesús te seguiré, Mi camino eres tu, Santa María del camino. \\ \\

\end{tabbing}

\vspace{-13mm}

\footnotetext[1]{Cfr. \emph{Caminaré} con ant. de entrada (Sal 12,6); cfr. \emph{Pueblo de Dios} (estribillo) con ant. de comunión (Sal 9,2-3); cfr. \emph{Vine a alabar} (est. 1) con 2 Cor 1,22. }
\footnotetext[2]{Cfr. \emph{Cuerpo y sangre de Jesús} (est. 2) 2 Cor 1,21-22; cfr. \emph{Quédate con nosotros} (est. 1) con Mc 2,1-12.}





%------------------------
\newpage
\setcounter{footnote}{0}
\thispagestyle{fancy}
\fancyhead{}
\fancyfoot{}
%\footskip=-1cm

\lhead{\changefont{cmss}{bx}{n} \small Revisión 2013}
  \chead{\changefont{cmss}{bx}{n} \small Domingo VIII - Durante el año}
\rhead{\changefont{cmss}{bx}{n} \small Ciclo B}
\rfoot{\changefont{cmss}{bx}{n}\large\thepage}

\vspace*{-10mm}

\begin{center}
{\large\it Discusión sobre el ayuno}
\end{center}

\vspace{-3mm}

\begin{tabbing}

{\changefont{cmss}{bx}{n} Entrada:\ \ \ \ \ }\=  Pueblo de Reyes\footnotemark[1], Vienen con alegría, Iglesia Peregrina de Dios. \\ \\


{\changefont{cmss}{bx}{n} Salmo:} \> 102 ant. 2 ``El amor del Señor...'' (estr. 1, 2, 4, 5 o 6). \\ \\ 

{\changefont{cmss}{bx}{n} Ofrendas:} \> Los frutos de la tierra\footnotemark[2], Este es nuestro pan\footnotemark[2], Te ofrecemos Padre nuestro (vidala), \\
\> Te presentamos, Bendito seas Señor. \\ \\


{\changefont{cmss}{bx}{n} Comunión:} \> No hay mayor amor\footnotemark[3], Yo soy el pan de vida\footnotemark[3], Cuerpo y Sangre de Jesús, \\
\> Vayamos a la mesa.  \\ \\

{\changefont{cmss}{bx}{n} Post-com.:} \> Mirarte sólo a tí, Cuántas gracias te debemos (P. Bevilacqua), Nuestro Maná,  \\
\> Adoremos a Dios. \\ \\

{\changefont{cmss}{bx}{n} Salida:} \> En medio de los Pueblos, Mi camino eres tu, Santa María del camino. \\ \\

\end{tabbing}

\vspace{-13mm}

\footnotetext[1]{\emph{Pueblo de Reyes} tiene contacto (fuerte) con todas las lecturas del día. }
\footnotetext[2]{Cfr. \emph{Los frutos de la tierra} (est. 2) Mc 2,18-22; cfr. con \emph{Este es nuestro pan} (est. 1) con Mc 2,21 (comparar el ``género nuevo'' con la ``continuación de su santa creación''.}
\footnotetext[3]{Cfr. \emph{No hay mayor amor} (est 1-3) y \emph{Yo soy el pan de vida} (est. 1) con Mc 18,22. }



%------------------------
\newpage
\setcounter{footnote}{0}
\thispagestyle{fancy}
\fancyhead{}
\fancyfoot{}
%\footskip=-1cm

\lhead{\changefont{cmss}{bx}{n} \small Revisión 2013}
  \chead{\changefont{cmss}{bx}{n} \small Domingo IX - Durante el año}
\rhead{\changefont{cmss}{bx}{n} \small Ciclo B}
\rfoot{\changefont{cmss}{bx}{n}\large\thepage}

\vspace*{-10mm}

\begin{center}
{\large\it Discusión sobre el sábado}
\end{center}

\vspace{-3mm}

\begin{tabbing}

{\changefont{cmss}{bx}{n} Entrada:\ \ \ \ \ }\=  Brilló la luz\footnotemark[1], Pueblo de Reyes\footnotemark[1], Vine a alabar\footnotemark[1], Pueblo de Dios. \\ \\


{\changefont{cmss}{bx}{n} Salmo:} \> 18b ant. 1 ``Tu palabra, Señor, ...'' (estr. 1, 2, 5 o 6). \\ \\ 

{\changefont{cmss}{bx}{n} Ofrendas:} \> Pan de vida y bebida de luz\footnotemark[2], Padre nuestro recibid\footnotemark[2], Bendeciré al Señor, \\
\> Bendito seas Señor. \\ \\


{\changefont{cmss}{bx}{n} Comunión:} \> Brilló la luz\footnotemark[1], Yo soy el pan de vida\footnotemark[3], Es mi Padre\footnotemark[3], Yo soy el camino\footnotemark[3]. \\ \\


{\changefont{cmss}{bx}{n} Post-com.:} \> Alabe todo el mundo (Taizé), Adoremos a Dios, Mirarte sólo a tí,     \\
\> Si el mismo pan comimos, Cuántas gracias te debemos (P. Bevilacqua). \\ \\

{\changefont{cmss}{bx}{n} Salida:} \> Canción del testigo\footnotemark[3], En medio de los Pueblos, Mi camino eres tu, \\
\> Madre de los peregrinos, Santa María del camino. \\ \\

\end{tabbing}

\vspace{-13mm}

\footnotetext[1]{Cfr. \emph{Brilló la luz} (estribillo) y \emph{Vine a alabar} con 2 Cor 4,6; cfr. \emph{Pueblo de Reyes} (est 2-8) con Mc 2,23-3,6; }
\footnotetext[2]{Cfr. \emph{Pan de vida y bebida de luz} (est. 1) 2 Cor 4,6; \emph{Padre nuestro recibid} (est. 3) proclama a Jesús como ``Señor'' y glorifica su nombre (cfr. con Mc 2,28).}
\footnotetext[3]{Cantos eucarísticos que exaltan la presencia real de Cristo en el pan consagrado; cfr. \emph{Canción del testigo} \mbox{(est. 1 y 2)} con 2 Cor 4,6-11.}




%------------------------
\newpage
\setcounter{footnote}{0}
\thispagestyle{fancy}
\fancyhead{}
\fancyfoot{}
%\footskip=-1cm

\lhead{\changefont{cmss}{bx}{n} \small Revisión 2013}
\chead{\changefont{cmss}{bx}{n} \small Domingo X - Durante el año}
\rhead{\changefont{cmss}{bx}{n} \small Ciclo B}
\rfoot{\changefont{cmss}{bx}{n}\large\thepage}

\vspace*{-10mm}

\begin{center}
{\large\it La verdadera familia de Jesús}
\end{center}

\vspace{-3mm}

\begin{tabbing}

{\changefont{cmss}{bx}{n} Entrada:\ \ \ \ \ }\= Somos la familia de Jesús\footnotemark[1], Iglesia peregrina\footnotemark[1], El Señor nos llama, Caminaré. \\ \\ 

{\changefont{cmss}{bx}{n} Salmo:} \> 129 ant. 1 ``Yo pongo mi esperanza en ti, Señor,...'' (estr. 1, 2, 3 o 4).\\
\> (antífona de reemplazo: Sal 24 ant. 1 ``A ti elevo mi alma...'') \\ \\ 

{\changefont{cmss}{bx}{n} Ofrendas:} \> Recibe oh Dios eterno\footnotemark[2], Comienza el sacrificio\footnotemark[2], Una espiga, \\
\> Sé como el grano de trigo. \\ \\

{\changefont{cmss}{bx}{n} Comunión:} \> Creo en ti Señor (Más cerca oh Dios)\footnotemark[3], Jesucristo danos de este pan\footnotemark[3],  \\
\> Oh buen Jesús\footnotemark[3], Escondido, Queremos ser Señor. \\ \\

{\changefont{cmss}{bx}{n} Post-com.:} \> Mirarte sólo a tí\footnotemark[4], Al atardecer de la vida\footnotemark[4], Nada te turbe (de Taizé). \\ \\

{\changefont{cmss}{bx}{n} Salida:} \> Salve María\footnotemark[5], Madre de los peregrinos, Santa María del camino, \\
\> Mi camino eres Tú. \\  \\

\end{tabbing}

\vspace{-15mm}

\footnotetext[1]{Cfr. \emph{Somos la familia de Jesús} con Mc 3,31-35 e \emph{Iglesia peregrina} (estr. 6 y 7) con la 2$^\circ$ lectura (2 Cor 5,1).}
\footnotetext[2]{Cfr. \emph{Recibe oh Dios eterno} (estr. 1) con Mc 3,35 y \emph{Comienza el sacrificio} con 2 Cor 4,13. }
\footnotetext[3]{Cfr. \emph{Creo en tí Señor} y \emph{Oh buen Jesús} con 2 Cor 4,13, \emph{Jesucristo danos de este pan} (estr. 4) con Mc 3,20-35. }
\footnotetext[4]{Cfr. \emph{Mirarte sólo a tí} con 2 Cor 4,18 y \emph{Al atardecer de la vida} con Mc 3,28-30.}
\footnotetext[5]{Debido a que se refiere al relato del pecado original en Gen 3,9-15.}

%------------------------
\newpage
\setcounter{footnote}{0}
\thispagestyle{fancy}
\fancyhead{}
\fancyfoot{}
%\footskip=-1cm

\lhead{\changefont{cmss}{bx}{n} \small Revisión 2013}
\chead{\changefont{cmss}{bx}{n} \small Domingo XI - Durante el año}
\rhead{\changefont{cmss}{bx}{n} \small Ciclo B}
\rfoot{\changefont{cmss}{bx}{n}\large\thepage}

\vspace*{-10mm}

\begin{center}
{\large\it La parábola de la semilla que crece por sí sola y del grano de mostaza }
\end{center}

\vspace{-3mm}

\begin{tabbing}

{\changefont{cmss}{bx}{n} Entrada:\ \ \ \ \ }\= Iglesia peregrina\footnotemark[1], Juntos como hermanos (estr. 2), Vienen con alegría, \\
\> Un pueblo que camina. \\ \\

{\changefont{cmss}{bx}{n} Salmo:} \> 102 ant. 2 ``El amor del Señor permanece...'' (estr. 1, 6 u 8).\\ \\

{\changefont{cmss}{bx}{n} Ofrendas:} \> Coplas de Yaraví\footnotemark[2], Sé como el grano de trigo, Una espiga, Nuestros dones. \\ \\

{\changefont{cmss}{bx}{n} Comunión:} \> Queremos ser Señor, Coplas de Yaraví\footnotemark[2], La canción de la Alianza,\\
\> Pescador de hombres.\\ \\

{\changefont{cmss}{bx}{n} Post-com.:} \> Tu fidelidad\footnotemark[3], Vaso nuevo (El Alfarero), Al atardecer de la vida. \\ \\

{\changefont{cmss}{bx}{n} Salida:} \> Vayan todos por el mundo\footnotemark[4], Anunciaremos tu Reino, Mi camino eres tú, \\
\> Santa María del camino.  \\ \\

\end{tabbing}

\vspace{-15mm}

\footnotetext[1]{Cfr. \emph{Iglesia peregrina} con Mc 4,26-34, Sal 91,13-14 y 2 Cor 5,6-8. }
\footnotetext[2]{Cfr. \emph{Coplas de Yaraví} (estr. 4) con Ez 17,24 y (estr. 3) con Mc 4,26-27.}
\footnotetext[3]{Cfr. con Sal 91,3.}
\footnotetext[4]{Cfr. estr. 1 con Mc 4,26-34. }


%------------------------
\newpage
\setcounter{footnote}{0}
\thispagestyle{fancy}
\fancyhead{}
\fancyfoot{}
%\footskip=-1cm

\lhead{\changefont{cmss}{bx}{n} \small Revisión 2013}
\chead{\changefont{cmss}{bx}{n} \small Domingo XII - Durante el año}
\rhead{\changefont{cmss}{bx}{n} \small Ciclo B}
\rfoot{\changefont{cmss}{bx}{n}\large\thepage}

\vspace*{-10mm}

\begin{center}
{\large\it La tempestad calmada }
\end{center}

\vspace{-3mm}

\begin{tabbing}

{\changefont{cmss}{bx}{n} Entrada:\ \ \ \ \ }\= Iglesia peregrina de Dios\footnotemark[1], Caminaré, Pueblo de Dios, Juntos como hermanos\footnotemark[2].   \\ \\

{\changefont{cmss}{bx}{n} Salmo:} \> 117 ant. 1 ``!`Demos gracias al Señor...!'' (estr. 2, 6 o 7).\\
\> (antífona de reemplazo: Sal 102 ant. 2 ``El amor del Señor permanece...'') \\ \\ 

{\changefont{cmss}{bx}{n} Ofrendas:} \> Bendeciré al Señor, Señor te ofrecemos\footnotemark[3], Comienza el sacrificio, \\
\> Te presentamos, Bendito seas. \\  \\

{\changefont{cmss}{bx}{n} Comunión:} \>  Creo en ti Señor (Más cerca oh Dios)\footnotemark[4], El Señor de Galilea, Yo soy el camino, \\
\> Bendeciré al Señor. \\ \\

{\changefont{cmss}{bx}{n} Post-com.:} \> El Señor es mi fortaleza (de Taizé), Nada te turbe (de Taizé), Mirarte sólo a ti. \\ \\

{\changefont{cmss}{bx}{n} Salida:} \> En medio de los pueblos, Mi camino eres tu, Oh Santísima\footnotemark[5], Salve oh Reina. \\
\>  \\ \\

\end{tabbing}

\vspace{-15mm}

\footnotetext[1]{Cfr. la 3$^\circ$ estrofa con el evangelio del día (Mc 4,35-41).}
\footnotetext[2]{Se prefiere la estrofa 2.}
\footnotetext[3]{Este canto tiene el contexto del Sal 88 y la Última Cena, pero se adapta muy bien al evangelio (cfr. Mc 4,41).}
\footnotetext[4]{Cfr. la 4$^\circ$ estrofa con el evangelio del día (Mc 4,35-41) y la 1$^\circ$ estrofa con el contexto de Job 38,1.8-11.}
\footnotetext[5]{Este canto era originalmente un \emph{himno marinero siciliano}. Quizás la asamblea desee cantarlo en este día.}

%------------------------
\newpage
\setcounter{footnote}{0}
\thispagestyle{fancy}
\fancyhead{}
\fancyfoot{}
%\footskip=-1cm

\lhead{\changefont{cmss}{bx}{n} \small Revisión 2013}
\chead{\changefont{cmss}{bx}{n} \small Domingo XIII - Durante el año}
\rhead{\changefont{cmss}{bx}{n} \small Ciclo B}
\rfoot{\changefont{cmss}{bx}{n}\large\thepage}

\vspace*{-10mm}

\begin{center}
{\large\it La resurrección de la hija de Jairo }
\end{center}

\vspace{-3mm}

\begin{tabbing}

{\changefont{cmss}{bx}{n} Entrada:\ \ \ \ \ }\= Pueblo de Dios, Caminaré, Vine a alabar, Juntos como hermanos (estr. 2).   \\ \\

{\changefont{cmss}{bx}{n} Salmo:} \> 94 ant. 1 ``!`Adoremos al Señor ...!'' (estr. 1 y 2).\\
\> (antífona de reemplazo: Sal 29 ant. 1 ``Te glorifico, Señor,...'') \\ \\ 

{\changefont{cmss}{bx}{n} Ofrendas:} \> Recibe oh Dios el pan\footnotemark[1], Padre nuestros recibid\footnotemark[1], Entre tus manos,  \\
\> Te presentamos. \\  \\

{\changefont{cmss}{bx}{n} Comunión:} \>  Creo en ti Señor (Más cerca oh Dios)\footnotemark[2], Como Cristo nos amó\footnotemark[2], Yo soy el camino,  \\
\>  Escondido, Quédate con nosotros.  \\ \\

{\changefont{cmss}{bx}{n} Post-com.:} \> Sé tu cual la mujer\footnotemark[3], El Señor es mi fortaleza (de Taizé), Adoremos a Dios,\\
\> Alabe todo el mundo (de Taizé). \\ \\

{\changefont{cmss}{bx}{n} Salida:} \> Oh María\footnotemark[4], Oh Santísima\footnotemark[4], Mi camino eres tu. \\
\>  \\ \\

\end{tabbing}

\vspace{-15mm}

\footnotetext[1]{Cfr. \emph{Recibe oh Dios} (est. 3) con Mc 5,22, \emph{Padre nuestro recibid}  (est. 3) con Mc 5,33-34,  .}
\footnotetext[2]{Cfr. \emph{Más cerca oh Dios} (est. 2 y 3) con Mc 5,34 y Mc 5,36, y \emph{Como Cristo nos amó} con 2 Cor 8,9.}
\footnotetext[3]{Cfr. con Mc 5,27-30.}
\footnotetext[4]{Parece muy necesario pedir la intercesión de María, sobre todo ante situaciones de fatalidad.}

%------------------------
\newpage
\setcounter{footnote}{0}
\thispagestyle{fancy}
\fancyhead{}
\fancyfoot{}
%\footskip=-1cm

\lhead{\changefont{cmss}{bx}{n} \small Revisión 2013}
\chead{\changefont{cmss}{bx}{n} \small Domingo XIV - Durante el año}
\rhead{\changefont{cmss}{bx}{n} \small Ciclo B}
\rfoot{\changefont{cmss}{bx}{n}\large\thepage}

\vspace*{-10mm}

\begin{center}
{\large\it Visita de Jesús a Nazaret }
\end{center}

\vspace{-3mm}

\begin{tabbing}

{\changefont{cmss}{bx}{n} Entrada:\ \ \ \ \ }\= Somos la familia de Jesús\footnotemark[1], Juntos como hermanos (est. 2), Iglesia peregrina\footnotemark[2], \\
\> Pueblo de reyes (est. 3)\footnotemark[2], El sermón de la montaña (est. 1).   \\ \\

{\changefont{cmss}{bx}{n} Salmo:} \> 122 ant. 1 ``Levanto mis ojos hacia ti...'' (estr. 1, 2 o 3).\\
\> (antífona de reemplazo: Sal 24 ant. 1 ``A ti elevo mi alma...'') \\ \\ 

{\changefont{cmss}{bx}{n} Ofrendas:} \> Te ofrecemos Padre nuestro (vidala)\footnotemark[3], Pan de vida y bebida de luz, \\ 
\> Bendeciré al Señor\footnotemark[3], Te presentamos, Bendito seas. \\  \\

{\changefont{cmss}{bx}{n} Comunión:} \>   Como Cristo nos amó\footnotemark[4], Creo en ti Señor (Más cerca oh Dios)\footnotemark[4], Vayamos a la mesa,   \\
\> Señor Jesús (sin est. 4), Yo soy el camino.  \\ \\

{\changefont{cmss}{bx}{n} Post-com.:} \> Mirarte sólo a ti, El Señor es mi fortaleza (de Taizé), Tu fidelidad.\\ \\

{\changefont{cmss}{bx}{n} Salida:} \> Madre de los peregrinos, Madre de nuestro pueblo (est. 3 y 7), Mi camino eres tu. \\
\>  \\ \\

\end{tabbing}

\vspace{-15mm}

\footnotetext[1]{Este canto confronta excelentemente el evangelio del día (Mc 6,3-4) con Mc 3,33-35.}
\footnotetext[2]{Cfr. \emph{Iglesia peregrina} (est. 4) y \emph{Pueblo de reyes} (est. 3) con Ez 2,5.}
\footnotetext[3]{Cfr. \emph{Te ofrecemos Padre nuestro} (est. 1) con 2 Cor 12,7 y \emph{Bendeciré al Señor} con la ant. de comunión (Sal 33,9).}
\footnotetext[4]{Cfr. \emph{Como Cristo nos amó} (est. 2) con Mc 6,3 y \emph{Creo en ti Señor}, que suplica ``aumenta mi fe'', con Mc 6,6.}






%------------------------
\newpage
\setcounter{footnote}{0}
\thispagestyle{fancy}
\fancyhead{}
\fancyfoot{}
%\footskip=-1cm

\lhead{\changefont{cmss}{bx}{n} \small Revisión 2013}
\chead{\changefont{cmss}{bx}{n} \small Domingo XV - Durante el año}
\rhead{\changefont{cmss}{bx}{n} \small Ciclo B}
\rfoot{\changefont{cmss}{bx}{n}\large\thepage}

\vspace*{-10mm}

\begin{center}
{\large\it Misión de los Doce }
\end{center}

\vspace{-3mm}

\begin{tabbing}

{\changefont{cmss}{bx}{n} Entrada:\ \ \ \ \ }\= Un pueblo que camina\footnotemark[1], Vienen con alegría, Iglesia peregrina\footnotemark[1], \\
\> Pueblo de Dios peregrino (de Matías Sagreras).   \\ \\

{\changefont{cmss}{bx}{n} Salmo:} \> 84 ant. 1 ``Señor, revélanos tu amor, ...'' (estr. 4, 5 o 6).\\ \\

{\changefont{cmss}{bx}{n} Ofrendas:} \> Te ofrecemos Padre nuestro (vidala)\footnotemark[2], Ofrenda de amor\footnotemark[2], Te presentamos, \\
\> Bendito seas. \\  \\

{\changefont{cmss}{bx}{n} Comunión:} \>  Mensajero de la paz\footnotemark[3], Cuerpo y sangre de Jesús\footnotemark[3], Queremos ser Señor\footnotemark[3], \\
\> Vayamos a la mesa.\\ \\

{\changefont{cmss}{bx}{n} Post-com.:} \> Cantemos hermanos (est. 1-2), Tan cerca de mí, Aleluia Cristo vino con su paz.\\ \\

{\changefont{cmss}{bx}{n} Salida:} \> Canción del misionero\footnotemark[3], Soy peregrino, En medio de los pueblos, \\
\> Santa María del camino. \\ \\

\end{tabbing}

\vspace{-15mm}

\footnotetext[1]{Cfr. \emph{Un pueblo que camina} con Mc 6,7-13; cfr. \emph{Iglesia peregrina} (est. 5) con Ef 1,10.}
\footnotetext[2]{Cfr. \emph{Te ofrecemos Padre nuestro} (est. 2) con Salmo 84,9-14; cfr. \emph{Ofrenda de amor} (est. 2) con Mc 6,7-13.}
\footnotetext[3]{Cfr. \emph{Mensajero de la paz} con Mc 6,7-13; cfr. \emph{Cuerpo y sangre de Jesús} (est. 2) con Ef 1,13; cfr. \emph{Queremos ser Señor} (est. 1) con Sal 84,11; cfr. \emph{Canción del misionero} con Mc 6,7-13.}






%------------------------
\newpage
\setcounter{footnote}{0}
\thispagestyle{fancy}
\fancyhead{}
\fancyfoot{}
%\footskip=-1cm

\lhead{\changefont{cmss}{bx}{n} \small Revisión 2013}
\chead{\changefont{cmss}{bx}{n} \small Domingo XVI - Durante el año}
\rhead{\changefont{cmss}{bx}{n} \small Ciclo B}
\rfoot{\changefont{cmss}{bx}{n}\large\thepage}

\vspace*{-10mm}

\begin{center}
{\large\it La primera multiplicación de los panes }
\end{center}

\vspace{-3mm}

\begin{tabbing}

{\changefont{cmss}{bx}{n} Entrada:\ \ \ \ \ }\= Callemos hermanos\footnotemark[1], Somos la familia de Jesús\footnotemark[1],  Un pueblo que camina, \\
\> Juntos como hermanos (est. 2).   \\ \\

{\changefont{cmss}{bx}{n} Salmo:} \> 22 ant. 1 ``El Señor es mi pastor ...'' (estr. 1, 2 o 3).\\ \\

{\changefont{cmss}{bx}{n} Ofrendas:} \> Al altar del Señor\footnotemark[2], Padre nuestro recibid, Pan de vida y bebida de luz, \\
\> Bendito seas. \\  \\

{\changefont{cmss}{bx}{n} Comunión:} \>  Oh santo altar\footnotemark[3], Quédate con nosotros, Yo soy el camino, Bendigamos al Señor, \\
\> Bendeciré al Señor.\\ \\

{\changefont{cmss}{bx}{n} Post-com.:} \> Mirarte sólo a tí, Cuántas gracias te debemos (P. Bevilacqua), \\
\> Si el mismo pan comimos, Las misericordias del Señor (Taizé).\\ \\

{\changefont{cmss}{bx}{n} Salida:} \> Jesús te seguiré\footnotemark[3], Mi camino eres tú, Canción del testigo, Cantemos hermanos.  \\ \\

\end{tabbing}

\vspace{-15mm}

\footnotetext[1]{Cfr. \emph{Callemos hermanos} con Mc 6,30-34; cfr. \emph{Somos la familia de Jesús} (estribillo) con Ef 2,6.}
\footnotetext[2]{Cfr. \emph{Al altar del Señor} (est. 1) con Salmo 22,5; cfr. \emph{Ofrenda de amor} (est. 2) con Mc 6,7-13.}
\footnotetext[3]{Cfr. \emph{Oh santo altar} (est. 4) con Ef 2,13-17; cfr. \emph{Jesús te seguiré} con Mc 30-34 y su paralelo Mt 14,13-21.}




%------------------------
\newpage
\setcounter{footnote}{0}
\thispagestyle{fancy}
\fancyhead{}
\fancyfoot{}
%\footskip=-1cm

\lhead{\changefont{cmss}{bx}{n} \small Revisión 2013}
\chead{\changefont{cmss}{bx}{n} \small Domingo XVII - Durante el año}
\rhead{\changefont{cmss}{bx}{n} \small Ciclo B}
\rfoot{\changefont{cmss}{bx}{n}\large\thepage}

\vspace*{-10mm}

\begin{center}
{\large\it La primera multiplicación de los panes }
\end{center}

\vspace{-3mm}

\begin{tabbing}

{\changefont{cmss}{bx}{n} Entrada:\ \ \ \ \ }\= Un solo Señor\footnotemark[1], El señor nos llama, Pueblo de Dios, Alabaré (est. 2). \\ \\

{\changefont{cmss}{bx}{n} Salmo:} \> 144 ant. 1 ``Te alabamos Señor ...'' (estr. 2, 6 o 7).\\ \\

{\changefont{cmss}{bx}{n} Ofrendas:} \> Un niño se te acercó\footnotemark[2], Señor te ofrecemos\footnotemark[2], Este es nuestro pan,  \\
\> Mira nuestra ofrenda, Bendito seas. \\ \\

{\changefont{cmss}{bx}{n} Comunión:} \>  Jesucristo dános de este pan\footnotemark[2], Jesús eucaristía\footnotemark[2], Panis angelicus\footnotemark[2], \\
\>  Yo soy el camino\footnotemark[2], Este es mi cuerpo\footnotemark[2].\\ \\

{\changefont{cmss}{bx}{n} Post-com.:} \> Adoremos a Dios, Tu fidelidad, Si el mismo pan comimos, Nuestro maná. \\ \\

{\changefont{cmss}{bx}{n} Salida:} \> En medio de los pueblos\footnotemark[4], Madre de nuestro pueblo, Canción del testigo.  \\ \\

\end{tabbing}

\vspace{-15mm}

\footnotetext[1]{Cfr. \emph{Un solo Señor} (estribillo) con Ef 4,5-6.}
\footnotetext[2]{Cfr. \emph{Un niño se te acercó} (est. 1 y 3) con Jn 6,9; cfr. \emph{Señor te ofrecemos} con ant. de entrada (Sal 67,36); \emph{Jesucristo danos de este pan} y \emph{Jesús eucaristía (Jesucristo Señor de la historia)} son muy apropiados para la multiplicación de los panes.}
\footnotetext[3]{Todos estos cantos son muy eucarísticos. }
\footnotetext[4]{Cfr. \emph{En medio de los pueblos} con Jn 6,14-15 y su versículo equivalente en Lc 7,16. }




%------------------------
\newpage
\setcounter{footnote}{0}
\thispagestyle{fancy}
\fancyhead{}
\fancyfoot{}
%\footskip=-1cm

\lhead{\changefont{cmss}{bx}{n} \small Revisión 2013}
\chead{\changefont{cmss}{bx}{n} \small Domingo XVIII - Durante el año}
\rhead{\changefont{cmss}{bx}{n} \small Ciclo B}
\rfoot{\changefont{cmss}{bx}{n}\large\thepage}

\vspace*{-10mm}

\begin{center}
{\large\it Discurso sobre el Pan de Vida }
\end{center}

\vspace{-3mm}

\begin{tabbing}

{\changefont{cmss}{bx}{n} Entrada:\ \ \ \ \ }\= Pueblo de Dios, Pueblo de Reyes (est. 1), Somos la familia de Jesús, \\
\> Iglesia peregrina (est. 6), Pueblo de Dios peregrino\footnotemark[1].\\ \\

{\changefont{cmss}{bx}{n} Salmo:} \> 94 ant. 1 ``Adoremos al Señor...'' (estr. 1, 2 o 3).\\ \\

{\changefont{cmss}{bx}{n} Ofrendas:} \> Te presentamos, Señor te ofrecemos\footnotemark[2], Este es nuestro pan, Bendeciré al Señor. \\  \\

{\changefont{cmss}{bx}{n} Comunión:} \>  Es mi Padre\footnotemark[3], Yo soy el Pan de Vida\footnotemark[3], Jesucristo danos de este pan\footnotemark[3],   \\
\>  Este es mi cuerpo, Yo soy el camino\footnotemark[3].\\ \\

{\changefont{cmss}{bx}{n} Post-com.:} \> Nuestro maná\footnotemark[3], Adoremos a Dios (est. 3), Cuántas gracias te debemos. \\ \\

{\changefont{cmss}{bx}{n} Salida:} \> Madre de nuestro pueblo, Madre de los peregrinos, Quiero decir que sí, \\
\> Mi camino eres tú.  \\ \\

\end{tabbing}

\vspace{-15mm}

\footnotetext[1]{Canto de Matías Sagreras.}
\footnotetext[2]{Este canto asegura ``tú eres nuestra fortaleza...'' y ``tú sólo eres bueno Señor'', en consonancia con la multiplicación de los panes y el discurso del Pan de Vida. }
\footnotetext[3]{Todos estos cantos hacen referencia al evangelio del día (Jn 6,24-35); cfr. \emph{Yo soy el Pan de Vida} con ant. comunión Jn 6,35; cfr. \emph{Este es mi cuerpo} (``verdadero manjar para los hombres...'') con ant. comunión Sab. 16,20. }




%------------------------
\newpage
\setcounter{footnote}{0}
\thispagestyle{fancy}
\fancyhead{}
\fancyfoot{}
%\footskip=-1cm

\lhead{\changefont{cmss}{bx}{n} \small Revisión 2013}
\chead{\changefont{cmss}{bx}{n} \small Domingo XIX - Durante el año}
\rhead{\changefont{cmss}{bx}{n} \small Ciclo B}
\rfoot{\changefont{cmss}{bx}{n}\large\thepage}

\vspace*{-10mm}

\begin{center}
{\large\it Discurso sobre el Pan de Vida }
\end{center}

\vspace{-3mm}

\begin{tabbing}

{\changefont{cmss}{bx}{n} Entrada:\ \ \ \ \ }\= Juntos como hermanos (est. 1)\footnotemark[1], Somos la familia de Jesús\footnotemark[1], Pueblo de Reyes, \\
\> Caminaré, Alabaré a mi Señor.\\ \\

{\changefont{cmss}{bx}{n} Salmo:} \> 33 ant. 1 ``Vayamos a gustar...'' (estr. 1, 3 o 4).\\ \\

{\changefont{cmss}{bx}{n} Ofrendas:} \> Señor te ofrecemos\footnotemark[2], Bendeciré al Señor\footnotemark[2], Pan de vida y bebida de luz\footnotemark[2], \\
\> Comienza el sacrificio, Un niño se te acercó. \\  \\

{\changefont{cmss}{bx}{n} Comunión:} \>  Panis angelicus\footnotemark[3], Yo soy el Pan de Vida\footnotemark[3], Es mi Padre\footnotemark[3], Este es mi Cuerpo\footnotemark[3],   \\
\>  Yo soy el camino\footnotemark[3], Cuerpo y Sangre de Jesús\footnotemark[3].\\ \\

{\changefont{cmss}{bx}{n} Post-com.:} \> Mirarte sólo a Ti\footnotemark[4], El Señor es mi fortaleza (Taizé)\footnotemark[4],  Nuestro maná\footnotemark[4]. \\ \\

{\changefont{cmss}{bx}{n} Salida:} \> Cantemos hermanos, Mi camino eres Tú, Soy peregrino, Santa María del camino. \\ \\

\end{tabbing}

\vspace{-15mm}

\footnotetext[1]{Cfr. \emph{Juntos como hermanos} (est. 1) con 1 Re 19,4-8; cfr. \emph{Somos la familia de Jesús} con 1 Re 19,4-8 y Jn 6,42.}
\footnotetext[2]{Cfr. \emph{Señor te ofrecemos} (est. 2 y 3) con 1 Re 19,4-8; cfr. \emph{Bendeciré al Señor} con Sal 33; cfr. \emph{Pan de vida y bebida de luz} con Ef 4,30-5,2. }
\footnotetext[3]{Todos estos cantos hacen referencia al evangelio del día (Jn 6,41-51); cfr. \emph{Cuerpo y Sangre de Jesús} (est. 2) con Ef 4,30. }
\footnotetext[4]{Cfr. \emph{Mirarte sólo a Ti} y \emph{El Señor es mi fortaleza} con 1 Re 19,4-8; cfr. \emph{Nuestro Maná} con Jn 6,41-51. }




%------------------------
\newpage
\setcounter{footnote}{0}
\thispagestyle{fancy}
\fancyhead{}
\fancyfoot{}
%\footskip=-1cm

\lhead{\changefont{cmss}{bx}{n} \small Revisión 2013}
\chead{\changefont{cmss}{bx}{n} \small Domingo XX - Durante el año}
\rhead{\changefont{cmss}{bx}{n} \small Ciclo B}
\rfoot{\changefont{cmss}{bx}{n}\large\thepage}

\vspace*{-10mm}

\begin{center}
{\large\it Discurso sobre el Pan de Vida }
\end{center}

\vspace{-3mm}

\begin{tabbing}

{\changefont{cmss}{bx}{n} Entrada:\ \ \ \ \ }\= Vienen con alegría 
\footnotemark[1], Que lindo llegar cantando\footnotemark[1], Vine a alabar, 
Alabaré.  \\ \\

{\changefont{cmss}{bx}{n} Salmo:} \> 33 ant. 1 ``Vayamos a gustar...'' (estr. 1, 
5 o 6).\\ \\

{\changefont{cmss}{bx}{n} Ofrendas:} \> Padre nuestro recibid, Señor te 
ofrecemos\footnotemark[2], Te presentamos, Bendeciré al Señor\footnotemark[2], 
\\
\> Bendito seas Señor.  \\  \\

{\changefont{cmss}{bx}{n} Comunión:} \>  Yo soy el Pan de Vida\footnotemark[3], 
Es mi Padre\footnotemark[3], Este es mi Cuerpo\footnotemark[3],  Yo soy el 
camino\footnotemark[3],   \\
\> Cuerpo y Sangre de Jesús\footnotemark[3].\\ \\

{\changefont{cmss}{bx}{n} Post-com.:} \> Adoremos a Dios (est. 
3)\footnotemark[4], Cuántas gracias te debemos\footnotemark[4], \\ 
\> Alabe todo el mundo (Taizé), Nuestro maná. \\ \\

{\changefont{cmss}{bx}{n} Salida:} \> Cantad a María\footnotemark[4], Canción 
del testigo, Cantemos hermanos,  Simple oración. \\ \\

\end{tabbing}

\vspace{-15mm}

\footnotetext[1]{Cfr. \emph{Vienen con alegría} (estribillo) y \emph{Que lindo 
llegar cantando} (estribillo) con Ef 5,19.}
\footnotetext[2]{\emph{Señor te ofrecemos} recuerda en su estribillo la 
``Cena Pascual''. Jn 6,51 corresponden a las palabras joánicas de la 
institución de la eucaristía; cfr. \emph{Bendeciré al Señor} con Sal 33. }
\footnotetext[3]{Todos estos cantos hacen referencia al evangelio del día (Jn 
6,51-58).}
\footnotetext[4]{Cfr. \emph{Adoremos a Dios} (est. 3) y \emph{Cuántas 
gracias te debemos} con Ef 5,20; cfr. \emph{Cantad a María} 
(est. 4) con Jn 6,51-58. }




%------------------------
\newpage
\setcounter{footnote}{0}
\thispagestyle{fancy}
\fancyhead{}
\fancyfoot{}
%\footskip=-1cm

\lhead{\changefont{cmss}{bx}{n} \small Revisión 2013}
\chead{\changefont{cmss}{bx}{n} \small Domingo XXI - Durante el año}
\rhead{\changefont{cmss}{bx}{n} \small Ciclo B}
\rfoot{\changefont{cmss}{bx}{n}\large\thepage}

\vspace*{-10mm}

\begin{center}
{\large\it La profesión de fe de Pedro }
\end{center}

\vspace{-3mm}

\begin{tabbing}

{\changefont{cmss}{bx}{n} Entrada:\ \ \ \ \ }\= Iglesia 
peregrina\footnotemark[1], Un solo Señor\footnotemark[1], Vine a alabar, Qué 
alegría.  \\ \\

{\changefont{cmss}{bx}{n} Salmo:} \> 33 ant. 1 ``Vayamos a gustar...'' (estr. 
1, 8 o 10).\\ \\

{\changefont{cmss}{bx}{n} Ofrendas:} \> Recibe oh Dios eterno\footnotemark[2], 
Sé como el grano de trigo\footnotemark[2], Una espiga\footnotemark[2], \\
\> Bendeciré al Señor\footnotemark[2].  \\  \\

{\changefont{cmss}{bx}{n} Comunión:} \>  Creo en ti Señor\footnotemark[3], Oh 
buen Jesús\footnotemark[3], El Señor de Galilea, Este es mi 
Cuerpo,     \\
\> Yo soy el camino.\\ \\

{\changefont{cmss}{bx}{n} Post-com.:} \> Creo en ti Señor\footnotemark[3], 
Mirarte sólo a ti, El alfarero, Si el mismo pan comimos. \\ \\

{\changefont{cmss}{bx}{n} Salida:} \> Soy peregrino\footnotemark[4], Mi 
camino eres tú\footnotemark[4], Canción del misionero, \\
\> Madre de los peregrinos. \\ \\

\end{tabbing}

\vspace{-15mm}

\footnotetext[1]{Cfr. \emph{Iglesia peregrina} (est. 2) con Ef 5,27.29-30 y Jn 
6,63; cfr. \emph{Un solo Señor} con Jn 6,65 y Ef 5,21-33.}
\footnotetext[2]{cfr. \emph{Recibe oh Dios eterno} (est. 1) con Jn 6,68; cfr. 
\emph{Una espiga} (est. 3) y \emph{Sé como el grano de trigo} (est. 1) con Ef. 
5,21-33; cfr. \emph{Bendeciré al Señor} con Sal 33. }
\footnotetext[3]{Cfr. \emph{Creo en Ti Señor} y \emph{Oh buen Jesús} con Jn 
6,68.}
\footnotetext[4]{Cfr. \emph{Soy peregrino} (estribillo) y \emph{Mi camino eres 
tú} con Jn 6,60-69. }


%------------------------
\newpage
\setcounter{footnote}{0}
\thispagestyle{fancy}
\fancyhead{}
\fancyfoot{}
%\footskip=-1cm

\lhead{\changefont{cmss}{bx}{n} \small Revisión 2013}
\chead{\changefont{cmss}{bx}{n} \small Domingo XXII - Durante el año}
\rhead{\changefont{cmss}{bx}{n} \small Ciclo B}
\rfoot{\changefont{cmss}{bx}{n}\large\thepage}

\vspace*{-10mm}

\begin{center}
{\large\it Discusión sobre las tradiciones }
\end{center}

\vspace{-3mm}

\begin{tabbing}

{\changefont{cmss}{bx}{n} Entrada:\ \ \ \ \ }\= El sermón de 
la montaña (est. 3)\footnotemark[1],$\,$Vine a 
alabar\footnotemark[1],$\,$Callemos 
hermanos\footnotemark[1],$\,$Caminaré\footnotemark[1].  \\ \\

{\changefont{cmss}{bx}{n} Salmo:} \> 18b ant. 1 ``Tu palabra, Señor, es la 
verdad ...'' (estr. 1, 2 o 3).\\ \\

{\changefont{cmss}{bx}{n} Ofrendas:} \> Coplas de Yaraví\footnotemark[2], 
Pan de vida y bebida de luz, Mira nuestra ofrenda, \\
\> Toma Señor nuestra vida, Te presentamos.  \\  \\

{\changefont{cmss}{bx}{n} Comunión:} \>  Queremos ser Señor\footnotemark[2], 
Simple oración, Vayamos a la mesa,      
\\
\> Bendigamos al Señor, Dios me dio a mi hermano.\\ \\

{\changefont{cmss}{bx}{n} Post-com.:} \> Al atardecer de la 
vida (est. 1)\footnotemark[2], Tan cerca de mí (est. 1), Todos unidos, \\
\> Cuantas gracias te debemos. \\ \\

{\changefont{cmss}{bx}{n} Salida:} \> En medio de los pueblos\footnotemark[3], 
Vayan todos por el mundo, Simple oración, \\
\> Mi camino eres tú, Santa María del camino. \\ \\

\end{tabbing}

\vspace{-15mm}

\footnotetext[1]{Cfr. \emph{El sermón de la montaña} (est. 3) con Deut 
4,1-2.6-8; cfr. \emph{Vine a alabar} con Mc 7,6; cfr. \emph{Callemos hermanos} 
con Sant 1,17-18.21b-22.27; cfr. \emph{Caminaré} (est, 1 y 4) con ant. de 
entrada (Sal 85,3.5).}
\footnotetext[2]{Cfr. \emph{Coplas de Yaraví} con Mc 7,1-8; cfr. 
\emph{Queremos ser Señor} (estribillo) con ant. de comunión del día (Mt 
5,9-10) y Mc 7,6; \emph{Al atardecer de la vida} (est. 1) con Sant 
1,17-18.21b-22.27. }
\footnotetext[3]{Cfr. \emph{En medio de los pueblos} (est. 2) con Sant. 
1,17-27.}






%------------------------
\newpage
\setcounter{footnote}{0}
\thispagestyle{fancy}
\fancyhead{}
\fancyfoot{}
%\footskip=-1cm

\lhead{\changefont{cmss}{bx}{n} \small Revisión 2013}
\chead{\changefont{cmss}{bx}{n} \small Domingo XXIII - Durante el año}
\rhead{\changefont{cmss}{bx}{n} \small Ciclo B}
\rfoot{\changefont{cmss}{bx}{n}\large\thepage}

\vspace*{-10mm}

\begin{center}
{\large\it Curación de un sordomudo }
\end{center}

\vspace{-3mm}

\begin{tabbing}

{\changefont{cmss}{bx}{n} Entrada:\ \ \ \ \ }\= Pueblo 
de Dios\footnotemark[1], Alabaré\footnotemark[1], Vienen con 
alegría, Qué alegría.  \\ \\

{\changefont{cmss}{bx}{n} Salmo:} \> 145 ant. 1 ``El Señor es fiel a su 
palabra, ...'' (estr. 1, 5 o 6).\\
\> (antífona de reemplazo: Sal 26 ant. 1 ``Cantaré y celebraré al Señor'') \\ 
\\ 

{\changefont{cmss}{bx}{n} Ofrendas:} \> Bendeciré al Señor\footnotemark[2], 
Te ofrecemos Padre nuestro (versión moderna),  \\
\> Recibe oh Dios eterno, Bendito seas.  \\  \\

{\changefont{cmss}{bx}{n} Comunión:} \>  Jesús te seguiré\footnotemark[2], 
Pescador de hombres\footnotemark[2], Yo soy el camino\footnotemark[2], El Señor 
de Galilea.\\ \\

{\changefont{cmss}{bx}{n} Post-com.:} \> Tan cerca de mí, Bendigamos al Señor 
(est. 1, 4), Cuantas gracias te debemos. \\ \\

{\changefont{cmss}{bx}{n} Salida:} \> Jesús te seguiré\footnotemark[2], 
Oh María, Oh Santísima, Mi camino eres tú, \\
\>  Santa María del camino. \\ \\

\end{tabbing}

\vspace{-15mm}

\footnotetext[1]{Cfr. \emph{Pueblo de Dios} (estribillo) y \emph{Alabaré} con 
Mc 7,36-37.}
\footnotetext[2]{Cfr. \emph{Bendeciré al Señor} con Mc 7,36-37; cfr. 
\emph{Jesús te seguiré} (est. 3) con Mc 7,31-37; \emph{Pescador de hombres} 
(est. 1) con Sant 2,1-7; cfr. \emph{Yo soy el camino} (est. 1) con ant. 
comunión del día (Jn 8,12). }







%------------------------
\newpage
\setcounter{footnote}{0}
\thispagestyle{fancy}
\fancyhead{}
\fancyfoot{}
%\footskip=-1cm

\lhead{\changefont{cmss}{bx}{n} \small Revisión 2013}
\chead{\changefont{cmss}{bx}{n} \small Domingo XXIV - Durante el año}
\rhead{\changefont{cmss}{bx}{n} \small Ciclo B}
\rfoot{\changefont{cmss}{bx}{n}\large\thepage}

\vspace*{-10mm}

\begin{center}
{\large\it La profesión de fe de Pedro y anuncio de la Pasión }
\end{center}

\vspace{-3mm}

\begin{tabbing}

{\changefont{cmss}{bx}{n} Entrada:\ \ \ \ \ }\= Caminaré\footnotemark[1], El 
Señor nos llama, Juntos como hermanos, Qué alegría.  \\ \\

{\changefont{cmss}{bx}{n} Salmo:} \> 129 ant. 1 ``Yo pongo mi esperanza en tí, 
Señor, ...'' (estr. 1, 3 o 4).\\
\> (antífona de reemplazo: Sal 29 ant. 1 ``Te glorifico Señor, porque me 
salvaste'') \\ 
\\ 

{\changefont{cmss}{bx}{n} Ofrendas:} \> Entre tus manos\footnotemark[2], 
 Mira nuestra ofrenda\footnotemark[2], Una espiga, Sé como el grano de trigo.  
\\ \\

{\changefont{cmss}{bx}{n} Comunión:} \>  Creo en tí (Más 
cerca oh Dios)\footnotemark[3], 
Oh buen Jesús\footnotemark[3], Escondido, Yo soy el camino, \\ 
\> Cuerpo y Sangre de Jesús.\\ \\

{\changefont{cmss}{bx}{n} Post-com.:} \> Creo en tí (est. 2), Mirarte 
sólo a tí, Tu fidelidad, Nada te turbe (Taizé). \\ \\

{\changefont{cmss}{bx}{n} Salida:} \> Mi camino eres tú, Canción del testigo, 
Madre de los peregrinos,  \\
\>  Santa María del camino. \\ \\

\end{tabbing}

\vspace{-15mm}

\footnotetext[1]{\emph{Caminaré} es una paráfrasis del salmo del día (Sal 114).}
\footnotetext[2]{Cfr. \emph{Entre tus manos} (estribillo y est. 1) con Mc 
8,31; cfr. \emph{Mira nuestra ofrenda} (est. 2) con Mc 8,37. }
\footnotetext[3]{Cfr. \emph{Creo en tí Señor (Más cerca oh Dios)} con Mc 8,29; 
cfr. \emph{Oh buen Jesús} (est. 1 y 2) con Mc 8,33-35. }






%------------------------
\newpage
\setcounter{footnote}{0}
\thispagestyle{fancy}
\fancyhead{}
\fancyfoot{}
%\footskip=-1cm

\lhead{\changefont{cmss}{bx}{n} \small Revisión 2013}
\chead{\changefont{cmss}{bx}{n} \small Domingo XXV - Durante el año}
\rhead{\changefont{cmss}{bx}{n} \small Ciclo B}
\rfoot{\changefont{cmss}{bx}{n}\large\thepage}

\vspace*{-10mm}

\begin{center}
{\large\it El segundo anuncio de la Pasión }
\end{center}

\vspace{-3mm}

\begin{tabbing}

{\changefont{cmss}{bx}{n} Entrada:\ \ \ \ \ }\= Brilló la luz\footnotemark[1], 
El Sermón de la montaña\footnotemark[1], El Señor nos llama, \\
\> Juntos como hermanos (est. 2), Vine a alabar.  \\ \\

{\changefont{cmss}{bx}{n} Salmo:} \> 30 ant. 1 ``En tus manos, Señor, ...'' 
(estr. 1, 2 u 8).\\ \\


{\changefont{cmss}{bx}{n} Ofrendas:} \> Los frutos de la 
tierra\footnotemark[2], Coplas de Yaraví\footnotemark[2], Te ofrecemos Padre 
nuestro (vidala), \\
\> Una espiga, Sé como el grano de trigo.  \\ \\

{\changefont{cmss}{bx}{n} Comunión:} \>  Queremos ser Señor\footnotemark[2], 
Vayamos a la mesa\footnotemark[2], Mensajero de la paz, Brilló 
la luz\footnotemark[1],  \\ 
\> Simple oración, Yo soy el camino.\\ \\

{\changefont{cmss}{bx}{n} Post-com.:} \> Bendigamos al Señor (cántico de 
caridad), Donde hay amor y caridad (Taizé)\footnotemark[2]. \\
\> Si el mismo pan comimos, El alfarero.\\ \\

{\changefont{cmss}{bx}{n} Salida:} \> Madre de los peregrinos,$\,$Simple 
oración,$\,$Vayan todos por el mundo,$\,$Soy peregrino. \\ \\

\end{tabbing}

\vspace{-15mm}

\footnotetext[1]{\emph{Brilló la luz} (Las Bienaventuranzas) (estribillo y est. 
4) con Mc 9,2-10 y Mc 9,30-37; cfr. \emph{El sermón de la montaña} (est. 1 y 
3) con Mc 9,33-37.}
\footnotetext[2]{Cfr. \emph{Los frutos de la tierra} (estribillo), 
\emph{Coplas de Yaraví} (est. 4) y \emph{Queremos ser Señor} (estribillo) con 
Mc 9,35; cfr. \emph{Vayamos a la mesa} (est. 3 y 4) con Mc 9,30-37; cfr. 
\emph{Donde hay amor y caridad} va muy bien porque Mc 
9,37 tiene su equivalente en Jn 13,20 en el contexto de la última cena. }









%------------------------
\newpage
\setcounter{footnote}{0}
\thispagestyle{fancy}
\fancyhead{}
\fancyfoot{}
%\footskip=-1cm

\lhead{\changefont{cmss}{bx}{n} \small Revisión 2013}
\chead{\changefont{cmss}{bx}{n} \small Domingo XXVI - Durante el año}
\rhead{\changefont{cmss}{bx}{n} \small Ciclo B}
\rfoot{\changefont{cmss}{bx}{n}\large\thepage}

\vspace*{-10mm}

\begin{center}
{\large\it La intolerancia de los Apóstoles y la gravedad del escándalo }
\end{center}

\vspace{-3mm}

\begin{tabbing}

{\changefont{cmss}{bx}{n} Entrada:\ \ \ \ \ }\= Iglesia 
peregrina\footnotemark[1], Juntos como hermanos (est. 2), 
El Sermón de la montaña,\\
\> Vine a alabar.  \\ \\

{\changefont{cmss}{bx}{n} Salmo:} \> 18b ant. 1 ``Tu palabra, Señor, es la 
verdad ...'' (estr. 1, 3 u 6).\\ \\


{\changefont{cmss}{bx}{n} Ofrendas:} \> Pan de vida y bebida 
de luz\footnotemark[1], Recibe oh Dios el pan, Al altar nos acercamos, \\
\> Coplas de Yaraví. \\ \\

{\changefont{cmss}{bx}{n} Comunión:} \>  Simple oración\footnotemark[2], 
Bendigamos al Señor\footnotemark[2], La canción de la 
Alianza\footnotemark[2],  \\
\> Mensajero de la paz, Si yo no tengo amor.  \\ \\


{\changefont{cmss}{bx}{n} Post-com.:} \>  Donde hay 
amor y caridad (Taizé), Al atardecer de la vida, \\
\> Si el mismo pan comimos, Adoremos a Dios.\\ \\

{\changefont{cmss}{bx}{n} Salida:} \> Vayan todos por el 
mundo\footnotemark[3], Santa María del camino, Oh María, Salve oh Reina. \\ \\

\end{tabbing}

\vspace{-15mm}

\footnotetext[1]{\emph{Iglesia peregrina} (estribillo) con Mc 9,43-48; cfr. 
\emph{Pan de vida y bebida de luz} (est. 1) con Mc 9,38-48.}
\footnotetext[2]{Cfr. \emph{Simple oración}, \emph{La canción de la Alianza} y 
\emph{Bendigamos al Señor (cántico de caridad)} con Mc 9,38-48. }
\footnotetext[3]{\emph{Vayan todos por el mundo} (est. 1) hace referencia a Mc 
9,49-50, que no se lee en este ciclo, pero está ligado al evangelio del día.}







%------------------------
\newpage
\setcounter{footnote}{0}
\thispagestyle{fancy}
\fancyhead{}
\fancyfoot{}
%\footskip=-1cm

\lhead{\changefont{cmss}{bx}{n} \small Revisión 2013}
\chead{\changefont{cmss}{bx}{n} \small Domingo XXVII - Durante el año}
\rhead{\changefont{cmss}{bx}{n} \small Ciclo B}
\rfoot{\changefont{cmss}{bx}{n}\large\thepage}

\vspace*{-10mm}

\begin{center}
{\large\it El matrimonio y el divorcio }
\end{center}

\vspace{-3mm}

\begin{tabbing}

{\changefont{cmss}{bx}{n} Entrada:\ \ \ \ \ }\= El Señor nos 
llama\footnotemark[1], Somos la familia de Jesús, 
Iglesia peregrina, \\
\> Juntos como hermanos.  \\ \\

{\changefont{cmss}{bx}{n} Salmo:} \> 127 ant. 1 ``Feliz quien ama al Señor ...'' 
(estr. 1 y 2).\\ \\


{\changefont{cmss}{bx}{n} Ofrendas:} \> Ofrenda de amor\footnotemark[1], Una 
espiga\footnotemark[1], Toma Señor nuestra vida, Te presentamos. \\ \\

{\changefont{cmss}{bx}{n} Comunión:} \>  La canción de la 
Alianza\footnotemark[2], Queremos ser Señor, Si yo no tengo amor, Escondido,\\
\> Bendigamos al Señor. \\  \\


{\changefont{cmss}{bx}{n} Post-com.:} \>  Si el mismo pan 
comimos\footnotemark[2], Tu fidelidad, Las misericordias del Señor 
(Taizé), \\
\>  Donde hay amor y caridad (Taizé), Adoremos a Dios.\\ \\

{\changefont{cmss}{bx}{n} Salida:} \> Madre de nuestro pueblo (est. 1, 2 , 3, 
7, 8), Madre de los peregrinos,  \\
\> Vayan todos por el 
mundo, Anunciaremos tu reino, Santa María del camino. 
\\ \\

\end{tabbing}

\vspace{-15mm}

\footnotetext[1]{Cfr. \emph{El Señor nos llama} (estribillo) con Mc 10,9; cfr. 
\emph{Ofrenda de amor (Por los niños)} (est. 1) con Mc 10,14; cfr. \emph{Una 
espiga} (est. 3 y 4) con aclamación al evangelio del día 1 Jn 4,12 y Mc 
10,2-16.}
\footnotetext[2]{Cfr. \emph{La canción de la Alianza} con aclamación al 
evangelio del día 1 Jn 4,12 y Mc 10,9; cfr. \emph{Si el mismo pan comimos} con 
ant. de comunión del día 1 Cor 10,17; cfr. \emph{Tu fidelidad} con Heb 
2,9-11. }







%------------------------
\newpage
\setcounter{footnote}{0}
\thispagestyle{fancy}
\fancyhead{}
\fancyfoot{}
%\footskip=-1cm

\lhead{\changefont{cmss}{bx}{n} \small Revisión 2013}
\chead{\changefont{cmss}{bx}{n} \small Domingo XXVIII - Durante el año}
\rhead{\changefont{cmss}{bx}{n} \small Ciclo B}
\rfoot{\changefont{cmss}{bx}{n}\large\thepage}

\vspace*{-10mm}

\begin{center}
{\large\it El hombre rico }
\end{center}

\vspace{-3mm}

\begin{tabbing}

{\changefont{cmss}{bx}{n} Entrada:\ \ \ \ \ }\= Pueblo de 
reyes\footnotemark[1], El sermón de la montaña\footnotemark[1], Somos la 
familia de Jesús,  \\
\> Iglesia peregrina, Vienen con alegría.  \\ \\

{\changefont{cmss}{bx}{n} Salmo:}\footnotemark[1] \> 89 ant. 1 ``Nuestra vida, 
Señor, pasa como un soplo ...'' 
(estr. 1, 3 y 5).\\ 
\> (antífona de reemplazo: Sal 26 ant. 1 ``Cantaré y celebraré al Señor'') \\ \\


{\changefont{cmss}{bx}{n} Ofrendas:} \> Los frutos de la 
tierra\footnotemark[2], Te ofrecemos oh Señor, Bendeciré al 
Señor\footnotemark[2],  \\
\>Recibe oh Dios el pan, Pan de vida y bebida de luz. \\ \\

{\changefont{cmss}{bx}{n} Comunión:} \>  Pescador de hombres\footnotemark[2], 
Bendeciré al Señor\footnotemark[2], Cuerpo y Sangre de Jesús,\\
\>  Yo soy el camino. \\  \\


{\changefont{cmss}{bx}{n} Post-com.:} \>  Mirarte sólo a ti, Vaso nuevo (El 
alfarero), Cuantas gracias te debemos.\\ \\

{\changefont{cmss}{bx}{n} Salida:} \> Canción del testigo\footnotemark[3], Mi 
camino eres tú, Anunciaremos tu reino, Soy peregrino.  \\ \\

\end{tabbing}

\vspace{-13mm}

\footnotetext[1]{Cfr. \emph{Pueblo de reyes} (estribillo y est. 1) con Sab 
7,7-11; cfr. \emph{El sermón de la montaña} (est. 1) con Sal 89,15 y con 
aclamación al evangelio Mt 5,3; se puede hacer el \emph{Aleluia} de 
\emph{Busca primero el Reino de Dios}.}
\footnotetext[2]{Cfr. \emph{Los frutos de la tierra} (estribillos) con Mc 10,18 
y Mc 10,27; cfr. \emph{Bendeciré al Señor} (est. 3) con ant. de comunión del 
día (Sal 33,11) y Mc 10,17-30; cfr. \emph{Pescador de hombres} (est 1 y 2) con 
Sab 7,7-11.  }
\footnotetext[3]{Cfr. \emph{Canción del testigo} (est. 1) con Sab 7,7-11.}








%------------------------
\newpage
\setcounter{footnote}{0}
\thispagestyle{fancy}
\fancyhead{}
\fancyfoot{}
%\footskip=-1cm

\lhead{\changefont{cmss}{bx}{n} \small Revisión 2013}
\chead{\changefont{cmss}{bx}{n} \small Domingo XXIX - Durante el año}
\rhead{\changefont{cmss}{bx}{n} \small Ciclo B}
\rfoot{\changefont{cmss}{bx}{n}\large\thepage}

\vspace*{-10mm}

\begin{center}
{\large\it La petición de Santiago y Juan }
\end{center}

\vspace{-3mm}

\begin{tabbing}

{\changefont{cmss}{bx}{n} Entrada:\ \ \ \ \ }\= Juntos como 
hermanos\footnotemark[1], El Señor nos llama\footnotemark[1], Iglesia 
peregrina,   \\
\>Somos la familia de Jesús, Caminaré.  \\ \\

{\changefont{cmss}{bx}{n} Salmo:} \> 32 ant. 2 ``Que descienda, Señor, ...'' 
(estr. 2, 7 y 8).\\ 
\> (antífona de reemplazo: Sal 102 ant. 2 ``El amor del Señor permanece 
...'') \\ \\


{\changefont{cmss}{bx}{n} Ofrendas:} \> Ofrenda de amor\footnotemark[2], 
Coplas de Yaraví\footnotemark[2], Pan de vida y bebida de luz, \\
\>Sé como el grano de trigo. \\ \\

{\changefont{cmss}{bx}{n} Comunión:} \>  Como Cristo nos amo\footnotemark[2], 
Jesucristo danos de este pan\footnotemark[2], Simple oración,  Creo en ti. \\  
\\


{\changefont{cmss}{bx}{n} Post-com.:} \>  Si el mismo pan comimos, Tu 
fidelidad, Las misericordias del Señor (Taizé), \\
\> Cuantas gracias te debemos.\\ \\

{\changefont{cmss}{bx}{n} Salida:} \> Madre de los peregrinos\footnotemark[3], 
Madre de nuestro pueblo, Salve María,\\ 
\> Oh María, Vayan todos por el mundo.  \\ \\

\end{tabbing}

\vspace{-15mm}

\footnotetext[1]{Cfr. \emph{Juntos como hermanos} (estribillo y est. 2) con 
Mc 10,43; cfr. \emph{El Señor nos llama} (estribillo) con Mc 10,43.45.}
\footnotetext[2]{Cfr. \emph{Ofrenda de amor (Por los niños)} (est. 1) con Mc 
10,42; cfr. \emph{Coplas de Yaraví} y \emph{Jesucristo danos de este pan} 
(estribillo) con Mc 10,43-45; cfr. \emph{Como Cristo nos amó} (est. 1) con Is 
53,10-11, Mc 10,45.  }
\footnotetext[3]{\emph{Madre de los peregrinos} para el día de la Madre 
(última estrofa).}







%------------------------
\newpage
\setcounter{footnote}{0}
\thispagestyle{fancy}
\fancyhead{}
\fancyfoot{}
%\footskip=-1cm

\lhead{\changefont{cmss}{bx}{n} \small Revisión 2013}
\chead{\changefont{cmss}{bx}{n} \small Domingo XXX - Durante el año}
\rhead{\changefont{cmss}{bx}{n} \small Ciclo B}
\rfoot{\changefont{cmss}{bx}{n}\large\thepage}

\vspace*{-10mm}

\begin{center}
{\large\it Curación de un ciego de Jericó }
\end{center}

\vspace{-3mm}

\begin{tabbing}

{\changefont{cmss}{bx}{n} Entrada:\ \ \ \ \ }\= Pueblo de Dios\footnotemark[1], 
Vienen con alegría\footnotemark[1], Qué alegría\footnotemark[1]. \\   \\

{\changefont{cmss}{bx}{n} Salmo:} \> 125 ant. 1 ``Los que siembran entre 
lágrimas, ...'' 
(estr. 1, 2 y 3).\\ 
\> (antífona de reemplazo: Sal 29 ant. 1 ``Te glorifico Señor ...'') \\ \\


{\changefont{cmss}{bx}{n} Ofrendas:} \> Pan de vida y bebida 
de luz\footnotemark[2], Padre nuestro recibid, Este es nuestro pan, \\
\>Te presentamos. \\ \\

{\changefont{cmss}{bx}{n} Comunión:} \>  Jesús te seguiré\footnotemark[2], 
El Señor de Galilea\footnotemark[2], Yo soy el camino, \\
\> Cuerpo y Sangre de Jesús. \\  
\\


{\changefont{cmss}{bx}{n} Post-com.:} \>  Adoremos a Dios, Alabe todo el mundo 
(Taizé), El Señor es mi fortaleza (Taizé).\\  \\

{\changefont{cmss}{bx}{n} Salida:} \> Soy peregrino\footnotemark[3], 
Canción del testigo, Mi camino eres tú, Anunciaremos tu Reino.\\ \\ 


\end{tabbing}

\vspace{-15mm}

\footnotetext[1]{Cfr. \emph{Pueblo de Dios} (estribillo) con 
Mc 10,46-52; cfr. \emph{Vienen con alegría} (estribillo) con Sal 125,6; cfr. 
\emph{Qué alegría} con ant. de entrada (Sal 104,3).}
\footnotetext[2]{\emph{Pan de vida y bebida de luz} habla de la luz como 
la que recuperó el ciego (Mc 10,46-52); cfr. \emph{Jesús te seguiré} (est. 3) y 
\emph{El Señor de Galilea} con Mc 10,46-52.  }
\footnotetext[3]{\emph{Soy peregrino} (estribillo) habla de la luz de la 
fe.}







%------------------------
\newpage
\setcounter{footnote}{0}
\thispagestyle{fancy}
\fancyhead{}
\fancyfoot{}
%\footskip=-1cm

\lhead{\changefont{cmss}{bx}{n} \small Revisión 2013}
\chead{\changefont{cmss}{bx}{n} \small Domingo XXXI - Durante el año}
\rhead{\changefont{cmss}{bx}{n} \small Ciclo B}
\rfoot{\changefont{cmss}{bx}{n}\large\thepage}

\vspace*{-10mm}

\begin{center}
{\large\it El mandamiento principal }
\end{center}

\vspace{-3mm}

\begin{tabbing}

{\changefont{cmss}{bx}{n} Entrada:\ \ \ \ \ }\= Un solo Señor\footnotemark[1], 
Pueblo de reyes\footnotemark[1], Caminaré, Vine a alabar. \\   \\

{\changefont{cmss}{bx}{n} Salmo:} \> 17 ant. 1 ``Te amo, Señor,...'' 
(estr. 1, 2 o 7).\\ 
\> (antífona de reemplazo: Sal 29 ant. 1 ``Te glorifico Señor porque me 
salvaste'') \\ \\


{\changefont{cmss}{bx}{n} Ofrendas:} \> Pan de vida y bebida de 
luz\footnotemark[2], Al altar nos 
acercamos\footnotemark[2], Te presentamos,  \\
\> Bendito seas. \\ \\

{\changefont{cmss}{bx}{n} Comunión:} \>  Bendigamos al Señor\footnotemark[2], 
Jesucristo danos de este pan, Cuerpo y Sangre de Jesús, \\
\> Yo soy el Pan de Vida, Yo soy el camino. \\  
\\


{\changefont{cmss}{bx}{n} Post-com.:} \> El Señor es mi 
fortaleza (Taizé), Donde hay amor y caridad (Taizé),  \\
\> Tan cerca de mí, Cuantas gracias te debemos.\\  \\

{\changefont{cmss}{bx}{n} Salida:} \> Anunciaremos tu Reino, Mi camino 
eres tú, En medio de los pueblos, \\
\> Simple oración.\\ \\ 


\end{tabbing}

\vspace{-11mm}

\footnotetext[1]{Cfr. \emph{Un solo Señor} (estribillo y est. 1) con 
Mc 12,28-34; cfr. \emph{Pueblo de reyes} (est. 4) con Heb 7,23-28.}
\footnotetext[2]{\emph{Al altar nos acercamos} (est. 3) con Mc 12,31; cfr. 
\emph{Bendigamos al Señor} y \emph{Pan de vida y bebida de luz} (est. 1 y 2) 
con Mc 12,28-34.}










%------------------------
\newpage
\setcounter{footnote}{0}
\thispagestyle{fancy}
\fancyhead{}
\fancyfoot{}
%\footskip=-1cm

\lhead{\changefont{cmss}{bx}{n} \small Revisión 2013}
\chead{\changefont{cmss}{bx}{n} \small Domingo XXXII - Durante el año}
\rhead{\changefont{cmss}{bx}{n} \small Ciclo B}
\rfoot{\changefont{cmss}{bx}{n}\large\thepage}

\vspace*{-10mm}

\begin{center}
{\large\it La ofrenda de la viuda }
\end{center}

\vspace{-3mm}

\begin{tabbing}

{\changefont{cmss}{bx}{n} Entrada:\ \ \ \ \ }\= El sermón de 
la montaña\footnotemark[1], Caminaré\footnotemark[1], Vienen con alegría, El 
Señor nos llama. \\   \\

{\changefont{cmss}{bx}{n} Salmo:} \> 145 ant. 1 ``El Señor es fiel a su 
Palabra, ...'' 
(estr. 4, 5 o 6).\\ 
\> (antífona de reemplazo: Sal 144 ant. 1 ``Te alabamos Señor ...'') \\ \\


{\changefont{cmss}{bx}{n} Ofrendas:} \> Mira nuestra ofrenda\footnotemark[2], 
Ofrenda de amor\footnotemark[2], Padre nuestro recibid, \\
\> Te ofrecemos Padre nuestro (vidala). \\ \\

{\changefont{cmss}{bx}{n} Comunión:} \>  Como Cristo nos 
amó\footnotemark[2], Vayamos a la mesa, Bendeciré al Señor, Este es mi cuerpo, 
\\
\> Cuerpo y Sangre de Jesús. \\  
\\


{\changefont{cmss}{bx}{n} Post-com.:} \>  Nada te turbe (Taizé), El Señor es mi 
fortaleza (Taizé), \\
\> Cuantas gracias te debemos, Mirarte sólo a ti.\\  \\

{\changefont{cmss}{bx}{n} Salida:} \> Santa María del Camino\footnotemark[3], 
Oh María\footnotemark[3], Oh Santísima\footnotemark[3], Salve 
María\footnotemark[3], \\
\> Salve oh Reina\footnotemark[3].\\ \\ 


\end{tabbing}

\vspace{-15mm}

\footnotetext[1]{Cfr. \emph{El sermón de la montaña} (est. 1) con 
Mc 12,41-44; cfr. \emph{Caminaré} (est. 1) con antífona de entrada (Sal 87,3).}
\footnotetext[2]{\emph{Mira nuestra ofrenda} con Mc 12,41-44; cfr. \emph{Ofrenda 
de amor} (est. 1) con 1 Rey 17,8-16; cfr. \emph{Como Cristo nos amó} (est. 2) 
con 1 Rey 17,8-16 y Mc 12,41-44.}
\footnotetext[3]{Todos estos cantos corresponden al Mes de María.}








%------------------------
\newpage
\setcounter{footnote}{0}
\thispagestyle{fancy}
\fancyhead{}
\fancyfoot{}
%\footskip=-1cm

\lhead{\changefont{cmss}{bx}{n} \small Revisión 2013}
\chead{\changefont{cmss}{bx}{n} \small Domingo XXXIII - Durante el año}
\rhead{\changefont{cmss}{bx}{n} \small Ciclo B}
\rfoot{\changefont{cmss}{bx}{n}\large\thepage}

\vspace*{-10mm}

\begin{center}
{\large\it La manifestación gloriosa del Hijo del hombre}
\end{center}

\vspace{-3mm}

\begin{tabbing}

{\changefont{cmss}{bx}{n} Entrada:\ \ \ \ \ }\= Alabaré\footnotemark[1], 
Iglesia peregrina\footnotemark[1], Somos la familia de Jesús\footnotemark[1], 
Junto como hermanos\footnotemark[1]. \\ \\

{\changefont{cmss}{bx}{n} Salmo:} \> 15 ant. 1 ``Tu eres, Señor, mi herencia, 
...'' (estr. 1, 2 o 5).\\ 
\> (antífona de reemplazo: Sal 15 ant. 3 ``Protégeme, Dios mío, ...'') \\ \\


{\changefont{cmss}{bx}{n} Ofrendas:} \> Recibe oh Dios eterno\footnotemark[2], 
Ofrenda de amor, Señor te ofrecemos, Bendeciré al Señor. \\ \\

{\changefont{cmss}{bx}{n} Comunión:} \>  Más cerca oh Dios\footnotemark[2], 
Yo soy el Pan de Vida, Cuerpo y Sangre de Jesús,  \\
\> Este es mi cuerpo, Es mi Padre. \\  \\


{\changefont{cmss}{bx}{n} Post-com.:} \>  Adoremos a Dios, Alabe todo el mundo 
(Taizé), Mirarte sólo a tí, \\
\> Cuantas gracias te debemos.\\  \\

{\changefont{cmss}{bx}{n} Salida:} \> Salve oh Reina\footnotemark[2], Oh 
Santísima\footnotemark[2], Oh María\footnotemark[2], Santa María del 
Camino\footnotemark[2].\\ \\ 


\end{tabbing}

\vspace{-13mm}

\footnotetext[1]{\emph{Alabaré} (est. 1) habla de los redimidos; cfr. 
\emph{Iglesia peregrina} (estribillo) con ant. de entrada (Jer 29,11.12.14); 
cfr. \emph{Somos la familia de Jesús} (est. 2) con el tema de la paz en Jer 
29,11-14; la melodía de \emph{Juntos como hermanos} \mbox{(est. 2)} corresponde 
a un 
spiritual que de llama \emph{My God what a morning} y hace referencia a las 
imagenes relatadas en \mbox{Mc 24,31}. }
\footnotetext[2]{ cfr. \emph{Recibe oh Dios eterno} con discurso 
esctológico en Mc 13,24-32; cfr. \emph{Más cerca oh Dios} (estribillo) con 
ant. de comunión (Sal 12,28); todos los cantos de salida 
corresponden al Mes de María.}










%------------------------
\newpage
\setcounter{footnote}{0}
\thispagestyle{fancy}
\fancyhead{}
\fancyfoot{}
%\footskip=-1cm

\lhead{\changefont{cmss}{bx}{n} \small Revisión 2013}
\chead{\changefont{cmss}{bx}{n} \small Domingo XXXIV - Durante el año}
\rhead{\changefont{cmss}{bx}{n} \small Ciclo B}
\rfoot{\changefont{cmss}{bx}{n}\large\thepage}

\vspace*{-10mm}

\begin{center}
{\large\it (Domingo de Cristo Rey) Jesús ante Pilato}
\end{center}

\vspace{-3mm}

\begin{tabbing}

{\changefont{cmss}{bx}{n} Entrada:\ \ \ \ \ }\= Rey de los 
reyes\footnotemark[1], Alabaré\footnotemark[1], Pueblo de los 
reyes\footnotemark[1],  Un solo Señor. \\ \\

{\changefont{cmss}{bx}{n} Salmo:} \> 92 ant. 1 ``Reina el Señor, alégrese la 
tierra ...'' (estr. 1, 3 o 4).\\ \\


{\changefont{cmss}{bx}{n} Ofrendas:} \> Padre nuestro recibid\footnotemark[2], 
Te ofrecemos oh Señor, Recibe oh Dios eterno, \\
\> Bendito seas Señor. \\ \\

{\changefont{cmss}{bx}{n} Comunión:} \>  Rey de los 
reyes\footnotemark[1], Cuerpo y Sangre de Jesús\footnotemark[2], Yo soy el Pan 
de Vida,   \\
\> Yo soy el camino, Vayamos a la mesa. \\  \\


{\changefont{cmss}{bx}{n} Post-com.:} \>  Alabe todo el mundo 
(Taizé), Adoremos a Dios, Tu fidelidad. \\ \\

{\changefont{cmss}{bx}{n} Salida:} \> Christus vincit\footnotemark[2], Oh 
Santísima\footnotemark[2], Madre de los peregrinos, \\
\> Madre de nuestro pueblo\footnotemark[2], Oh María\footnotemark[2].\\ \\ 


\end{tabbing}

\vspace{-13mm}

\footnotetext[1]{Cfr. \emph{Pueblo de los reyes} (estribillo y est. 3) y 
\emph{Rey de los reyes} con ant. de entrada (Apoc 1,6) y aclamación al 
evangelio (Mc 11,9-10); cfr. \emph{Alabaré} con ant. de entrada Apoc 5,12. }
\footnotetext[2]{ cfr. \emph{Padre nuestro recibid} (est. 4) con aclamación 
al evangelio (Mc 11,9-10); cfr. \emph{Cuerpo y Sangre de Jeús} (est. 3) con 
ant. de entrada (Apoc 1,6); \emph{Christus vincit} es muy tradicional para 
este día de Cristo Rey; todos los cantos a la Virgen corresponden al Mes de 
María.}













%------------------------
\newpage
\setcounter{footnote}{0}
\thispagestyle{fancy}
\fancyhead{}
\fancyfoot{}
%\footskip=-1cm

\lhead{\changefont{cmss}{bx}{n} \small Revisión 2013}
\chead{\changefont{cmss}{bx}{n} \small Fiesta del 2 de febrero - Presentación del Señor}
\rhead{\changefont{cmss}{bx}{n} \small Ciclo B}
\rfoot{\changefont{cmss}{bx}{n}\large\thepage}

\vspace*{-11mm}

\begin{center}
{\large\it La presentación de Jesús en el Templo }
\end{center}

\vspace{-3mm}

\begin{tabbing}

{\changefont{cmss}{bx}{n} Entrada:\ \ \ \ \ }\= Sal 23 (ant. 1)\footnotemark[1], Pueblo de reyes\footnotemark[1], Pueblo de Dios, Alabaré (est.  2 o 3). \\ \\


{\changefont{cmss}{bx}{n} Salmo:} \> 23 ant. 2 ``Felices los que son fieles al Señor...'' (estr. 1, 2 o 3). \\ 
\> (antífona de reemplazo: Sal 147 ant. 1 ``Glorifica al Señor Jerusalén...'') \\ \\

{\changefont{cmss}{bx}{n} Ofrendas:} \> Recibe oh Dios eterno\footnotemark[2], Toma Señor nuestra vida\footnotemark[2], Te ofrecemos Padre nuestro  \\
\> (vidala), Padre nuestro recibid. \\ \\ 

{\changefont{cmss}{bx}{n} Comunión:} \> Más cerca oh Dios\footnotemark[3], Pueblo de reyes\footnotemark[1], Como Cristo nos amó\footnotemark[3], Este es mi cuerpo, \\
\>  Bendeciré al Señor. \\  \\

{\changefont{cmss}{bx}{n} Post-com.:} \> Aleluia Cristo vino con su paz, Adoremos a Dios, Alabe todo el mundo (Taizé).\\ \\

{\changefont{cmss}{bx}{n} Salida:} \> Madre de nuestro pueblo (est. 5)\footnotemark[4], Canción del testigo\footnotemark[4], Soy peregrino. \\ \\


\end{tabbing}

\vspace{-15mm}

\footnotetext[1]{Se ingresa en procesión con las candelas encendidas. Si se canta \emph{Pueblo de reyes}, hacer especialmente est. 1, 2  y 3.  }
\footnotetext[2]{Cfr. \emph{Recibe oh Dios eterno} (est. 1) y \emph{Toma Señor nuestra vida} con Mal 3,3-4.}
\footnotetext[3]{Cfr. \emph{Más cerca oh Dios (Creo en ti Señor)} (est. 3) con el cánto de Simeón (Lc 2,29-32); cfr. \emph{Como Cristo nos amó} (est. 2) con  Heb. 2,17-18.  }
\footnotetext[4]{Cfr. \emph{Madre de nuestro pueblo} (est. 5) con Lc 2,22-40); el protagonista  de \emph{Canción del testigo} puede ser identificado, en cierto aspecto, con Simeón (Lc 2,22-40).}



%------------------------
\newpage
\setcounter{footnote}{0}
\thispagestyle{fancy}
\fancyhead{}
\fancyfoot{}
%\footskip=-1cm

\lhead{\changefont{cmss}{bx}{n} \small Revisión 2013}
\chead{\changefont{cmss}{bx}{n} \small Solemnidad del 8 de mayo - Nuestra Señora de Luján}
\rhead{\changefont{cmss}{bx}{n} \small Ciclo B}
\rfoot{\changefont{cmss}{bx}{n}\large\thepage}

\vspace*{-11mm}

\begin{center}
{\large\it Jesús y su madre }
\end{center}

\vspace{-3mm}

\begin{tabbing}

{\changefont{cmss}{bx}{n} Entrada:\ \ \ \ \ }\= Somos un pueblo que camina, Madre de los peregrinos, Pueblo de Dios peregrino\footnotemark[1],\\
\> La Virgen María nos reúne, Feliz de ti María, Iglesia peregrina de Dios.\\ \\  


{\changefont{cmss}{bx}{n} Salmo:} \> Magnificat. ``El Señor hizo en mí maravillas...'' (todo) \\ \\

{\changefont{cmss}{bx}{n} Ofrendas:} \> Bendeciré al Señor\footnotemark[2], Te ofrecemos oh Señor, Te presentamos, Bendito seas. \\ \\ 

{\changefont{cmss}{bx}{n} Comunión:} \> Ave Verum, Jesucristo danos de este pan\footnotemark[3], Bendeciré al Señor\footnotemark[2], Este es mi cuerpo, \\
\> Mi alma glorifica\footnotemark[3]. \\  \\

{\changefont{cmss}{bx}{n} Post-com.:} \> Quiero decir que sí, Bendita sea tu pureza, Bajo tu amparo (P. Bevilacqua).\\ \\

{\changefont{cmss}{bx}{n} Salida:} \> Madre de los peregrinos, Madre de nuestro pueblo, Oh María, Cantad a María,\\
\> Canto de María, Santa María del camino.\\ \\

\end{tabbing}

\vspace{-15mm}

\footnotetext[1]{Canto reciente compuesto por Matías Sagreras y grabado por el Grupo de Música Litúrgica (P. Esteban Sacchi). }
\footnotetext[2]{Cfr. \emph{Bendeciré al Señor} con Lc 1,46-55.}
\footnotetext[3]{\emph{Jesucristo danos de este pan} menciona a María en su est. 4; cfr. \emph{Mi alma glorifica} con Lc 1,46-55.  }

%------------------------
\newpage
\setcounter{footnote}{0}
\thispagestyle{fancy}
\fancyhead{}
\fancyfoot{}
%\footskip=-1cm

\lhead{\changefont{cmss}{bx}{n} \small Revisión 2013}
\chead{\changefont{cmss}{bx}{n} \small Solemnidad del 29 de junio - San Pedro y San Pablo}
\rhead{\changefont{cmss}{bx}{n} \small Ciclo B}
\rfoot{\changefont{cmss}{bx}{n}\large\thepage}

\vspace*{-11mm}

\begin{center}
{\large\it La profesión de fe de Pedro}
\end{center}

\vspace{-3mm}

\begin{tabbing}

{\changefont{cmss}{bx}{n} Entrada:\ \ \ \ \ }\= Cante la Iglesia\footnotemark[1], Iglesia peregrina, Un pueblo que camina, Vienen con alegría.\\ \\  


{\changefont{cmss}{bx}{n} Salmo:} \> 33 ant. 1 ``Vayamos a gustar la bondad del Señor'' (1, 2, 3 o 4) \\ \\

{\changefont{cmss}{bx}{n} Ofrendas:} \> Pan de vida y bebida de luz,$\,$Recibe oh Dios el pan\footnotemark[2],$\,$Te presentamos,$\,$Bendito seas. \\ \\ 

{\changefont{cmss}{bx}{n} Comunión:} \> Más cerca oh Dios\footnotemark[3], Pescador de hombres\footnotemark[3], Bendeciré al Señor\footnotemark[3],\\ \> El Señor de Galilea. \\  \\

{\changefont{cmss}{bx}{n} Post-com.:} \> Si el mismo pan comimos, El Señor es mi fortaleza (Taizé), Mirarte sólo a ti\footnotemark[4].\\ \\

{\changefont{cmss}{bx}{n} Salida:} \> En medio de los pueblos, Mi camino eres tú\footnotemark[4], Canción del testigo, \\ 
\> Vayan todos por el mundo. \\ \\

\end{tabbing}

\vspace{-15mm}

\footnotetext[1]{Canto para el día de \emph{todos} los Santos, en particular para San Pedro y San Pablo. }
\footnotetext[2]{\emph{Recibe oh Dios el pan} (est. 3) pide por los difuntos.  }
\footnotetext[3]{Cfr. \emph{Más cerca oh Dios} con Jn 21,18-19 (martirio de Pedro); cfr. \emph{Pescador de hombres} (est. 1 y 2) con Jn 21,19 y Hech 3,6; cfr. \emph{Bendeciré al Señor} con Sal 33,2-9.}
\footnotetext[4]{Cfr. \emph{Mirarte sólo a ti} y \emph{Mi camino eres tú} son perfectamente aplicables a la vida de los apóstoles Pedro y Pablo.}

%------------------------
\newpage
\setcounter{footnote}{0}
\thispagestyle{fancy}
\fancyhead{}
\fancyfoot{}
%\footskip=-1cm

\lhead{\changefont{cmss}{bx}{n} \small Revisión 2013}
\chead{\changefont{cmss}{bx}{n} \small Solemnidad del 15 de agosto - Asunción de la Virgen María}
\rhead{\changefont{cmss}{bx}{n} \small Ciclo B}
\rfoot{\changefont{cmss}{bx}{n}\large\thepage}

\vspace*{-11mm}

\begin{center}
{\large\it El canto de la Virgen María }
\end{center}

\vspace{-3mm}

\begin{tabbing}

{\changefont{cmss}{bx}{n} Entrada:\ \ \ \ \ }\= Feliz de ti María\footnotemark[1], La Virgen María nos reúne, Que alegría.\\ \\  


{\changefont{cmss}{bx}{n} Salmo:} \> 44 con ant. ``A tu diestra, Señor, resplandece la Reina'' (música de Sal 30 ant. 1) \\ 
\> (estr. 5 y 7 del Sal 44) \\ \\

{\changefont{cmss}{bx}{n} Ofrendas:} \> Bendeciré al Señor\footnotemark[2], Pan de vida y bebida de luz\footnotemark[2], Te presentamos, Bendito seas. \\ \\ 

{\changefont{cmss}{bx}{n} Comunión:} \> Bendeciré al Señor\footnotemark[2],$\,$Jesucristo danos de este pan\footnotemark[3],$\,$Mi alma glorifica\footnotemark[3],$\,$Es mi Padre. \\  \\

{\changefont{cmss}{bx}{n} Post-com.:} \> Quiero decir que sí, Bendita sea tu pureza.\\ \\

{\changefont{cmss}{bx}{n} Salida:} \> Un día la veré\footnotemark[4], Cantad a María, Los cielos la tierra\footnotemark[4], Canto de María.\\ \\

\end{tabbing}

\vspace{-15mm}

\footnotetext[1]{Cfr. \emph{Feliz de ti María} (est. 5) con la antífona de entrada Apoc 12,1 y con evangelio de la Vigilia Lc 11,27-28. }
\footnotetext[2]{Cfr. \emph{Bendeciré al Señor} con Lc 1,46-55; \emph{Pan de vida y bebida de luz} con 1 Cor 15,20-27.}
\footnotetext[3]{\emph{Jesucristo danos de este pan} menciona a María en su est. 4; cfr. \emph{Mi alma glorifica} con Lc 1,46-55; cfr. \emph{Es mi Padre} (est. 3) con 1 Cor 15,20-27.  }
\footnotetext[4]{ Cfr. \emph{Un día la veré} (est. 1) con Sal 44,18; \emph{Los cielos, la tierra} (est. 2) con Lc 1,39-43. }

%------------------------
\newpage
\setcounter{footnote}{0}
\thispagestyle{fancy}
\fancyhead{}
\fancyfoot{}
%\footskip=-1cm

\lhead{\changefont{cmss}{bx}{n} \small Revisión 2013}
\chead{\changefont{cmss}{bx}{n} \small Fiesta del 14 de septiembre - Exaltación de la Santa Cruz}
\rhead{\changefont{cmss}{bx}{n} \small Ciclo B}
\rfoot{\changefont{cmss}{bx}{n}\large\thepage}

\vspace*{-11mm}

\begin{center}
{\large\it El diálogo de Jesús con Nicodemo }
\end{center}

\vspace{-3mm}

\begin{tabbing}

{\changefont{cmss}{bx}{n} Entrada:\ \ \ \ \ }\=  Cruz de Cristo\footnotemark[1], Somos la familia de Jesús\footnotemark[1], Juntos como hermanos (est. 1), \\
\> Caminaré (est. 1). \\ \\


{\changefont{cmss}{bx}{n} Salmo:} \> 17 ant. 1 ``Te amo, Señor, mi fuerza...'' (estr. 1, 2 o 4). \\ 
\> (antífona de reemplazo: Sal 29 ant. 1 ``Te glorifico, Señor, ...'' del P. Bevilacqua) \\ \\

{\changefont{cmss}{bx}{n} Ofrendas:} \> Este es nuestro pan\footnotemark[1], Te ofrecemos Padre nuestro (nuevo)\footnotemark[1], Mira nuestra ofrenda,   \\
\> Bendito seas, Te presentamos. \\ \\ 

{\changefont{cmss}{bx}{n} Comunión:} \> Más cerca oh Dios\footnotemark[2], En memoria tuya\footnotemark[2], Jesús la imagen de Dios Padre\footnotemark[3], \\
\> En la postrera cena,  Como Cristo nos amó\footnotemark[2], Jesucristo danos de este pan.  \\  \\

{\changefont{cmss}{bx}{n} Post-com.:} \> Tu fidelidad, Adoremos a Dios, Mirarte solo a ti.\\ \\

{\changefont{cmss}{bx}{n} Salida:} \> Madre de nuestro pueblo (est. 1 y 9)\footnotemark[1], En medio de los pueblos, Mi camino eres tú \\
\> Santa María del camino, Salve oh Reina. \\ \\


\end{tabbing}

\vspace{-15mm}

\footnotetext[1]{\emph{Cruz de Cristo} se acostumbra cantar en cuaresma, pero su texto es perfecto para este día. Los demás cantos también hacen referencia directa a la cruz del Señor.   }
\footnotetext[2]{Cfr. \emph{Más cerca oh Dios} (est. 1 y 5) y \emph{Como Cristo nos amó} (est. 3) con Jn 3,13-17; cfr. \emph{Jesús la imagen de Dios Padre} con Fil 2,6-11; cfr. \emph{En memoria tuya} (est. 5 y 6) con Fil. 2,6-11. }



%------------------------
\newpage
\setcounter{footnote}{0}
\thispagestyle{fancy}
\fancyhead{}
\fancyfoot{}
%\footskip=-1cm

\lhead{\changefont{cmss}{bx}{n} \small Revisión 2013}
\chead{\changefont{cmss}{bx}{n} \small 1 de noviembre - Solemnidad de todos 
los santos}
\rhead{\changefont{cmss}{bx}{n} \small Ciclo B}
\rfoot{\changefont{cmss}{bx}{n}\large\thepage}

\vspace*{-11mm}

\begin{center}
{\large\it Las Bienaventuranzas }
\end{center}

\vspace{-3mm}

\begin{tabbing}

{\changefont{cmss}{bx}{n} Entrada:\ \ \ \ \ }\=  Cante la 
Iglesia\footnotemark[1], Brilló la luz\footnotemark[1], 
Alabaré\footnotemark[1], Iglesia peregrina. \\ \\


{\changefont{cmss}{bx}{n} Salmo:} \> 23 ant. 2 ``Felices los que son 
fieles...'' (estr. 1, 2 o 3). \\ 
\> (antífona de reemplazo: Sal 24 ant. 1 ``A ti elevo mi alma, ...'') \\ \\

{\changefont{cmss}{bx}{n} Ofrendas:} \> Recibe oh Dios el pan\footnotemark[2], 
Bendeciré al Señor, Te presentamos. \\  \\

{\changefont{cmss}{bx}{n} Comunión:} \> Brilló la luz\footnotemark[1], 
Cuerpo y Sangre de Jesús, Vayamos a la mesa, Es mi Padre.\\ \\

{\changefont{cmss}{bx}{n} Post-com.:} \> La misericordia del Señor (Taizé), Al 
atardecer de la vida, Adoremos a Dios.\\ \\

{\changefont{cmss}{bx}{n} Salida:} \> Cante la 
Iglesia\footnotemark[1], Anunciaremos tu Reino, Canción del testigo, \\
\> Salve oh Reina\footnotemark[3], En medio de los pueblos. \\ \\


\end{tabbing}

\vspace{-15mm}

\footnotetext[1]{\emph{Cante la Iglesia} es un canto especial para este día; 
cfr. \emph{Brilló la luz} con Mt 5,1-12; cfr. \emph{Alabaré} (est. 1) con 
\mbox{Apoc 72-14}. }
\footnotetext[2]{ \emph{Recibe oh Dios el pan} (est. 3) pide por los difuntos. }
\footnotetext[3]{Cfr. \emph{Salve oh Reina} (est. 2) habla del 
final de este ``destierro''.}



%------------------------
\newpage
\setcounter{footnote}{0}
\thispagestyle{fancy}
\fancyhead{}
\fancyfoot{}
%\footskip=-1cm

\lhead{\changefont{cmss}{bx}{n} \small Revisión 2013}
\chead{\changefont{cmss}{bx}{n} \small 2 de noviembre - Conmemoración de todos 
los fieles difuntos}
\rhead{\changefont{cmss}{bx}{n} \small Ciclo B}
\rfoot{\changefont{cmss}{bx}{n}\large\thepage}

\vspace*{-11mm}

\begin{center}
{\large\it El anuncio de la resurrección }
\end{center}

\vspace{-3mm}

\begin{tabbing}

{\changefont{cmss}{bx}{n} Entrada:\ \ \ \ \ }\=  Hacia ti morada 
santa\footnotemark[1], Brilló la luz\footnotemark[1], Sal 129 (ant. 1), Sal 83 
(ant. 1), \\
\> Juntos como hermanos (est. 2). \\ \\


{\changefont{cmss}{bx}{n} Salmo:} \> 26 ant. 2 ``El Señor es mi luz, mi 
salvación...'' (estr. 1, 3, 5 o 7). \\ \\

{\changefont{cmss}{bx}{n} Ofrendas:} \> Recibe oh Dios el pan\footnotemark[2], 
Sé como el grano de trigo\footnotemark[2], Mira nuestra ofrenda. \\  \\

{\changefont{cmss}{bx}{n} Comunión:} \> Yo soy el pan de vida\footnotemark[3], 
Más cerca oh Dios\footnotemark[3], Brilló la luz\footnotemark[1], Es mi Padre,\\
\> Vayamos a la mesa. \\  \\

{\changefont{cmss}{bx}{n} Post-com.:} \> Mirarte sólo a ti\footnotemark[4], La 
misericordia del Señor (Taizé), Nada te turbe (Taizé).\\ \\

{\changefont{cmss}{bx}{n} Salida:} \> Soy peregrino\footnotemark[4], Salve oh 
Reina\footnotemark[4], En medio de los pueblos. \\ \\


\end{tabbing}

\vspace{-15mm}

\footnotetext[1]{Cfr. \emph{Hacia ti morada santa} se acostumbra para misas por 
los difuntos; \emph{Brilló la luz} es muy apropiada por las bienaventuranzas. }
\footnotetext[2]{ \emph{Recibe oh Dios el pan} (est. 3) pide por los difuntos; 
\emph{Sé como el grano de trigo} (est. 4) habla de la vuelta al Padre. }
\footnotetext[3]{Cfr. \emph{Yo soy el pan de vida} con ant. comunión (Jn 
11,25-26); \emph{Más cerca oh Dios} (est. 3) pide morar cerca del Señor.  }
\footnotetext[4]{Cfr. \emph{Mirarte sólo a ti} con Sal 26,8.13; cfr. \emph{Soy 
peregrino} (est. 4) con Apoc 21,2-5; \emph{Salve oh Reina} (est. 2) habla del 
final de este ``destierro''.}



%------------------------
\newpage
\setcounter{footnote}{0}
\thispagestyle{fancy}
\fancyhead{}
\fancyfoot{}
%\footskip=-1cm

\lhead{\changefont{cmss}{bx}{n} \small Revisión 2013}
\chead{\changefont{cmss}{bx}{n} \small Fiesta del 9 de noviembre - Dedicación de San Juan de Letrán}
\rhead{\changefont{cmss}{bx}{n} \small Ciclo B}
\rfoot{\changefont{cmss}{bx}{n}\large\thepage}

\vspace*{-11mm}

\begin{center}
{\large\it La expulsión de los vendedores del Templo }
\end{center}

\vspace{-3mm}

\begin{tabbing}

{\changefont{cmss}{bx}{n} Entrada:\ \ \ \ \ }\=  Qué alegría\footnotemark[1], Pueblo de Dios\footnotemark[1], Pueblo de Reyes\footnotemark[1], Iglesia peregrina de Dios. \\ \\


{\changefont{cmss}{bx}{n} Salmo:} \> 94 ant. 1 ``Adoremos al Señor...'' (estr. 1 y 3). \\ \\

{\changefont{cmss}{bx}{n} Ofrendas:} \> Al altar del Señor, Te ofrecemos oh Señor, Te presentamos, Bendeciré al Señor. \\  \\

{\changefont{cmss}{bx}{n} Comunión:} \> Vayamos a la mesa\footnotemark[2], Cuerpo y sangre de Jesús\footnotemark[2], Este es mi cuerpo\footnotemark[2], \\
\> Yo soy el camino\footnotemark[2]. \\ \\

{\changefont{cmss}{bx}{n} Post-com.:} \> Adoremos a Dios, Alabe todo el mundo (Taizé), Cuantas gracias te debemos, \\
\> Si el mismo pan comimos.\\ \\

{\changefont{cmss}{bx}{n} Salida:} \> Soy peregrino\footnotemark[3], Madre de los peregrinos, Madre de nuestro pueblo, \\
\> Salve María,  Santa María del camino. \\ \\


\end{tabbing}

\vspace{-15mm}

\footnotetext[1]{Cfr. \emph{Qué alegría} con ant. de entrada (Apoc 21,2); cfr. \emph{Pueblo de Dios} (estribillo) con Sal 45,9; cfr. \emph{Pueblo de reyes} (est. 2 y 5) con ant. de entrada (Apoc 21,2) y 1 Cor 3,11. }
\footnotetext[2]{ Todos estos son cantos eucarísticos. Es bueno aprovechar estas fiestas, que no tienen canto específico, para cantarlos.  }
\footnotetext[3]{Cfr. \emph{Soy peregrino} con (est. 4) con Apoc 21,2-5; el resto de los cantos son marianos por el mes de María.  }






%------------------------
\newpage
\setcounter{footnote}{0}
\thispagestyle{fancy}
\fancyhead{}
\fancyfoot{}
%\footskip=-1cm

\lhead{\changefont{cmss}{bx}{n} \small Revisión 2013}
\chead{\changefont{cmss}{bx}{n} \small Memoria del 11 de noviembre - San Martín 
de Tours}
\rhead{\changefont{cmss}{bx}{n} \small Ciclo B}
\rfoot{\changefont{cmss}{bx}{n}\large\thepage}

\vspace*{-11mm}

\begin{center}
{\large\it El juicio final }
\end{center}

\vspace{-3mm}

\begin{tabbing}

{\changefont{cmss}{bx}{n} Entrada:\ \ \ \ \ }\=  El sermón de 
la montaña\footnotemark[1], Brilló la luz\footnotemark[1], El Señor nos llama, 
Vine a alabar. \\ \\


{\changefont{cmss}{bx}{n} Salmo:} \> 118 ant. 2 ``Felices los que escuchan...'' 
(estr. 1 y 2). \\ 
\> (antífona de reemplazo: Sal 102 ant. 2 ``El amor del Señor, ...'') \\ \\

{\changefont{cmss}{bx}{n} Ofrendas:} \> Mira nuestra ofrenda, Este es nuestro 
pan, Te presentamos, Bendeciré al Señor. \\  \\

{\changefont{cmss}{bx}{n} Comunión:} \> Queremos ser Señor\footnotemark[2], 
Yo soy el camino, Creo en ti (Más cerca oh Dios). \\ \\

{\changefont{cmss}{bx}{n} Post-com.:} \> Al atardecer de la vida, Nada te turbe 
(Taizé), Mirarte sólo a ti, \\
\> Si el mismo pan comimos.\\ \\

{\changefont{cmss}{bx}{n} Salida:} \> Santa María del camino, Madre de 
los peregrinos, Madre de nuestro pueblo. \\ \\


\end{tabbing}

\vspace{-15mm}

\footnotetext[1]{\emph{El sermón de la montaña} y \emph{Brilló la luz} 
corresponden a las bienaventuranzas, muy apropiado a la vida de San Martín de 
Tours (y al juicio final). }
\footnotetext[2]{ \emph{Queremos ser Señor} está en línea con la vida del 
santo al proclamar ``queremos ser Señor servidores de verdad''; cfr. \emph{Al 
altardecer de la vida} (estr. 1) con Mt 25,31-40.  }










%------------------------
\newpage
\thispagestyle{empty}

%\begin{center}
%{\changefont{cmss}{bx}{n} \Huge VADEMECUM}\\
%\vspace{6mm}
%{\changefont{cmss}{bx}{n} \Huge de cantos litúrgicos}
%\end{center}

\vspace*{30mm}

\begin{center}
{\changefont{cmss}{bx}{n} \Huge CICLO C  (Lucas)}
\end{center}
%------------------------
\newpage
\thispagestyle{empty}


%------------------------
\newpage
\setcounter{footnote}{0}
\thispagestyle{fancy}
\fancyhead{}
\fancyfoot{}
%\footskip=-1cm

\lhead{\changefont{cmss}{bx}{n} \small Revisión 2013}
\chead{\changefont{cmss}{bx}{n} \small Domingo I - Adviento}
\rhead{\changefont{cmss}{bx}{n} \small Ciclo C}
\rfoot{\changefont{cmss}{bx}{n}\large\thepage}

\begin{center}
{\large\it La manifestación gloriosa del Hijo del hombre}
\end{center}

\vspace{3mm}

\begin{tabbing}

{\changefont{cmss}{bx}{n} Entrada:\ \ \ \ \ }\= Juntos como hermanos (estr. 2)\footnotemark[1], Despertemos llega Cristo, Toda la tierra espera, \\ 
 \> Señor a ti clamamos (se puede usar para el encendido de la corona de Adviento). \\ \\

{\changefont{cmss}{bx}{n} Salmo:} \> 24 ant. 1 ``!`A ti elevo mi alma, a ti mi Dios y Señor!'' (estr. 3,5 o 6). \\ \\

{\changefont{cmss}{bx}{n} Ofrendas:} \> Padre nuestro recibid\footnotemark[2], Pan de vida y bebida de luz\footnotemark[2], Toda la tierra espera.\\ \\

{\changefont{cmss}{bx}{n} Comunión:} \> Yo soy el camino, Bendeciré al Señor.\\ \\

{\changefont{cmss}{bx}{n} Post-com.:} \> Al atardecer de la vida\footnotemark[3], Nada te turbe (de Taizé).\\ \\

{\changefont{cmss}{bx}{n} Salida:} \> Santa María del camino.  \\  \\

\end{tabbing}

\footnotetext[1]{Ver comentario a \textbf{Juntos como hermanos}. }
\footnotetext[2]{Se hacen eco de la 2$^\circ$ lectura (1 Tes 3,12). }
\footnotetext[3]{Va bien con el evangelio (Lc 21,36).}
%------------------------
\newpage
\setcounter{footnote}{0}
\thispagestyle{fancy}
\fancyhead{}
\fancyfoot{}
%\footskip=-1cm


\lhead{\changefont{cmss}{bx}{n} \small Revisión 2013}
\chead{\changefont{cmss}{bx}{n} \small Domingo II - Adviento}
\rhead{\changefont{cmss}{bx}{n} \small Ciclo C}
\rfoot{\changefont{cmss}{bx}{n}\large\thepage}

\begin{center}
{\large\it La predicación de Juan el Bautista}
\end{center}

\vspace{3mm}


\begin{tabbing}

{\changefont{cmss}{bx}{n} Entrada:\ \ \ \ \ }\= Toda la tierra espera (estr. 1 y 3)\footnotemark[1], Despertemos llega Cristo, \\ 
 \> Señor a ti clamamos (se puede usar para el encendido de la corona de Adviento). \\ \\

{\changefont{cmss}{bx}{n} Salmo:} \> 125 ant. 1 ``Los que siembran entre lágrimas, cantando cosecharán'' (estr. 1, 2 o 3). \\
 \> (antífona de reemplazo: Sal 24 ant. 1 ``!`A ti elevo mi alma, a ti mi Dios y Señor!'') \\ \\

{\changefont{cmss}{bx}{n} Ofrendas:} \> Toda la tierra espera (si no se usa de entrada), Saber que vendrás\footnotemark[2].\\ \\

{\changefont{cmss}{bx}{n} Comunión:} \> Creo en ti Señor\footnotemark[3], Mensajero de la paz\footnotemark[3], Yo soy el camino.\\ \\

{\changefont{cmss}{bx}{n} Post-com.:} \> Adoremos a Dios (estr. 3), Mirarte sólo a ti, Nada te turbe, El Señor es mi fortaleza.\\ \\

{\changefont{cmss}{bx}{n} Salida:} \> Madre de los peregrinos.  \\  \\

\end{tabbing}

\vspace{-3mm}

\footnotetext[1]{El evangelio hace referencia a Is 40,3-5 (est. 1 y 3 de \textbf{Toda la tierra espera}). }
\footnotetext[2]{Ver inconvenientes detallados en el comentario a \textbf{Saber que vendrás}. }
\footnotetext[3]{Debido a los textos del evangelio Lc 3,6 y Lc 3,3, respectivamente.}

%------------------------
\newpage
\setcounter{footnote}{0}
\thispagestyle{fancy}
\fancyhead{}
\fancyfoot{}
%\footskip=-1cm


\lhead{\changefont{cmss}{bx}{n} \small Revisión 2013}
\chead{\changefont{cmss}{bx}{n} \small Solemnidad del 8 de diciembre}
\rhead{\changefont{cmss}{bx}{n} \small Ciclo C}
\rfoot{\changefont{cmss}{bx}{n}\large\thepage}

\begin{center}
{\large\it Inmaculada Concepción de María}
\end{center}

%\vspace{3mm}


\begin{tabbing}

{\changefont{cmss}{bx}{n} Entrada:\ \ \ \ \ }\= Feliz de ti María, La Virgen María nos reúne. \\ \\

{\changefont{cmss}{bx}{n} Salmo:} \> 97 ant. 1 ``Cantemos al Señor un canto nuevo, aleluia...'' (estr. 1, 2 o 3). \\ 
\> (antífona de reemplazo: Sal 95 ``Cantemos al Señor un canto nuevo''  \\
\> del P. José Bevilacqua)\\ \\

{\changefont{cmss}{bx}{n} Ofrendas:} \> Coplas de Yaraví, Bendeciré al Señor.   \\ \\


{\changefont{cmss}{bx}{n} Comunión:} \>  Mi alma glorifica, Jesucristo danos de este pan\footnotemark[1].\\ \\

{\changefont{cmss}{bx}{n} Post-com.:} \> Bendita sea tu pureza, Quiero decir que sí.\\ \\

{\changefont{cmss}{bx}{n} Salida:} \> Toda de Dios, Los cielos, la tierra, El ángel vino de los cielos,\\
\> Oh Santísima, Oh María, Madre de nuestro pueblo. \\  \\

\end{tabbing}

\vspace{-10mm}

\footnotetext[1]{Porque en su última estrofa se refiere a María. }


%------------------------
\newpage
\setcounter{footnote}{0}
\thispagestyle{fancy}
\fancyhead{}
\fancyfoot{}
%\footskip=-1cm

\lhead{\changefont{cmss}{bx}{n} \small Revisión 2013}
\chead{\changefont{cmss}{bx}{n} \small Domingo III - Adviento}
\rhead{\changefont{cmss}{bx}{n} \small Ciclo C}
\rfoot{\changefont{cmss}{bx}{n}\large\thepage}

\begin{center}
{\large\it (Domingo de Gaudete) La predicación de Juan el Bautista }
\end{center}

\vspace{3mm}


\begin{tabbing}

{\changefont{cmss}{bx}{n} Entrada:\ \ \ \ \ }\= Que alegría (Sal 121)\footnotemark[1], Vienen con alegría\footnotemark[1], Despertemos llega Cristo, \\ 
 \> Señor a ti clamamos (se puede usar para el encendido de la corona de Adviento). \\ \\

{\changefont{cmss}{bx}{n} Salmo:} \> 84 ant. 1 ``Señor, revélanos tu amor, concédenos tu salvación'' (estr 1, 4 o 6). \\ \\

{\changefont{cmss}{bx}{n} Ofrendas:} \> Toda la tierra espera, Pan de vida y bebida de luz, Al altar del Señor.\\ \\

{\changefont{cmss}{bx}{n} Comunión:} \> El viñador\footnotemark[2], Creo en ti Señor, Yo soy el camino, Bendeciré al Señor.\\ \\

{\changefont{cmss}{bx}{n} Post-com.:} \> Tan cerca de mí\footnotemark[3], El Señor es mi fortaleza.\\ \\

{\changefont{cmss}{bx}{n} Salida:} \> Salve María, Salve oh Reina, Quiero decir que sí.  \\  \\

\end{tabbing}

\vspace{-3mm}

\footnotetext[1]{Por ser domingo de \emph{Gaudete}. }
\footnotetext[2]{Cfr. Lc 3,17 del evangelio del día. }
\footnotetext[3]{Porque se hace el anuncio de que el Salvador está cerca.}

%------------------------
\newpage
\setcounter{footnote}{0}
\thispagestyle{fancy}
\fancyhead{}
\fancyfoot{}
%\footskip=-1cm

\lhead{\changefont{cmss}{bx}{n} \small Revisión 2013}
\chead{\changefont{cmss}{bx}{n} \small Domingo IV - Adviento}
\rhead{\changefont{cmss}{bx}{n} \small Ciclo C}
\rfoot{\changefont{cmss}{bx}{n}\large\thepage}

\vspace*{-8mm}

\begin{center}
{\large\it La visita de María a Isabel }
\end{center}

%\vspace{1mm}


\begin{tabbing}

{\changefont{cmss}{bx}{n} Entrada:\ \ \ \ \ }\= Despertemos llega Cristo, Señor a ti clamamos, Toda la tierra espera.  \\ 
 \> (Señor a ti clamamos se puede usar para el encendido de la corona de Adviento) \\ \\

{\changefont{cmss}{bx}{n} Salmo:} \> 79 ant. 3 ``Míranos, Señor, ven a salvarnos...'' (est. 1, 6 o 7). \\ 
\> (antífona de reemplazo: Sal 84 ant. 1 ``Señor, revélanos tu amor...'') \\ \\

{\changefont{cmss}{bx}{n} Ofrendas:} \> Saber que vendrás, Toda la tierra espera, Pan de vida y bebida de luz,\\
\>  Te ofrecemos Padre nuestro (II)\footnotemark[3].\\ \\

{\changefont{cmss}{bx}{n} Comunión:} \> Mi alma glorifica, El viñador\footnotemark[1], Jesucristo danos de este pan\footnotemark[2], Creo en ti Señor, \\
\> Bendeciré al Señor. \\ \\ 

{\changefont{cmss}{bx}{n} Post-com.:} \> Adoremos a Dios (estr. 1), Alabe todo el mundo (de Taizé).\\ \\

{\changefont{cmss}{bx}{n} Salida:} \> Santa María del camino\footnotemark[4], Madre de los peregrinos.  \\  \\

\end{tabbing}

\vspace{-12mm}

\footnotetext[1]{En referencia al salmo 79 ``ven a visitar tu vid''. }
\footnotetext[2]{Porque en su última estrofa habla de María.}
\footnotetext[3]{Debido a su texto de Adviento en la estr. 3, que dice: ``anhelando libertad.'' }
\footnotetext[4]{Ver comentario de \textbf{Santa María del camino}. }

%------------------------
\newpage
\setcounter{footnote}{0}
\thispagestyle{fancy}
\fancyhead{}
\fancyfoot{}
%\footskip=-1cm

\lhead{\changefont{cmss}{bx}{n} \small Revisión 2013}
\chead{\changefont{cmss}{bx}{n} \small Solemnidad del 25 de diciembre - Natividad del Señor}
\rhead{\changefont{cmss}{bx}{n} \small Ciclo C}
\rfoot{\changefont{cmss}{bx}{n}\large\thepage}

\begin{center}
{\large\it Prólogo al evangelio según San Juan }
\end{center}

\vspace{1mm}


\begin{tabbing}

{\changefont{cmss}{bx}{n} Entrada:\ \ \ \ \ }\= Ha nacido el Rey del cielo\footnotemark[1], Gloria eterna\footnotemark[2], Mundo feliz\footnotemark[1], Pastores de la montaña\footnotemark[1], \\
\> Cristianos vayamos\footnotemark[1], Vamos pastorcitos\footnotemark[1], Ya llegó la Nochebuena\footnotemark[1].  \\  \\

{\changefont{cmss}{bx}{n} Salmo:} \> 97 con ant. del Sal 95  ``Hoy nos ha nacido un salvador...'' (est. 1, 5 o 6). \\ \\

{\changefont{cmss}{bx}{n} Ofrendas:} \> Cristianos vayamos\footnotemark[1], Vamos pastorcitos\footnotemark[1], Ya llegó la Nochebuena\footnotemark[1].\\ \\

{\changefont{cmss}{bx}{n} Comunión:} \> Pastores de la montaña\footnotemark[1], El pan de Belén\footnotemark[3], Noche de paz\footnotemark[3],\\
\> La peregrinación (A la huella)\footnotemark[3], Zamba de la Navidad. \\ \\

{\changefont{cmss}{bx}{n} Post-com.:} \> Noche anunciada\footnotemark[3], ?`Qué niño es éste?, Adoremos a Dios (estr. 1), \\
\> Alabe todo el mundo (de Taizé).\\ \\

{\changefont{cmss}{bx}{n} Salida:} \> Gloria eterna\footnotemark[2], Pastores de la montaña\footnotemark[1], Mundo feliz\footnotemark[1], Entonen tiernos cánticos.  \\  \\

\end{tabbing}

\vspace{-14mm}

\footnotetext[1]{Cfr. \emph{Ya llegó la Nochebuena} con Mt 1,25, \emph{Mundo feliz} con Sal 97, \emph{Pastores de la montaña} con Lc 2,8-16, \\ \emph{Vamos pastorcitos} con Lc 2,8-20 y \emph{Cristianos vayamos} con Lc 2,15-18. }
\footnotetext[2]{Ver comentario a \textbf{Gloria eterna}. Cfr. con Is 9,1-2 (1$^\circ$ lectura de la misa de la noche). }
\footnotetext[3]{Cfr. \emph{El pan de Belén} con Mt 1,23, \emph{La peregrinación} con Lc 2,6-7, \emph{Noche de paz} y \emph{Noche anunciada} con Lc 2,6-20.}

%------------------------
\newpage
\setcounter{footnote}{0}
\thispagestyle{fancy}
\fancyhead{}
\fancyfoot{}
%\footskip=-1cm

\lhead{\changefont{cmss}{bx}{n} \small Revisión 2013}
\chead{\changefont{cmss}{bx}{n} \small Domingo de la octava de Navidad}
\rhead{\changefont{cmss}{bx}{n} \small Ciclo C}
\rfoot{\changefont{cmss}{bx}{n}\large\thepage}

\begin{center}
{\large\it (La Sagrada Familia) Jesús entre los doctores de la Ley  }
\end{center}

\vspace{1mm}


\begin{tabbing}

{\changefont{cmss}{bx}{n} Entrada:\ \ \ \ \ }\= Ya llegó la Nochebuena (texto modificado), Gloria eterna, Mundo feliz, \\ 
\> Ha nacido el Rey del cielo, Pastores de la montaña, Cristianos vayamos,   \\ 
\> Vamos pastorcitos.\\ \\

{\changefont{cmss}{bx}{n} Salmo:} \> 127 ant. 1 ``!`Feliz quien ama al Señor...!'' (est. 1, 2). \\ \\

{\changefont{cmss}{bx}{n} Ofrendas:} \> Cristianos vayamos, Vamos pastorcitos, Ya llegó la Nochebuena.\\ \\

{\changefont{cmss}{bx}{n} Comunión:} \> La peregrinación (A la huella)\footnotemark[1], Noche de paz, El pan de Belén,  \\
\>  Pastores de la montaña, Zamba de la Navidad. \\ \\

{\changefont{cmss}{bx}{n} Post-com.:} \> Noche anunciada, ?`Qué niño es éste?, Adoremos a Dios (estr. 3), \\
\> Alabe todo el mundo (de Taizé).\\ \\

{\changefont{cmss}{bx}{n} Salida:} \> Madre de nuestro pueblo (estr. 1,4,5,6)\footnotemark[2], Gloria eterna, Pastores de la montaña,\\
\>  Mundo feliz, Entonen tiernos cánticos.  \\  \\

\end{tabbing}

\vspace{-12mm}

\footnotetext[1]{Por tratarse de la Fiesta de La Sagrada Familia. }
\footnotetext[2]{Cfr. con Lc 2,39-52. }

%------------------------
\newpage
\setcounter{footnote}{0}
\thispagestyle{fancy}
\fancyhead{}
\fancyfoot{}
%\footskip=-1cm

\lhead{\changefont{cmss}{bx}{n} \small Revisión 2013}
\chead{\changefont{cmss}{bx}{n} \small Solemnidad del 1$^\circ$ de enero}
\rhead{\changefont{cmss}{bx}{n} \small Ciclo C}
\rfoot{\changefont{cmss}{bx}{n}\large\thepage}

\begin{center}
{\large\it Santa María Madre de Dios }
\end{center}

%\vspace{1mm}


\begin{tabbing}

{\changefont{cmss}{bx}{n} Entrada:\ \ \ \ \ }\= María Madre de Dios\footnotemark[1], Feliz de ti María (est. 1,2)\footnotemark[1], La Virgen María nos reúne\footnotemark[1], \\ 
\> Gloria eterna\footnotemark[2].\\ \\

{\changefont{cmss}{bx}{n} Salmo:} \> 66 ant. 1 ``!`A tí, Señor, te alabe la tierra...!'' (est. 1, 2). \\
\> (antífona de reemplazo: Sal 144 ant. 1 ``Te alabamos, Señor, ...'') \\ \\

{\changefont{cmss}{bx}{n} Ofrendas:} \> Bendeciré al Señor, Cristianos vayamos, Vamos pastorcitos.\\ \\

{\changefont{cmss}{bx}{n} Comunión:} \> Simple oración\footnotemark[3], La peregrinación (A la huella), El pan de Belén,  \\
\>  Pastores de la montaña. \\ \\

{\changefont{cmss}{bx}{n} Post-com.:} \> Quiero decir que sí\footnotemark[1], Noche anunciada, Adoremos a Dios (estr. 3), \\
\> Alabe todo el mundo (de Taizé).\\ \\

{\changefont{cmss}{bx}{n} Salida:} \> Salve María\footnotemark[1], Oh María\footnotemark[1], Junto a ti María\footnotemark[1],  \\
\> Madre de nuestro pueblo (estr. 1,4,5)\footnotemark[1], Gloria eterna\footnotemark[2].\\ \\

\end{tabbing}

\vspace{-14mm}
\footnotetext[1]{Por ser la solemnidad de Santa María Madre de Dios.}
\footnotetext[2]{Debido a que la antífona de entrada del día coincide con \emph{Gloria eterna} en que hacen referencia a Is 9,2. }
\footnotetext[3]{Debido a que es la Jornada mundial por la paz, en concordancia en el estribillo de \emph{Simple oración}. }

%------------------------
\newpage
\setcounter{footnote}{0}
\thispagestyle{fancy}
\fancyhead{}
\fancyfoot{}
%\footskip=-1cm

\lhead{\changefont{cmss}{bx}{n} \small Revisión 2013}
\chead{\changefont{cmss}{bx}{n} \small Domingo II - Navidad}
\rhead{\changefont{cmss}{bx}{n} \small Ciclo C}
\rfoot{\changefont{cmss}{bx}{n}\large\thepage}

\begin{center}
{\large\it Prólogo al evangelio según San Juan }
\end{center}

\vspace{1mm}


\begin{tabbing}

{\changefont{cmss}{bx}{n} Entrada\footnotemark[1]:\ \ \ \ \ }\= Ha nacido el Rey del cielo, Gloria eterna, Mundo feliz, Pastores de la montaña, \\
\> Cristianos vayamos, Vamos pastorcitos.  \\  \\

{\changefont{cmss}{bx}{n} Salmo:} \> 147 ant. 1 ``Glorifica al Señor Jerusalén...'' (est. 1, 2 o 4). \\ \\

{\changefont{cmss}{bx}{n} Ofrendas:} \> Vamos pastorcitos, Cristianos vayamos, Ya llegó la Nochebuena.\\ \\

{\changefont{cmss}{bx}{n} Comunión:} \> El pan de Belén, Pastores de la montaña, La peregrinación (A la huella). \\  \\

{\changefont{cmss}{bx}{n} Post-com.:} \> Adoremos a Dios (estr. 1), Alabe todo el mundo (de Taizé), Noche anunciada.\\ \\

{\changefont{cmss}{bx}{n} Salida:} \> Gloria eterna, Pastores de la montaña, Mundo feliz, Entonen tiernos cánticos, \\
\>  Oh Santísima.  \\  \\

\end{tabbing}

\vspace{-11mm}

\footnotetext[1]{Se repiten muchos de los cantos del día de Navidad. El evangelio del día es el mismo que el de Navidad. }

%------------------------
\newpage
\setcounter{footnote}{0}
\thispagestyle{fancy}
\fancyhead{}
\fancyfoot{}
%\footskip=-1cm

\lhead{\changefont{cmss}{bx}{n} \small Revisión 2013}
\chead{\changefont{cmss}{bx}{n} \small Solemnidad del 6 de enero - Navidad }
\rhead{\changefont{cmss}{bx}{n} \small Ciclo C}
\rfoot{\changefont{cmss}{bx}{n}\large\thepage}

\begin{center}
{\large\it (La Epifanía del Señor) La visita de los magos }
\end{center}

%\vspace{1mm}

\begin{tabbing}

{\changefont{cmss}{bx}{n} Entrada:\ \ \ \ \ }\= Mundo feliz, Ha nacido el Rey del cielo, Gloria eterna,    \\ 
\> Ya llegó la Nochebuena (texto modificado)\footnotemark[1], Cristianos vayamos\footnotemark[2].   \\ \\

{\changefont{cmss}{bx}{n} Salmo:} \> 71 ant. 1 ``Tú eres, Señor, el único Rey...'' (est. 1, 4 , 5 o 6). \\
\> (antífona de reemplazo: Sal 144 ant. 1 ``Te alabamos Señor...'') \\ \\

{\changefont{cmss}{bx}{n} Ofrendas:} \> Cristianos vayamos\footnotemark[2], Ya llegó la Nochebuena\footnotemark[1], Vamos pastorcitos.\\ \\

{\changefont{cmss}{bx}{n} Comunión:} \> Los reyes magos, El pan de Belén, La peregrinación (A la huella),    \\
\>  Pastores de la montaña, Zamba de la Navidad, Noche de paz. \\ \\

{\changefont{cmss}{bx}{n} Post-com.:} \> Adoremos a Dios (estr. 1), Alabe todo el mundo (de Taizé), ?`Qué niño es éste?,   \\
\> Noche anunciada.\\ \\

{\changefont{cmss}{bx}{n} Salida:} \> Gloria eterna, Pastores de la montaña, Mundo feliz, Entonen tiernos cánticos,  \\
\>  Madre de nuestro pueblo (estr. 1 y 4).  \\  \\

\end{tabbing}

\vspace{-11mm}

\footnotetext[1]{Con el texto modificado: ``Vamos todos a esperarlo'' por ``Vamos todos a adorarlo'', y ``va a nacer'' por ``ya nació''. }
\footnotetext[2]{Cfr. con Mt 2,2 y Mt 2,11. }

%------------------------
\newpage
\setcounter{footnote}{0}
\thispagestyle{fancy}
\fancyhead{}
\fancyfoot{}
%\footskip=-1cm

\lhead{\changefont{cmss}{bx}{n} \small Revisión 2013}
\chead{\changefont{cmss}{bx}{n} \small Domingo después del 6 de enero - Navidad }
\rhead{\changefont{cmss}{bx}{n} \small Ciclo C}
\rfoot{\changefont{cmss}{bx}{n}\large\thepage}

\begin{center}
{\large\it El bautismo de Jesús }
\end{center}

%\vspace{1mm}

\begin{tabbing}

{\changefont{cmss}{bx}{n} Entrada:\ \ \ \ \ }\= Gloria eterna,  Brilló la luz, Un sólo Señor (estr. 3)\footnotemark[1], Pueblo de Dios,    \\ 
\> Vine a alabar.   \\ \\

{\changefont{cmss}{bx}{n} Salmo:} \> 32 ant. 1 ``Cantemos todos al Señor, aleluia...'' (est. 1, 3 o 8). \\
\> (antífona de reemplazo: Sal 26 ant. 1 ``Cantaré y celebraré al Señor.'') \\ \\

{\changefont{cmss}{bx}{n} Ofrendas:} \> Padre nuestro recibid, Te ofrecemos oh Señor, Te presentamos, Bendito seas.\\ \\

{\changefont{cmss}{bx}{n} Comunión:} \> Brilló la luz, Yo soy el camino, Bendeciré al Señor, Como Cristo nos amó.  \\  \\

{\changefont{cmss}{bx}{n} Post-com.:} \> Tu fidelidad, Tan cerca de mí, Adoremos a Dios, Alabe todo el mundo (de Taizé).   \\ \\

{\changefont{cmss}{bx}{n} Salida:} \> Gloria eterna, Mi camino eres tú, Canción del testigo, Anunciaremos tu Reino. \\  \\

\end{tabbing}

\vspace{-11mm}

\footnotetext[1]{Debido a que habla de ``un solo Dios y Padre'', pero siempre que quede claro en la asamblea que el bautismo del Señor (bautismo de penitencia) no es igual a nuestro bautismo. }

%------------------------
\newpage
\setcounter{footnote}{0}
\thispagestyle{fancy}
\fancyhead{}
\fancyfoot{}
%\footskip=-1cm

\lhead{\changefont{cmss}{bx}{n} \small Revisión 2013}
\chead{\changefont{cmss}{bx}{n} \small Miércoles de Ceniza}
\rhead{\changefont{cmss}{bx}{n} \small Ciclo C}
\rfoot{\changefont{cmss}{bx}{n}\large\thepage}

\begin{center}
{\large\it La limosna, la oración, el ayuno }
\end{center}

\vspace{1mm}


\begin{tabbing}

{\changefont{cmss}{bx}{n} Entrada:\ \ \ \ \ }\= Dice el Señor conviértanse\footnotemark[1], Perdón Señor, Sí me levantaré.  \\ \\

{\changefont{cmss}{bx}{n} Salmo:} \> 50 ant. 1 ``Piedad, Señor, pecamos contra ti'' (est. 1, 2, 6, 7 u 8). \\ \\

{\changefont{cmss}{bx}{n} Cenizas:} \> Perdón oh Dios mío, Perdón Señor, Sí me levantaré, Vuelve a mí. \\ \\

{\changefont{cmss}{bx}{n} Ofrendas:} \> Recibe oh Dios el pan, Te ofrecemos Padre nuestro (vidala), \\
\>  Recibe oh Padre Santo, Coplas de Yaraví. \\ \\

{\changefont{cmss}{bx}{n} Comunión:} \> Oh buen Jesús, Sí me levantaré, Bienaventurados los pobres, \\
\>  Creo en ti Señor, Vuelve a mí.\\ \\

{\changefont{cmss}{bx}{n} Post-com.:} \> Vaso nuevo (El alfarero), Todos unidos, Nada te turbe (de Taizé).\\ \\

{\changefont{cmss}{bx}{n} Salida:} \> Soy peregrino, Virgen de la esperanza.  \\  \\

\end{tabbing}

\vspace{-11mm}

\footnotetext[1]{Aclamación al evangelio, pero que se puede usar de entrada (cfr. Joel 2, 12-18).  }

%------------------------
\newpage
\setcounter{footnote}{0}
\thispagestyle{fancy}
\fancyhead{}
\fancyfoot{}
%\footskip=-1cm

\lhead{\changefont{cmss}{bx}{n} \small Revisión 2013}
\chead{\changefont{cmss}{bx}{n} \small Domingo I - Cuaresma}
\rhead{\changefont{cmss}{bx}{n} \small Ciclo C}
\rfoot{\changefont{cmss}{bx}{n}\large\thepage}


\begin{center}
{\large\it Las tentaciones de Jesús en el desierto }
\end{center}

\vspace{-1mm}



\begin{tabbing}

{\changefont{cmss}{bx}{n} Entrada:\ \ \ \ \ }\= Perdón Señor\footnotemark[1], Juntos como hermanos, Sí me levantaré, Perdón oh Dios mío.\\ \\

{\changefont{cmss}{bx}{n} Salmo:} \> 24 ant. 1 ``A ti elevo mi alma...'' (estr. del salmo 90: 1, 6 o 7) \\ 
\> (antífona de reemplazo: Sal 15 ``Protégeme, Dios mío, porque en ti me refugio'' \\
\> del P. José Bevilacqua) \\ \\

{\changefont{cmss}{bx}{n} Ofrendas:} \> Este es nuestro pan\footnotemark[2], Recibe oh Dios el pan, Mira nuestra ofrenda,   \\
\>  Te ofrecemos Padre nuestro (vidala).  \\ \\

{\changefont{cmss}{bx}{n} Comunión:} \> Dios es fiel\footnotemark[3], Sí me levantaré, Creo en ti Señor, Bienaventurados los pobres, \\
\> Hambre de Dios, Oh buen Jesús. \\ \\

{\changefont{cmss}{bx}{n} Post-com.:} \> Tu fidelidad\footnotemark[3], Nada te turbe (de Taizé).\\ \\

{\changefont{cmss}{bx}{n} Salida:} \> Santa María del Amén\footnotemark[4], Soy peregrino, Virgen de la esperanza.  \\  \\

\end{tabbing}

\vspace{-14mm}

\footnotetext[1]{Cfr. estrofas 1 con la 1$^\circ$ lectura (Deut. 26,7-8).  }
\footnotetext[2]{Cfr. estrofas 2 con la 1$^\circ$ lectura (Deut. 26,1-11).  }
\footnotetext[3]{Por el contexto de la obra mesiánica de Jesús en favor de la Alianza.  }
\footnotetext[4]{Por habla de nuestras propias debilidades. }

%------------------------
\newpage
\setcounter{footnote}{0}
\thispagestyle{fancy}
\fancyhead{}
\fancyfoot{}
%\footskip=-1cm

\lhead{\changefont{cmss}{bx}{n} \small Revisión 2013}
\chead{\changefont{cmss}{bx}{n} \small Domingo II - Cuaresma}
\rhead{\changefont{cmss}{bx}{n} \small Ciclo C}
\rfoot{\changefont{cmss}{bx}{n}\large\thepage}

\begin{center}
{\large\it La transfiguración de Jesús }
\end{center}

\vspace{1mm}


\begin{tabbing}

{\changefont{cmss}{bx}{n} Entrada:\ \ \ \ \ }\= Brilló la luz, Cruz de Cristo\footnotemark[1], Juntos como hermanos, Perdón Señor,   \\
\> Sí me levantaré (estr. 3), Vuélvenos tu rostro (est. 1 y 3).\\ \\

{\changefont{cmss}{bx}{n} Salmo:} \> 26 ant. 2 ``El Señor es mi luz, mi salvación...'' (estrofas 1, 5 o 7)  \\ \\

{\changefont{cmss}{bx}{n} Ofrendas:} \> Pan de vida y bebida de luz, Este es nuestro pan, Recibe oh Dios el pan,  \\
\>  Te ofrecemos Padre nuestro (vidala). \\ \\

{\changefont{cmss}{bx}{n} Comunión:} \> En memoria tuya\footnotemark[2], Brilló la luz, Dios es fiel\footnotemark[2], Creo en ti Señor\footnotemark[3],   \\
\>   Vuelve a mí\footnotemark[2].\\ \\

{\changefont{cmss}{bx}{n} Post-com.:} \> Tu fidelidad\footnotemark[2], Más cerca oh Dios de ti\footnotemark[4].\\ \\

{\changefont{cmss}{bx}{n} Salida:} \> Madre de nuestro pueblo (estr. 9), Virgen de la esperanza.  \\  \\

\end{tabbing}

\vspace{-11mm}

\footnotetext[1]{Cfr. la presentación de la cruz con Lc 9,30-31 (Moisés y Elías hablando de la Pasión).  }
\footnotetext[2]{Por el contexto de la obra mesiánica de Jesús en favor de la Alianza.  }
\footnotetext[3]{Con el texto de \emph{Cantemos hermanos con amor}.  }
\footnotetext[4]{Texto tradicional de \emph{Creo en ti Señor}.  }

%------------------------
\newpage
\setcounter{footnote}{0}
\thispagestyle{fancy}
\fancyhead{}
\fancyfoot{}
%\footskip=-1cm

\lhead{\changefont{cmss}{bx}{n} \small Revisión 2013}
\chead{\changefont{cmss}{bx}{n} \small Domingo III - Cuaresma}
\rhead{\changefont{cmss}{bx}{n} \small Ciclo C}
\rfoot{\changefont{cmss}{bx}{n}\large\thepage}

\begin{center}
{\large\it La parábola de la higuera estéril}
\end{center}

\vspace{1mm}


\begin{tabbing}

{\changefont{cmss}{bx}{n} Entrada:\ \ \ \ \ }\= Vuélvenos tu rostro, Sí me levantaré, Perdón Señor, Perdón oh Dios mío.\\   \\

{\changefont{cmss}{bx}{n} Salmo:} \> 102 ant. 2 ``El amor del Señor permanece para siempre'' (estr. 1, 2, 3 o 5). \\ \\

{\changefont{cmss}{bx}{n} Ofrendas:} \> Recibe oh Dios eterno, Coplas de Yaraví, Recibe oh Dios el pan, \\ 
\> Señor Jesús.  \\ \\

{\changefont{cmss}{bx}{n} Comunión:} \> Bienaventurados los pobres, En memoria tuya.  \\ \\

{\changefont{cmss}{bx}{n} Post-com.:} \> Tu fidelidad, (silencio).\\ \\

{\changefont{cmss}{bx}{n} Salida:} \> Virgen de la esperanza (estr. 1,2,5), Soy peregrino, Santa María del Amén.  \\  \\

\end{tabbing}

\vspace{-11mm}

%------------------------
\newpage
\setcounter{footnote}{0}
\thispagestyle{fancy}
\fancyhead{}
\fancyfoot{}
%\footskip=-1cm

\lhead{\changefont{cmss}{bx}{n} \small Revisión 2013}
\chead{\changefont{cmss}{bx}{n} \small Domingo IV - Cuaresma}
\rhead{\changefont{cmss}{bx}{n} \small Ciclo C}
\rfoot{\changefont{cmss}{bx}{n}\large\thepage}

\begin{center}
{\large\it La parábola del hijo pródigo (o el padre misericordioso)}
\end{center}

\vspace{1mm}


\begin{tabbing}

{\changefont{cmss}{bx}{n} Entrada:\ \ \ \ \ }\= Qué alegría\footnotemark[1], Vienen con alegría\footnotemark[1], Sí me levantaré, Perdón Señor.\\  \\

{\changefont{cmss}{bx}{n} Salmo:} \> 33 ant. 1 ``Vayamos a gustar la bondad del Señor'' (estr. 1, 2 o 3). \\ \\

{\changefont{cmss}{bx}{n} Ofrendas:} \> Coplas de Yaraví, Mira nuestra ofrenda, Pan de vida y bebida de luz.\\ \\ 

{\changefont{cmss}{bx}{n} Comunión:} \> Sí me levantaré (estr. 1-12), Vuelve a mí, En memoria tuya, Dios es fiel, \\
\>  Hambre de Dios.  \\ \\

{\changefont{cmss}{bx}{n} Post-com.:} \> Creo en ti (estr. 1)\footnotemark[2], (silencio).\\ \\

{\changefont{cmss}{bx}{n} Salida:} \> Soy peregrino\footnotemark[3], En medio de los pueblos (estr. 1,2)\footnotemark[3].  \\  \\

\end{tabbing}

\vspace{-11mm}

\footnotetext[1]{Por ser domingo de Laetare.  }
\footnotetext[2]{Con el texto tradicional de \emph{Más cerca oh Dios de ti}.  }
\footnotetext[3]{Cfr. \emph{Soy peregrino} y \emph{En medio de los pueblos} con la 1$^\circ$ lectura del día.  }

%------------------------
\newpage
\setcounter{footnote}{0}
\thispagestyle{fancy}
\fancyhead{}
\fancyfoot{}
%\footskip=-1cm

\lhead{\changefont{cmss}{bx}{n} \small Revisión 2013}
\chead{\changefont{cmss}{bx}{n} \small Domingo V - Cuaresma}
\rhead{\changefont{cmss}{bx}{n} \small Ciclo C}
\rfoot{\changefont{cmss}{bx}{n}\large\thepage}

\begin{center}
{\large\it La mujer adúltera}
\end{center}

\vspace{1mm}


\begin{tabbing}

{\changefont{cmss}{bx}{n} Entrada:\ \ \ \ \ }\= Cruz de Cristo, Perdón Señor, Perdón oh Dios mío (estribillo),  \\
\> Vuélvenos tu rostro. \\ \\

{\changefont{cmss}{bx}{n} Salmo:} \> 125 ant. 1 ``Los que siembran entre lágrimas...'' (estr. todas). \\ 
\> (antífona de reemplazo: Sal 84 ant. 1 ``Señor, revélanos tu amor...'') \\ \\

{\changefont{cmss}{bx}{n} Ofrendas:} \> Señor Jesús\footnotemark[1], Entre tus manos, Sé como el grano de trigo, Una espiga, \\ 
\>  Recibe oh Dios el pan. \\ \\ 

{\changefont{cmss}{bx}{n} Comunión:} \> Vuelve a mí\footnotemark[2], Cruz de Cristo, Antes de ser llevado a la muerte,  \\
\> En memoria tuya.  \\ \\

{\changefont{cmss}{bx}{n} Post-com.:} \>  Vaso nuevo (El alfarero), Nada te turbe (de Taizé), (silencio).\\ \\

{\changefont{cmss}{bx}{n} Salida:} \> Virgen de la esperanza (estr. 1,2,5), Santa María del amén, Soy peregrino, \\
\> Madre de nuestro pueblo (estr. 9).  \\  \\

\end{tabbing}

\vspace{-12mm}

\footnotetext[1]{Cfr. la 5$^\circ$ estrofa con con la 2$^\circ$ lectura, especialmente Flp 3,13-14.}
\footnotetext[2]{Debido a la aclamación al evangelio (Jl 2,12-13). }

%------------------------
\newpage
\setcounter{footnote}{0}
\thispagestyle{fancy}
\fancyhead{}
\fancyfoot{}
%\footskip=-1cm

\lhead{\changefont{cmss}{bx}{n} \small Revisión 2013}
\chead{\changefont{cmss}{bx}{n} \small Domingo de Ramos - Cuaresma}
\rhead{\changefont{cmss}{bx}{n} \small Ciclo C}
\rfoot{\changefont{cmss}{bx}{n}\large\thepage}

\begin{center}
{\large\it (Domingo de Pasión) La Pasión de Jesús}
\end{center}

%\vspace{1mm}


\begin{tabbing}

{\changefont{cmss}{bx}{n} Entrada:\ \ \ \ \ }\= Canta Jerusalén, Arriba nuestros ramos, Con ramos en las manos (Sal 23 ant. 3),\\
\> Bendito el que viene (Sal 46 ant. 2).\\   \\

{\changefont{cmss}{bx}{n} Salmo:} \> 21 ant. 1 ``Dios mío, no me abandones...'' (estr. 1, 3, 7, 8 o 9). \\ \\

{\changefont{cmss}{bx}{n} Ofrendas:} \> Este es nuestro pan\footnotemark[1], Te ofrecemos Padre nuestro (vidala)\footnotemark[1], Coplas de Yaraví, \\
\> Una espiga, Sé como el grano de trigo, Zamba del grano de trigo.\\ \\ 

{\changefont{cmss}{bx}{n} Comunión:} \> Rey de los reyes, En la postrera cena, Antes de ser llevado a la muerte,\\
\> En memoria tuya, Jesucristo danos de este pan, Queremos ser Señor\footnotemark[2].  \\ \\

{\changefont{cmss}{bx}{n} Post-com.:} \> Victoria tu reinarás, Creo en ti Señor\footnotemark[3], Tu fidelidad, (silencio).\\ \\

{\changefont{cmss}{bx}{n} Salida:} \> Virgen de la esperanza (estr. 1,2,5), Cristo Jesús (estr. 2,3).\\ \\

\end{tabbing}

\vspace{-14mm}

\footnotetext[1]{Cfr. 2$^\circ$ estrofa con Lc 22,17-20. }
\footnotetext[2]{Sólo en caso de necesidad. }
\footnotetext[3]{Con el texto tradicional de \emph{Más cerca oh Dios de ti}.  }





%------------------------
\newpage
\setcounter{footnote}{0}
\thispagestyle{fancy}
\fancyhead{}
\fancyfoot{}
%\footskip=-1cm

\lhead{\changefont{cmss}{bx}{n} \small Revisión 2013}
\chead{\changefont{cmss}{bx}{n} \small Jueves Santo - Cena del Señor}
\rhead{\changefont{cmss}{bx}{n} \small Ciclo C}
\rfoot{\changefont{cmss}{bx}{n}\large\thepage}

\begin{center}
{\large\it (Triduo pascual) El lavatorio de los pies}
\end{center}

%\vspace{1mm}


\begin{tabbing}

{\changefont{cmss}{bx}{n} Entrada\footnotemark[1]:\ \ \ \ \ }\= Me pongo en tus 
manos oh Señor, El Señor nos llama.\\   \\

{\changefont{cmss}{bx}{n} Salmo:} \> 115 ant. 1 ``?`Con qué pagaré al 
Señor...?'' (estr. 1 y 2). \\ \\

{\changefont{cmss}{bx}{n} Aclamación:} \> Si yo el maestro (dos 
veces). \\ \\

{\changefont{cmss}{bx}{n} Lavatorio:} \> Un mandamiento nuevo, Si yo el 
maestro (dos veces). \\ \\

{\changefont{cmss}{bx}{n} Ofrendas:} \> Los frutos de la tierra.\\ \\ 

{\changefont{cmss}{bx}{n} Comunión:} \> Antes de ser llevado a la muerte,$\,$No 
hay mayor amor,$\,$Dios me dio a mi hermano,\\
\> Memorial.  \\ \\

{\changefont{cmss}{bx}{n} Procesión:} \> Cantemos al amor 
de los amores, Yo soy el camino.\\ \\

{\changefont{cmss}{bx}{n} Adoración:} \> Tan sublime sacramento (Tantum 
ergo).\\ \\

\end{tabbing}

\vspace{-14mm}

\footnotetext[1]{Se canta el \emph{Gloria} (con sentimiento porque no se 
cantará más hasta la Vigilia pascual. }





%------------------------
\newpage
\setcounter{footnote}{0}
\thispagestyle{fancy}
\fancyhead{}
\fancyfoot{}
%\footskip=-1cm

\lhead{\changefont{cmss}{bx}{n} \small Revisión 2013}
\chead{\changefont{cmss}{bx}{n} \small Viernes Santo - Pasión del Señor}
\rhead{\changefont{cmss}{bx}{n} \small Ciclo C}
\rfoot{\changefont{cmss}{bx}{n}\large\thepage}

\begin{center}
{\large\it (Triduo pascual) La Pasión del Señor}
\end{center}

%\vspace{1mm}


\begin{tabbing}

{\changefont{cmss}{bx}{n} Entrada:\ \ \ \ \ }\= (en 
silencio; postración silenciosa).\\   \\

{\changefont{cmss}{bx}{n} Salmo:} \> 30 ant. 1 ``En tus manos 
Señor...'' (estr. 1, 2, 5 u 8). \\ \\

{\changefont{cmss}{bx}{n} Himno:} \> Jesús la imagen de Dios Padre. \\ \\

{\changefont{cmss}{bx}{n} Colecta:} \> Oh víctima inmolada. \\ \\

{\changefont{cmss}{bx}{n} Cruz:} \> antífona ``Te adoramos Cristo y 
te bendecimos...'' (se hace 3 veces). \\ \\

{\changefont{cmss}{bx}{n} Dolorosa:} \> Junto a la cruz, Madre de 
nuestro pueblo \\ \\

{\changefont{cmss}{bx}{n} Comunión:} \> Más cerca oh Dios,$\,$Perdón 
Señor misericordia,$\,$Salmo 50. \\ \\

{\changefont{cmss}{bx}{n} Ador. cruz:} \> Cruz de Cristo (Es la cruz), 
Coplas de soledad, Oh víctima inmolada, Salmo 41.\\ \\

\end{tabbing}

\vspace{-14mm}






%------------------------
\newpage
\setcounter{footnote}{0}
\thispagestyle{fancy}
\fancyhead{}
\fancyfoot{}
%\footskip=-1cm

\lhead{\changefont{cmss}{bx}{n} \small Revisión 2013}
\chead{\changefont{cmss}{bx}{n} \small Domingo de Pascua}
\rhead{\changefont{cmss}{bx}{n} \small Ciclo C}
\rfoot{\changefont{cmss}{bx}{n}\large\thepage}

\begin{center}
{\large\it El sepulcro vacío }
\end{center}

\vspace{1mm}


\begin{tabbing}

{\changefont{cmss}{bx}{n} Entrada:\ \ \ \ \ }\= Encendamos el cirio pascual, Esta es la luz de Cristo, Hoy la Iglesia victoriosa, \\
\> Que resuene por la tierra (si no se usa en otro momento).\\ \\

{\changefont{cmss}{bx}{n} Salmo:}\footnotemark[1] \> 117 ant. 2 ``Este es el día que hizo el Señor, aleluia,...'' (estr. 1, 7 u 8). \\ \\

{\changefont{cmss}{bx}{n} Aspersión:} \> Nueva vida, Esta es el agua pura, Tu agua bendita, Un solo Señor,\\ 
\>  Que resuene por la tierra. \\ \\

{\changefont{cmss}{bx}{n} Ofrendas:} \> Te ofrecemos oh Señor, Pan de vida y bebida de luz.\\ \\

{\changefont{cmss}{bx}{n} Comunión:} \> Gloria al Señor ha llegado la pascua, Que resuene por la tierra, \\
\> La gran noticia, Resucitó. \\ \\

{\changefont{cmss}{bx}{n} Post-com.:} \> Vive Jesús el Señor, Alabe todo el mundo (de Taizé).\\ \\

{\changefont{cmss}{bx}{n} Salida:} \> Alégrate María, Suenen campanas, Toda la tierra levante su voz.  \\  \\

\end{tabbing}

\vspace{-10mm}

\footnotetext[1]{Luego de la segunda lectura hay Secuencia pascual.}


%------------------------
\newpage
\setcounter{footnote}{0}
\thispagestyle{fancy}
\fancyhead{}
\fancyfoot{}
%\footskip=-1cm

\lhead{\changefont{cmss}{bx}{n} \small Revisión 2013}
\chead{\changefont{cmss}{bx}{n} \small Domingo II - Pascua}
\rhead{\changefont{cmss}{bx}{n} \small Ciclo C}
\rfoot{\changefont{cmss}{bx}{n}\large\thepage}

\vspace*{-10mm}

\begin{center}
{\large\it (Domingo de la Misericordia) La incredulidad de Tomás }
\end{center}

\vspace{-1mm}


\begin{tabbing}

{\changefont{cmss}{bx}{n} Entrada:\ \ \ \ \ }\= Resucitó, Hoy la Iglesia victoriosa, Encendamos el	 cirio pascual,  \\ 
 \> Esta es la luz de Cristo. \\ \\

{\changefont{cmss}{bx}{n} Salmo:} \> 117 ant. 1 ``Demos gracias al Señor porque es bueno...'' (estr. 1, 7 u 8). \\ 
\> (antífona de reemplazo: Sal 117 ant. 2 ``Este es el día que hizo el Señor...'') \\ \\

{\changefont{cmss}{bx}{n} Ofrendas:} \> Señor te ofrecemos\footnotemark[1], Pan de vida y bebida de luz, Te presentamos. \\ \\

{\changefont{cmss}{bx}{n} Comunión:} \> No hay mayor amor\footnotemark[2], Resucitó, Gloria al Señor ha llagado la pascua, \\
\> La gran noticia, Que resuene por la tierra. \\ \\

{\changefont{cmss}{bx}{n} Post-com.:} \> Tan cerca de mí\footnotemark[3], Adoremos a Dios (estr. 1)\footnotemark[3], Más cerca oh Dios (estr. 2)\\ 
\> Vive Jesús, Alabe todo el mundo, Gloria al Señor ha llagado la pascua. \\ \\

{\changefont{cmss}{bx}{n} Salida:} \> Toda la tierra levante su voz, Hoy la Iglesia victoriosa, Alégrate María, \\
\> Cantad a María, Suenen campanas. \\  \\

\end{tabbing}

\vspace{-12mm}

\footnotetext[1]{Es perfecto para la Divina Misericordia: ``?`quién podrá cantar tus misericordias...?''.  }
\footnotetext[2]{Cfr. la estrofa 3 con Jn 20,25-27, la estrofa 4 con Jn 20,19 y la estrofa 5 con Jn 20,21-23.}
\footnotetext[3]{Cfr. \textbf{Tan cerca de mí} con Jn 20,27-28 y \textbf{Adoremos a Dios} con Jn 20,28. }


%------------------------
\newpage
\setcounter{footnote}{0}
\thispagestyle{fancy}
\fancyhead{}
\fancyfoot{}
%\footskip=-1cm

\lhead{\changefont{cmss}{bx}{n} \small Revisión 2013}
\chead{\changefont{cmss}{bx}{n} \small Domingo III - Pascua}
\rhead{\changefont{cmss}{bx}{n} \small Ciclo C}
\rfoot{\changefont{cmss}{bx}{n}\large\thepage}

\vspace*{-10mm}

\begin{center}
{\large\it Aparición junto al mar de Tiberíades }
\end{center}

\vspace{-1mm}


\begin{tabbing}

{\changefont{cmss}{bx}{n} Entrada:\ \ \ \ \ }\= Pueblo de Dios\footnotemark[1], Encendamos el cirio pascual,  Esta es la luz de Cristo\footnotemark[2]. \\ \\ 

{\changefont{cmss}{bx}{n} Salmo:} \> 17 ant. 1 ``Te amo, Señor, mi fuerza y mi refugio...'' (estr. 1, 2, 4 o 7). \\ 
\> (antífona de reemplazo: Sal 29 ``Te glorifico, Señor, porque me salvaste'' \\
\> del P. José Bevilacqua) \\ \\

{\changefont{cmss}{bx}{n} Ofrendas:} \> Padre nuestro recibid\footnotemark[3], Pan de vida y bebida de luz, Te presentamos. \\ \\

{\changefont{cmss}{bx}{n} Comunión:} \> Cuerpo y Sangre de Jesús\footnotemark[4], La gran noticia, Que resuene por la tierra, \\
\> Resucitó.\\ \\

{\changefont{cmss}{bx}{n} Post-com.:} \> Adoremos a Dios (estr. 1 y 2)\footnotemark[3], Vive Jesús el Señor, \\ 
\> Alabe todo el mundo (de Taizé). \\ \\

{\changefont{cmss}{bx}{n} Salida:} \> Suenen campanas, Toda la tierra levante su voz, Alégrate María, \\
\> Cantad a María. \\  \\

\end{tabbing}

\vspace{-14mm}

\footnotetext[1]{Cfr. las lecturas del día Hech 5,41 Sal 29,2 Apoc 5,12-14, Jn 21,7. Este último exclama: ``!`Es el Señor!''.}
\footnotetext[2]{Cfr. 1$^\circ$ lectura Hech 5,27-41.}
\footnotetext[3]{Cfr. con la 2$^\circ$ lectura (Apoc 5,11-14). }
\footnotetext[4]{Cfr. Apoc 5,9-10, versículos previos a la 2$^\circ$ lectura.}


%------------------------
\newpage
\setcounter{footnote}{0}
\thispagestyle{fancy}
\fancyhead{}
\fancyfoot{}
%\footskip=-1cm

\lhead{\changefont{cmss}{bx}{n} \small Revisión 2013}
\chead{\changefont{cmss}{bx}{n} \small Domingo IV - Pascua}
\rhead{\changefont{cmss}{bx}{n} \small Ciclo C}
\rfoot{\changefont{cmss}{bx}{n}\large\thepage}

\vspace*{-10mm}

\begin{center}
{\large\it (Domingo del Buen Pastor) Jesús, Hijo de Dios }
\end{center}

\vspace{-1mm}


\begin{tabbing}

{\changefont{cmss}{bx}{n} Entrada\footnotemark[1]:\ \ \ \ \ }\= El Señor nos llama (estr. 2), Pueblo de Reyes (estr. 6), Pueblo de Dios,   \\ 
\> Encendamos el cirio pascual,  Esta es la luz de Cristo.\\ \\ 

{\changefont{cmss}{bx}{n} Salmo\footnotemark[2]:} \> 99 ant. 1 ``Lleguemos hasta el Señor...'' (estr. 1, 2, o 4). \\ 
\> (antífona de reemplazo: Sal 22 ant. 1 ``El Señor es mi Pastor...'') \\ \\

{\changefont{cmss}{bx}{n} Ofrendas:} \> Te ofrecemos Padre nuestro (vidala)\footnotemark[3], Los frutos de la tierra\footnotemark[3],  \\
\> Pan de vida y bebida de luz, Te presentamos. \\ \\

{\changefont{cmss}{bx}{n} Comunión:} \> Yo soy el camino, Pueblo de Reyes, No hay mayor amor. \\ \\

{\changefont{cmss}{bx}{n} Post-com.:} \> Cantemos hermanos (estr. 1), Vive Jesús, Alabe todo el mundo (de Taizé). \\ \\

{\changefont{cmss}{bx}{n} Salida:} \> Cantemos hermanos (todo), Suenen campanas, Alégrate María, \\
\> Vayan todos por el mundo (estr. 2). \\  \\

\end{tabbing}

\vspace{-14mm}

\footnotetext[1]{Cfr. todos estos cantos con las lecturas del día, especialmente:Hech 13,47 (luz de las naciones), \\ Sal 99,2-3 (el pueblo de Dios) y Jn 10,27 (la voz del Pastor).}
\footnotetext[2]{Recordar que es conveniente que el Aleluia sea el de \emph{Yo soy el Maestro y el Pastor} de Néstor Gallego.}
\footnotetext[3]{Cfr. con la 2$^\circ$ lectura (Apoc 7,17). }

%------------------------
\newpage
\setcounter{footnote}{0}
\thispagestyle{fancy}
\fancyhead{}
\fancyfoot{}
%\footskip=-1cm

\lhead{\changefont{cmss}{bx}{n} \small Revisión 2013}
\chead{\changefont{cmss}{bx}{n} \small Domingo V - Pascua}
\rhead{\changefont{cmss}{bx}{n} \small Ciclo C}
\rfoot{\changefont{cmss}{bx}{n}\large\thepage}

\vspace*{-10mm}

\begin{center}
{\large\it Jesús anuncia su glorificación y el mandamiento del amor }
\end{center}

\vspace{-1mm}

\begin{tabbing}

{\changefont{cmss}{bx}{n} Entrada:\ \ \ \ \ }\= Pueblo de Dios, Un solo Señor (estr. 1,3)\footnotemark[1], Que resuene por la tierra,  \\ 
\>  Vine a alabar. \\ \\

{\changefont{cmss}{bx}{n} Salmo:} \> 144 ant. 1 ``Te alabamos Señor y bendecimos tu Nombre'' (estr. 4 y 5). \\ \\ 

{\changefont{cmss}{bx}{n} Ofrendas:} \> Sé como el grano de trigo\footnotemark[2], Padre nuestro recibid\footnotemark[2], Te ofrecemos oh Señor, \\
\> Te presentamos. \\ \\

{\changefont{cmss}{bx}{n} Comunión\footnotemark[3]:} \> Cuerpo y Sangre de Jesús, Un mandamiento nuevo, Como Cristo nos amó,  \\
\> Dios me dio a mi hermano, Ven hermano, Yo soy el camino. \\ \\

{\changefont{cmss}{bx}{n} Post-com.:} \> Adoremos a Dios (estr. 2), Vive Jesús, Alabe todo el mundo (de Taizé). \\ \\

{\changefont{cmss}{bx}{n} Salida:} \> Cantemos hermanos (todo), Soy peregrino (estr. 1,4)\footnotemark[4], Cantad a María. \\  \\

\end{tabbing}

\vspace{-14mm}

\footnotetext[1]{Cfr. con el evangelio del día (Jn 13,31-35).}
\footnotetext[2]{Cfr. \emph{Sé como el grano de trigo} con Jn 13,31-35 (|| Jn 12,27-28), \emph{Padre nuestro recibid} con Sal 144,1-2 y Jn 13,31-32. }
\footnotetext[3]{Cfr. todos estos cantos son equivalentes. Cfr. Jn 13,31-35 (y paralelos Jn 12,23.27-28; 7,33; 8,21; 15,12). }
\footnotetext[4]{Cfr. la 4$^\circ$ estrofa de \emph{Soy peregrino} con la 2$^\circ$ lectura (Apoc 21,1-5).}

%------------------------
\newpage
\setcounter{footnote}{0}
\thispagestyle{fancy}
\fancyhead{}
\fancyfoot{}
%\footskip=-1cm

\lhead{\changefont{cmss}{bx}{n} \small Revisión 2013}
\chead{\changefont{cmss}{bx}{n} \small Domingo VI - Pascua}
\rhead{\changefont{cmss}{bx}{n} \small Ciclo C}
\rfoot{\changefont{cmss}{bx}{n}\large\thepage}

\vspace*{-10mm}

\begin{center}
{\large\it La promesa del Espíritu Santo }
\end{center}

\vspace{-1mm}

\begin{tabbing}

{\changefont{cmss}{bx}{n} Entrada:\ \ \ \ \ }\= Pueblo de Dios, Un solo Señor (estr. 1,3)\footnotemark[1], Que resuene por la tierra,  \\ 
\>  Vine a alabar, Vienen con alegría. \\ \\

{\changefont{cmss}{bx}{n} Salmo:} \> 66 ant. 1 ``!`A ti, Señor, te alabe la tierra y los pueblos...!'' (estr. 1, 2 o 3).\\
\> (antífona de reemplazo: Sal 144 ant. 1 ``Te alabamos Señor...'') \\ \\ 

{\changefont{cmss}{bx}{n} Ofrendas:} \> Pan de vida y bebida de luz, Te ofrecemos Padre nuestro (vidala)\footnotemark[2],  \\
\> Te ofrecemos oh Señor, Te presentamos. \\ \\

{\changefont{cmss}{bx}{n} Comunión:} \> No hay mayor amor\footnotemark[3], La canción de la Alianza\footnotemark[4], Cuerpo y Sangre de Jesús, \\
\> Un mandamiento nuevo, Yo soy el camino. \\ \\

{\changefont{cmss}{bx}{n} Post-com.:} \> Nada te turbe, Alabe todo el mundo, Adoremos a Dios (estr. 2,3), \\
\> Vive Jesús, El Señor es mi fortaleza (de Taizé), Creo en ti Señor (estr 2,3). \\ \\

{\changefont{cmss}{bx}{n} Salida:} \> Cantad a María, Alégrate María, Que resuene por la tierra, Cantemos hermanos. \\  \\

\end{tabbing}

\vspace{-14mm}

\footnotetext[1]{Cfr. con el evangelio del día (Jn 14,23-29).}
\footnotetext[2]{Cfr. con el evangelio del día, especialmente lo referido a los términos de ``paz'' y ``alegría''. }
\footnotetext[3]{Cfr. la 2$^\circ$ estrofa con Jn 14,23 y la 3$^\circ$ estrofa con Jn 14,27. }
\footnotetext[4]{Existe una semejanza importante entre el evangelio del día y 1 Jn 4,6-21. }

%------------------------
\newpage
\setcounter{footnote}{0}
\thispagestyle{fancy}
\fancyhead{}
\fancyfoot{}
%\footskip=-1cm

\lhead{\changefont{cmss}{bx}{n} \small Revisión 2013}
\chead{\changefont{cmss}{bx}{n} \small Ascensión del Señor}
\rhead{\changefont{cmss}{bx}{n} \small Ciclo C}
\rfoot{\changefont{cmss}{bx}{n}\large\thepage}

\vspace*{-15mm}

\begin{center}
%{\large\it \'Ultimas instrucciones de Jesús y la ascensión }
\end{center}

\vspace{-6mm}

\begin{tabbing}

{\changefont{cmss}{bx}{n} Entrada:\ \ \ \ \ }\= Suenen cantos de alegría\footnotemark[1], Un solo Señor (estr. 1,3), Vienen con alegría,  \\ 
\>  Iglesia peregrina. \\ \\

{\changefont{cmss}{bx}{n} Salmo\footnotemark[2]:} \> 46 ant. 1 ``Cantemos al Señor, nuestro Dios, aleluia...'' (estr. 1, 3 o 4).\\
\> (antífona de reemplazo: Sal 147 ant. 1 ``Glorifica al Señor Jerusalén...'') \\ \\ 

{\changefont{cmss}{bx}{n} Ofrendas:} \> Nuestros dones\footnotemark[3], Bendeciré al Señor\footnotemark[3], Al altar del Señor, Te presentamos,  \\
\> Te ofrecemos oh Señor, Bendito seas. \\ \\

{\changefont{cmss}{bx}{n} Comunión:} \> Jesucristo danos de este pan\footnotemark[4], Oh buen Jesús\footnotemark[4], Bendeciré al Señor, \\
\> Yo soy el camino. \\ \\

{\changefont{cmss}{bx}{n} Post-com.:} \> Adoremos a Dios\footnotemark[5], Tu fidelidad\footnotemark[5], Vive Jesús, Alabe todo el mundo (de Taizé). \\ \\

{\changefont{cmss}{bx}{n} Salida:} \> Suenen cantos de alegría\footnotemark[1], Canción del testigo\footnotemark[6], Vayan todos por el mundo\footnotemark[6],\\
\> Anunciaremos tu Reino, Cantemos hermanos. \\  \\

\end{tabbing}

\vspace{-12mm}

\footnotetext[1]{Cfr. \emph{Suenen cantos de alegría} con Hech 1,9-11, y \emph{Vienen con alegría} con Hech 1,8 y Lc 24,52-53.}
\footnotetext[2]{Se puede cantar el Aleluia del Padre Catena con la antífona ``Vayan por el mundo, aununcien mi Reino...''. }
\footnotetext[3]{Cfr. 3$^\circ$ estrofa de \emph{Nuestros dones} con Sal 46,2 y Lc 24,52-53, y \emph{Bendeciré al Señor} (Sal 33) con Sal 46,2. }
\footnotetext[4]{Cfr. \emph{Jesucristo danos de este pan} con Hech 1,11 y Heb 10,19-23, y \emph{Oh buen Jesús} con Heb. 10,12.19-23. }
\footnotetext[5]{Cfr. \emph{Adoremos a Dios} con Lc 24,52-53, y \emph{Tu fidelidad} con Heb 10,23.}
\footnotetext[6]{Cfr. \emph{Canción del testigo} con Hech 1,8 y Lc 24,48, y \emph{Vayan todos por el mundo} con Lc 24,46-48.}

%------------------------
\newpage
\setcounter{footnote}{0}
\thispagestyle{fancy}
\fancyhead{}
\fancyfoot{}
%\footskip=-1cm

\lhead{\changefont{cmss}{bx}{n} \small Revisión 2013}
\chead{\changefont{cmss}{bx}{n} \small Domingo de Pentecostés}
\rhead{\changefont{cmss}{bx}{n} \small Ciclo C}
\rfoot{\changefont{cmss}{bx}{n}\large\thepage}

\vspace{-1mm}

\begin{center}
{\large\it Apariciones de Jesús a los discípulos  }
\end{center}

\vspace{-1mm}


\begin{tabbing}

{\changefont{cmss}{bx}{n} Entrada:\ \ \ \ \ }\= Hoy tu Espíritu Señor\footnotemark[1], Espíritu divino, Pueblo de Reyes\footnotemark[2], Vine a alabar. \\ \\

{\changefont{cmss}{bx}{n} Salmo:}\footnotemark[3] \> 103 (estr. 1, 7 o 13) con música ant. del Sal 144 ``Envía tu Espíritu, Señor, ...''. \\ \\

{\changefont{cmss}{bx}{n} Ofrendas:} \> Ven de lo alto, Espíritu divino, Una espiga\footnotemark[4], Coplas de Yaraví, Te presentamos, \\
\> Bendito seas.\\ \\

{\changefont{cmss}{bx}{n} Comunión:} \> Soplo de Dios viviente, Envíanos Padre\footnotemark[5], Ven Espíritu Santo, Maranathá\footnotemark[5]. \\ \\

{\changefont{cmss}{bx}{n} Post-com.:} \> Ven oh Santo Espíritu\footnotemark[1] (de Taizé), Espíritu de Dios.\\ \\

{\changefont{cmss}{bx}{n} Salida:} \> Madre de nuestro pueblo\footnotemark[6] (estr.10), En medio de los pueblos\footnotemark[1], \\
\> Canción del misionero.  \\  \\

\end{tabbing}

\vspace{-15mm}

\footnotetext[1]{Cfr. con la la 1$^\circ$ lectura (Hech 2,1-11).}
\footnotetext[2]{Cfr. con Ex 19,6 que se lee como 2$^\circ$ lectura en la Vigilia de Pentecostés.}
\footnotetext[3]{Luego de la segunda lectura hay Secuencia de Pentecostés+Aleluia.}
\footnotetext[4]{Porque Pentecostés corresponde a la fiesta hebrea de las \emph{Siete Semanas}, o de las primicias (Lev 23,16).}
\footnotetext[5]{Cfr. \emph{Envíanos Padre} con Jn 20,22 y Hech 2,1-4, y \emph{Maranathá} con Hech 2,1-11 y 1 Cor 12,3-7.}
\footnotetext[6]{Cfr. con Hech 2,12-14.}

%------------------------
\newpage
\setcounter{footnote}{0}
\thispagestyle{fancy}
\fancyhead{}
\fancyfoot{}
%\footskip=-1cm

\lhead{\changefont{cmss}{bx}{n} \small Revisión 2013}
\chead{\changefont{cmss}{bx}{n} \small Domingo de Santísima Trinidad}
\rhead{\changefont{cmss}{bx}{n} \small Ciclo C}
\rfoot{\changefont{cmss}{bx}{n}\large\thepage}

\vspace{-1mm}

\begin{center}
{\large\it La misión del Espíritu Santo}
\end{center}

\vspace{-1mm}


\begin{tabbing}

{\changefont{cmss}{bx}{n} Entrada:\ \ \ \ \ }\= Un solo Señor (estr. 1), El Señor nos llama (estr. 3), Pueblo de Dios,  \\ 
\> Vuélvenos tu rostro\footnotemark[1], Vine a alabar. \\ \\

{\changefont{cmss}{bx}{n} Salmo\footnotemark[2]:} \> 8 ant. 1 ``¡`Oh Señor, nuestro Dios, que admirable...'' (todo). \\ 
\> (antífona de reemplazo: Sal 144 ant. 1 ``Te alabamos Señor...'') \\ \\ 

{\changefont{cmss}{bx}{n} Ofrendas:} \> Padre nuestro recibid\footnotemark[3], Ven de lo alto, Una espiga, Te presentamos, Bendito seas.\\ \\

{\changefont{cmss}{bx}{n} Comunión:} \> Cuerpo y Sangre de Jesús, Es mi Padre, Escondido, Yo soy el pan de vida,\\
\> Vayamos a la mesa, Este es mi Cuerpo. \\ \\

{\changefont{cmss}{bx}{n} Post-com.:} \> Adoremos a Dios (estr. 1), Tu fidelidad, Alabe todo el mundo (de Taizé).\\ \\

{\changefont{cmss}{bx}{n} Salida:} \> Mi camino eres tú\footnotemark[3], Vayan todos por el mundo\footnotemark[4], Canción del testigo, \\
\> Canción del misionero.  \\  \\

\end{tabbing}

\vspace{-15mm}

\footnotetext[1]{Se refiere a la Santísima Trinidad, pero muchas veces se lo asocia como canto de cuaresma.}
\footnotetext[2]{Se puede cantar el Aleluia del P. Catena con la antífona ``Dios es nuestro Padre, Jesús nuestro hermano...''.}
\footnotetext[3]{Porque glorifica a la Santísima Trinidad.}
\footnotetext[4]{Porque se refiere al evangelio de este día para el ciclo B (Mt 28,16-20).}


%------------------------
\newpage
\setcounter{footnote}{0}
\thispagestyle{fancy}
\fancyhead{}
\fancyfoot{}
%\footskip=-1cm

\lhead{\changefont{cmss}{bx}{n} \small Revisión 2013}
\chead{\changefont{cmss}{bx}{n} \small Domingo del Santísimo Cuerpo y Sangre de Cristo}
\rhead{\changefont{cmss}{bx}{n} \small Ciclo C}
\rfoot{\changefont{cmss}{bx}{n}\large\thepage}

\vspace{-5mm}

%\begin{center}
%{\large\it La multiplicación de los panes}
%\end{center}

\vspace{-5mm}


\begin{tabbing}

{\changefont{cmss}{bx}{n} Entrada:\ \ \ \ \ }\= El Señor nos llama (estr. 2), Iglesia peregrina, Somos la familia de Jesús. \\ \\

{\changefont{cmss}{bx}{n} Salmo:} \> 109 ant. 1 ``Tu eres Sacerdote para siempre...'' (todo). \\ 
\> (antífona de reemplazo: Sal 147 ant. 1 ``Glorifica al Señor, Jerusalén...'') \\ \\ 

{\changefont{cmss}{bx}{n} Ofrendas:} \> Un niño se te acercó\footnotemark[1], Los frutos de la tierra, Recibe oh Dios eterno,\\
\> Padre nuestro recibid, Te presentamos\footnotemark[2], Bendito seas\footnotemark[2].\\ \\

{\changefont{cmss}{bx}{n} Comunión:} \> Panis angelicus, Cuerpo y Sangre de Jesús\footnotemark[3], En memoria tuya\footnotemark[3], Es mi Padre\footnotemark[4],\\
\> Jesús eucaristía\footnotemark[1], Yo soy el pan de vida\footnotemark[4], Escondido, Este es mi cuerpo\footnotemark[4], \\
\> En la postrera cena\footnotemark[4], Vayamos a la mesa\footnotemark[4]. \\ \\

\vspace{-1mm}

{\changefont{cmss}{bx}{n} Post-com.:} \> Nuestro maná, Adoremos a Dios, Tantum Ergo (Tan sublime Sacramento), \\
\> Alabado sea el Santísimo, Te adoramos hostia divina.\\ \\

\vspace{-1mm}

{\changefont{cmss}{bx}{n} Salida:} \> Canción del testigo, En medio de los pueblos, Mi camino eres tú,   \\
\> Anunciaremos tu Reino, Vayan todos por el mundo.  \\  \\

\end{tabbing}

\vspace{-17mm}

\footnotetext[1]{Cfr. con Lc 11,13 y su equivalente Jn 6,9.}
\footnotetext[2]{Nos recuerda, adem\'as, la bendici\'on de Abrám de la 1$^\circ$ lectura (Gen 14, 18-20).}
\footnotetext[3]{Cfr. con la 2$^\circ$ lectura (1 Cor 11,23-29).}
\footnotetext[4]{Cfr. con la antífona del Aleluia (Jn 6,51).}

%------------------------
\newpage
\setcounter{footnote}{0}
\thispagestyle{fancy}
\fancyhead{}
\fancyfoot{}
%\footskip=-1cm

\lhead{\changefont{cmss}{bx}{n} \small Revisión 2013}
\chead{\changefont{cmss}{bx}{n} \small Domingo II - Durante el año}
\rhead{\changefont{cmss}{bx}{n} \small Ciclo C}
\rfoot{\changefont{cmss}{bx}{n}\large\thepage}

\vspace*{-13mm}

\begin{center}
{\large\it Las bodas de Caná }
\end{center}

\vspace{-3mm}

\begin{tabbing}

{\changefont{cmss}{bx}{n} Entrada:\ \ \ \ \ }\= Un solo Señor\footnotemark[1], El Señor nos llama\footnotemark[2], Pueblo de Dios\footnotemark[1], Qué alegría, Vine a alabar.  \\ \\

{\changefont{cmss}{bx}{n} Salmo:} \> 97 ant. 1 ``Cantemos al Señor...'' (estr. 1, 2 o 3).\\ 
\> (antífona de reemplazo: Sal 95 ant. 1 ``Cantemos al Señor...'' del P. Bevilacqua.) \\ \\ 

{\changefont{cmss}{bx}{n} Ofrendas:} \> Al altar del Señor\footnotemark[3], Te ofrecemos oh Señor, Pan de vida y bebida de luz,  \\
\> Te presentamos, Bendito seas. \\ \\

{\changefont{cmss}{bx}{n} Comunión:} \> Creo en ti Señor\footnotemark[4], Yo soy el camino, Pescador de hombres, Escondido, \\
\> Bendeciré al Señor.  \\ \\

{\changefont{cmss}{bx}{n} Post-com.:} \> Tu fidelidad, Mirarte sólo a ti, Alabe todo el mundo (de Taizé), \\ 
\> Cuántas gracias te debemos, \\ \\

{\changefont{cmss}{bx}{n} Salida:} \> Madre de nuestro pueblo\footnotemark[5], Anunciaremos tu Reino\footnotemark[5], Mi camino eres tú.\\ \\

\end{tabbing}

\vspace{-15mm}

\footnotetext[1]{Cfr.  \emph{Un sólo Señor} con 1 Cor 12,4-6 y el estribillo de \emph{Pueblo de Dios} con Sal 95,3 y Jn 2,11. }
\footnotetext[2]{Este canto hace referencia a la parábola del banquete nupcial (Mt 22,1-14 y Lc 14,16-24). Es un contexto similar al de las bodas de Caná (Jn 2,1-11) aunque el mensaje es otro. }
\footnotetext[3]{Cfr. con la antífona de comunión del día (Sal 22,5) y Sal 95,8-9. }
\footnotetext[4]{Cfr. \emph{Creo en ti Señor (Más cerca oh Dios de ti)} con Jn 2,11 y su 5$^\circ$ estrofa con la ``hora de Jesús'' (Jn 2,4). }
\footnotetext[5]{Cfr. \emph{Madre de nuestro pueblo}  (estr. 1 y 8) con evangelio del día (Jn 2,1-11) y \emph{Anunciaremos tu Reino} con Sal 95,3.}


%------------------------
\newpage
\setcounter{footnote}{0}
\thispagestyle{fancy}
\fancyhead{}
\fancyfoot{}
%\footskip=-1cm

\lhead{\changefont{cmss}{bx}{n} \small Revisión 2013}
\chead{\changefont{cmss}{bx}{n} \small Domingo III - Durante el año}
\rhead{\changefont{cmss}{bx}{n} \small Ciclo C}
\rfoot{\changefont{cmss}{bx}{n}\large\thepage}

\vspace*{-13mm}

\begin{center}
{\large\it Enseñanza de Jesús en Nazaret }
\end{center}

\vspace{-3mm}

\begin{tabbing}

{\changefont{cmss}{bx}{n} Entrada:\ \ \ \ \ }\= El sermón de la montaña\footnotemark[1], Juntos como hermanos (est. 2), Vienen con alegría\footnotemark[1], \\ \> Iglesia peregrina\footnotemark[1], Vine a alabar.  \\ \\

{\changefont{cmss}{bx}{n} Salmo:} \> 18b ant. 1 ``Tu palabra Señor...'' (estr. 1, 2, 3 o 7).\\ \\

{\changefont{cmss}{bx}{n} Ofrendas:} \> Te ofrecemos Padre nuestro (vidala), Recibe oh Dios el pan, Te presentamos, \\
\>  Bendito seas. \\ \\

{\changefont{cmss}{bx}{n} Comunión:} \>  Yo soy el camino\footnotemark[2], Vayamos a la mesa, Jesús te seguiré, Bendeciré al Señor. \\ \\

{\changefont{cmss}{bx}{n} Post-com.:} \> Tu fidelidad, Si el mismo pan comimos, Alabe todo el mundo (de Taizé), \\ 
\> Adoremos a Dios. \\ \\

{\changefont{cmss}{bx}{n} Salida:} \> Mi camino eres tu, Jesús te seguiré, En medio de los pueblos, \\
\> Anunciaremos tu reino.\\ \\

\end{tabbing}

\vspace{-15mm}

\footnotetext[1]{Cfr.  \emph{El sermón de la montaña} (estribillo) con Neh 8,8 y con Sal 18b,8; cfr. \emph{Vienen con alegría} con Lc 4,18; \mbox{cfr. \emph{Iglesia peregrina}} (est. 1) con 1 Cor 12,12-14.27. }
\footnotetext[2]{Cfr. \emph{Yo soy el camino} con Lc 4,18. }


%------------------------
\newpage
\setcounter{footnote}{0}
\thispagestyle{fancy}
\fancyhead{}
\fancyfoot{}
%\footskip=-1cm

\lhead{\changefont{cmss}{bx}{n} \small Revisión 2013}
\chead{\changefont{cmss}{bx}{n} \small Domingo IV - Durante el año}
\rhead{\changefont{cmss}{bx}{n} \small Ciclo C}
\rfoot{\changefont{cmss}{bx}{n}\large\thepage}

\vspace*{-13mm}

\begin{center}
{\large\it Enseñanza de Jesús en Nazaret (cont.) }
\end{center}

\vspace{-3mm}

\begin{tabbing}

{\changefont{cmss}{bx}{n} Entrada:\ \ \ \ \ }\= Vienen con 
alegría\footnotemark[1], El Señor nos llama, Somos la familia de Jesús, 
\\
\> Juntos como hermanos (est. 2) \\ \\

{\changefont{cmss}{bx}{n} Salmo:} \> 17 ant. 1 ``Te amo Señor, mi fuerza y mi 
refugio...'' (estr. 1, 2, 5 u 8).\\ 
\> (antífona de reemplazo: Sal 29 ant. 1 ``Te glorifico, Señor,...'' del P. 
Bevilacqua) \\ \\ 

{\changefont{cmss}{bx}{n} Ofrendas:} \> Ofrenda de amor\footnotemark[1], Te 
ofrecemos Padre nuestro (vidala), Recibe oh Dios el pan,  \\
\>  Bendeciré al Señor, Te presentamos. \\ \\

{\changefont{cmss}{bx}{n} Comunión:} \>  Queremos ser Señor\footnotemark[2], 
Quédate con nosotros\footnotemark[2], Si yo no tengo amor\footnotemark[2], \\
\> Como Crsisto nos amó, Bendeciré al Señor. \\ \\

{\changefont{cmss}{bx}{n} Post-com.:} \> El alfarero (Vaso nuevo), Tan cerca de 
mí, Nada te turbe (Taizé), \\ 
\> El Señor es mi fortaleza (Taizé). \\ \\

{\changefont{cmss}{bx}{n} Salida:} \> En medio de los pueblos, Canción del 
misionero, Jesús te seguiré,\\
\> Santa María del camino. \\ \\


\end{tabbing}

\vspace{-15mm}

\footnotetext[1]{Cfr.  \emph{Vienen con alegría} (est. 1) con Lc 4,22 y 
con Sal 70,15.17; me parece que \emph{Ofrenda de amor} hace una buena síntesis 
entre 1 Cor 13,1-13 y Lc 4,21-30. }
\footnotetext[2]{La 2$^\circ$ lectura es el Himno al amor de San Pablo (1 Cor 
13,1-13), que se corresponde con \emph{Si yo no tengo amor}. Pero me parece 
que \emph{Queremos ser Señor} y \emph{Quédate con nosotros} se ajustan mejor 
al mensaje de este domingo. }





%------------------------
\newpage
\setcounter{footnote}{0}
\thispagestyle{fancy}
\fancyhead{}
\fancyfoot{}
%\footskip=-1cm

\lhead{\changefont{cmss}{bx}{n} \small Revisión 2013}
\chead{\changefont{cmss}{bx}{n} \small Domingo V - Durante el año}
\rhead{\changefont{cmss}{bx}{n} \small Ciclo C}
\rfoot{\changefont{cmss}{bx}{n}\large\thepage}

\vspace*{-13mm}

\begin{center}
{\large\it La pesca milagrosa }
\end{center}

\vspace{-3mm}

\begin{tabbing}

{\changefont{cmss}{bx}{n} Entrada:\ \ \ \ \ }\= Vine a alabar\footnotemark[1], 
Caminaré\footnotemark[1], Vienen con alegría, El Señor nos 
llama. \\ \\

{\changefont{cmss}{bx}{n} Salmo:} \> 137 ant. 1 ``Te doy gracias Señor por tu 
amor...'' (estr. 1, 2, 3 o 5).\\ 
\> (antífona de reemplazo: Sal 84 ant. 1 ``Señor, revélanos tu amor,...'') \\ 
\\ 

{\changefont{cmss}{bx}{n} Ofrendas:} \> Recibe oh Dios eterno\footnotemark[1], 
Al altar nos acercamos, Te ofrecemos Padre nuestro   \\
\> (vidala), Bendito seas. \\ \\

{\changefont{cmss}{bx}{n} Comunión:} \>  Pescador de hombres\footnotemark[2], 
Jesús te seguiré, Como Cristo nos amó\footnotemark[2], \\
\> Oh buen Jesús\footnotemark[2]. \\ \\

{\changefont{cmss}{bx}{n} Post-com.:} \> Adoremos a Dios\footnotemark[2], Tan 
cerca de mí, Nada te turbe (Taizé). \\ \\

{\changefont{cmss}{bx}{n} Salida:} \> Canción del testigo\footnotemark[2], 
Jesús te seguiré, En medio de los pueblos, \\
\> Oh María, Santa María del camino. \\ \\


\end{tabbing}

\vspace{-15mm}

\footnotetext[1]{Cfr.  \emph{Vine a alabar} (estribillo y est. 1) con Lc 5,10 y 
con ant. de entrada (Sal 94,6-7); cfr. \emph{Caminaré} con Sal 137,3; cfr. 
\emph{Recibe oh Dios eterno} (est. 2) con Lc 5,8. }
\footnotetext[2]{Cfr. \emph{Pescador de hombres} con Lc 5,10; cfr. \emph{Como 
Crsito nos amó} con 1 Cor 15, 3-8. 11; cfr. \emph{Oh buen Jesús} \mbox{(est. 
2)} con Lc 5,8; cfr. \emph{Adoremos a Dios} con ant. de entrada (Sal 94,6-7); 
cfr. \emph{Canción del testigo} (est. 1 y 2) con Is 6,1-8. }





%------------------------
\newpage
\setcounter{footnote}{0}
\thispagestyle{fancy}
\fancyhead{}
\fancyfoot{}
%\footskip=-1cm

\lhead{\changefont{cmss}{bx}{n} \small Revisión 2013}
\chead{\changefont{cmss}{bx}{n} \small Domingo X - Durante el año}
\rhead{\changefont{cmss}{bx}{n} \small Ciclo C}
\rfoot{\changefont{cmss}{bx}{n}\large\thepage}

\vspace*{-10mm}

\begin{center}
{\large\it Resurrección del hijo de una viuda }
\end{center}

\vspace{-3mm}

\begin{tabbing}

{\changefont{cmss}{bx}{n} Entrada:\ \ \ \ \ }\= Pueblo de Dios, Vienen con alegría, Juntos como hermanos (est.2),  \\ 
\>  Alabaré a mi Señor (estr. 2). \\ \\

{\changefont{cmss}{bx}{n} Salmo:} \> 17 ant. 1 ``Te amo, Señor, mi fuerza...'' (estr. 1, 4, 5 o 7).\\
\> (antífona de reemplazo: Sal 29 ant. 1 ``Te glorifico, Señor,...'' del P. Bevilacqua) \\ \\ 


{\changefont{cmss}{bx}{n} Ofrendas:} \> Recibe oh Dios el pan\footnotemark[1], Ofrenda de amor (Por los niños)\footnotemark[2], Mira nuestra ofrenda, \\
\> Bendito seas Señor. \\ \\

{\changefont{cmss}{bx}{n} Comunión:} \> Creo en ti Señor (Más cerca oh Dios)\footnotemark[3], Bendeciré al Señor, Yo soy el camino, \\
\> El Señor de Galilea. \\ \\

{\changefont{cmss}{bx}{n} Post-com.:} \> El Señor es mi fortaleza\footnotemark[4], Mirarte sólo a Tí, Cántico de caridad (estr. 1,4 o 5). \\ \\

{\changefont{cmss}{bx}{n} Salida:} \> Soy peregrino, Anunciaremos tu Reino, Santa María del camino.  \\ \\

\end{tabbing}

\vspace{-15mm}

\footnotetext[1]{Cfr. la 3$^\circ$ estrofa con 1 Rey 17,21-22. }
\footnotetext[2]{Debido a que en su 1$^\circ$ y 3$^\circ$ estrofa recuerda a los que sufren y a los difuntos.}
\footnotetext[3]{Cfr. \emph{Creo en ti} con la antífona de comunión del día (Sal 17,3), \emph{Bendeciré al Señor} (Sal 33) con Sal 17 y \emph{Yo soy el camino} con Lc 7,16.}
\footnotetext[4]{Debido al milagro de Elías (1 Rey 17,17-24) y de Jesús (Lc 7,11-17).}


%------------------------
\newpage
\setcounter{footnote}{0}
\thispagestyle{fancy}
\fancyhead{}
\fancyfoot{}
%\footskip=-1cm

\lhead{\changefont{cmss}{bx}{n} \small Revisión 2013}
\chead{\changefont{cmss}{bx}{n} \small Domingo XI - Durante el año}
\rhead{\changefont{cmss}{bx}{n} \small Ciclo C}
\rfoot{\changefont{cmss}{bx}{n}\large\thepage}

\vspace*{-10mm}

\begin{center}
{\large\it La pecadora perdonada }
\end{center}

\vspace{-3mm}

\begin{tabbing}

{\changefont{cmss}{bx}{n} Entrada:\ \ \ \ \ }\= El Sermón de la montaña\footnotemark[1], Contritos nos postramos\footnotemark[2], Me pongo en tus manos  \\ 
\> (estribillo sólo), Vine a alabar. \\ \\

{\changefont{cmss}{bx}{n} Salmo:} \> 50 ant. 1 ``Piedad, Señor, pecamos...'' (estr. 2, 4, 5, 6, 7 u 8).\\ \\

{\changefont{cmss}{bx}{n} Ofrendas:} \> Entre tus manos\footnotemark[3], Recibe oh Dios el pan\footnotemark[3], Te ofrecemos Padre nuestro (vidala)\footnotemark[3], \\
\> Padre nuestro recibid, Toma Señor nuestra vida. \\ \\

{\changefont{cmss}{bx}{n} Comunión:} \> Oh buen Jesús\footnotemark[4], Canción de la Alianza\footnotemark[4], Queremos ser Señor\footnotemark[4], Coplas de Yaraví.\\ \\

{\changefont{cmss}{bx}{n} Post-com.:} \> Mirarte sólo a Tí, Vaso nuevo (El alfarero), Cuántas gracias te debemos. \\ \\

{\changefont{cmss}{bx}{n} Salida:} \> Oh Santísima, Oh María, Mi camino eres tú, Simple oración.  \\ \\

\end{tabbing}

\vspace{-15mm}

\footnotetext[1]{Cfr. el estribillo con Gal 2,16.19-21 y la 1$^\circ$ estrofa con Lc 7,38. }
\footnotetext[2]{Debido a que revive el gesto de la pecadora de Lc 7,38.}
\footnotetext[3]{Cfr. \emph{Entre tus manos} con Lc 7,38, la 3$^\circ$ estrofa de \emph{Recibe oh Dios el pan} con Lc 7,36-8,3 y la 3$^\circ$ estrofa de \emph{Te ofrecemos Padre nuestro} con Lc 7,38.}
\footnotetext[4]{Cfr. \emph{Oh buen Jesús} con Lc 7,38 cuando dice ``indigno soy confieso avergonzado''. Cfr. \emph{La canción de la Alianza} (estr. 2) con Gal 2,16.19-21 y con la aclamación al evangelio (1 Jn 4,10b). Cfr. \emph{Queremos ser Señor} (estr. 3) con Gal 2,16.19-21. }


%------------------------
\newpage
\setcounter{footnote}{0}
\thispagestyle{fancy}
\fancyhead{}
\fancyfoot{}
%\footskip=-1cm

\lhead{\changefont{cmss}{bx}{n} \small Revisión 2013}
\chead{\changefont{cmss}{bx}{n} \small Domingo XII - Durante el año}
\rhead{\changefont{cmss}{bx}{n} \small Ciclo C}
\rfoot{\changefont{cmss}{bx}{n}\large\thepage}

\vspace*{-10mm}

\begin{center}
{\large\it La profesión de fe de Pedro }
\end{center}

\vspace{-3mm}

\begin{tabbing}

{\changefont{cmss}{bx}{n} Entrada:\ \ \ \ \ }\= Pueblo de reyes (estr. 3), Me pongo en tus manos (estribillo)\footnotemark[1], Que alegría,  \\ 
\> Juntos como hermanos (estr. 2). \\ \\

{\changefont{cmss}{bx}{n} Salmo:} \> 62 ant. 1 ``Señor, mi Dios, te busco...'' (estr. 1, 2 o 3).\\ 
\> (antífona de reemplazo: Sal 41 ant. 1 ``Mi alma tiene sed de Dios...'') \\ \\

{\changefont{cmss}{bx}{n} Ofrendas:} \> Comienza al sacrificio, Este es nuestro pan, Señor te ofrecemos, \\
\> Mira nuestra ofrenda. \\ \\

{\changefont{cmss}{bx}{n} Comunión:} \> Creo en ti Señor\footnotemark[2], Yo soy el camino\footnotemark[2], Jesucristo danos de este pan, \\
\> Este es mi cuerpo. \\ \\

{\changefont{cmss}{bx}{n} Post-com.:} \> Creo en ti Señor, Tu fidelidad, Si el mismo pan comimos (estribillo),\\
\> El Señor es mi fortaleza (de Taizé). \\ \\

{\changefont{cmss}{bx}{n} Salida:} \> En medio de los pueblos, Anunciaremos tu reino, Oh María, Salve oh Reina.  \\ \\

\end{tabbing}

\vspace{-15mm}

\footnotetext[1]{Cfr. con el evangelio del día Lc 9,18-24. }
\footnotetext[2]{Cfr. \emph{Creo en ti (Más cerca oh Dios de ti)} con Lc 9,23-24, \emph{Yo soy el camino} con antífonas del aleluia (Jn 10,27) y de comunión (Jn 10,11-15).}

%------------------------
\newpage
\setcounter{footnote}{0}
\thispagestyle{fancy}
\fancyhead{}
\fancyfoot{}
%\footskip=-1cm

\lhead{\changefont{cmss}{bx}{n} \small Revisión 2013}
\chead{\changefont{cmss}{bx}{n} \small Domingo XIII - Durante el año}
\rhead{\changefont{cmss}{bx}{n} \small Ciclo C}
\rfoot{\changefont{cmss}{bx}{n}\large\thepage}

\vspace*{-10mm}

\begin{center}
{\large\it Las exigencias de la vocación apostólica }
\end{center}

\vspace{-3mm}

\begin{tabbing}

{\changefont{cmss}{bx}{n} Entrada:\ \ \ \ \ }\= Que alegría\footnotemark[1], Vienen con alegría\footnotemark[1], Un pueblo que camina, Callemos hermanos\footnotemark[1],   \\ 
\> Juntos como hermanos (estr. 2). \\ \\

{\changefont{cmss}{bx}{n} Salmo:} \> 15 ant. 1 ``Tu eres, Señor, mi herencia...'' (estr. 1, 3, 4 o 5).\\ \\

{\changefont{cmss}{bx}{n} Ofrendas:} \> Ofrenda de amor (Por los niños)\footnotemark[2], Recibe oh Dios eterno, Mira nuestra ofrenda, \\
\> Toma Señor nuestra vida, Bendito seas. \\ \\

{\changefont{cmss}{bx}{n} Comunión:} \> Jesús te seguiré\footnotemark[3], Mensajero de la paz\footnotemark[4], Queremos ser Señor\footnotemark[3], Simple oración. \\ \\

{\changefont{cmss}{bx}{n} Post-com.:} \> Mirarte sólo a ti\footnotemark[4], Quiero decir que sí, Nada te turbe (de Taizé). \\ \\

{\changefont{cmss}{bx}{n} Salida:} \> Anunciaremos tu reino\footnotemark[4], Quiero decir que sí, Santa María del camino, \\
\> Mi camino eres tu, Simple oración.  \\ \\

\end{tabbing}

\vspace{-15mm}

\footnotetext[1]{Cfr. \emph{Que alegría} con Lc 9,51, \emph{Vienen con alegría} con Lc 9,52 y \emph{Callemos hermanos} con la aclamación al evangelio (1 Sam 3,9 y Jn 6,68). }
\footnotetext[2]{Cfr. en la 2$^\circ$ estrofa cuando dice ``por los pueblos que no te conocen'' con Lc 9,53-56.}
\footnotetext[3]{Cfr. \emph{Jesús te seguiré} con Lc 9,57, \emph{Queremos ser Señor} con Gal 5,13 y \emph{Pescador de hombres} con Lc 9,59.}
\footnotetext[4]{Cfr. \emph{Mirarte sólo a ti} con Lc 9,61-62 y \emph{Anunciaremos tu reino} con Lc 9,60.}

%------------------------
\newpage
\setcounter{footnote}{0}
\thispagestyle{fancy}
\fancyhead{}
\fancyfoot{}
%\footskip=-1cm

\lhead{\changefont{cmss}{bx}{n} \small Revisión 2013}
\chead{\changefont{cmss}{bx}{n} \small Domingo XIV - Durante el año}
\rhead{\changefont{cmss}{bx}{n} \small Ciclo C}
\rfoot{\changefont{cmss}{bx}{n}\large\thepage}

\vspace*{-11mm}

\begin{center}
{\large\it La misión de los setenta y dos discípulos }
\end{center}

\vspace{-3mm}

\begin{tabbing}

{\changefont{cmss}{bx}{n} Entrada:\ \ \ \ \ }\= Pueblo de Dios\footnotemark[1], Vienen con alegría\footnotemark[1], Que alegría\footnotemark[1],    \\ 
\> Un pueblo que camina, Somos la familia de Jesús (est. 2). \\ \\

{\changefont{cmss}{bx}{n} Salmo:} \> 65 ant. 1 ``Todo el mundo cante la gloria de Dios...'' (estr. 1, 2, 3 o 7).\\ 
\> (antífona de reemplazo: Sal 26 ant. 1 ``Cantaré y celebraré al Señor'') \\ \\

{\changefont{cmss}{bx}{n} Ofrendas:} \> Padre nuestro recibid\footnotemark[2], Bendeciré al Señor, Te ofrecemos Padre nuestro (vidala), \\
\> Coplas de Yaraví. \\ \\

{\changefont{cmss}{bx}{n} Comunión:} \> Mensajero de la paz\footnotemark[3], Queremos ser Señor\footnotemark[3], Bendeciré al Señor, Simple oración. \\ \\

{\changefont{cmss}{bx}{n} Post-com.:} \> Mirarte sólo a ti, Cántico de caridad (Bendigamos al Señor),\\
\>  Cuantas gracias te debemos, Alabe todo el mundo (de Taizé). \\ \\

{\changefont{cmss}{bx}{n} Salida:} \> Canción del misionero\footnotemark[4], Soy peregrino\footnotemark[4], Vayan todos por el mundo,  \\
\> Madre de los peregrinos. \\ \\

\end{tabbing}

\vspace{-15mm}

\footnotetext[1]{Cfr. \emph{Pueblo de Dios} (est. 1) con Sal 65,1-5, y \emph{Que alegría} (est. 3 y 4) con Lc 10,1-12 y con Gal 6,16. }
\footnotetext[2]{Cfr. \emph{Padre nuestro recibid} (est. 3 y 4) con Gal 6,17 y Sal 65, respectivamente.}
\footnotetext[3]{Cfr. \emph{Mensajero de la paz} con Lc 10,1-12.17-20 y \emph{Queremos ser Señor} (est. 1) con Lc 10,1-12 y Gal 6,16.}
\footnotetext[4]{Cfr. \emph{Canción del msionero} con Lc 10,1-9 y \emph{Soy peregrino} (est. 2) con Lc 10,1-12 y Gal 6,16.}

%------------------------
\newpage
\setcounter{footnote}{0}
\thispagestyle{fancy}
\fancyhead{}
\fancyfoot{}
%\footskip=-1cm

\lhead{\changefont{cmss}{bx}{n} \small Revisión 2013}
\chead{\changefont{cmss}{bx}{n} \small Domingo XV - Durante el año}
\rhead{\changefont{cmss}{bx}{n} \small Ciclo C}
\rfoot{\changefont{cmss}{bx}{n}\large\thepage}

\vspace*{-11mm}

\begin{center}
{\large\it La parábola del buen samaritano }
\end{center}

\vspace{-3mm}

\begin{tabbing}

{\changefont{cmss}{bx}{n} Entrada:\ \ \ \ \ }\= Vine a alabar\footnotemark[1], El sermón de la montaña\footnotemark[1], Iglesia peregrina\footnotemark[1],  Vienen con alegría\footnotemark[1]. \\ \\

{\changefont{cmss}{bx}{n} Salmo:} \> 18b ant. 1 ``Tu palabra Señor es la verdad...'' (estr. 1, 2, 3 o 4).\\ \\

{\changefont{cmss}{bx}{n} Ofrendas:} \> Ofrenda de amor (Por los niños)\footnotemark[2], Bendeciré al Señor, Recibe oh Dios eterno\footnotemark[2], \\ 
\> Te ofrecemos Padre nuestro (misa nicarag\"uense). \\ \\

{\changefont{cmss}{bx}{n} Comunión:} \> Tan cerca de mí\footnotemark[3], Dios me dio a mi hermano\footnotemark[3], La canción de la Alianza\footnotemark[3], \\
\> Es mi Padre\footnotemark[3], Yo soy el pan de vida\footnotemark[3], Oh buen Jesús\footnotemark[3], Si yo no tengo amor. \\ \\

{\changefont{cmss}{bx}{n} Post-com.:} \> Tan cerca de mí (estribillo), Ubi caritas (de Taizé), \\
\>  Cántico de caridad (Bendigamos al Señor), Si el mismo pan comimos. \\ \\

{\changefont{cmss}{bx}{n} Salida:} \> Vayan todos por el mundo, Madre de los peregrinos, Santa María del camino,  \\
\> Simple oración. \\ \\

\end{tabbing}

\vspace{-15mm}

\footnotetext[1]{Cfr. \emph{Vine a alabar} con Deut 30,14,  \emph{El sermón de la montaña} con Deut 30,10.14 y Lc 10,26-28, \emph{Iglesia peregrina} (est. 1 y 6) con Col 1,18-20 y \emph{Vienen con alegría} con Lc 10,33. }
\footnotetext[2]{Cfr. \emph{Ofrenda de amor} con Lc 10,29-37, \emph{Recibe oh Dios} con Col 1,18-20.}
\footnotetext[3]{Cfr. \emph{Tan cerca de mí} con Deut 30,14. Cfr. \emph{Yo soy el pan de vida}, \emph{La canción de la Alianza} y \emph{Dios me dio a mi hermano} con Lc 10,25, Lc 10,27-28 y Lc 10,29-37 respectivamente. Cfr. \emph{Oh buen Jesús} con Col 1,20. }


%------------------------
\newpage
\setcounter{footnote}{0}
\thispagestyle{fancy}
\fancyhead{}
\fancyfoot{}
%\footskip=-1cm

\lhead{\changefont{cmss}{bx}{n} \small Revisión 2013}
\chead{\changefont{cmss}{bx}{n} \small Domingo XVI - Durante el año}
\rhead{\changefont{cmss}{bx}{n} \small Ciclo C}
\rfoot{\changefont{cmss}{bx}{n}\large\thepage}

\vspace*{-11mm}

\begin{center}
{\large\it El encuentro de Jesús con Marta y María }
\end{center}

\vspace{-3mm}

\begin{tabbing}

{\changefont{cmss}{bx}{n} Entrada:\ \ \ \ \ }\= El sermón de la montaña\footnotemark[1], El Señor nos llama\footnotemark[1], Callemos hermanos\footnotemark[1],\\  \> Vienen con alegría. \\ \\

{\changefont{cmss}{bx}{n} Salmo:} \> 1 ant. 1 ``Feliz el hombre que siempre medita...'' (estr. 1 y 2).\\ \\

{\changefont{cmss}{bx}{n} Ofrendas:} \> Este es nuestro pan\footnotemark[2], Mira nuestra ofrenda\footnotemark[2], Una espiga\footnotemark[2], Padre nuestro recibid\footnotemark[2]. \\ \\ 

{\changefont{cmss}{bx}{n} Comunión:} \> Creo en ti Señor (Más cerca oh Dios)\footnotemark[3], Este es mi cuerpo\footnotemark[3], Queremos ser Señor\footnotemark[3], \\
\> Jesucristo danos de este pan. \\ \\

{\changefont{cmss}{bx}{n} Post-com.:} \> Mirarte sólo a ti\footnotemark[4], El Señor es mi fortaleza (de Taizé), Si el mismo pan comimos. \\ \\

{\changefont{cmss}{bx}{n} Salida:} \> Simple oración\footnotemark[4], Mi camino eres tu\footnotemark[4], Madre de los peregrinos,  \\
\> Santa maría del camino. \\ \\

\end{tabbing}

\vspace{-15mm}

\footnotetext[1]{Cfr. \emph{El sermón de la montaña} (est. 1 y 2) con Gen 18,10 y Lc 10,39.41-42, \emph{El Señor nos llama} (especialmente \mbox{est. 2 y 3}) con Gen 18,1-3 y Lc 10,38-42 y \emph{Callemos hermanos} con Lc 10,38-42. }
\footnotetext[2]{Cfr. \emph{Este es nuestro pan} (estribillo y est. 1) con Gen 18,5-6, \emph{Mira nuestra ofrenda} (estribillo) con Col 1,24, \mbox{\emph{Una espiga}} (est. 3) con Gen 18,5-6, y \emph{Padre nuestro recibid} (est. 3) con Col 1,24.}
\footnotetext[3]{Cfr. \emph{Creo en ti Señor} (est. 1) con Lc 10,39, (est. 3) con Lc 10,41 y (est. 4) con Col 1,24, y \emph{Este es mi cuerpo} \mbox{(est. 1)} con Gen 18,5 y (est. 3) con Lc 10,40. }
\footnotetext[4]{Cfr. \emph{Simple oración} con Lc 10,38-42; \emph{Mirarte sólo a ti} y \emph{Mi camino eres tu} con Lc 10,39. }

%------------------------
\newpage
\setcounter{footnote}{0}
\thispagestyle{fancy}
\fancyhead{}
\fancyfoot{}
%\footskip=-1cm

\lhead{\changefont{cmss}{bx}{n} \small Revisión 2013}
\chead{\changefont{cmss}{bx}{n} \small Domingo XVII - Durante el año}
\rhead{\changefont{cmss}{bx}{n} \small Ciclo C}
\rfoot{\changefont{cmss}{bx}{n}\large\thepage}

\vspace*{-11mm}

\begin{center}
{\large\it El Padrenuestro y la parábola del amigo insistente }
\end{center}

\vspace{-3mm}

\begin{tabbing}

{\changefont{cmss}{bx}{n} Entrada:\ \ \ \ \ }\= El sermón de la montaña\footnotemark[1], Pueblo de Dios\footnotemark[1], Caminaré\footnotemark[1], Juntos como hermanos\footnotemark[1], \\  
\> Vine a alabar\footnotemark[1]. \\ \\

{\changefont{cmss}{bx}{n} Salmo:} \> 137 ant. 1 ``Te doy gracias, Señor, por tu amor...'' (estr. 1, 2, 4 o 5).\\ 
\> (antífona de reemplazo: Sal 84 ant. 1: ``Señor, revélanos tu amor,...'') \\ \\

{\changefont{cmss}{bx}{n} Ofrendas:} \> Recibe oh Dios el pan, Toma Señor nuestra vida, Ofrenda de amor (Por los niños), \\ 
\> Bendeciré al Señor. \\ \\ 

{\changefont{cmss}{bx}{n} Comunión:} \> Simple oración\footnotemark[2],$\,$Como Cristo nos amó\footnotemark[2],$\,$Quédate con nosotros\footnotemark[2],$\,$Vayamos a la mesa. \\ \\

{\changefont{cmss}{bx}{n} Post-com.:} \> El Señor es mi fortaleza (de Taizé)\footnotemark[3], Nada te turbe (de Taizé), Mirarte sólo a ti,\\
\> Cuántas gracias te debemos. \\ \\

{\changefont{cmss}{bx}{n} Salida:} \> Simple oración\footnotemark[2], Oh María, Oh Santísima, Salve oh Reina. \\ \\

\end{tabbing}

\vspace{-15mm}

\footnotetext[1]{Cfr. \emph{El sermón de la montaña} (est. 1) con Sal 137,6, (est. 3) con Mt 5,7-8; \emph{Pueblo de Dios} (estribillo) con \mbox{Sal 137,8} y (estr 1) con Lc 11,1; \emph{Caminaré} (est. 1) con Sal 137,3; \emph{Juntos como hermanos} (est. 2) con Lc 11,1 y con Sal 137,3; \emph{Vine a alabar} con Col 2,12-14. }
\footnotetext[2]{Cfr. \emph{Simple oración} (est. 4) con Lc 11,9-10; \emph{Como Cristo nos amó} (comienzo de cada estrofa) con Col 2,14 y \mbox{Lc 11,13}; \mbox{\emph{Quédate con nosotros}} (est. 4) con insistencia de Abraham en Gen 18, 20-21. 23-32.}
\footnotetext[3]{Cfr. con Sal 137,3. }


%------------------------
\newpage
\setcounter{footnote}{0}
\thispagestyle{fancy}
\fancyhead{}
\fancyfoot{}
%\footskip=-1cm

\lhead{\changefont{cmss}{bx}{n} \small Revisión 2013}
\chead{\changefont{cmss}{bx}{n} \small Domingo XVIII - Durante el año}
\rhead{\changefont{cmss}{bx}{n} \small Ciclo C}
\rfoot{\changefont{cmss}{bx}{n}\large\thepage}

\vspace*{-11mm}

\begin{center}
{\large\it La parábola del rico insensato }
\end{center}

\vspace{-3mm}

\begin{tabbing}

{\changefont{cmss}{bx}{n} Entrada:\ \ \ \ \ }\= Brilló la luz\footnotemark[1], El sermón de la montaña\footnotemark[1], Vine a alabar, Vienen con alegría.\\ \\  


{\changefont{cmss}{bx}{n} Salmo:} \> 89 ant. 1 ``Nuestra vida, Señor, pasa como un soplo...'' (estr. 2, 4 o 5).\\ 
\> (antífona de reemplazo: Sal 15 ant.3: ``Protégeme, Dios mío, porque en tí...'') \\ \\

{\changefont{cmss}{bx}{n} Ofrendas:} \> Coplas de Yaraví\footnotemark[2], Mira nuestra ofrenda\footnotemark[2], Padre nuestro recibid\footnotemark[2],  \\ 
\> Bendito seas Señor, Te presentamos. \\ \\ 

{\changefont{cmss}{bx}{n} Comunión:} \> Brilló la luz\footnotemark[1], Quédate con nosotros\footnotemark[3], Queremos ser Señor\footnotemark[3], Este es mi cuerpo\footnotemark[3], \\
\> Es mi Padre\footnotemark[3], Yo soy el pan de vida\footnotemark[3]. \\ \\

{\changefont{cmss}{bx}{n} Post-com.:} \> Al atardecer de la vida (est. 2), Nada te turbe (de Taizé), Vaso nuevo\footnotemark[3]. \\ \\

{\changefont{cmss}{bx}{n} Salida:} \> Mi camino eres tu\footnotemark[4], Madre de los peregrinos\footnotemark[4], Anunciaremos tu Reino, \\
\> Santa María del camino. \\ \\

\end{tabbing}

\vspace{-15mm}

\footnotetext[1]{Cfr. \emph{Brilló la luz} (est. 1) con Lc 12,15.21, Col 3,5 y Mt 5,3; \emph{El sermón de la montaña} con Col 3,5 y Lc 12,13-15. }
\footnotetext[2]{Cfr. \emph{Coplas de Yaraví} (est. 4) con Col 3,5, Lc 12,15 y la aclamación al evangelio Mt 5,3; \emph{Mira nuestra ofrenda} con Lc 12,21 con Lc 21,3-4); \mbox{\emph{Padre nuestro recibid}} (est. 3) con Ecl 2,22-23.}
\footnotetext[3]{Cfr. \emph{Quédate con nosotros} (est. 2 y 3) con Col 3,5 , Lc 12,13-15 y Lc 12,20-21; \emph{Queremos ser Señor} con Lc 12,13-15; \emph{Este es mi cuerpo}, \emph{Es mi Padre} y \emph{Yo soy el pan de vida} con Sab 16,20 y Jn 6,35; \emph{Vaso nuevo} con Col 3,5.  }
\footnotetext[4]{Cfr. \emph{Mi camino eres tu} con Col 3,5 y \emph{Madre de los peregrinos} (est 2) con Lc 12,13-21. }


%------------------------
\newpage
\setcounter{footnote}{0}
\thispagestyle{fancy}
\fancyhead{}
\fancyfoot{}
%\footskip=-1cm

\lhead{\changefont{cmss}{bx}{n} \small Revisión 2013}
\chead{\changefont{cmss}{bx}{n} \small Domingo XIX - Durante el año}
\rhead{\changefont{cmss}{bx}{n} \small Ciclo C}
\rfoot{\changefont{cmss}{bx}{n}\large\thepage}

\vspace*{-11mm}

\begin{center}
{\large\it Exhortación a la vigilancia y a la fidelidad }
\end{center}

\vspace{-3mm}

\begin{tabbing}

{\changefont{cmss}{bx}{n} Entrada:\ \ \ \ \ }\= Pueblo de Dios\footnotemark[1], Pueblo de reyes\footnotemark[1], El Señor nos llama\footnotemark[1], Un solo Señor\footnotemark[1].\\ \\  


{\changefont{cmss}{bx}{n} Salmo:} \> 32 ant. 1 ``Cantemos todos al Señor, aleluia,...'' (estr. 7 y 8).\\ 
\> (antífona de reemplazo: Sal 26 ant.1: ``Cantaré y celebraré al Señor'') \\ \\

{\changefont{cmss}{bx}{n} Ofrendas:} \> Comienza el sacrificio\footnotemark[2], Los frutos de la tierra\footnotemark[2], Te ofrecemos Padre nuestro \footnotemark[2]  \\ 
\> (versión moderna), Bendeciré al Señor. \\ \\ 

{\changefont{cmss}{bx}{n} Comunión:} \> Cuerpo y sangre de Jesús\footnotemark[3], Queremos ser Señor\footnotemark[3], Jesucristo danos de este pan\footnotemark[3],  \\
\> Creo en ti Señor (Más cerca oh Dios), Oh buen Jesús. \\ \\

{\changefont{cmss}{bx}{n} Post-com.:} \> Al atardecer de la vida, Nada te turbe (Taizé), El Señor es mi fortaleza (Taizé), \\
\> Mirarte sólo a ti. \\ \\

{\changefont{cmss}{bx}{n} Salida:} \> Soy peregrino\footnotemark[3],$\,$Canción del misionero,$\,$Mi camino eres tu,$\,$Santa María del camino.\\ \\

\end{tabbing}

\vspace{-15mm}

\footnotetext[1]{Cfr. \emph{Pueblo de Dios} (est. 1) y \emph{Pueblo de reyes} con Sab 18,9 y Sal 32,12; \emph{El Señor nos llama} hace refrencia a Mt 22,1-14 y Lc 14,16-24, que tiene analogías con Lc 12, 32-48; \emph{Un solo Señor} (est. 3) con Lc 12,37-40. }
\footnotetext[2]{Cfr. \emph{Comienza el sacrificio} (est. 2) con Sab 18,5-9, Heb 11, 1-2.8-19 y Lc 12,37; \emph{Los frutos de la tierra} (súplica al final de cada estrofa) con Lc 12,37-38; \mbox{\emph{Te ofrecemos Padre nuestro}} (est. 4) con Lc 12,37-43.}
\footnotetext[3]{Cfr. \emph{Cuerpo y Sangre de Jesús} (est. 2) con Sal 32,12; \emph{Queremos ser Señor} (estribillo) con Lc 12,42-43; \emph{Jesucristo danos de este pan} (est. 4) con Sab 18,9 y Sal 32,12; \emph{Soy peregrino} con Heb 11,13-19.  }

%------------------------
\newpage
\setcounter{footnote}{0}
\thispagestyle{fancy}
\fancyhead{}
\fancyfoot{}
%\footskip=-1cm

\lhead{\changefont{cmss}{bx}{n} \small Revisión 2013}
\chead{\changefont{cmss}{bx}{n} \small Domingo XX - Durante el año}
\rhead{\changefont{cmss}{bx}{n} \small Ciclo C}
\rfoot{\changefont{cmss}{bx}{n}\large\thepage}

\vspace*{-11mm}

\begin{center}
{\large\it Jesús, signo de contradicción }
\end{center}

\vspace{-3mm}

\begin{tabbing}

{\changefont{cmss}{bx}{n} Entrada:\ \ \ \ \ }\= Un solo Señor\footnotemark[1], Somos la familia de Jesús\footnotemark[1], Caminaré\footnotemark[1], Que alegría.\\ \\  


{\changefont{cmss}{bx}{n} Salmo:} \> 17 ant. 1 ``Te amo, Señor, mi fuerza ...'' (estr. 1, 2 o 4).\\ 
\> (antífona de reemplazo: Sal 30 ant.1: ``En tus manos, Señor,...'') \\ \\

{\changefont{cmss}{bx}{n} Ofrendas:} \> Recibe oh Dios eterno\footnotemark[2], Entre tus manos, Nuestros dones, \\
\> Te ofrecemos Padre nuestro (vidala). \\  \\ 

{\changefont{cmss}{bx}{n} Comunión:} \> Yo soy el camino\footnotemark[3], Cuerpo y sangre de Jesús\footnotemark[3], Es mi Padre, Vayamos a la mesa,  \\
\> Este es mi cuerpo. \\ \\

{\changefont{cmss}{bx}{n} Post-com.:} \> Tu fidelidad\footnotemark[4], Adoremos a Dios, Alabe todo el mundo (Taizé), \\
\> Cuantas gracias te debemos. \\ \\

{\changefont{cmss}{bx}{n} Salida:} \> Oh María, Salve oh Reina, En medio de los pueblos, Mi camino eres tú, \\
\> Santa María del camino.\\ \\

\end{tabbing}

\vspace{-15mm}

\footnotetext[1]{Cfr. \emph{Un solo Señor} (estribillo) con Lc 12,50; \emph{Somos la familia de Jesús} con Lc 12,52-53 y éste, a su vez, con \mbox{Mt 12,48-50}, Mc 3,33-35 o Lc 8,21; \emph{Caminaré} (paráfrasis del Sal 114) con Sal 39,2-4.18. }
\footnotetext[2]{Cfr. \emph{Recibe oh Dios eterno} (est. 2 y 3) con Lc 12,50-53.}
\footnotetext[3]{Cfr. \emph{Yo soy el camino} con la aclamación del día Jn 10,27; \emph{Cuerpo y sangre de Jesús} (est. 1 y 2) con Lc 12,49-53. }
\footnotetext[4]{Cfr. \emph{Tu fidelidad} con Lc 12,49.} 

%------------------------
\newpage
\setcounter{footnote}{0}
\thispagestyle{fancy}
\fancyhead{}
\fancyfoot{}
%\footskip=-1cm

\lhead{\changefont{cmss}{bx}{n} \small Revisión 2013}
\chead{\changefont{cmss}{bx}{n} \small Domingo XXI - Durante el año}
\rhead{\changefont{cmss}{bx}{n} \small Ciclo C}
\rfoot{\changefont{cmss}{bx}{n}\large\thepage}

\vspace*{-11mm}

\begin{center}
{\large\it Los nuevos elegidos del reino }
\end{center}

\vspace{-3mm}

\begin{tabbing}

{\changefont{cmss}{bx}{n} Entrada:\ \ \ \ \ }\= Juntos como hermanos (est. 2), Qué alegría\footnotemark[1], Vienen con alegría.\\ \\  


{\changefont{cmss}{bx}{n} Salmo:} \> 65 ant. 1 ``Todo el mundo cante la gloria de Dios ...'' (estr. 1, 4 o 6).\\ 
\> (antífona de reemplazo: Sal 144 ant.1: ``Te alabamos, Señor,...'') \\ \\

{\changefont{cmss}{bx}{n} Ofrendas:} \> Los frutos de la tierra\footnotemark[2], Toma Señor nuestra vida\footnotemark[2], Al altar nos acercamos, \\
\> Pan de vida y bebida de luz. \\  \\ 

{\changefont{cmss}{bx}{n} Comunión:} \> Queremos ser Señor\footnotemark[3], Yo soy el camino\footnotemark[3], Yo soy el pan de vida\footnotemark[3], Es mi Padre\footnotemark[3],  \\
\> Vayamos a la mesa. \\ \\

{\changefont{cmss}{bx}{n} Post-com.:} \> Al atardecer de la vida (est. 1), Cántico de caridad (Bendigamos al Señor),  \\
\> Si el mismo pan comimos, Mirarte sólo a ti Señor, Nada te turbe (Taizé). \\ \\

{\changefont{cmss}{bx}{n} Salida:} \> Vayan todos por el mundo\footnotemark[4], Anunciaremos tu reino, Mi camino eres tú\footnotemark[3], \\
\> Simple oración.\\ \\

\end{tabbing}

\vspace{-15mm}

\footnotetext[1]{Cfr. \emph{Que alegría} (estribillo) con Lc 13,22. }
\footnotetext[2]{Cfr. \emph{Los frutos de la tierra} (est. 1) y \emph{Toma Señor nuestra vida} (est. 3) con la ant. comunión (Sal 103,13-15).}
\footnotetext[3]{Cfr. \emph{Queremos ser Señor} (estribillo) con Lc 13,30; \emph{Yo soy el camino} y \emph{Mi camino eres tu} con la aclamación \mbox{(Jn 14,6)}; cfr. \emph{Yo soy el pan de vida} y \emph{Es mi Padre} con la antífona de comunión (Jn 6,55). }
\footnotetext[4]{Cfr. \emph{Vayan todos por el mundo} (estribillo) con antífona del salmo (Mc 16,1).}



%------------------------
\newpage
\setcounter{footnote}{0}
\thispagestyle{fancy}
\fancyhead{}
\fancyfoot{}
%\footskip=-1cm

\lhead{\changefont{cmss}{bx}{n} \small Revisión 2013}
\chead{\changefont{cmss}{bx}{n} \small Domingo XXII - Durante el año}
\rhead{\changefont{cmss}{bx}{n} \small Ciclo C}
\rfoot{\changefont{cmss}{bx}{n}\large\thepage}

\vspace*{-11mm}

\begin{center}
{\large\it La humildad cristiana }
\end{center}

\vspace{-3mm}

\begin{tabbing}

{\changefont{cmss}{bx}{n} Entrada:\ \ \ \ \ }\= El sermón de la montaña\footnotemark[1], Pueblo de reyes\footnotemark[1], El Señor nos llama, Vine a alabar.\\ \\  


{\changefont{cmss}{bx}{n} Salmo:} \> 32 ant. 1 ``Cantemos todos al Señor ...'' (estr. 1, 7 u 8).\\ 
\> (antífona de reemplazo: Sal 26 ant.1: ``Cantaré y celebraré...'') \\ \\

{\changefont{cmss}{bx}{n} Ofrendas:} \> Mira nuestra ofrenda\footnotemark[2], Coplas de Yaraví\footnotemark[2], Padre nuestro recibid, \\
\> Pan de vida y bebida de luz. \\  \\ 

{\changefont{cmss}{bx}{n} Comunión:} \> Queremos ser Señor, Bendeciré al Señor\footnotemark[3], Panis angelicus\footnotemark[3], Pescador de hombres\footnotemark[3]. \\ \\

{\changefont{cmss}{bx}{n} Post-com.:} \> Vaso nuevo (El alfarero), Si el mismo pan comimos, Mirarte sólo a ti. \\ \\

{\changefont{cmss}{bx}{n} Salida:} \> Madre de los peregrinos\footnotemark[4], Junto a ti María\footnotemark[4], Simple oración, \\
\> Mi camino eres tu.\\ \\

\end{tabbing}

\vspace{-15mm}

\footnotetext[1]{Cfr. \emph{El sermón de la montaña} (estribillo) con Eclo 3,29, (est, 1) con Lc 14,11, (est. 2 y 4) con ant. comunión (Mt 5,9-10). \emph{Pueblo de reyes} (est. 2) con aclamación (Mt 11,29).  }
\footnotetext[2]{Cfr. \emph{Mira nuestra ofrenda} con Sal 67,6; \emph{Coplas de Yaraví} (est. 4) con Eclo 3,18 y Lc 14,11.}
\footnotetext[3]{Cfr. \emph{Bendeciré al Señor} (est. 1, 3 y 4) con Sal 67,4-5.10-11; cfr. \emph{Panis angelicus} con aclamación (Mt 11,29); cfr. \emph{Pescador de hombres} (est. 1 y 2) con Eclo 3,17-18 y Sal 67,11.}
\footnotetext[4]{Cfr. \emph{Madre de los peregrinos} (est. 1) y \emph{Junto a to María} (est. 2) con Eclo 3,17-18 y Lc 14,11.}





%------------------------
\newpage
\setcounter{footnote}{0}
\thispagestyle{fancy}
\fancyhead{}
\fancyfoot{}
%\footskip=-1cm

\lhead{\changefont{cmss}{bx}{n} \small Revisión 2013}
\chead{\changefont{cmss}{bx}{n} \small Domingo XXIII - Durante el año}
\rhead{\changefont{cmss}{bx}{n} \small Ciclo C}
\rfoot{\changefont{cmss}{bx}{n}\large\thepage}

\vspace*{-11mm}

\begin{center}
{\large\it La necesidad del desprendimiento }
\end{center}

\vspace{-3mm}

\begin{tabbing}

{\changefont{cmss}{bx}{n} Entrada:\ \ \ \ \ }\= Somos la familia de Jesús\footnotemark[1], Vienen con alegría, Vine a alabar.\\ \\  


{\changefont{cmss}{bx}{n} Salmo:} \> 89 ant. 1 ``Nuestra vida, Señor, pasa como un soplo...'' (estr. 2, 3 o 5).\\ 
\> (antífona de reemplazo: Sal 15 ``Protégeme, Dios mío, porque en ti me refugio'') \\ \\

{\changefont{cmss}{bx}{n} Ofrendas:} \> Mira nuestra ofrenda\footnotemark[2], Te ofrecemos Padre nuestro\footnotemark[2] (vidala),  \\ 
\> Toma Señor nuestra vida, Pan de vida y bebida de luz. \\ \\ 

{\changefont{cmss}{bx}{n} Comunión:} \> Creo en ti Señor (Más cerca oh Dios)\footnotemark[3], Queremos ser Señor\footnotemark[3], Jesús te seguiré\footnotemark[3],  \\
\> Yo soy el camino. \\ \\

{\changefont{cmss}{bx}{n} Post-com.:} \> Mirarte sólo a ti Señor, Nada te turbe (Taizé), Adoremos a Dios. \\ \\

{\changefont{cmss}{bx}{n} Salida:} \> Soy peregrino, Canción del testigo\footnotemark[4], Madre de los peregrinos.\\ \\

\end{tabbing}

\vspace{-15mm}

\footnotetext[1]{Se basa en Lc 8,21. Sin embargo, se correponde con evangelio del día (Lc 14,26). }
\footnotetext[2]{Cfr. \emph{Mira nuestra ofrenda} (est. 2) con Lc 14,33 y \emph{Te ofrecemos Padre nuestro} (est. 1) con Lc 14,27 y Fil 15-17.}
\footnotetext[3]{Cfr. \emph{Creo en ti Señor} (est. 1 y 5) con Lc 14,27; \emph{Queremos ser Señor} (estribillo y est. 1) con Lc 14,25-33; \emph{Jesús te seguiré} (est. 2) con Sab 9,18 y la aclamación al evangelio Sal 118,135.  }
\footnotetext[4]{Cfr. con Sal 89,14.}


%------------------------
\newpage
\setcounter{footnote}{0}
\thispagestyle{fancy}
\fancyhead{}
\fancyfoot{}
%\footskip=-1cm

\lhead{\changefont{cmss}{bx}{n} \small Revisión 2013}
\chead{\changefont{cmss}{bx}{n} \small Domingo XXIV - Durante el año}
\rhead{\changefont{cmss}{bx}{n} \small Ciclo C}
\rfoot{\changefont{cmss}{bx}{n}\large\thepage}

\vspace*{-11mm}

\begin{center}
{\large\it Las parábolas de la misericordia de Dios }
\end{center}

\vspace{-3mm}

\begin{tabbing}

{\changefont{cmss}{bx}{n} Entrada:\ \ \ \ \ }\= El Señor nos llama\footnotemark[1], Iglesia peregrina, Juntos como hermanos, Vine a alabar.\\ \\  


{\changefont{cmss}{bx}{n} Salmo:} \> 50 ant. 1 ``Piedad Señor ...'' (estr. 1, 6 o 9).\\ \\

{\changefont{cmss}{bx}{n} Ofrendas:} \> Señor te ofrecemos\footnotemark[2], Pan de vida y bebida de luz\footnotemark[2], Recibe oh Dios el pan\footnotemark[2], \\
\> Te ofrecemos Padre nuestro (vidala)\footnotemark[2], Al altar nos acercamos. \\  \\ 

{\changefont{cmss}{bx}{n} Comunión:} \> Yo soy el camino\footnotemark[3], Cuerpo y sangre de Jesús\footnotemark[3], Vayamos a la mesa\footnotemark[3],\\
\>  Este es mi cuerpo, Como Cristo nos amó. \\ \\

{\changefont{cmss}{bx}{n} Post-com.:} \> Si el mismo pan comimos\footnotemark[3], Cántico de caridad (Bendigamos al Señor),  \\
\> Ubi cárita (Taizé), Tu fidelidad, Cuántas gracias te debemos, Adoremos a Dios. \\ \\

{\changefont{cmss}{bx}{n} Salida:} \> En medio de los pueblos, Mi camino eres tu, Salve oh Reina, Salve María, \\
\> Oh María, Oh Santísima.\\ \\

\end{tabbing}

\vspace{-15mm}

\footnotetext[1]{Cfr. estrofa 1 con Lc 15,1-7 y con el relato en Mt 22,1-14 y Lc 14,16-24. }
\footnotetext[2]{Cfr. \emph{Señor te ofrecemos} con Sal 50,19, la aclamación al evangelio 2 Cor 5,19 y el evangelio, espcialmente \mbox{Lc 15,1-7}; \emph{Pan de vida y bebida de luz} (est 2 y 3) con Lc 15,11-32; cfr. \emph{Recibe oh Dios} (est. 1 y 2) y \emph{Te ofrecemos Padre nuestro} (est. 2) con Sal 50,19. }
\footnotetext[3]{Cfr. \emph{Yo soy el camino} (est. 3) con Lc 15,4-7; cfr. \emph{Cuerpo y sangre de Jesús} (est. 4) y \emph{Si el mismo pan comimos } con ant. comunión (1 Cor 10,16); cfr. \emph{Vayamos a la mesa} (est. 2) con Lc 15,31-33.}


%------------------------
\newpage
\setcounter{footnote}{0}
\thispagestyle{fancy}
\fancyhead{}
\fancyfoot{}
%\footskip=-1cm

\lhead{\changefont{cmss}{bx}{n} \small Revisión 2013}
\chead{\changefont{cmss}{bx}{n} \small Domingo XXV - Durante el año}
\rhead{\changefont{cmss}{bx}{n} \small Ciclo C}
\rfoot{\changefont{cmss}{bx}{n}\large\thepage}

\vspace*{-11mm}

\begin{center}
{\large\it La parábola del administrador sagaz }
\end{center}

\vspace{-3mm}

\begin{tabbing}

{\changefont{cmss}{bx}{n} Entrada:\ \ \ \ \ }\= Brilló la luz\footnotemark[1], El sermón de la montaña\footnotemark[1], Caminaré, Qué alegría.\\ \\  


{\changefont{cmss}{bx}{n} Salmo:} \> 32 ant. 1 ``Cantemos todos al Señor ...'' (estr. 1, 7 u 8).\\ 
\> (antífona de reemplazo: Sal 26 ant. 1 ``Cantaré y celebraré al Señor'' \footnotemark[2]) \\ \\

{\changefont{cmss}{bx}{n} Ofrendas:} \> Bendeciré al Señor\footnotemark[3], Nuestros dones, Una espiga, Pan de vida y bebida de luz, \\
\> Te presentamos, Bendito seas. \\  \\ 

{\changefont{cmss}{bx}{n} Comunión:} \> Simple oración, Yo soy el camino\footnotemark[3], Quédate con nosotros\footnotemark[3], Vayamos a la mesa. \\ \\

{\changefont{cmss}{bx}{n} Post-com.:} \> Nada te turbe (Taizé)\footnotemark[3], Al atardecer de la vida, Nuestro maná, Cuántas gracias    \\
\>  te debemos, Adoremos a Dios, Alabe todo el mundo (Taizé). \\ \\

{\changefont{cmss}{bx}{n} Salida:} \> Anunciaremos tu reino\footnotemark[4], Canción del misionero, Simple oración, Soy peregrino. \\ \\

\end{tabbing}

\vspace{-15mm}

\footnotetext[1]{Cfr. \emph{Brilló la luz} (est. 1 y 5) con Lc 16,9. Para \emph{El sermón de la montaña} se prefiere cantar la estrofa 3. }
\footnotetext[2]{En lugar del Sal 26 ant. 1, se puede cantar Sal 99 ant. 1. ``Lleguemos hasta el Señor...''. }
\footnotetext[3]{Cfr. \emph{Bendeciré al Señor} (est. 3 y 4) con Lc 16,1-13; cfr. \emph{Yo soy el camino} (est. 3) con antífona de comunión \mbox{(Jn 10,14)}; cfr. \emph{Quédate con nosotros} con Am 8,4-7 y Lc 16,1-3; cfr. \emph{Nada te turbe} con Lc 16,13.}
\footnotetext[4]{Cfr. \emph{Anunciaremos tu reino} con Lc 16,9. }


%------------------------
\newpage
\setcounter{footnote}{0}
\thispagestyle{fancy}
\fancyhead{}
\fancyfoot{}
%\footskip=-1cm

\lhead{\changefont{cmss}{bx}{n} \small Revisión 2013}
\chead{\changefont{cmss}{bx}{n} \small Domingo XXVI - Durante el año}
\rhead{\changefont{cmss}{bx}{n} \small Ciclo C}
\rfoot{\changefont{cmss}{bx}{n}\large\thepage}

\vspace*{-11mm}

\begin{center}
{\large\it La parábola del hombre rico y el pobre Lázaro }
\end{center}

\vspace{-3mm}

\begin{tabbing}

{\changefont{cmss}{bx}{n} Entrada:\ \ \ \ \ }\= Pueblo de reyes (est. 3)\footnotemark[1], El sermón de la montaña\footnotemark[1], Pueblo de Dios,\\
\> Me pongo en tus manos (estribillo sólo), El Señor nos llama (est. 3).\\ \\  

{\changefont{cmss}{bx}{n} Salmo:} \> 145 ant. 1 ``El Señor es fiel a su Palabra ...'' (estr. 2, 4 o 5).\\ 
\> (antífona de reemplazo: Sal 26 ant. 1 ``Cantaré y celebraré al Señor'' \footnotemark[1]) \\ \\

{\changefont{cmss}{bx}{n} Ofrendas:} \> Padre nuestro recibid, Una espiga\footnotemark[2], Ofrenda de amor\footnotemark[2], Bendeciré al Señor, \\
\> Mira nuestra ofrenda, Bendito seas Señor. \\  \\ 

{\changefont{cmss}{bx}{n} Comunión:} \> Creo en ti Señor (Más cerca oh Dios)\footnotemark[3], Como Cristo nos amó\footnotemark[3], \\
\> Yo soy el pan de vida\footnotemark[3], Yo soy el camino. \\ \\

{\changefont{cmss}{bx}{n} Post-com.:} \> Tu fidelidad\footnotemark[4], El Señor es mi fortaleza (Taizé), Al atardecer de la vida. \\ \\

{\changefont{cmss}{bx}{n} Salida:} \> Canción del testigo\footnotemark[4], Anunciaremos tu reino, Madre de los peregrinos. \\ \\

\end{tabbing}

\vspace{-15mm}

\footnotetext[1]{Cfr. \emph{Pueblo de reyes} (estribillo) con 1 Tim 6,15 y est. 3 con Lc 16,19-31; cfr. \emph{El sermón de la montaña} (est. 1 y 2) con Lc 16,25, y estribillo con Lc 16,29; en lugar del Sal 26 ant. 1, se puede cantar Sal 99 ant. 1.}
\footnotetext[2]{Cfr. \emph{Una espiga} (est 2 y 4) con Lc 16,19-31; cfr. \emph{Ofrenda de amor (Por los niños)} (est. 1) con Lc 16.19-31.}
\footnotetext[3]{Cfr. \emph{Creo en ti Señor} (est. 1, 3 y 4) con Lc 16, 25 y con 1 Tim 6,12; cfr. \emph{Como Cristo nos amó} con 1 Tim 6,11-16 y Lc 16,19-31; cfr. \emph{Yo soy el pan de vida} (estribillo y est. 2 y 3) con Lc 16,22. }
\footnotetext[4]{Cfr. \emph{Tu fidelidad} con Lc 16,31 y el paralelismo en Jn 12,7-8; \emph{Canción del testigo} (est. 2) con Sal 145,14. }



%------------------------
\newpage
\setcounter{footnote}{0}
\thispagestyle{fancy}
\fancyhead{}
\fancyfoot{}
%\footskip=-1cm

\lhead{\changefont{cmss}{bx}{n} \small Revisión 2013}
\chead{\changefont{cmss}{bx}{n} \small Domingo XXVII - Durante el año}
\rhead{\changefont{cmss}{bx}{n} \small Ciclo C}
\rfoot{\changefont{cmss}{bx}{n}\large\thepage}

\vspace*{-11mm}

\begin{center}
{\large\it El poder de la fe y la parábola del servidor humilde }
\end{center}

\vspace{-3mm}

\begin{tabbing}

{\changefont{cmss}{bx}{n} Entrada:\ \ \ \ \ }\= El Señor nos llama\footnotemark[1], Juntos como hermanos (est. 2)\footnotemark[1], Un pueblo que camina\footnotemark[1], \\  
\> El sermón de la montaña\footnotemark[1], Vienen con alegría. \\ \\


{\changefont{cmss}{bx}{n} Salmo:} \> 94 ant. 1 ``Adoremos al Señor ...'' (estr. 1 y 3).\\  \\

{\changefont{cmss}{bx}{n} Ofrendas:} \> Al altar nos acercamos\footnotemark[2], Comienza el sacrificio\footnotemark[2], Los frutos de la tierra\footnotemark[2],  \\
\> Una espiga\footnotemark[2]. \\  \\ 

{\changefont{cmss}{bx}{n} Comunión:} \> Queremos ser Señor\footnotemark[3], Creo en ti Señor\footnotemark[3], Escondido\footnotemark[3], Jesús eucaristía\footnotemark[3].  \\ \\


{\changefont{cmss}{bx}{n} Post-com.:} \> Tu fidelidad\footnotemark[4], El Señor es mi fortaleza, Creo en ti, Si el mismo pan comimos\footnotemark[4].    \\ \\

{\changefont{cmss}{bx}{n} Salida:} \> Soy peregrino\footnotemark[5], Simple oración\footnotemark[3], Vayan por el mundo\footnotemark[2], Madre de los peregrinos\footnotemark[5]. \\ \\

\end{tabbing}

\vspace{-15mm}

\footnotetext[1]{Cfr. \emph{El Señor nos llama} (est. 3), \emph{Juntos como hermanos} (est. 2) y \emph{Un pueblo que camina} (est. 3) con Lc 17,5; \mbox{cfr. \emph{El sermón de la montaña}} (est. 1) con Lc 17,10. }
\footnotetext[2]{Cfr. \emph{Al altar nos acercamos} (est. 3) con Lc 17,3-4; cfr. \emph{Comienza el sacrificio} (est. 2) con Hab 2,4 y Lc 17,5; \mbox{cfr. \emph{Los frutos de la tierra}} con Lc 17,10; cfr. \emph{Una espiga} (est. 4) y \emph{Vayan todos por el mundo} (estribillo) con Lc 17,3-6. }
\footnotetext[3]{Cfr. \emph{Queremos ser Señor} (estribillo) con Lc 17,10; cfr. \emph{Creo en ti Señor} (est. 2) con Lc 17,5; cfr. \emph{Jesús eucaristía} (est. 3) y \emph{Simple oración} (est. 1 y 4) con Lc 17,3-4.}
\footnotetext[4]{Cfr. \emph{Tu fidelidad} con Hab 2,3 y ant. comun. Lam 3,25; cfr. \emph{Si el mismo pan comimos} con ant. comun. 1 Cor 10,17.}
\footnotetext[5]{Cfr. \emph{Soy peregrino} (estribillo) con Lc 17,5; cfr. \emph{Madre de los peregrinos} (est. 1) con con Lc 17,5.10. }


%------------------------
\newpage
\setcounter{footnote}{0}
\thispagestyle{fancy}
\fancyhead{}
\fancyfoot{}
%\footskip=-1cm

\lhead{\changefont{cmss}{bx}{n} \small Revisión 2013}
\chead{\changefont{cmss}{bx}{n} \small Domingo XXVIII - Durante el año}
\rhead{\changefont{cmss}{bx}{n} \small Ciclo C}
\rfoot{\changefont{cmss}{bx}{n}\large\thepage}

\vspace*{-11mm}

\begin{center}
{\large\it La curación de los diez leprosos }
\end{center}

\vspace{-3mm}

\begin{tabbing}

{\changefont{cmss}{bx}{n} Entrada:\ \ \ \ \ }\= Pueblo de Dios\footnotemark[1], Brilló la luz\footnotemark[1], Juntos como hermanos (est. 2), Vine a alabar. \\ \\ 


{\changefont{cmss}{bx}{n} Salmo:} \> 97 ant. 2 ``El Señor ha triunfado...'' (estr. 1, 2 o 3)\footnotemark[2].\\ 
\> (antífona de reemplazo: Sal 95 ant. 1 ``Cantamos al Señor un canto nuevo'') \\ \\

{\changefont{cmss}{bx}{n} Ofrendas:} \> Señor te ofrecemos\footnotemark[3],$\,$Al altar del Señor\footnotemark[3],$\,$Bendeciré al Señor\footnotemark[3],$\,$Padre nuestro recibid.  \\ \\

{\changefont{cmss}{bx}{n} Comunión:} \> Es mi Padre\footnotemark[4], Creo en ti Señor\footnotemark[4], Yo soy el camino\footnotemark[4], Bendeciré al Señor\footnotemark[3],  \\ 
\> Jesús te seguiré\footnotemark[4].\\ \\


{\changefont{cmss}{bx}{n} Post-com.:} \> Tu fidelidad, El Señor es mi fortaleza (Taizé), Alabe todo el mundo (Taizé), \\ 
\> Cuantas gracias te debemos, Adoremos a Dios.    \\ \\

{\changefont{cmss}{bx}{n} Salida:} \> Cantemos hermanos\footnotemark[5], Jesús te seguiré, Canción del testigo, Mi camino eres tu. \\ \\

\end{tabbing}

\vspace{-13mm}

\footnotetext[1]{Cfr. \emph{Pueblo de Dios} (estribillo) con Sal 97,1; \emph{Brilló la luz} (est. 1) con antífona de comunión del día Sal 33,11; \emph{Juntos como hermanos} (est. 2) con Lc 17,19. }
\footnotetext[2]{Si se prefiere, se puede reemplazar la antífona por la ant. 1: ``Cantemos al Señor un canto nuevo...''.}
\footnotetext[3]{Cfr. \emph{Señor te ofrecemos} con Lc 17,11-19; cfr. \emph{Al altar del Señor} (est. 3) con Lc 17,13; cfr. \emph{Bendeciré al Señor} con antífona de comunión del día Sal 33,11.}
\footnotetext[4]{Cfr. \emph{Es mi Padre} y \emph{Creo en ti Señor} (Más cerca oh Dios) con Lc 17,19; cfr. \emph{Jesús te seguiré} (est. 3) con \mbox{Lc 17,11-19}.}
\footnotetext[5]{Cfr. \emph{Cantemos hermanos} con Sal 97,1. }


%------------------------
\newpage
\setcounter{footnote}{0}
\thispagestyle{fancy}
\fancyhead{}
\fancyfoot{}
%\footskip=-1cm

\lhead{\changefont{cmss}{bx}{n} \small Revisión 2013}
\chead{\changefont{cmss}{bx}{n} \small Domingo XXIX - Durante el año}
\rhead{\changefont{cmss}{bx}{n} \small Ciclo C}
\rfoot{\changefont{cmss}{bx}{n}\large\thepage}

\vspace*{-11mm}

\begin{center}
{\large\it La parábola del juez y la viuda }
\end{center}

\vspace{-3mm}

\begin{tabbing}

{\changefont{cmss}{bx}{n} Entrada:\ \ \ \ \ }\= Caminaré\footnotemark[1], Juntos como hermanos (est. 2)\footnotemark[1], Qué alegría, Vine a alabar. \\ \\ 


{\changefont{cmss}{bx}{n} Salmo:} \> 120 ant. 1 ``Nuestra ayuda nos viene...'' (estr. 1, 2, 3 o 4).\\ 
\> (antífona de reemplazo: Sal 26 ant. 2 ``El Señor es mi luz mi salvación...'') \\ \\

{\changefont{cmss}{bx}{n} Ofrendas:} \> Comienza el sacrificio\footnotemark[1], Mira nuestra ofrenda, Recibe oh Dios el pan, \\
\> Toma Señor nuestra vida, Te presentamos, Bendito seas.  \\ \\

{\changefont{cmss}{bx}{n} Comunión:} \> Quédate con nosotros\footnotemark[2], Simple oración\footnotemark[2], Queremos ser Señor, Yo soy el camino. \\  \\ 


{\changefont{cmss}{bx}{n} Post-com.:} \> Mirarte sólo a ti Señor, Tu fidelidad, Adoremos a Dios (est. 3), \\ 
\> Cuantas gracias te debemos.    \\ \\

{\changefont{cmss}{bx}{n} Salida:} \> Salve oh Reina, Canción del misionero\footnotemark[3], Mi camino eres tu, Sta. María del camino. \\ \\

\end{tabbing}

\vspace{-15mm}

\footnotetext[1]{Cfr. \emph{Caminaré} (est. 1) con antífona de entrada del día Sal 16,6 y con Sal 120,2-3; \emph{Juntos como hermanos} (est. 2) con Sal 120,2-3 y con Lc 18,1; cfr. \emph{Comienza el sacrificio} (est. 2) con Lc 18,8. }
\footnotetext[2]{Cfr. \emph{Quédate con nosotros} est. 3 con Lc 18,2-6 y la est. 4 con Lc 18,1; cfr. \emph{Simple oración} con 2 Tim 3,17. }
\footnotetext[3]{Cfr. \emph{Canción del misionero} con 2 Tim 3,14-4,2. }


%------------------------
\newpage
\setcounter{footnote}{0}
\thispagestyle{fancy}
\fancyhead{}
\fancyfoot{}
%\footskip=-1cm

\lhead{\changefont{cmss}{bx}{n} \small Revisión 2013}
\chead{\changefont{cmss}{bx}{n} \small Domingo XXX - Durante el año}
\rhead{\changefont{cmss}{bx}{n} \small Ciclo C}
\rfoot{\changefont{cmss}{bx}{n}\large\thepage}

\vspace*{-11mm}

\begin{center}
{\large\it La parábola del fariseo y el publicano }
\end{center}

\vspace{-3mm}

\begin{tabbing}

{\changefont{cmss}{bx}{n} Entrada:\ \ \ \ \ }\= El sermón de la montaña\footnotemark[1], Pueblo de reyes\footnotemark[1], Juntos como hermanos (est. 2)\footnotemark[1], \\
\> Qué alegría\footnotemark[1]. \\ \\ 


{\changefont{cmss}{bx}{n} Salmo:} \> 33 ant. 1 ``Vayamos a gustar...'' (o bien ant. 2) (estr. 1, 8, 9 o 10).\\ \\

{\changefont{cmss}{bx}{n} Ofrendas:} \> Recibe oh Dios al pan\footnotemark[1], Bendito seas, Los frutos de la tierra, \\
\> Pan de vida y bebida de luz.  \\ \\

{\changefont{cmss}{bx}{n} Comunión:} \> Oh buen Jesús\footnotemark[2], Bendeciré al Señor\footnotemark[2], La canción de la Alianza,\\
\> Cuerpo y sangre de Jesús, Este es mi cuerpo. \\  \\ 


{\changefont{cmss}{bx}{n} Post-com.:} \> Cántico de caridad (Bendigamos al Señor), Si el mismo pan comimos,  \\ 
\> El Señor es mi fortaleza (de Taizé), Adoremos a Dios.    \\ \\

{\changefont{cmss}{bx}{n} Salida:} \> Cantemos hermanos\footnotemark[3], Canción del testigo\footnotemark[3], Vayan todos por el mundo\footnotemark[3], \\
\> Canción del misionero. \\ \\

\end{tabbing}

\vspace{-15mm}

\footnotetext[1]{Cfr. \emph{El sermón de la montaña} (est. 1) con Eclo 35, 12-14.16-18, Sal 33,2-3 y Lc 18,13-14;\emph{Pueblo de reyes}, (estribillo) con Sal 33,2 (cantar especialmente estrofa 4); \emph{Juntos como hermanos} (est. 2) con con Lc 18,13-14; \mbox{cfr. \emph{Qué alegría}} con  antífona de entrada del día Sal 104,3; cfr. \emph{Recibe oh Dios al pan} con Lc 18,13-14. }
\footnotetext[2]{Cfr. \emph{Oh buen Jesús} (est. 2) con Lc 18,13-14; cfr. \emph{Bendeciré al Señor} con Sal 33. }
\footnotetext[3]{Cfr. \emph{Cantemos hermanos} (est. 2) con Lc 18,13-14; \emph{Canción del testigo} y \emph{Vayan todos} con 2 Tim 4,6-8.16-18. }

%------------------------
\newpage
\setcounter{footnote}{0}
\thispagestyle{fancy}
\fancyhead{}
\fancyfoot{}
%\footskip=-1cm

\lhead{\changefont{cmss}{bx}{n} \small Revisión 2013}
\chead{\changefont{cmss}{bx}{n} \small Domingo XXXI - Durante el año}
\rhead{\changefont{cmss}{bx}{n} \small Ciclo C}
\rfoot{\changefont{cmss}{bx}{n}\large\thepage}

\vspace*{-11mm}

\begin{center}
{\large\it La conversión de Zaqueo }
\end{center}

\vspace{-3mm}

\begin{tabbing}

{\changefont{cmss}{bx}{n} Entrada:\ \ \ \ \ }\= El Señor nos llama\footnotemark[1], Vine a alabar\footnotemark[1], Vienen con alegría\footnotemark[1], \\
\> Iglesia peregrina de Dios. \\ \\ 


{\changefont{cmss}{bx}{n} Salmo:} \> 144 ant. 1 ``Te alabamos Señor'' (estr. 1, 4 o 5).\\ \\

{\changefont{cmss}{bx}{n} Ofrendas:} \> Pan de vida y bebida de luz\footnotemark[2], Este es nuestro pan, Te presentamos, \\
\> Bendeciré al Señor.  \\ \\

{\changefont{cmss}{bx}{n} Comunión:} \> Como Cristo nos amó\footnotemark[3], Jesucristo danos de este pan, Cuerpo y sangre de Jesús,\\
\> Este es mi cuerpo. \\  \\ 


{\changefont{cmss}{bx}{n} Post-com.:} \> Cuántas gracias te debemos, Vaso nuevo (El alfarero), Al atardecer de la vida,  \\ 
\> Tan cerca de mi.    \\ \\

{\changefont{cmss}{bx}{n} Salida:} \> En medio de los pueblos, Cantemos hermanos, Mi camino eres tu, Soy peregrino, \\
\> Anunciaremos tu reino. \\ \\

\end{tabbing}

\vspace{-15mm}

\footnotetext[1]{Cfr. \emph{El Señor nos llama} (est. 1) con Sal 144,1-2 y Lc 19,1-2;\emph{Vine a alabar}, (estribillo) con Sal 144,1; \mbox{\emph{Vienen con alegría}} (est. 1) con Lc 19,1-10; }
\footnotetext[2]{Cfr. \emph{Pan de vida y bebida de luz} (est. 2 y 3) con Lc 19,1-10. }
\footnotetext[3]{Cfr. \emph{Como Cristo nos amó} (est. 2) con la aclamación al evangelio Jn 3,16. }


%------------------------
\newpage
\setcounter{footnote}{0}
\thispagestyle{fancy}
\fancyhead{}
\fancyfoot{}
%\footskip=-1cm

\lhead{\changefont{cmss}{bx}{n} \small Revisión 2013}
\chead{\changefont{cmss}{bx}{n} \small Domingo XXXII - Durante el año}
\rhead{\changefont{cmss}{bx}{n} \small Ciclo C}
\rfoot{\changefont{cmss}{bx}{n}\large\thepage}

\vspace*{-11mm}

\begin{center}
{\large\it Discusión sobre la resurrección de los muertos }
\end{center}

\vspace{-3mm}

\begin{tabbing}

{\changefont{cmss}{bx}{n} Entrada:\ \ \ \ \ }\= Caminaré\footnotemark[1], Alabaré, Un pueblo que camina, Qué alegría. \\ \\


{\changefont{cmss}{bx}{n} Salmo:} \> 17 ant. 1 ``Te amo Señor, mi fuerza...'' (estr. 1,2,4,6 o 7).\\
\> (antífona de reemplazo: Sal 24 ant. 1 ``!`A ti elevo mi alma,...'') \\ \\

{\changefont{cmss}{bx}{n} Ofrendas:} \> Comienza el sacrificio\footnotemark[1], Recibe oh Dios eterno, Los frutos de la tierra, \\
\> Te presentamos, Bendito seas.  \\ \\

{\changefont{cmss}{bx}{n} Comunión:} \> Yo soy el pan de vida\footnotemark[2], Es mi Padre\footnotemark[2], Como Cristo nos amó, Escondido, \\
\> Yo soy el camino, Queremos ser Señor\footnotemark[2]. \\  \\ 


{\changefont{cmss}{bx}{n} Post-com.:} \> El Señor es mi fortaleza (Taizé)\footnotemark[3], Alabe todo el mundo (Taizé), Adoremos a Dios.  \\ \\

{\changefont{cmss}{bx}{n} Salida:} \> Soy peregrino\footnotemark[3], Salve oh Reina\footnotemark[4], Salve María\footnotemark[4], Oh María\footnotemark[4]. \\  \\

\end{tabbing}

\vspace{-15mm}

\footnotetext[1]{Cfr. \emph{Caminaré} con Sal 16,1.5-6; cfr. \emph{Comienza el sacrificio} (est. 2) con 2 Mac 7,2. }
\footnotetext[2]{Cfr. \emph{Yo soy el pan de vida} (estribillo) con Lc 20,34-37; \emph{Es mi Padre} (est. 3) con Lc 20,37-38; cfr.\emph{Como Cristo nos amó} (est 4) con Lc 20,38; cfr. \emph{Queremos ser Señor} (est. 3) con Lc 20,37-38. }
\footnotetext[3]{Cfr. \emph{El Señor es mi fortaleza} y \emph{Soy peregrino} con 1$^\circ$ lectura (2 Mac 7,1-14). }
\footnotetext[4]{Si estamos en el mes preparatorio para el 8 de diciembre, conviene cantar un canto a María. }

%------------------------
\newpage
\setcounter{footnote}{0}
\thispagestyle{fancy}
\fancyhead{}
\fancyfoot{}
%\footskip=-1cm

\lhead{\changefont{cmss}{bx}{n} \small Revisión 2013}
\chead{\changefont{cmss}{bx}{n} \small Domingo XXXIII - Durante el año}
\rhead{\changefont{cmss}{bx}{n} \small Ciclo C}
\rfoot{\changefont{cmss}{bx}{n}\large\thepage}

\vspace*{-11mm}

\begin{center}
{\large\it Los signos precursores del fin }
\end{center}

\vspace{-3mm}

\begin{tabbing}

{\changefont{cmss}{bx}{n} Entrada:\ \ \ \ \ }\= Iglesia peregrina, Pueblo de Dios peregrino\footnotemark[1], Qué alegría, Vine a alabar. \\ \\


{\changefont{cmss}{bx}{n} Salmo:} \> 97 ant. 2 ``El Señor ha triunfado...'' ó ant. 1 ``Cantemos al Señor...'' (estr. 4,5 o 6).\\
\> (antífona de reemplazo: Sal 95 ``Cantemos al Señor un canto nuevo'' - Bevilacqua) \\ \\

{\changefont{cmss}{bx}{n} Ofrendas:} \> Mira nuestra ofrenda, Este es nuestro pan\footnotemark[2], Comienza el sacrificio\footnotemark[2],  \\
\> Los frutos de la tierra.  \\ \\

{\changefont{cmss}{bx}{n} Comunión:} \> No hay mayor amor, Cuerpo y sangre de Jesús, Jesucristo danos de este pan, \\
\> Este es mi cuerpo. \\  \\ 


{\changefont{cmss}{bx}{n} Post-com.:} \> Si el mismo pan comimos, Tu fidelidad, Mirarte sólo a ti, Adoremos a Dios.  \\ \\

{\changefont{cmss}{bx}{n} Salida:} \> Santa María del camino\footnotemark[3], Madre de los peregrinos\footnotemark[3], Madre de nuestro pueblo  \\
\> (est. 1 y 3)\footnotemark[3], Quiero decir que sí, Soy peregrino\footnotemark[4]. \\  \\

\end{tabbing}

\vspace{-15mm}

\footnotetext[1]{Canto compuesto por Matías Sagreras, organista de la Basílica del Santísimo Sacramento. }
\footnotetext[2]{Cfr. \emph{Mira nuestra ofrenda} (estribillo)  con Lc 21,1-5; cfr. \emph{Este es nuestro pan} (est. 2) con la aclamación al evangelio del día (Lc 21,28); cfr. \emph{Comienza el sacrificio} (est. 2) con Lc 21,19. }
\footnotetext[3]{Por estar en el mes de María. }
\footnotetext[4]{Cfr. \emph{Soy peregrino} (est. 3) con Lc 21,19. }



%------------------------
\newpage
\setcounter{footnote}{0}
\thispagestyle{fancy}
\fancyhead{}
\fancyfoot{}
%\footskip=-1cm

\lhead{\changefont{cmss}{bx}{n} \small Revisión 2013}
\chead{\changefont{cmss}{bx}{n} \small Domingo XXXIV - Durante el año}
\rhead{\changefont{cmss}{bx}{n} \small Ciclo C}
\rfoot{\changefont{cmss}{bx}{n}\large\thepage}

\vspace*{-11mm}

\begin{center}
{\large\it (Domingo de Cristo Rey) El buen ladrón }
\end{center}

\vspace{-3mm}

\begin{tabbing}

{\changefont{cmss}{bx}{n} Entrada:\ \ \ \ \ }\= Pueblo de reyes\footnotemark[1], Pueblo de Dios\footnotemark[1], Alabaré\footnotemark[1], Qué alegría\footnotemark[1]. \\ \\


{\changefont{cmss}{bx}{n} Salmo:} \> 121 ant. 1 ``!`Como me alegré...'' (estr. 1, 2, 3 o 5).\\
\> (reemplazo: Sal 92 ant. 1 ``Reina el Señor...'' con estr. 1, 3 o 4) \\ \\

{\changefont{cmss}{bx}{n} Ofrendas:} \> Padre nuestro recibid\footnotemark[2], Te ofrecemos oh Señor, Bendito seas, Al altar del Señor.\\ \\

{\changefont{cmss}{bx}{n} Comunión:} \> Rey de los reyes\footnotemark[3], Panis angelicus\footnotemark[3], Cuerpo y sangre de Jesús\footnotemark[3],\\
\> Vayamos a la mesa, Este es mi cuerpo. \\  \\ 


{\changefont{cmss}{bx}{n} Post-com.:} \> Alabe todo el mundo (de Taizé), Adoremos a Dios, Padre por tu bondad\footnotemark[4].\\ \\ 

{\changefont{cmss}{bx}{n} Salida:} \> Christus vincit, Oh María, Oh Santísima, Madre de los peregrinos, \\
\> Madre de nuestro pueblo, Mi camino eres tu. \\ \\

\end{tabbing}

\vspace{-15mm}

\footnotetext[1]{Cfr. \emph{Pueblo de reyes} (estribillo) con antífona de entrada Apoc 1,6, est. 3 con Lc 23,35-43, est. 8 y 9
 con \mbox{Col 1,12.18-20}; \emph{Pueblo de Dios}, (est. 1) con Col 1,12; \emph{Alabaré} con antifona de entrada Apoc 5,12; \mbox{cfr. \emph{Qué alegría}} con  \mbox{Sal 121} (cantarlo, sobre todo, si se reemplaza el salmo por Sal 92). }
\footnotetext[2]{Cfr. \emph{Padre nuestro recibid} (estr. 4) con aclamación al evangelio Mc 11,9-10. }
\footnotetext[3]{Cfr. \emph{Rey de los reyes} con Col 1,12-13 y la antífona de entrada Apoc 1,6; cfr. \emph{Panis angelicus} con evangelio del día Lc 23,35-43;  \emph{Cuerpo y sangre de Jesús} (est. 1) con Lc 23,35-43 y (est. 3) con la antífona de entrada Apoc 1,6. }
\footnotetext[4]{La melodía de \emph{Padre por tu bondad} corresponde a \emph{Jesus, remember me} de la comunidad de Taizé.}

%------------------------
\newpage
\setcounter{footnote}{0}
\thispagestyle{fancy}
\fancyhead{}
\fancyfoot{}
%\footskip=-1cm

\lhead{\changefont{cmss}{bx}{n} \small Revisión 2013}
\chead{\changefont{cmss}{bx}{n} \small Fiesta del 2 de febrero - Presentación del Señor}
\rhead{\changefont{cmss}{bx}{n} \small Ciclo C}
\rfoot{\changefont{cmss}{bx}{n}\large\thepage}

\vspace*{-11mm}

\begin{center}
{\large\it La presentación de Jesús en el Templo }
\end{center}

\vspace{-3mm}

\begin{tabbing}

{\changefont{cmss}{bx}{n} Entrada:\ \ \ \ \ }\= Sal 23 (ant. 1)\footnotemark[1], Pueblo de reyes\footnotemark[1], Pueblo de Dios, Alabaré (est.  2 o 3). \\ \\


{\changefont{cmss}{bx}{n} Salmo:} \> 23 ant. 2 ``Felices los que son fieles al Señor...'' (estr. 1, 2 o 3). \\ 
\> (antífona de reemplazo: Sal 147 ant. 1 ``Glorifica al Señor Jerusalén...'') \\ \\

{\changefont{cmss}{bx}{n} Ofrendas:} \> Recibe oh Dios eterno\footnotemark[2], Toma Señor nuestra vida\footnotemark[2], Te ofrecemos Padre nuestro  \\
\> (vidala), Padre nuestro recibid. \\ \\ 

{\changefont{cmss}{bx}{n} Comunión:} \> Más cerca oh Dios\footnotemark[3], Pueblo de reyes\footnotemark[1], Como Cristo nos amó\footnotemark[3], Este es mi cuerpo, \\
\>  Bendeciré al Señor. \\  \\

{\changefont{cmss}{bx}{n} Post-com.:} \> Aleluia Cristo vino con su paz, Adoremos a Dios, Alabe todo el mundo (Taizé).\\ \\

{\changefont{cmss}{bx}{n} Salida:} \> Madre de nuestro pueblo (est. 5)\footnotemark[4], Canción del testigo\footnotemark[4], Soy peregrino. \\ \\


\end{tabbing}

\vspace{-15mm}

\footnotetext[1]{Se ingresa en procesión con las candelas encendidas. Si se canta \emph{Pueblo de reyes}, hacer especialmente est. 1, 2  y 3.  }
\footnotetext[2]{Cfr. \emph{Recibe oh Dios eterno} (est. 1) y \emph{Toma Señor nuestra vida} con Mal 3,3-4.}
\footnotetext[3]{Cfr. \emph{Más cerca oh Dios (Creo en ti Señor)} (est. 3) con el cánto de Simeón (Lc 2,29-32); cfr. \emph{Como Cristo nos amó} (est. 2) con  Heb. 2,17-18.  }
\footnotetext[4]{Cfr. \emph{Madre de nuestro pueblo} (est. 5) con Lc 2,22-40); el protagonista  de \emph{Canción del testigo} puede ser identificado, en cierto aspecto, con Simeón (Lc 2,22-40).}


%------------------------
\newpage
\setcounter{footnote}{0}
\thispagestyle{fancy}
\fancyhead{}
\fancyfoot{}
%\footskip=-1cm

\lhead{\changefont{cmss}{bx}{n} \small Revisión 2013}
\chead{\changefont{cmss}{bx}{n} \small Solemnidad del 8 de mayo - Nuestra Señora de Luján}
\rhead{\changefont{cmss}{bx}{n} \small Ciclo C}
\rfoot{\changefont{cmss}{bx}{n}\large\thepage}

\vspace*{-11mm}

\begin{center}
{\large\it Jesús y su madre }
\end{center}

\vspace{-3mm}

\begin{tabbing}

{\changefont{cmss}{bx}{n} Entrada:\ \ \ \ \ }\= Somos un pueblo que camina, Madre de los peregrinos, Pueblo de Dios peregrino\footnotemark[1],\\
\> La Virgen María nos reúne, Feliz de ti María, Iglesia peregrina de Dios.\\ \\  


{\changefont{cmss}{bx}{n} Salmo:} \> Magnificat. ``El Señor hizo en mí maravillas...'' (todo) \\ \\

{\changefont{cmss}{bx}{n} Ofrendas:} \> Bendeciré al Señor\footnotemark[2], Te ofrecemos oh Señor, Te presentamos, Bendito seas. \\ \\ 

{\changefont{cmss}{bx}{n} Comunión:} \> Ave Verum, Jesucristo danos de este pan\footnotemark[3], Bendeciré al Señor\footnotemark[2], Este es mi cuerpo, \\
\> Mi alma glorifica\footnotemark[3]. \\  \\

{\changefont{cmss}{bx}{n} Post-com.:} \> Quiero decir que sí, Bendita sea tu pureza, Bajo tu amparo (P. Bevilacqua).\\ \\

{\changefont{cmss}{bx}{n} Salida:} \> Madre de los peregrinos, Madre de nuestro pueblo, Oh María, Cantad a María,\\
\> Canto de María, Santa María del camino.\\ \\

\end{tabbing}

\vspace{-15mm}

\footnotetext[1]{Canto reciente compuesto por Matías Sagreras y grabado por el Grupo de Música Litúrgica (P. Esteban Sacchi). }
\footnotetext[2]{Cfr. \emph{Bendeciré al Señor} con Lc 1,46-55.}
\footnotetext[3]{\emph{Jesucristo danos de este pan} menciona a María en su est. 4; cfr. \emph{Mi alma glorifica} con Lc 1,46-55.  }

%------------------------
\newpage
\setcounter{footnote}{0}
\thispagestyle{fancy}
\fancyhead{}
\fancyfoot{}
%\footskip=-1cm

\lhead{\changefont{cmss}{bx}{n} \small Revisión 2013}
\chead{\changefont{cmss}{bx}{n} \small Solemnidad del 29 de junio - San Pedro y San Pablo}
\rhead{\changefont{cmss}{bx}{n} \small Ciclo C}
\rfoot{\changefont{cmss}{bx}{n}\large\thepage}

\vspace*{-11mm}

\begin{center}
{\large\it La profesión de fe de Pedro}
\end{center}

\vspace{-3mm}

\begin{tabbing}

{\changefont{cmss}{bx}{n} Entrada:\ \ \ \ \ }\= Cante la Iglesia\footnotemark[1], Iglesia peregrina, Un pueblo que camina, Vienen con alegría.\\ \\  


{\changefont{cmss}{bx}{n} Salmo:} \> 33 ant. 1 ``Vayamos a gustar la bondad del Señor'' (1, 2, 3 o 4) \\ \\

{\changefont{cmss}{bx}{n} Ofrendas:} \> Pan de vida y bebida de luz,$\,$Recibe oh Dios el pan\footnotemark[2],$\,$Te presentamos,$\,$Bendito seas. \\ \\ 

{\changefont{cmss}{bx}{n} Comunión:} \> Más cerca oh Dios\footnotemark[3], Pescador de hombres\footnotemark[3], Bendeciré al Señor\footnotemark[3],\\ \> El Señor de Galilea. \\  \\

{\changefont{cmss}{bx}{n} Post-com.:} \> Si el mismo pan comimos, El Señor es mi fortaleza (Taizé), Mirarte sólo a ti\footnotemark[4].\\ \\

{\changefont{cmss}{bx}{n} Salida:} \> En medio de los pueblos, Mi camino eres tú\footnotemark[4], Canción del testigo, \\ 
\> Vayan todos por el mundo. \\ \\

\end{tabbing}

\vspace{-15mm}

\footnotetext[1]{Canto para el día de \emph{todos} los Santos, en particular para San Pedro y San Pablo. }
\footnotetext[2]{\emph{Recibe oh Dios el pan} (est. 3) pide por los difuntos.  }
\footnotetext[3]{Cfr. \emph{Más cerca oh Dios} con Jn 21,18-19 (martirio de Pedro); cfr. \emph{Pescador de hombres} (est. 1 y 2) con Jn 21,19 y Hech 3,6; cfr. \emph{Bendeciré al Señor} con Sal 33,2-9.}
\footnotetext[4]{Cfr. \emph{Mirarte sólo a ti} y \emph{Mi camino eres tú} son perfectamente aplicables a la vida de los apóstoles Pedro y Pablo.}

%------------------------
\newpage
\setcounter{footnote}{0}
\thispagestyle{fancy}
\fancyhead{}
\fancyfoot{}
%\footskip=-1cm

\lhead{\changefont{cmss}{bx}{n} \small Revisión 2013}
\chead{\changefont{cmss}{bx}{n} \small Solemnidad del 15 de agosto - Asunción de la Virgen María}
\rhead{\changefont{cmss}{bx}{n} \small Ciclo C}
\rfoot{\changefont{cmss}{bx}{n}\large\thepage}

\vspace*{-11mm}

\begin{center}
{\large\it El canto de la Virgen María }
\end{center}

\vspace{-3mm}

\begin{tabbing}

{\changefont{cmss}{bx}{n} Entrada:\ \ \ \ \ }\= Feliz de ti María\footnotemark[1], La Virgen María nos reúne, Que alegría.\\ \\  


{\changefont{cmss}{bx}{n} Salmo:} \> 44 con ant. ``A tu diestra, Señor, resplandece la Reina'' (música de Sal 30 ant. 1) \\ 
\> (estr. 5 y 7 del Sal 44) \\ \\

{\changefont{cmss}{bx}{n} Ofrendas:} \> Bendeciré al Señor\footnotemark[2], Pan de vida y bebida de luz\footnotemark[2], Te presentamos, Bendito seas. \\ \\ 

{\changefont{cmss}{bx}{n} Comunión:} \> Bendeciré al Señor\footnotemark[2],$\,$Jesucristo danos de este pan\footnotemark[3],$\,$Mi alma glorifica\footnotemark[3],$\,$Es mi Padre. \\  \\

{\changefont{cmss}{bx}{n} Post-com.:} \> Quiero decir que sí, Bendita sea tu pureza.\\ \\

{\changefont{cmss}{bx}{n} Salida:} \> Un día la veré\footnotemark[4], Cantad a María, Los cielos la tierra\footnotemark[4], Canto de María.\\ \\

\end{tabbing}

\vspace{-15mm}

\footnotetext[1]{Cfr. \emph{Feliz de ti María} (est. 5) con la antífona de entrada Apoc 12,1 y con evangelio de la Vigilia Lc 11,27-28. }
\footnotetext[2]{Cfr. \emph{Bendeciré al Señor} con Lc 1,46-55; \emph{Pan de vida y bebida de luz} con 1 Cor 15,20-27.}
\footnotetext[3]{\emph{Jesucristo danos de este pan} menciona a María en su est. 4; cfr. \emph{Mi alma glorifica} con Lc 1,46-55; cfr. \emph{Es mi Padre} (est. 3) con 1 Cor 15,20-27.  }
\footnotetext[4]{ Cfr. \emph{Un día la veré} (est. 1) con Sal 44,18; \emph{Los cielos, la tierra} (est. 2) con Lc 1,39-43. }

%------------------------
\newpage
\setcounter{footnote}{0}
\thispagestyle{fancy}
\fancyhead{}
\fancyfoot{}
%\footskip=-1cm

\lhead{\changefont{cmss}{bx}{n} \small Revisión 2013}
\chead{\changefont{cmss}{bx}{n} \small Fiesta del 14 de septiembre - Exaltación de la Santa Cruz}
\rhead{\changefont{cmss}{bx}{n} \small Ciclo C}
\rfoot{\changefont{cmss}{bx}{n}\large\thepage}

\vspace*{-11mm}

\begin{center}
{\large\it El diálogo de Jesús con Nicodemo }
\end{center}

\vspace{-3mm}

\begin{tabbing}

{\changefont{cmss}{bx}{n} Entrada:\ \ \ \ \ }\=  Cruz de Cristo\footnotemark[1], Somos la familia de Jesús\footnotemark[1], Juntos como hermanos (est. 1), \\
\> Caminaré (est. 1). \\ \\


{\changefont{cmss}{bx}{n} Salmo:} \> 17 ant. 1 ``Te amo, Señor, mi fuerza...'' (estr. 1, 2 o 4). \\ 
\> (antífona de reemplazo: Sal 29 ant. 1 ``Te glorifico, Señor, ...'' del P. Bevilacqua) \\ \\

{\changefont{cmss}{bx}{n} Ofrendas:} \> Este es nuestro pan\footnotemark[1], Te ofrecemos Padre nuestro (nuevo)\footnotemark[1], Mira nuestra ofrenda,   \\
\> Bendito seas, Te presentamos. \\ \\ 

{\changefont{cmss}{bx}{n} Comunión:} \> Más cerca oh Dios\footnotemark[2], En memoria tuya\footnotemark[2], Jesús la imagen de Dios Padre\footnotemark[3], \\
\> En la postrera cena,  Como Cristo nos amó\footnotemark[2], Jesucristo danos de este pan.  \\  \\

{\changefont{cmss}{bx}{n} Post-com.:} \> Tu fidelidad, Adoremos a Dios, Mirarte solo a ti.\\ \\

{\changefont{cmss}{bx}{n} Salida:} \> Madre de nuestro pueblo (est. 1 y 9)\footnotemark[1], En medio de los pueblos, Mi camino eres tú \\
\> Santa María del camino, Salve oh Reina. \\ \\


\end{tabbing}

\vspace{-15mm}

\footnotetext[1]{\emph{Cruz de Cristo} se acostumbra cantar en cuaresma, pero su texto es perfecto para este día. Los demás cantos también hacen referencia directa a la cruz del Señor.   }
\footnotetext[2]{Cfr. \emph{Más cerca oh Dios} (est. 1 y 5) y \emph{Como Cristo nos amó} (est. 3) con Jn 3,13-17; cfr. \emph{Jesús la imagen de Dios Padre} con Fil 2,6-11; cfr. \emph{En memoria tuya} (est. 5 y 6) con Fil. 2,6-11. }



%------------------------
\newpage
\setcounter{footnote}{0}
\thispagestyle{fancy}
\fancyhead{}
\fancyfoot{}
%\footskip=-1cm

\lhead{\changefont{cmss}{bx}{n} \small Revisión 2013}
\chead{\changefont{cmss}{bx}{n} \small 1 de noviembre - Solemnidad de todos 
los santos}
\rhead{\changefont{cmss}{bx}{n} \small Ciclo C}
\rfoot{\changefont{cmss}{bx}{n}\large\thepage}

\vspace*{-11mm}

\begin{center}
{\large\it Las Bienaventuranzas }
\end{center}

\vspace{-3mm}

\begin{tabbing}

{\changefont{cmss}{bx}{n} Entrada:\ \ \ \ \ }\=  Cante la 
Iglesia\footnotemark[1], Brilló la luz\footnotemark[1], 
Alabaré\footnotemark[1], Iglesia peregrina. \\ \\


{\changefont{cmss}{bx}{n} Salmo:} \> 23 ant. 2 ``Felices los que son 
fieles...'' (estr. 1, 2 o 3). \\ 
\> (antífona de reemplazo: Sal 24 ant. 1 ``A ti elevo mi alma, ...'') \\ \\

{\changefont{cmss}{bx}{n} Ofrendas:} \> Recibe oh Dios el pan\footnotemark[2], 
Bendeciré al Señor, Te presentamos. \\  \\

{\changefont{cmss}{bx}{n} Comunión:} \> Brilló la luz\footnotemark[1], 
Cuerpo y Sangre de Jesús, Vayamos a la mesa, Es mi Padre.\\ \\

{\changefont{cmss}{bx}{n} Post-com.:} \> La misericordia del Señor (Taizé), Al 
atardecer de la vida, Adoremos a Dios.\\ \\

{\changefont{cmss}{bx}{n} Salida:} \> Cante la 
Iglesia\footnotemark[1], Anunciaremos tu Reino, Canción del testigo, \\
\> Salve oh Reina\footnotemark[3], En medio de los pueblos. \\ \\


\end{tabbing}

\vspace{-15mm}

\footnotetext[1]{\emph{Cante la Iglesia} es un canto especial para este día; 
cfr. \emph{Brilló la luz} con Mt 5,1-12; cfr. \emph{Alabaré} (est. 1) con 
\mbox{Apoc 72-14}. }
\footnotetext[2]{ \emph{Recibe oh Dios el pan} (est. 3) pide por los difuntos. }
\footnotetext[3]{Cfr. \emph{Salve oh Reina} (est. 2) habla del 
final de este ``destierro''.}




%------------------------
\newpage
\setcounter{footnote}{0}
\thispagestyle{fancy}
\fancyhead{}
\fancyfoot{}
%\footskip=-1cm

\lhead{\changefont{cmss}{bx}{n} \small Revisión 2013}
\chead{\changefont{cmss}{bx}{n} \small 2 de noviembre - Conmemoración de todos los fieles difuntos}
\rhead{\changefont{cmss}{bx}{n} \small Ciclo C}
\rfoot{\changefont{cmss}{bx}{n}\large\thepage}

\vspace*{-11mm}

\begin{center}
{\large\it El anuncio de la resurrección }
\end{center}

\vspace{-3mm}

\begin{tabbing}

{\changefont{cmss}{bx}{n} Entrada:\ \ \ \ \ }\=  Hacia ti morada santa\footnotemark[1], Brilló la luz\footnotemark[1], Sal 129 (ant. 1), Sal 83 (ant. 1), \\
\> Juntos como hermanos (est. 2). \\ \\


{\changefont{cmss}{bx}{n} Salmo:} \> 26 ant. 2 ``El Señor es mi luz, mi salvación...'' (estr. 1, 3, 5 o 7). \\ \\

{\changefont{cmss}{bx}{n} Ofrendas:} \> Recibe oh Dios el pan\footnotemark[2], Sé como el grano de trigo\footnotemark[2], Mira nuestra ofrenda. \\  \\

{\changefont{cmss}{bx}{n} Comunión:} \> Yo soy el pan de vida\footnotemark[3], Más cerca oh Dios\footnotemark[3], Brilló la luz\footnotemark[1], Es mi Padre,\\
\> Vayamos a la mesa. \\  \\

{\changefont{cmss}{bx}{n} Post-com.:} \> Mirarte sólo a ti\footnotemark[4], La misericordia del Señor (Taizé), Nada te turbe (Taizé).\\ \\

{\changefont{cmss}{bx}{n} Salida:} \> Soy peregrino\footnotemark[4], Salve oh Reina\footnotemark[4], En medio de los pueblos. \\ \\


\end{tabbing}

\vspace{-15mm}

\footnotetext[1]{Cfr. \emph{Hacia ti morada santa} se acostumbra para misas por los difuntos; \emph{Brilló la luz} es muy apropiada por las bienaventuranzas. }
\footnotetext[2]{ \emph{Recibe oh Dios el pan} (est. 3) pide por los difuntos; \emph{Sé como el grano de trigo} (est. 4) habla de la vuelta al Padre. }
\footnotetext[3]{Cfr. \emph{Yo soy el pan de vida} con ant. comunión (Jn 11,25-26); \emph{Más cerca oh Dios} (est. 3) pide morar cerca del Señor.  }
\footnotetext[4]{Cfr. \emph{Mirarte sólo a ti} con Sal 26,8.13; cfr. \emph{Soy peregrino} (est. 4) con Apoc 21,2-5; \emph{Salve oh Reina} (est. 2) habla del final de este ``destierro''.}



%------------------------
\newpage
\setcounter{footnote}{0}
\thispagestyle{fancy}
\fancyhead{}
\fancyfoot{}
%\footskip=-1cm

\lhead{\changefont{cmss}{bx}{n} \small Revisión 2013}
\chead{\changefont{cmss}{bx}{n} \small Fiesta del 9 de noviembre - Dedicación de San Juan de Letrán}
\rhead{\changefont{cmss}{bx}{n} \small Ciclo C}
\rfoot{\changefont{cmss}{bx}{n}\large\thepage}

\vspace*{-11mm}

\begin{center}
{\large\it La expulsión de los vendedores del Templo }
\end{center}

\vspace{-3mm}

\begin{tabbing}

{\changefont{cmss}{bx}{n} Entrada:\ \ \ \ \ }\=  Qué alegría\footnotemark[1], Pueblo de Dios\footnotemark[1], Pueblo de Reyes\footnotemark[1], Iglesia peregrina de Dios. \\ \\


{\changefont{cmss}{bx}{n} Salmo:} \> 94 ant. 1 ``Adoremos al Señor...'' (estr. 1 y 3). \\ \\

{\changefont{cmss}{bx}{n} Ofrendas:} \> Al altar del Señor, Te ofrecemos oh Señor, Te presentamos, Bendeciré al Señor. \\  \\

{\changefont{cmss}{bx}{n} Comunión:} \> Vayamos a la mesa\footnotemark[2], Cuerpo y sangre de Jesús\footnotemark[2], Este es mi cuerpo\footnotemark[2], \\
\> Yo soy el camino\footnotemark[2]. \\ \\

{\changefont{cmss}{bx}{n} Post-com.:} \> Adoremos a Dios, Alabe todo el mundo (Taizé), Cuantas gracias te debemos, \\
\> Si el mismo pan comimos.\\ \\

{\changefont{cmss}{bx}{n} Salida:} \> Soy peregrino\footnotemark[3], Madre de los peregrinos, Madre de nuestro pueblo, \\
\> Salve María,  Santa María del camino. \\ \\


\end{tabbing}

\vspace{-15mm}

\footnotetext[1]{Cfr. \emph{Qué alegría} con ant. de entrada (Apoc 21,2); cfr. \emph{Pueblo de Dios} (estribillo) con Sal 45,9; cfr. \emph{Pueblo de reyes} (est. 2 y 5) con ant. de entrada (Apoc 21,2) y 1 Cor 3,11. }
\footnotetext[2]{ Todos estos son cantos eucarísticos. Es bueno aprovechar estas fiestas, que no tienen canto específico, para cantarlos.  }
\footnotetext[3]{Cfr. \emph{Soy peregrino} con (est. 4) con Apoc 21,2-5; el resto de los cantos son marianos por el mes de María.  }





%------------------------
\newpage
\setcounter{footnote}{0}
\thispagestyle{fancy}
\fancyhead{}
\fancyfoot{}
%\footskip=-1cm

\lhead{\changefont{cmss}{bx}{n} \small Revisión 2013}
\chead{\changefont{cmss}{bx}{n} \small Memoria del 11 de noviembre - San Martín 
de Tours}
\rhead{\changefont{cmss}{bx}{n} \small Ciclo C}
\rfoot{\changefont{cmss}{bx}{n}\large\thepage}

\vspace*{-11mm}

\begin{center}
{\large\it El juicio final }
\end{center}

\vspace{-3mm}

\begin{tabbing}

{\changefont{cmss}{bx}{n} Entrada:\ \ \ \ \ }\=  El sermón de 
la montaña\footnotemark[1], Brilló la luz\footnotemark[1], El Señor nos llama, 
Vine a alabar. \\ \\


{\changefont{cmss}{bx}{n} Salmo:} \> 118 ant. 2 ``Felices los que escuchan...'' 
(estr. 1 y 2). \\ 
\> (antífona de reemplazo: Sal 102 ant. 2 ``El amor del Señor, ...'') \\ \\

{\changefont{cmss}{bx}{n} Ofrendas:} \> Mira nuestra ofrenda, Este es nuestro 
pan, Te presentamos, Bendeciré al Señor. \\  \\

{\changefont{cmss}{bx}{n} Comunión:} \> Queremos ser Señor\footnotemark[2], 
Yo soy el camino, Creo en ti (Más cerca oh Dios). \\ \\

{\changefont{cmss}{bx}{n} Post-com.:} \> Al atardecer de la vida, Nada te turbe 
(Taizé), Mirarte sólo a ti, \\
\> Si el mismo pan comimos.\\ \\

{\changefont{cmss}{bx}{n} Salida:} \> Santa María del camino, Madre de 
los peregrinos, Madre de nuestro pueblo. \\ \\


\end{tabbing}

\vspace{-15mm}

\footnotetext[1]{\emph{El sermón de la montaña} y \emph{Brilló la luz} 
corresponden a las bienaventuranzas, muy apropiado a la vida de San Martín de 
Tours (y al juicio final). }
\footnotetext[2]{ \emph{Queremos ser Señor} está en línea con la vida del 
santo al proclamar ``queremos ser Señor servidores de verdad''; cfr. \emph{Al 
altardecer de la vida} (estr. 1) con Mt 25,31-40.  }











%------------------------
\newpage
\thispagestyle{empty}

%\begin{center}
%{\changefont{cmss}{bx}{n} \Huge VADEMECUM}\\
%\vspace{6mm}
%{\changefont{cmss}{bx}{n} \Huge de cantos litúrgicos}
%\end{center}

\vspace*{30mm}

\begin{center}
{\changefont{cmss}{bx}{n} \Huge Comentarios a los cantos}
\end{center}
%------------------
\newpage

\small
\setcounter{footnote}{0}
\pagestyle{fancy}
\fancyhead{}
\fancyfoot{}
%\footskip=-1cm

\lhead{\changefont{cmss}{bx}{n} \small Revisión 2013}
\chead{\changefont{cmss}{bx}{n} \small Comentarios}
\rhead{\changefont{cmss}{bx}{n} \small Guillermo Frank}
\rfoot{\changefont{cmss}{bx}{n}\large\thepage}

\section*{\small Adoremos a Dios} \noindent\footnotesize Melod\'\i a surgida, quiz\'as, en el Movimiento de la Palabra (dato a confirmar). Para mayores referencias sobre el Movimiento de la Palabra, ver \textbf{La cancion de la Alianza}.
\section*{\small Alabado sea el Santisimo} \noindent\footnotesize El texto comienza con la ant\'\i fona latina \textit{Adoremus in aeternum Sanctissimum Sacramentum} y luego invoca la inmaculada concepci\'on de Mar\'\i a con palabras que recuerdan al himno latino \textit{Te Matrem Dei laudamus}. Ambos son muy antiguos, y hay documentaci\'on de comienzos del siglo XVII sobre el uso corriente de su traducci\'on \textit{Alabado sea el Sant\'\i simo Sacramento} (San Juan de Ribera, 1609). La melod\'\i a est\'a formada por compases, con un tiempo fuerte seguido de dos d\'ebiles, d\'andole caracter\'\i sticas de \textit{vals} (seg\'un \textit{Oregon Catholic Press}) u otras melod\'\i as populares europeas (\textit{polka}, \textit{monferrina}, etc). Comparte cierto parentesco con melod\'\i as austr\'\i acas del siglo XVIII como \textit{Te deum, Katholisches Gesangbuch} (Viena, 1774) \textbf{Cruz de Cristo}. Sin embargo, todas estas semejanzas podr\'\i an deberse a adaptaciones posteriores de la melod\'\i a. El 
Padre Jos\'e Bevilacqua reconoce un aire franc\'es en la melod\'\i a. En los antiguos cancioneros \emph{Cantiques spirituels divis\'es en deux parties} de 1776 y \emph{Recueil de cantiques spirituels \`{a} l'usage des petits s\'eminaires et autres maisons d'\'education} de 1833 aparece una melod\'ia bajo el nombre \emph{Ouvrages du Seigneur}, que bien merece ser tenida en cuenta como un antecedente de la melod\'\i a (en la opini\'on del Padre Jos\'e Bevilacqua). Por ese motivo se la cataloga como del siglo XVIII (la versi\'on actual es la aparecida en el cancionero \textit{Gloria al Se\~nor} de 1955).
\section*{\small Alabe todo el mundo} \noindent\footnotesize El texto corresponde a la traducción de la ant\'\i fona latina \emph{Laudate omnes gentes, laudate dominum} (Sal 116,1). La melod\'\i a es de la comunidad ecum\'enica de Taiz\'e.
\section*{\small Al atardecer} \noindent\footnotesize El texto del estribillo est\'a basado en los \emph{Avisos espirituales nro. 60} (oraci\'on del alma enamorada) de San Juan de la Cruz (1542-1591). El texto de las estrofas se inspira en pasajes evang\'elicos (ver referencias b\'\i blicas). 
\section*{\small Al\'egrate Mar\'\i a} \noindent\footnotesize Basado en el himno latino \textit{Regina Caeli} (oraci\'on del siglo XII atribuida a San Gregorio Magno). Sustituye el rezo del \textit{Angelus} en el tiempo pascual. 
\section*{\small Arriba los corazones} \noindent\footnotesize San Agust\'\i n vincul\'o este ``sursum corda'' con Col 3,1.
\section*{\small Arriba nuestros ramos} \noindent\footnotesize Canto para la procesi\'on de ramos en el domingo de Pasi\'on. Ver nota \textbf{Canta Jerusalen}.
\section*{\small Bendeciré al Senor} \noindent\footnotesize Texto tambi\'en conocido como \textit{Prueben que bueno}. El texto corresponde al Sal. 33. La Dra. Delia Geraghty segura (en comunicaci\'on personal) que el autor de esta melod\'\i a es An\'\i bal Pappagallo, del Movimiento de la Palabra de Dios. El a\~no de composici\'on se ubica entre 1975 y 1978. Es, en cierto sentido y de acuerdo con la Dra. Mirta Ridruejo, un canto hermano de \textit{La canci\'on de la Alianza}. Ver comentario \textbf{La cancion de la Alianza}.   
\section*{\small Bendici\'on de San Francisco} \noindent Texto conocido como \textit{Bendici\'on de San Francisco}.
\section*{\small Bendigamos al Señor} \noindent Melodía del siglo XVIII atribuida a Pierre de Corbeil, obispo de Sens (1200-1222). Esta melodía acompañaba al himno latino \textit{Orientis Partibus}, que se cantaba en la \emph{Fete de l’Ane} (Fiesta del Asno, 14 de enero) y se relacionaba con el relato evangélico de la huida a Egipto. También, la misma melodía se usó para acompañar el himno latino \emph{Concordi Laetitia} en honor a la Virgen María. 
\section*{\small Bendita sea tu pureza} \noindent\footnotesize Oraci\'on tradicional de la Iglesia. Existen varias melod\'\i as para acompa\~nar a este texto. La primera fue compuesta por Francisco de Madina Igarz\'abal (1907-1972) y apareci\'o en el cancionero \emph{Gloria al Se\~nor} (1955). La segunda corresponde al chamam\'e de Francisco Cerimele, aparecido en el cancionero \emph{Cantemos hermanos con amor II} (1981). Una tercera melod\'\i a es el chamam\'e \emph{Kil\'ometro 11} de Mario del Tr\'ansito Cocomarola (1918-1974), compuesto alrededor del a\~no 1940.
\section*{\small Bendito seas} \noindent Texto lit\'urgico basado en la oraci\'on rab\'\i nica del \textit{hamotzi}.
\section*{\small Bienaventurados} \noindent Melod\'\i a \textit{picardy} (villancico franc\'es del siglo XVII) seg\'un se comenta en \textbf{Junto a la cruz}. El texto corresponde a Alberto Taul\'e.
\section*{\small Cancion del misionero} \noindent Tambi\'en conocido como \textit{Alma misionera}. Este canto fue compuesto por Enrique Garc\'\i a V\'elez, sacerdote cat\'olico de la Arquidi\'ocesis de Le\'on, Guanajuato, M\'exico (padrenri@hotmail.com). Al respecto, el mismo Padre Enrique Garc\'\i a V\'elez comenta: ``Creo que recib\'\i\ una verdadera inspiraci\'on en el a\~no de 1994, cuando para un Congreso Nacional Misonero (CONAMI) nos pidieron una composici\'on para la misi\'on que fuera motivadora para dicho congreso, el cual se llev\'o a cabo en 1995, en mi ciudad natal, Le\'on, Guanajuato. Yo creo que de ah\'\i\ se ha ido extendiendo para muchas partes del mundo, ya que hab\'\i a gente de casi todos los continentes. Yo, hasta ahora que soy sacerdote, me enter\'e que las composiciones hab\'\i a que registrarlas y yo munca lo hice por lo que probablemente alguien se adjudica la canci\'on o le hayan cambiado algunas cosas. A fin de cuentas ese canto es para llevar el evangelio de Cristo.''
\section*{\small Cantad a Mar\'\i a} \noindent\footnotesize La referencia Ecli 43,9 se refiere a Mar\'\i a como ``estrella del alba'', seg\'un el an\'alisis de San Antonio de Padua (\textit{Sermones}). La melod\'\i a de este canto es de Lorenzo Perosi (1872-1956), con el t\'\i tulo original en italiano \emph{Lodate Maria}. Este compositor ocup\'o el cargo de director de la Capella Musicale Pontificia Sistina, nombrado por Leon XIII en 1898. 
\section*{\small Canta Jerusalen} \noindent\footnotesize Corresponde a la ant\'\i fona latina \textit{Lauda Jerusalem} (Sal 147,12). Esta ant\'\i fona se canta durante la procesi\'on de ramos (domingo de Pasi\'on). Ella alaba a Dios ``que envi\'o m\'ultiples palabras a su pueblo, y que, finalmente en esta etapa definitiva, nos ha enviado su Palabra, que es el Hijo'' (L. Alonso Sch\"okel). Por eso, en Sal 147,15 se proclama ``su palabra corre veloz''.  
\section*{\small Cante la Iglesia} \noindent\footnotesize Mat\'\i as Hern\'an Sagreras (mhsagreras@gmail.com), organista de la Bas\'\i lica Sant\'\i simo Sacramento, comenta respecto de este canto: ``el himno se llama \textit{O Pater sancte}. Es un himno ingl\'es para el domingo de Trinidad. Sus primeras l\'\i neas son \textit{Father most holy, mercifull and loving...} La tonada es conocida como \textit{Angers}, con su m\'etrica 11.11.11.5. La melod\'\i a est\'a tomada de la colecci\'on de J.B. Croft.'' La versi\'on m\'as antigua conocida de \textit{Angers} hasta el momento corresponde a la del Antifonario de Chartres del a\~no 1784.
\section*{\small Cantemos al amor} \noindent\footnotesize Himno oficial del XXII Congreso Eucar\'\i stico Internacional (Madrid, 1911). El texto es del Padre Restituto del Valle Ruiz y la m\'usica de Ignacio Busca Sagastizabal. 
\section*{\small Cántico de Caridad} \noindent (ver \textit{Bendigamos al Señor}).
\section*{\small Canto de Maria} \noindent\footnotesize Canto \textit{Magnificat}.
\section*{\small Christus vincit} \noindent\footnotesize El texto corresponde al himno ambrosiano del siglo VIII, con aclamaciones a Cristo Rey de Gloria. La melod\'\i a  se us\'o en Radio Vaticana y se le atribuye su autor\'\i a a Jan Kunc (1933). 
\section*{\small Como Cristo nos am\'o} \noindent\footnotesize Melod\'\i a \emph{Bye and bye} de Charles Albert Tindley (EEUU, 1905). Tambi\'en se la conoce por \textit{El Se\~nor nos da su amor}. 
\section*{\small Coplas de soledad} \noindent\footnotesize Inspirado en la secuencia \textit{Stabat Mater}.
\section*{\small Coplas Yaravi} La primera estrofa se basa en la oraci\'on del fil\'osofo Rabindranath Tagore (India 1861-1941) en su obra Gitanjali.
\section*{\small Cristianos vayamos} \noindent\footnotesize Basado en el himno \textit{Adestes Fideles}. Los manuscritos m\'as antiguos de este himno son del a\~no 1640, encontrados en el palacio del rey Juan IV de Protugal. Sin embargo, la versi\'on m\'as conocida es la de John Francis Wade de 1760. 
\section*{\small Cristo Jesus} Texto compuesto para XXXII Congreso Eucar\'\i stico Internacional (Buenos Aires, 1934).
\section*{\small Cruz de Cristo} \noindent\footnotesize Melod\'\i a de \textit{Te deum, Katholisches Gesangbuch} (Viena, 1774) con el texto \textit{Grosser Gott, wir loben Dich} atribuida a Ignaz Franz. El texto de Ast\'oviza se inspira en \textit{Ecce lignum crucis} que es la ant\'\i fona de adoraci\'on de la cruz del Viernes Santo.
\section*{\small Cuantas gracias te debemos} \noindent\footnotesize Canto (texto y m\'usica del Padre Jos\'e Bevilacqua). En comunicaci\'on personal (a\~no 2013), el Padre Jos\'e Bevilacqua comenta: ``se me inspir\'o en una acci\'on de gracias en d\'\i a del Sagrado Coraz\'on, hace muchos a\~nos en el c\'antico de \mbox{Is 12}''. Efectivamente, la estrofa 1 se corresponde con Is 12,1 y la estrofa 2 con Is 12,3.
\section*{\small Cuerpo y Sangre de Jes\'us} \noindent\footnotesize En la primera estrofa, la palabra ``promesas'' corresponde al domingo de Pascua. Si se la reemplaza por ``esperanza'' corresponder\'\i a a \mbox{1 Cor 11,2-6}.
\section*{\small Despertemos llega Cristo} \noindent\footnotesize El texto de este canto es de Osvaldo Catena. La melod\'\i a del estribillo es una variaci\'on del espiritual negro \emph{Lord, I want to be a Christian}, posiblemente realizada por Osvaldo Catena para adaptarla mejor el texto en espa\~nol. La melod\'\i a apareci\'o publicada en \emph{Folk Songs of the American Negro} (1907), compilada por los hermanos Frederick Work y John W. Work, Jr. Tambi\'en se encuentra en la edici\'on de 1918 de \emph{Religious Folk Songs of the Negro as Sung on the Plantations}, arreglada por los directores musicales del \emph{Hampton Normal and Agricultural Institute}, Hampton, Virginia. M\'as recientemente, apareci\'o publicado en \emph{American Negro Songs: 230 folk songs and spirituals, religious and secular} de John Wesley Work III (1940).  En \emph{Negro Slave Songs in the United States} (1953), Miles Mark Fisher sugiere que este \emph{spiritual} podr\'\i a haberse originado en Virginia alrededor de 1750 debido a 
textos que contienen palabras similares. Desde el punto de vista musical, la melod\'\i a recuerda algunas \emph{Child ballads} de los Apalaches (cf. algunas variantes de \emph{Rare Willie Drowned in Yarrow}, Child ballad nro. 215). Esto es consistente con el origen propuesto por Miles Mark Fisher. En la adaptaci\'on de Osvaldo Catena se conserva la caracter\'\i stica de \emph{call-and-response} t\'\i pica de muchas melod\'\i as afro-americanas (como por ejemplo, \emph{Go down Moses}). 
\section*{\small Dice el Senor, convi\'ertanse} \noindent\footnotesize Aclamaci\'on al evangelio para el tiempo de cuaresma. 
\section*{\small El angel vino} \noindent\footnotesize El texto de este canto corresponde a la oraci\'on del \textit{Angelus}. Est\'a especialmente aconsejado para la liturgia de la Anunciaci\'on (25 de marzo). Se cree que la melod\'\i a es de origen polaco, pero este dato no ha podido ser confirmado. Respecto de c\'omo lleg\'o este canto a la Argentina, nos dice el bolet\'\i n lit\'urgico de la Parroquia San Gabriel Arc\'angel, dirigido por el Padre Ricardo Dotro: ``Este canto... nos ha llegado a trav\'es del Paraguay. Lo cantaban los seminaristas paraguayos en el seminario de Bs. As. hace m\'as de cincuenta a\~nos como un canto t\'\i pico de all\'a, en particular para la fiesta de la Inmaculada Concepción, que es venerada all\'\i\ a orillas del lago Caacup\'e, en pleno Adviento. La melod\'\i a es pegadiza y f\'acil de recordar. El tono de Mi mayor permite darle una sonoridad sencilla, alegre y solemne al mismo tiempo. La recuper\'o el P. Osvaldo Catena, el gran recopilador santafesino, en su libro \textit{
Cantemos hermanos con amor (II)} de 1984. El texto, adem\'as de referirse al evangelio de Lucas cap\'\i tulo I, a veces literalmente, es la versi\'on cantada de la oraci\'on conocida como \textit{Angelus}. Esa oraci\'on se usaba rezar tres veces por d\'\i a, a la ma\~nana, al mediod\'\i a y a la tarde, a\~nadiendo despu\'es de cada s\'uplica un Ave María. La segunda estrofa es palabra por palabra la respuesta de la Pur\'\i sima Virgen al anuncio del Mensajero divino. La tercera est\'a tomada del evangelio de Juan, cap\'\i tulo I vers. 14. El aire del texto es po\'etico pero de estilo popular, pienso que gestado por la gente que encontr\'o en esa melod\'\i a europea, la ocasi\'on de inventar un poema que rima bien y es agradable. El estribillo es especial: a Mar\'\i a se la llama \textit{Virgen Madre, Se\~nora nuestra} en una actitud de profundo respeto de quienes se consideran sus \textit{hijos} y la creen \textit{estrella que anuncia la salvaci\'on}. Mar\'\i a en esta poes\'\i a popular es como el lucero 
del alba que brilla cuando ya el sol ha salido. La Iglesia la llamar\'a en 1979 \textit{estrella de la evangelización}, con una expresi\'on m\'as t\'ecnica y teol\'ogica. El canto no es s\'olo un c\'antico mariano m\'as, sino un aut\'entico canto de Adviento y al preparar el cantoral, Orlando F. Barbieri lo coloc\'o como primer canto del Adviento, con una acertada convicci\'on.'' (Parroquia San Gabriel Arc\'angel, 1$^\circ$ de octubre de 2006)
\section*{\small El pueblo de Dios} \noindent Canto atribuida a Ricardo Roscelli, Di\'ocesis de San Isidro, Argentina. Las referencias m\'as lejanas ubican a este canto dentro de la d\'ecada de 1980.
\section*{\small Encendamos el cirio} \noindent Basado en la ant\'\i fona de procesi\'on de la Vigilia Pascual \textit{Lumen Christi, deo gratias}, en el \textit{Exultet} o Preg\'on Pascual (Ex 13,21-22, Num 14,14, Sab 18,3 se refieren a la ``columna de fuego'', Zac 14,6-9, Sal 138,12, Lc 2,32, Apoc 21,25 se refieren a la luz en la noche) y en el himno \textit{O filii et filiae} (Mt 28,2-8 y 1 Cor 15,54-57).
\section*{\small En memoria tuya} \noindent Contiene una frase del himno latino \textit{Tantum ergo}.
\section*{\small Escondido} \noindent Texto inspirado en el himno latino \textit{Adoro te Devote}, escrito por Santo Tom\'as de Aquino (1225-1274).  
\section*{\small Esta es la luz de Cristo} \noindent Es una versi\'on en espa\~nol de la canci\'on norteamericana \emph{This little light of mine}, compuesta por Harry Dixon Loes (1892-1965) hacia el a\~no 1920. Ambos, el texto original en ingl\'es y la melod\'\i a corresponden a \'el. Las referencias b\'\i blicas Is 49,6 y Hech 13,47 se refieren a la luz de Cristo, que ilumina y salva al mundo. El resto de las referencias ponen el \'enfasis en que cada cristiano es reflejo de esa luz en el mundo. El mensaje de este canto lo hace apropiado para la renovaci\'on de promesas bautismales (domingo de Pascua).
\section*{\small Feliz de ti Mar\'\i a} La melod\'\i a de este canto fue compuesta por Yosef Hadar (1926-2006) alrededor del a\~no 1956, bajo el t\'\i tulo \emph{Erev Shel Shoshanim}. El t\'\i tulo se traduce como \emph{Noche de rosas} o \emph{Noche de lirios}. Tambi\'en se tradujo al \'arabe como \emph{Yarus(O, rose!)}. El texto original de \emph{Erev Shel Shoshanim} es un poema de amor escrito por Moshe Dor (1932-). Luego, la melod\'\i a fue usada con otros textos como \emph{Feliz de Ti Mar\'\i a} de Osvaldo Catena, \emph{B\'enissez le Seigneur, vous tous serviteurs} de la Comunidad de l'Emmanuel, \emph{Tiell\"a ken vaeltaa} de Juhani Forsberg o \emph{Adventtilaulu} del grupo finland\'es Maaseudun Tulevaisuus. En el caso del texto de Osvaldo Catena, la elecci\'on de la melod\'\i a corresponde a la siguiente identificaci\'on de Mar\'\i a: ``como un lirio entre los cardos es mi amada entre las j\'ovenes'' \mbox{(Cant. 2,2).}
\section*{\small Gloria a Dios I} \noindent\footnotesize Gloria de Osvaldo Catena, nro. 45 de \textit{Cantemos hermanos con amor I}. El estribillo de este canto corresponde a Lc 2,14.
\section*{\small Gloria al Senor ha llegado la pascua} \noindent\footnotesize Inspirado en el himno latino \textit{Pascha nostrum}, adaptado por Osvaldo Catena. Este himno se puede incluir como parte del Aleluia pascual, en el siguiente orden: \mbox{aleluia$\rightarrow$pascha nostrum$\rightarrow$aleluia.} La melod\'\i a de este canto corresponde a un fragmento del poema sinf\'onico \emph{Finlandia} (Op. 26) de Jean Sibelius, compuesto en el a\~no 1899. Esta misma melod\'\i a tambi\'en canta con el texto \emph{Nuestra Oraci\'on} de M. Baz\'an, como canto para el ofertorio. 
\section*{\small Gloria eterna} \noindent\footnotesize La melod\'\i a corresponde a \textit{See, the conquering hero comes}, del acto III (nro. 58) del oratorio \textit{Judas Maccab\"aus} de G. Fr. H\"andel (oratorio del a\~no 1746). Sin embargo, en la versi\'on de 1746 a\'un no apareci\'o esta melod\'\i a, sino que H\"andel la incorpor\'o en 1751 a partir de su oratorio \textit{Joshua}.  All\'\i\ musicalizaba la victoria de Otoniel, mientras que en \textit{Judas Maccab\"aus} se refiere a la victoria de Judas sobre Lisias y la posterior purificaci\'on del Templo y dedicaci\'on de su altar. El texto \textit{Gloria eterna} (inspirado en el himno latino \textit{Canticourm Iubilo}) relaciona la purificaci\'on del Templo de Jerusal\'en con el nacimiento de Cristo. ``Alg\'un exegeta observa, adem\'as, que el d\'\i a 25 del noveno mes (1 Mac 4,52) se celebraba la fiesta de la Dedicaci\'on del Templo, instituida por Judas Macabeo en el 164 A.C. La coincidencia de fechas significar\'\i a entonces que con Jes\'us, 
aparecido como luz de Dios en la noche, se realiza verdaderamente la consagraci\'on del templo, el Adviento de Dios a esta tierra'' (Benedicto XVI, 2009). En algunos cancioneros esta melod\'\i a se usa como canto de resurrecci\'on, con el texto \textit{Thine be the glory} (traducido como \textit{A ti la gloria}).
\section*{\small Ha nacido el rey} \noindent\footnotesize Texto basado en el himno \textit{Hark! The Herald Angels sing} de Charles Wesley, aparecido en \textit{Hymns and Sacred Poems} (1739) bajo el t\'\i tulo \textit{Hymn for Christmas-Day}. La melod\'\i a es de Felix Mendelssohn, compuesta en 1840 como parte de \textit{Festgesang (Gutenberg cantata)} para conmemorar la invensi\'on de la imprenta. 
\section*{\small Himno a la cruz} \noindent\footnotesize La referencia b\'\i blica de este canto fue sugerida por el Grupo Pueblo de Dios.
\section*{\small Hoy la Iglesia} \noindent\footnotesize M\'usica del salmo 35 de \textit{Aulcuns pseaumes et cantiques mys en chant} (salterio calvinista de Ginebra, Estrasburgo, 1539), del salmo 67 (mismo salterio, edici\'on de 1562), del himno pascual \textit{Lasst uns erfreuen herlich sehr} (himnario jesuita \textit{Catholische Geistliche Kirchenges\"ange}, Colonia, 1623), en el salmo 135 \textit{Confess Jehovah} (salterio de Henry Ainsworth, 1612) y otros (con sus variaciones). Esta melod\'\i a sirvi\'o para distintos cantos como \textit{Hoy la Iglesia victoriosa}, \textit{Suenen cantos de alegr\'\i a}, \textit{Cristo Jes\'us resucit\'o}, \textit{Oh criaturas del Se\~nor} (texto de San Francisco de As\'\i s, 1225) o en ingl\'e \textit{All Creatures of Our God and King}, \textit{Naciones todas alabad} (texto de Isaac Watts) y \textit{Hymnum canamus gloriae} (``Cantemos un himno de gloria'' de San Beda el Venerable, 673-735). El texto \textit{Hoy la Iglesia victoriosa} se inspira en \textit{Christ is risen 
today} de Charles Wesley (1739).
\section*{\small Jesus Eucaristia} \noindent\footnotesize Himno del X Congreso Eucar\'\i stico Nacional de 2004. 
\section*{\small Junto a la cruz}  Esta melod\'\i a se conoce como \textit{picardy}, un villancico franc\'es del siglo XVII. Apareci\'o publicado en \textit{Chansons Populaires des Provences de France} (vol. IV, Paris, 1860). La melod\'\i a fue una transcripci\'on de Madame Pierre Dupont en Champfleury-Wekerlin bajo el t\'\i tulo \textit{Jesus Christ s'habille en pauvre}, a partir de una canci\'on popular proveniente de Picardy (provincia del norte de Francia). El texto es de Osvaldo Catena, pero tambi\'en se usa el texto de Alberto Taul\'e \textbf{Bienaventurados}. 
\section*{\small Junto a ti Maria} \noindent\footnotesize Ver nota \textbf{Madre de los peregrinos} sobre la maternidad espiritual de Mar\'\i a.
\section*{\small Juntos como hermanos} \noindent\footnotesize La melod\'\i a corresponde al espiritual negro \textit{My Lord, what a morning}, cuyo texto se basa en Mt 24,29-30. Se cree que Ces\'areo Gabar\'ain (1936-1991) se inspir\'o en este pasaje para \textit{Juntos como hermanos}. La melod\'\i a \textit{My Lord, what a morning} fue compuesta por John W. Work III (1901-1967) para el \emph{Festival of Music and Art} de 1956. 
\section*{\small La canci\'on de la Alianza} \noindent\footnotesize Este canto surgi\'o en el Movimiento de la Palabra de Dios (movimiento creado por el Padre Ricardo M\'artensen entre 1973 y 1974, barrio de Flores, Buenos Aires). Su autora es Mirta E. Ridruejo (ridruejom@fibertel.com.ar). De acuerdo con la autora, esta composici\'on fue el resultado de un pedido del Padre Ricardo M\'artensen para musicalizar el texto 1 Jn 4. La obra se finaliz\'o entre 1975 y 1976.
\section*{\small La Virgen Mar\'\i a nos re\'une} \noindent\footnotesize El texto de este canto es ambiguo en referencia a Mar\'\i a, pero quiz\'as se apoya en los dos textos citados, tanto para el banquete (figura de la Eucarist\'\i a), como en la oraci\'on (y escucha de la Palabra).
\section*{\small Los cielos la tierra} \noindent\footnotesize Esta melod\'\i a se conoce como \textit{Himno de Lourdes}. Su origen es desconocido, aunque se cree que proviene de los Pirineos franceses. Abbe Gaignet compuso un himno a Mar\'\i a Inmaculada sobre esta melod\'\i a (Grenoble, 1882). El texto de \textit{Los cielos, la tierra} nos recuerda el \textit{Ave Mar\'\i a de Lourdes}.
\section*{\small Madre de los peregrinos} \noindent\footnotesize Texto para el d\'\i a de la madre (\'ultima estrofa). Este texto, adem\'as, explora algunos aspectos de la tradici\'on de la Iglesia sobre la maternidad espiritual de Mar\'\i a. Por ejemplo, seg\'un Origenes ``en efecto, el que es perfecto ya no vive (por s\'\i\ mismo), es Cristo quien vive en \'el; y porque Cristo vive en \'el, le es dicho a Mar\'\i a con respecto a \'el: \textit{he ah\'\i\ a tu Hijo, a Cristo}'' (Comm. in Joan., t 1,6;PG 14,32 AB).
\section*{\small Maranath\'a} \noindent\footnotesize Tambi\'en conocido como \textit{Ven Esp\'\i ritu de Dios}. Tanto el texto como la melod\'\i a corresponden al canta-autor argentino Ariel Glaser. 
\section*{\small Mar\'\i a, Madre de Dios} \noindent\footnotesize Este canto es de autor an\'onimo. Sin embargo, figura bajo el t\'\i tulo de \textit{Oh Mar\'\i a} en el cancionero \textit{Cantemos en esp\'\i ritu y en verdad} del Movimiento de la Palabra de Dios (re-edici\'on del a\~no 2011). Esto hace suponer que su autor podr\'\i a pertenecer a dicho movimiento. La partitura de este canto corresponde a una transcripci\'on realizada varios a\~nos antes en la parroquia Inmaculada Concepci\'on de Belgrano.    
\section*{\small Mas cerca oh Dios} \noindent\footnotesize Texto inspirado en el poema ingl\'es \textit{Nearer, My God, to Thee} de Sarah Flower Adams (1841), a su vez, basado en Gen 28,11-19. La versi\'on castellana se aparta un poco de \'esta. Las referencias b\'\i blicas corresponden a la versi\'on castellana (ver \textit{Gloria al Se\~nor 2} de 1980 o \textit{Cantar y orar} de 1998). Osvaldo Catena escribi\'o un texto alternativo bajo el t\'\i tulo \textit{Creo en ti, Se\~nor}. La melod\'\i a corresponde al himno \textit{Bethany} de Lowell Mason (1856). 
\section*{\small Mensajero de la paz} \noindent\footnotesize En el nro. 308 de \textit{Cantemos hermanos con amor I} figura como de autor an\'onimo. Este dato debe cambiarse por R. Gonz\'alez. En una comunicaci\'on personal, el Padre Jos\'e Bevilacqua se\~nal\'o que tiempo despu\'es de publicarse \textit{Cantemos hermanos con amor I}, recibi\'o una carta del autor. Aparentemente, el canto fue compuesto en C\'ordoba, Argentina.
\section*{\small Me pongo en tus manos} \noindent\footnotesize Texto basado en \textit{Padre, me pongo en tus manos} (oraci\'on de abandono) de Charles de Foucauld (1858-1916). 
\section*{\small Mira nuestra ofrenda} \noindent\footnotesize Melod\'\i a de origen italiano, conocida como \textit{Guarda questa offerta}. Se la atribuye al grupo GEN rosso, del movimiento de los focolares. El movimiento focolar fue fundado por Chiara Lubich (1920-2008) en 1943 y aprobado por la Santa Sede en 1962. Su objetivo es impulsar el ecumenismo y el di\'alogo entre personas. Los grupos GEN (Nueva Generaci\'on) son el resultado de un llamado de Chiara Lubich a los j\'ovenes en 1966.
\section*{\small Mirarte solo a ti Senor} \noindent\footnotesize La referencia b\'\i blica Lc 9,61-62 corresponde a reemplazar la palabra ``mirarte'' por ``seguirte''. El texto original corresponde a \textit{We worship only You} (traducido por \textit{Seguirte s\'olo a T\'\i }) de Jorge E. Maldonado y publicado en ``The Covenant Hymnal''. La melod\'\i a es an\'onima (quiz\'as o\'\i da por Jorge E. Maldonado en Ecuador).  
\section*{\small Mundo feliz} \noindent\footnotesize La melod\'\i a de \textit{Joy to the World} (traducido por ``Mundo feliz'') es atribuida a G. Fr. H\"andel (1685-1759). Esto se debe a que parte del texto coincide con el recitativo \textit{Comfort Ye} de su oratorio \textit{Messiah} y las primeras cuatro notas coinciden con \textit{Lift up your heads} y \textit{Glory to God} de ese mismo oratorio. La melod\'\i a fue posteriormente arreglada  y adaptada al texto de Isaac Watts \textit{Joy to the World} por Lowell Mason (1839).
\section*{\small Noche de paz} \noindent\footnotesize Traducci\'on de \textit{Stille Nacht, heilige Nacht}, escrito por Joseph Mohr y con melod\'\i a de Franz Xaver Gruber (texto de Joseph Mohr del a\~no 1816).
\section*{\small Nuestro mana} \noindent\footnotesize Texto y melod\'\i a de Guillermo Ortiz (Parroquia Inmaculada Concepci\'on, Belgrano, Buenos Aires, a\~no 2007). Su uso lit\'urgico est\'a prescripto para adoraci\'on eucar\'\i stica o como ant\'\i fona post-comuni\'on durante la misa.
\section*{\small Nueva vida} \noindent\footnotesize Texto y m\'usica de Ces\'areo Gabar\'ain (1973). Este canto se emplea en la liturgia bautismal porque expresa claramente los efectos de este sacramento y sus gestos. 
\section*{\small Oh buen Jesus} \noindent\footnotesize Texto y m\'usica del Hno. Le\'on de Jes\'us. 
\section*{\small Oh Maria} \noindent\footnotesize La melod\'\i a de \emph{Oh Mar\'\i a} se encontr\'o en el cancionero \emph{Paderborn Gesangbuch} (Paderborn-Alemania, 1765). Se la conoce como melod\'\i a \emph{Daily, daily} (de Paderborn) porque tradicionalmente acompa\~n\'o el texto \emph{Daily, daily sing to Mary} (traducci\'on del hinmo latino \emph{Omni die dic Mariae}). Se cree, sin embargo, que la melod\'\i a puede ser anterior, quiz\'as de Francia o Alemania. Con el tiempo, la melod\'\i a se us\'o alternativamente con diferentes textos, como \emph{Ye servants of God, your Master proclaim}, \emph{Ye who own the Faith of Jesus} (esta es una de las varias versiones presentadas por el himnario anglicano) o en una de las versiones de \emph{There's a wideness in God's mercy}. En el caso del himno latino \emph{Omni die dic Mariae}, este texto se atribuye a San Bernado de Cluny (monje benedictino del siglo XII) pero su difusi\'on corresponde a San Casimiro (1458-1483). Aqu\'\i , en Argentina, esta melod\'\i 
a se puede cantar tambi\'en con el texto de \textit{Virgencita de Luj\'an} para el 8 de mayo o en fiestas patrias. 
\section*{\small Oh Santisima} \noindent\footnotesize El texto corresponde al himno latino \textit{O Sanctissima} (cantado en las fiestas marianas).  La melod\'\i a es conocida como himno ``marinero siciliano''. Se public\'o por primera vez en \emph{The European Magazine and London Review}, Vol. 22, pp.385-386 (noviembre de 1792). Se titul\'o \emph{Sicilian Mariners's Hymn to the Virgin}. El Dr. Byron E. Underwood destaca (en su art\'\i culo \emph{The Earliest Source of the Sicilian Mariners' Hymn}, publicado en \emph{The Hymn} 27:3 (julio de 1976) p.75-84) que es posible que Charles Burney (1727-1814) haya oido la melodi\'\i a en sus viajes musicales por Italia. Otros sostienen que fue el poeta alem\'an Gottfried Herder quien la oy\'o durante su viaje a Italia en 1788-1789.  La melod\'\i a se us\'o tambi\'en para el villancico alem\'an \emph{O Du Fr\"ohliche}. Se la conoce adem\'as con otros textos bajo los t\'\i tulos \emph{Lord, dismiss us with Thy blessing} y \emph{Saviour Like a Shepherd Lead Us}. Por 
otro lado, se cree que la melod\'\i a de \emph{We shall Overcome} (con texto de Charles Tindley) se basa en este himno ``marinero siciliano''. Ludwig van Beethoven realiz\'o su propio arreglo de la melod\'\i a \textit{O Sanctissima} en su colecci\'on de \emph{12 Canciones folkl\'oricas variadas} (WoO 157/4). 
\section*{\small Oh Victima inmolada} \noindent\footnotesize Versi\'on de Monse\~nor Enrique Rau, basado en el hinmo latino \textit{Ave verum} y en el texto \textit{Ach Herr, mich armen S\"under} de Cyriakus Schneegass (1597). La melod\'\i a es de Hans Leo Hassler (1601), armonizado por J. S. Bach para la \textit{Pasi\'on seg\'un San Mateo}. Esta melod\'\i a apareci\'o publicada en la colecci\'on de himnos \textit{Praxis Pietatis Melica} (1656) de Johann Cr\"uger, con el t\'\i tulo \textit{O Haupt voll Blut und Wunden} (traducido como \textit{Oh rostro ensangrentado}) y basada en el himno latino \textit{Salve mundi salutare}. Este himno latino tambi\'en se conoce como \textit{Jesu dulcis amor meus} (con algunas alteraciones). El texto \textit{Ach Herr, mich armen S\"under} se basa en el salmo 6, que la tradici\'on cristiana lo incluye dentro de los siete salmos llamados ``penitenciales'' (6,31,37,50,101,129,142). Por eso est\'a prescripto para el Viernes Santo (\textit{O V\'\i ctima inmolada}) y tercer 
domingo despu\'es de Sant\'\i sima Trinidad (\textit{Ach Herr, mich armen S\"under}).
\section*{\small Panis angelicus} \noindent\footnotesize Este texto es uno de los tres himnos escritos por Santo Tom\'as de Aquino para la Fiesta de Corpus Christi (siglo XIII). En 1872, C\'esar Franck compuso la melod\'\i a actual y la incorpor\'o en su \textit{Messe \`a troi voix}.
\section*{\small Pastores de la montana} \noindent\footnotesize La melod\'\i a corresponde al villancico franc\'es \textit{Les anges dans nos campagnes} del siglo XVIII (o quiz\'as anterior). El texto original franc\'es fue traducido a distintos idiomas y sufri\'o ciertas adaptaciones. En espa\~nol existen las versiones \textit{Pastores de la monta\~na}, \textit{\'Angeles cantando est\'an}, \textit{C\'antico angelical} y otras. Tanto el texto original como sus traducciones contienen la versi\'on latina de la alabanza hecha por una multitud del ''ej\'ercito celestial'': \textit{!`Gloria in excelsis Deo!} (Lc 2,14). En la versi\'on \textit{Pastores de la monta\~na}, la tercera estrofa incluye adem\'as la afirmaci\'on ``Dios de Dios y luz de luz'' (\textit{Deum de Deo, lumen de lumine}) del credo Niceno-Constantinopolitano (a\~nos 325 y 381).
\section*{\small Pescador de hombres} \noindent\footnotesize Texto y m\'usica de Ces\'areo Gabar\'ain Azurmendi (1936-1991). Este canto se public\'o en 1979 y se difundi\'o hasta convertirse en uno de los dos cantos favoritos de Juan Pablo II (el otro lleva la melod\'\i a de \textit{Victoria, tu reinar\'as}). El Padre Ireneusz Chmielewski, misionero y predicador en Polonia, llev\'o la canci\'on por toda Polonia, ense\~n\'andola a la gente, especialmente a la juventud polaca. All\'\i\ se la conoci\'o con el nombre de \textit{Barka}. Es probable que Juan Pablo II la escuchara y recordara de aquella \'epoca.

\section*{\small Pueblo de Dios} \noindent\footnotesize La melod\'\i a corresponde al villancico de navidad \emph{Dulce Bel\'en} u \textit{Oi Betleem}, atribuido al Padre Jos\'e Antonio Donostia (1886-1956) por ser \'el quien lo public\'o junto con otras obras. Sin embargo, Patxi Oroz afirma que esta melod\'\i a se encuentra documentada ya desde 1736 bajo el texto \textit{La fourmi et la sauterelle} (ver Musiker, cuadernos de m\'usica, nro.15 p. 53-72, a\~no 2007). La melod\'\i a tambi\'en fue tomada por el Rev. Sabine Baring-Gould (1834-1924) para el himno \textit{The infant king} o \textit{Sing lullaby}. Cesar Franck (1822-1890) la utiliz\'o como m\'usica de ofertorio. Por su origen, se cree que la melod\'\i a es de la regi\'on vasca-francesa. 
\section*{\small Pueblo de reyes} \noindent\footnotesize Las referencias b\'\i blicas dadas son las que inspiraron la ant\'\i fona. Las estrofas ``recogen los diferentes t\'\i tulos mesi\'anicos y reales que la Escritura aplica a Cristo'' (Lucien Deiss). Ver cancionero \textit{Pueblo de Reyes} de Lucien Deiss para mayores referencias.
\section*{\small Que nino es este} \noindent\footnotesize Melod\'\i a de \textit{Greensleeves}, aparecida en Inglaterra hacia 1580. \textit{?`Qu\'e ni\~no es este?} es un texto adaptado del himno \textit{What child is this?} de William Chatterton Dix (1837-1898).
\section*{\small Que resuene por la tierra} \noindent\footnotesize Melod\'\i a an\'onima publicada en Londres en 1708, en la colecci\'on de himnos llamada \textit{Lyra Davidica}. Este texto es de Osvaldo Catena, inspirado en el \textit{Easter Hymn} de Samuel Longfellow.
\section*{\small Saber que vendras} \noindent\footnotesize La melod\'\i a de este canto corresponde a \textit{Blowin' in the wind} de R. A. Zimmermann (Bob Dylan). El texto es de Jes\'us Garc\'\i a Torralba. Este canto apareci\'o publicado en el cancionero \textit{Adviento y liberaci\'on} de Juan Antonio Espinosa (1970). En 1997, Juan Pablo II le respondi\'o p\'ublicamente a Bob Dylan respecto del texto original de  \textit{Blowin' in the wind}. \\

\noindent El texto de Torralba no sigue la doctrina católica, dado que en su segunda estrofa dice que pedimos ``por el odio de los que mueren sin fe''. La Iglesia pide por los que no tienen fe el Viernes Santo, pero \emph{nunca} avala su ``odio''. Como se dijo más arriba, este texto salió publicado en un cancionero llamado \textit{Adviento y liberación} conducido por el sacerdote Espinosa, seguidor de la ``teoría de la liberación''. Esta teoría de los '70 fue declarada herética hace unos años por justificar la violencia y por tener ``serias falencias metodológicas'' en su construcción. Ante estas circunstancias, el canto fue incluido en nuestro cancionero ``porque a la gente le gusta'', pero se modificó la frase más polémica, para no ser contraria a la doctrina católica.

\section*{\small Salve Maria} \noindent\footnotesize Melod\'\i a atribuida a los grupos GEN, Movimiento Focolar \textbf{Mira nuestra ofrenda}. El texto corresponde a \textit{Salve Regina}. Ver nota complementaria de \textit{Salve oh Reina}.
\section*{\small Salve oh Reina} \noindent\footnotesize Inspirado en el himno latino \textit{Salve Regina} atribuido a Hermann Contractus (1013-1054). Prescripto para rezarse durante el tiempo ordinario, desde Sant\'\i sima Trinidad hasta Cristo Rey. El Papa Leon XIII prescribi\'o su recitaci\'on despu\'es de cada misa rezada.
\section*{\small Santa Maria del camino} \noindent\footnotesize No est\'a clara la referencia b\'\i blica de este texto, pero la devoci\'on a Santa Mar\'\i a del Camino se corresponde con la visitaci\'on a su prima Isabel. 
\section*{\small Santo} \noindent\footnotesize Las referencias Is 6,3 y Apoc 4,8 corresponden al \textit{Sanctus}. Sal 117,26 y Mt 21,9 corresponden al \textit{Benedictus}. Ambos se cantan juntos en la liturgia actual. El \emph{Santo j\'oven} surgi\'o en el Movimiento de la Palabra y fue compuesto por Sergio Sanchez y Patricia (Lule) Yaquino.
\section*{\small S\'e como el grano} \noindent\footnotesize Melod\'\i a compuesta por Mar\'\i a Marta Bianco, del Movimiento de la Palabra de Dios (este dato fue suministrado por la Dra. Mirtha Ridruejo y confirmado por Delia Geraghty). Ver comentario a \textbf{La canci\'on de la Alianza}.
\section*{\small Secuencia Pentecostes} \noindent\footnotesize Basado en la secuencia \textit{Veni, Sancte Spiritus}, atribuido a Stephen Langton (1150-1228), Arzobispo de Canterbury.
\section*{\small Secuencia Stabat Mater} \noindent\footnotesize Texto atribuido a Inocencio III y Jacopone da Todi. Este texto fue adaptado para cantarse con la melod\'\i a de N\'estor Gallego.
\section*{\small Senor a ti clamamos} \noindent\footnotesize Melod\'\i a del villancico de navidad \textit{Laissez paistre vos Bestes} de Marc-Antoine Charpentier (1643-1704).
\section*{\small Senor haz de nosotros} \noindent\footnotesize El texto corresponde a la oraci\'on de San Francisco de As\'\i s.  
\section*{\small Si yo el Maestro} La melod\'\i a de este canto es la misma que \textit{Me pongo en tus manos} de Ra\'ul Canali. El estribillo es reemplazado por la referencia b\'\i blica \mbox{Jn 13,14}. Las estrofas, si se cantan, son las mismas de \textit{Me pongo en tus manos}.
\section*{\small Si yo no tengo amor} \noindent\footnotesize El texto corresponde a una adaptaci\'on del \textit{Himno de la caridad} de San Pablo (1 Cor 13,1-13). Tanto el texto (adaptado) y la m\'usica se atribuyen al Padre Jorge Baylach Planas (1922-2001). La \'unica referencia a esta autor\'\i a se encuentra dada en el cancionero \emph{Cantemos al Dios de la vida} del Padre Sergio Gruppo (misionero de la Consolata, 1930-2002), 4$^\circ$ edici\'on, Santa Fe de Bogot\'a, Colombia (1996). All\'\i\ se consigna como autor (de texto y m\'usica) a ``J. Baylach''. Este mismo nombre aparece en la obra \emph{Se\~nor, ten piedad de nosotros} del mismo cancionero, con la indicaci\'on ``J. Baylach - Ecuador''. Podr\'\i a tratarse de Jorge Baylach, o bien, Jos\'e-Oriol Baylach (hermano de Jorge Baylach), ambos sacerdotes de la familia vicenciana y llegados a Ecuador en el a\~no 1946, provenientes de Espa\~na. Sin embargo, Jorge Baylach fue quien tuvo actuaci\'on musical. De acuerdo con la \emph{Semblanza del P. Jorge 
Baylach Planas, sacerdote de la Congregaci\'on de la Misi\'on} escrita por el P. Gonzalo Mart\'\i nez S., C.M. (\emph{CLAPVI Vicentian Journals and publications, Paper 111 (2001)}), la producci\'on musical m\'as intensa del P. Jorge Baylach comenz\'o en el a\~no 1958. Para la d\'ecada del '70 el canto \emph{Si yo no tengo amor} hab\'ia aparecido en Buenos Aires, seg\'un indica el Dr. Juan Veniard en \emph{La m\'usica en la Iglesia}, CIAFIC ediciones, Bs.As. (2011), p\'agina 311. En 1981 ya se hab\'\i a grabado para los libros de canto \emph{Cantemos hermanos con amor}, seg\'un se\~nala el Padre Jos\'e Bevilacqua. Por lo tanto, es probable que \emph{Si yo no tengo amor} haya sido compuesto en la d\'ecada del '60 y difundido extensamente en la d\'ecada del '70. Hay que se\~nalar que tanto en \emph{Cantemos hermanos con amor} como en las publicaciones de \emph{Oregon Catholic Press} y \emph{GIA publications}, este canto figura como an\'onimo. El Padre Jos\'e Bevilacqua, en comunicaci\'on personal, lo ubicaba 
como de origen espa\~nol, al ``estilo'' de los cantos del Camino Neocatecumenal. Evidentemente, el ``estilo'' es propio de la m\'usica religiosa surgida entre los a\~nos '60 y '70, precisamente cuando comenz\'o el Camino Neocatecumenal. Pero, adem\'as, hay que aclarar que Jorge Baylach naci\'o y creci\'o en Barcelona, Espa\~na, y desarroll\'o sus estudios en Francia y Madrid. Esto explicar\'\i a la confusi\'on con el origen espa\~nol del canto, a pesar de estar m\'as difundido en latino-am\'erica (por haberse compuesto en Ecuador). En Espa\~na, este canto se reemplaza por \emph{Si me falta el amor} de Francisco Palaz\'on.



\section*{\small Soplo de Dios} \noindent\footnotesize El texto corresponde a Osvaldo Catena. La melod\'\i a es conocida como \emph{V\aa rvindar friska} (brisas de primavera), de origen tradicional sueco. Esta melod\'\i a se canta en suecia en la noche del 30 de abril, fiesta de \emph{Valborgsmassoafton} (noche de Walpurgis) para marcar el fin del invierno y el comienzo de la primavera (de ah\'\i\ su nombre). Es una fiesta antigua escandinava de origen pagano, aunque en Europa se cristianiz\'o luego de que el 1 de mayo del a\~no 870 fuera el d\'\i a en que se transladaran las reliquias de Santa Walpurgis. La melod\'\i a acompa\~na tambi\'en a otros textos como \emph{Den stackars Anna eller Molltoner fr\aa n Norrland} de la poeta sueca Julia Nyberg (1784-1854), \emph{Kiefern im Wind} de Walter Scherf (1920-) y \emph{Vorvindar gladir} de una danza t\'\i pica de Islandia. Se suele clasificar a la melod\'\i a como \emph{hambo-polska} o \emph{leksand-polska} seg\'un la variante de que se trate. El compositor 
sueco Gunner Fredrik De Frumerie (1908-1987) realiz\'o unas variaciones sinf\'onicas (Op. 25) sobre la melod\'\i a de \emph{V\aa rvindar friska} (en el a\~no 1940-41). 
\section*{\small Soy peregrino} \noindent\footnotesize El texto corresponde a Osvaldo Catena, inspirado en el himno afro-americano llamado \emph{Poor wayfaring stranger}, o, \emph{I am a poor wayfaring stranger}, o simplemente, \emph{Wayfaring stranger}. El texto original del himno es de origen desconocido, aunque se atribuye a Richard Allen (1760-1831), quien lo habr\'\i a compuesto ya cercano a su muerte. Richard Allen fund\'o la \emph{African Methodist Episcopal Church} en 1816. Si bien esta autor\'\i a no puede ser verificada, se cree m\'as probable que provenga de una tradici\'on oral surgida alrededor de 1780, pero publicada tard\'\i amente en 1867 (\emph{Revival and camp meeting minstrel}, Philadelphia - Perkinpine \& Higgins, p\'ag. 272). La melod\'\i a aparece impresa por primera vez en 1882 (\emph{Plantation melodies, book of negro folk songs} por el Rev. Marshall W. Taylor). Sin embargo, la versi\'on de 1882 no es la que comunmente se utiliza en los himnarios y cancioneros, sino el arreglo de 
Charlie D. Tillman en \emph{The Revival} de 1891. Para obtener m\'as detalles al respecto, se puede consultar el art\'\i culo de John F. Garst, \emph{Poor Wayfaring Stranger-Early Publications}, en \emph{ The Hymn} v. 31, no. 2, p.97-101 (1980).
\section*{\small Suenen campanas} \noindent\footnotesize Inspirado en el \textit{Exultet} o Preg\'on Pascual.
\section*{\small Suenen cantos} Ver nota complementaria de \textit{Hoy la Iglesia victoriosa} respecto de la melod\'\i a (misma melod\'\i a). El texto est\'a inspirado en \textit{Hymnum canamus gloriae} de San Beda el Venerable (673-735). Es un himno para Ascenci\'on del Se\~nor.
\section*{\small Tan cerca de m\'\i } \noindent\footnotesize La letra y la m\'usica de este canto son de Luis Alfredo D\'\i az Britos, incluido en su disco \emph{Baja a Dios de las nubes} de 1979. Sin embargo, el Padre Ces\'areo Gabar\'ain lo transcribi\'o en su \'album \emph{La misa es una fiesta} desconociendo su autor\'\i a, por lo que debi\'o firmarlo como ``pentecostales - C. Gabar\'ain''. Aparentemente, el Padre Gabar\'ain recibi\'o la melod\'\i a de las iglesias pentecostales y de forma oral, sin una partitura escrita. El texto original del canto (de Luis Alfredo D\'\i az Britos) encuentra algunas peque\~nas variantes en la transcripci\'on de Gabar\'ain. Un cambio notorio es ``No busques a Cristo en lo alto'' (Luis Alfredo D\'\i az) por ``Ya no busco a Cristo en las alturas'' (Garab\'ain) y la supresi\'on de ``llenos de ceguera espiritual''. El texto as\'\i\ arreglado por Garab\'ain intenta eliminar cualquier motivo de discusi\'on dogm\'atica cat\'olica.
\section*{\small Tan sublime sacramento} \noindent\footnotesize Esta melod\'\i a es uno de los varios \textit{Tantum ergo} que compuso Franz Joseph Haydn (1732-1809) y lleva el n\'umero de cat\'alogo (Hoboken-Verzeichnis) Hob~XXIII~c:G5 bajo el t\'\i tulo \textit{Pange lingua} (Pange lingua gloriosi, adagio en Sol Mayor). El texto corresponde a Santo Tomas de Aquino (a\~no 1264). ``Tantum ergo'' son las palabras iniciales de la pen\'ultima estrofa del himno de v\'\i speras \textit{Pange lingua} para la fiesta de Corpus Christi. Las dos \'ultimas estrofas de este himno se prescriben para la bendici\'on del Sant\'\i simo Sacramento. Adem\'as, recitar este himno durante el Triduo pascual goza de indulgencia plenaria. En su traducci\'on castellana, el \textit{Tantum ergo} figura como \textit{Tan sublime sacramento} o simplemente \textit{Himno eucar\'\i stico}.
\section*{\small Te adoramos Cristo} \noindent\footnotesize El texto es la traducci\'on  de la antigua ant\'\i fona latina \textit{Adoramus te, Christe, et benedicimus tibi, quia per sanctam crucem tuam redemisti mundum}. La melod\'\i a actual fue compuesta por Humberto Facal. Se usa como aclamaci\'on al evangelio durante el tiempo de cuaresma, en el rezo del via crucis y alternativamente para la adoraci\'on de la cruz el viernes santo. 
\section*{\small Te adoramos} \noindent\footnotesize Melod\'\i a compuesta por el Mro. Giuseppe Caruana (1880-1931) para la Fiesta de Corpus Christi, durante el XXIV Congreso Eucar\'\i stico Internacional de Malta (1913).  La melod\'\i a fue encargada por Dun Karm Psaila (1871-1961), quien compuso el texto en idioma malt\'es, con el t\'\i tulo \textit{Nadurawk Ja Hobz tas-Sema}. El himno pas\'o a Italia bajo el t\'\i tulo \textit{T'adoriam ostia divina}.   
\section*{\small Te ofrecemos Padre nuestro I (vidala)} \noindent\footnotesize Las referencias b\'\i blicas de este canto corresponden a la versi\'on del cancionero \textit{Cantemos hermanos con amor I}, con m\'usica de Manuel Acosta Villafa\~ne (1902-1956). La melod\'\i a original se titula \textit{Vidala del Culampaj\'a}. Respecto de su autor, Don Manuel Acosta Villafa\~ne, el profesor Ra\'ul E. Cano nos informa: ``No es tan conocido en cambio el hecho de que curs\'o estudios en el seminario, donde lleg\'o hasta tercer a\~no. Fue sin dudas el suyo, siempre un esp\'\i ritu prof\'undamente religioso, lo cual no s\'olo se evidenci\'o en los serenos y bondadosos rasgos de su car\'acter y en los actos de toda su vida, sino tambi\'en en sus canciones.''
\section*{\small Te ofrecemos Padre nuestro II}  \noindent\footnotesize Las referencias b\'\i blicas de este canto corresponden al texto de Carlos Mej\'\i a Godoy. Si bien la obra original tiene una melod\'\i a propia (tambi\'en compuesta por Carlos Mej\'\i a Godoy para la \textit{Misa Campesina Nicarag\"uense}), aqu\'\i\ se canta con una melod\'\i a  an\'onima (transcripta por Luis Caparra). La melod\'\i a original puede encontrarse en \textit{El himnario presbiterano}. El texto completo de la misa nicarag\"uense no debe considerarse exeg\'etico, ya que presenta serios cuestionamientos respecto de su fidelidad al Evangelio, espec\'\i ficamente en cuanto al mensaje de amor dado por Cristo.
\section*{\small Toda de Dios} \noindent\footnotesize Obra compuesta por el Padre Osvaldo Catena (1920-1986). El Padre Jos\'e Bevilacqua nos informa que \'esta fua la \'ultima composici\'on del Padre Catena antes de fallecer.
\section*{\small Toda la tierra levante} \noindent\footnotesize Basado en \textit{O filii et filiae}, variaciones sobre la melod\'\i a publicada en Paris en 1623, en la colecci\'on \textit{Airs sur les hymnes sacrez, odes et 	no\"els}. El texto de \textit{O filii et filiae} se atribuye a Jean Tisserant. 
\section*{\small Un dia la vere} \noindent\footnotesize El texto de Jos\'e Costamagna est\'a basado en el poema \textit{Andr\`o a vederla un d\`\i}. Este poema con la melod\'\i a de \textit{Un d\'\i a la ver\'e} fue compuesto por Pierre (Pietro) Janin, padre de la Sociedad de Mar\'\i a, fundada por Jean-Claude Colin (1790-1875). En abril de 1853, la comunidad del colegio Marista de Langogne (sur de Francia) se dirige en peregrinaci\'on a Puy para la clausura del jubileo. El padre Janin en la v\'\i spera, camina a pie en la nieve, con otros padres. En esta ocasi\'on canta con tal empe\~no que sufre una p\'erdida de la voz durante 15 d\'\i as, en los cuales escribe un largo poema en el que pide a la Virgen le devuelva la voz  y pueda cantar piadosamente en su honor ``porque mis canciones, ya sabes, ellas son mi vida, la forma en que me expreso, mi oraci\'on (...) todos mis acentos son para t\'\i''. Unas semanas m\'as tarde, cerrando el mes de Mar\'\i a (mayo), compone en el aire de lo que es una pastoral (\
textit{le ciel en est le prix, il cielo ne \`e il premio}), un canto llamado entonces \textit{J'irais la voir, un jour} (\textit{Andr\`o a vederla un d\`\i}). El t\'\i tulo original era \textit{Un radieux espoir} (\textit{Un radiante futuro}) y el texto comprend\'\i a cinco estrofas, la misma cantidad de letras que forma el nombre Mar\'\i a. Este canto pronto se populariz\'o por toda Italia. En el texto encontramos cierta referencia a la frase ``y despu\'es de este destierro mu\'estranos a Jes\'us'' proveniente del himno latino  \textit{Salve Regina}. El contenido general de la obra lo hace apropiado para la fiesta de la Asunci\'on de Mar\'\i a (15 de agosto).
\section*{\small Victoria} \noindent\footnotesize Esta referencia b\'\i blica se debe a la tradici\'on patr\'\i stica. San Justino, Tertuliano, San Agustin, Casiodoro y otros, interpretaron ``el Se\~nor reina'' (Sal 95,10) por ``el Se\~nor reina desde lo alto del madero''. El himno de Venancio Fortunato \textit{Vexilla Regis} usa esta interpretaci\'on y \textit{Victoria, tu reinar\'as} se inspira en \'el. Este himno se cantaba en lugar del Aleluia, el Viernes Santo. El himno \textit{O crux, ave, spes unica} est\'a emparentado con el \textit{Vexilla Regis}. El autor del texto actual de \textit{Victoria} figura como an\'onimo en \textit{Cantemos hermanos con amor II} y debe corregirse por Manuel Fernandez. El texto original, en franc\'es, corresponde a Andr\'e Losay (1913-1984). El mismo Andr\'e Losay fue quien, estando en el campo de prisioneros Stalag III B (F\"urstenberg-Oder) en 1940, escuch\'o la melod\'\i a de unos prisioneros polacos y la transcribi\'o (agreg\'andole el texto de su autor\'\i a y titul\'
andola \emph{Victoire}). Posteriormente, el Padre David Julien la revis\'o y armoniz\'o. En muchos cancioneros figura como compuesta por David Julien. Sin embargo, la melod\'\i a original entonada por los prisioneros era \textit{G\'oralu, czy ci nie \.{z}al}, que hace referencia a las lamentaciones de un monta\~n\'es que debe dejar su tierra natal. Se cree que esta melod\'\i a es originaria de las monta\~nas \textit{Tatry}, al sur de Polonia. El compositor Jozef Swider (1930-) realiz\'o un arreglo coral sobre esta melod\'\i a.    
\section*{\small Vine a alabar a Dios} \noindent\footnotesize Basado en el gospel \textit{I just came to praise the Lord} de Wayne Romero (1975).
\section*{\small Vuelve a mi} \noindent\footnotesize Adaptaci\'on del texto y la melod\'\i a \textit{Hosea} de Gregory Norbet o.s.b. (1972).



\end{document}          
